% !TeX program = lualatex
% Using VimTeX, you need to reload the plugin (\lx) after having saved the document in order to use LuaLaTeX (thanks to the line above)

\documentclass[a4paper]{article}

% Expanded on 2025-10-24 at 01:33:03.

\usepackage{../../style}

\title{Quantum crypto}
\author{Joachim Favre}
\date{Vendredi 24 octobre 2025}

\begin{document}
\maketitle

\lecture{12}{2025-10-16}{Used to be alèd}{
\begin{itemize}[left=0pt]
    \item Complete proof of the leftover hashing lemma.
    \item Explanation of the pretty-good measurement theorem.
\end{itemize}

}

\begin{parag}{}
    \begin{subparag}{Complete proof}
        We now make a complete proof, with an arbitrary $E$. It is not necessary to be able to redo this proof, but it is good to understand the simplified proof from above.

        Hence, let $\rho_{XE} = \sum_{x} p_x \ket{x}\bra{x}_X \otimes \rho_x^E$ be an arbitrary $k$-source. We define: 
        \[\rho_{XYE} = \sum_{x} p_x \ket{x}\bra{x}_X \otimes \left(\frac{1}{2^d} \sum_{y} \ket{y}\bra{y}\right) \otimes \rho_x^E,\]
        \[\rho_{ZYE} = \sum_{x, y} p_k \ket{\text{Est}\left(x, y\right)}\bra{\text{Est}\left(x, y\right)} \otimes \frac{1}{2^d} \ket{y}\bra{y} \otimes \rho_x^E.\]

        Our goal is to show that, leaving $\rho = \sum_{x} p_x \rho_x$: 
        \[S = \left\|\rho_{ZYE} - \frac{1}{2^m} I_Z \otimes \frac{1}{2^d} I_Y \otimes \rho\right\|_{tr} \leq \epsilon.\]
        Note that we can write: 
        \[\rho_{ZYE} = \sum_{z, y} \ket{z}\bra{z} \otimes \ket{y}\bra{y} \otimes \rho_{zy}, \mathspace \text{where } \rho_{zy} = \frac{1}{2^d} \sum_{x: f_y\left(x\right) = z} p_x \rho_x.\]
        
        Hence, by properties of the trace distance: 
        \autoeq{S = \left\|\sum_{z, y} \ket{z}\bra{z} \otimes \ket{y}\bra{y} \otimes \rho_{zy} - \frac{1}{2^{d + m}} \ket{z}\bra{z} \otimes \ket{y}\bra{y} \otimes \rho\right\|_{tr} = \sum_{z, y} \left\|\ket{z}\bra{z} \otimes \ket{y}\bra{y} \otimes \rho_{zy} - \frac{1}{2^{d + m}} \ket{z}\bra{z} \otimes \ket{y}\bra{y} \otimes \rho\right\|_{tr} = \sum_{z, y} \left\|\rho_{zy} - \frac{1}{2^{d + m}} \rho\right\|_{tr}}
        
        Now, Cauchy-Schwarz tells us that $\Tr\left(X Y^{\dagger}\right) \leq \sqrt{\Tr\left(X X^{\dagger}\right) \Tr\left(Y Y^{\dagger}\right)}$. Hence: 
        \autoeq{\left\|X\right\|_1 = \Tr\left(\left|X\right|\right) = \Tr\left(\rho^{1/4} \rho^{-1/4} \left|X\right| \rho^{-1/4} \rho^{1/4}\right) = \Tr\left(\rho^{-1/4} \left|X\right| \rho^{-1/4} \rho^{1/2}\right) \leq \sqrt{\Tr\left(\left(\rho^{-1/4} \left|X\right| \rho^{-1/4}\right)^2\right)} \underbrace{\sqrt{\Tr\left(\left(\rho^{1/2}\right)^2\right)}}_{= 1} = \sqrt{\Tr\left(\rho^{-1/2} \left|X\right| \rho^{-1/2} \left|X\right|\right)}.}

        This tells us that: 
        \autoeq{S = \frac{1}{2}\sum_{z, y} \left\|\rho_{zy} - \frac{1}{2^{d + m}} \rho\right\|_1 \leq \frac{1}{2} \sum_{z, y} \underbrace{\sqrt{\Tr\left(\rho^{-1/2} \left(\rho_{zy} - \frac{1}{2^{d + m}} \rho\right) \rho^{-1/2} \left(\rho_{zy} - \frac{1}{2^{d + m}} \rho\right)\right)}}_{= a_i} \underbrace{1}_{= b_i}.}
        
        We can again apply Cauchy-Schwarz, to get: 
        \autoeq{S \leq \frac{1}{2} \sqrt{\sum_{z, y} \Tr\left(\rho^{-1/2} \left(\rho_{zy} - \frac{1}{2^{d + m}} \rho\right) \rho^{-1/2} \left(\rho_{zy} - \frac{1}{2^{d + m}} \rho\right)\right)}\cdot \sqrt{2^{d + m}} = \frac{\sqrt{2^{d + m}}}{2} \sqrt{\sum_{z, y} \Tr\left(\rho^{-1/2} \rho_{zy} \rho^{-1/2} \rho_{zy}\right) - \frac{1}{2^{d+m}} \Tr\left(\rho\right)}.}
        
        We know that $\Tr\left(\rho\right) = 1$. So, we are only left with the inner sum. Using the definition of $\rho_{zy}$, the inner sum is: 
        \autoeq{T = \sum_{z, y} \sum_{\substack{x: f_y\left(x\right) = z \\ x': f_y\left(x'\right) = z}} \frac{1}{2^{2d}} p_x p_{x'} \Tr\left(\rho^{-1/2} \rho_x \rho^{-1/2} \rho_{x'}\right) = \frac{1}{2^{2d}} \sum_{z, y} \sum_{\substack{x: f_y\left(x\right) = z \\ x' \neq x: f_y\left(x'\right) = z}} p_x p_{x'} \Tr\left(\rho^{-1/2} \rho_x \rho^{-1/2} \rho_{x'}\right) \fakeequal + \frac{1}{2^{2d}} \sum_{z, y}\sum_{x: f_y\left(x\right) = z}  \Tr\left(\rho^{-1/2} p_x \rho_x \rho^{-1/2} p_{x} \rho_{x}\right)}

        Using the exact same reasoning as in the simplified proof, we then find:
        \autoeq{T = \frac{1}{2^{d + m}} \sum_{x \neq x'} p_x p_{x'} \Tr\left(\rho^{-1/2} \rho_x \rho^{-1/2} \rho_{x'}\right) \fakeequal + \frac{1}{2^d} \sum_{x} p_{x}^2 \Tr\left(\rho^{-1/2} \rho_x \rho^{-1/2} \rho_{x}\right).}

        We have two sums to bound, which we can do in a similar way to the simplified proof. The first one is easy, since all terms are non-negative, using the fact that $\rho = \sum_{x} p_x \rho_x$:
        \autoeq{T_1 = \sum_{x \neq x'} p_x p_{x'} \Tr\left(\rho^{-1/2} \rho_x \rho^{-1/2} \rho_{x'}\right) \leq \sum_{x, x'} p_x p_{x'} \Tr\left(\rho^{-1/2} \rho_x \rho^{-1/2} \rho_{x'}\right) = \Tr\left(\rho^{-1/2} \rho \rho^{-1/2} \rho\right) = \Tr\left(\rho\right) = 1.}

        The second sum is slightly more tricky. Let $M_x = \rho^{-1/2} p_x \rho_x \rho^{-1/2}$, and note that we can write: 
        \[T_2 = \sum_{x} p_x \Tr\left(\rho^{-1/2} p_x \rho_x \rho^{-1/2} \rho_{x}\right) = \sum_{x} p_x \Tr\left(M_x \rho_x\right).\]
        
        Now, we make two key observations. First, $M_x$ is a valid POVM. We can notably indeed check that, using $\rho = \sum_{x} p_x \rho_x$, then $\sum_{x} M_x = \rho^{-1/2} \sum_{x} p_x \rho_x \rho^{-1/2} = I$. This POVM is called a pretty-good measurement and has a nice interpretation, which we will explore in the following paragraphs.

        Moreover, recall that:
        \[P_{guess}\left(K \suchthat E\right) = \max_{\left\{M_x\right\} \text{ POVM}} \sum_{x} p_x \Tr\left(M_x \rho_x\right).\]

        This thus tells us that, since $P_{guess}\left(X \suchthat E\right)$ is the maximisation of the expression above over all POVMs and that $\left\{M_x\right\}$ is a specific POVM: 
        \autoeq{T_2 =  \sum_{x} p_x \Tr\left(M_x \rho_x\right) \leq \max_{\left\{M_x\right\} \text{ POVM}} \sum_{x} p_x \Tr\left(M_x \rho_x\right) = P_{guess}\left(K \suchthat E\right) = 2^{- H_{min}\left(K \suchthat E\right)} \leq 2^{-k}.}
        
        We thus found that: 
        \[T = \frac{1}{2^{d + m}} T_1 + \frac{1}{2^d} T_2 \leq \frac{1}{2^{d+m}}\cdot 1 + \frac{1}{2^d} 2^{-k}.\]

        We can thus finish the proof exactly like we did in the simplified proof:
        \autoeq{S \leq \frac{\sqrt{2^{d + m}}}{2} \sqrt{T - \frac{1}{2^{d+m}} \Tr\left(\rho\right)} \leq \frac{\sqrt{2^{d + m}}}{2} \sqrt{\frac{1}{2^{d + m}} + \frac{1}{2^{d + k}} - \frac{1}{2^{d+m}}} = \frac{2^{\left(d - k\right)/2}}{2} = \frac{\epsilon}{2}.}

        \qed
    \end{subparag}
\end{parag}

\begin{parag}{Distinguishability problem}
    Let $\rho_1, \ldots, \rho_t$ be mixed states. The goal is to find the best POVM to distinguish them: 
    \[p_{succ} = \sup_{\left\{M_x\right\} \text{ POVM}} \sum_{x} p_x \Tr\left(M_x \rho_x\right).\]

    \begin{subparag}{Remark 1}
        Note that, by definition, $p_{succ} = P_{guess}\left(K \suchthat E\right)$ if $\rho_{KE} = \sum_{k} p_k \ket{k}\bra{k} \otimes \rho_k$.
    \end{subparag}
    
    \begin{subparag}{Remark 2}
        We showed that, when $t = 2$, this is linked to the trace distance. There is however no such clean solution when $t > 2$.
    \end{subparag}
\end{parag}

\begin{parag}{Theorem: Pretty-good measurement}
    We consider the following POVM to solve the distinguishability problem: 
    \[M_x = \rho^{-1/2} \rho_x \rho^{-1/2}, \mathspace \text{where } \rho = \sum_{x} \rho_x.\]

    This POVM is named a \important{pretty-good measurement} (PGM), because it is not optimal, but works pretty well: 
    \[p_{succ}\left(OPT\right)^2 \leq p_{succ}\left(PGM\right) = \sum_{x} p_x \Tr\left(\rho^{-1/2} \rho_x \rho^{-1/2} \rho_x\right) \leq p_{succ}\left(OPT\right).\]

    \begin{subparag}{Remark 1}
        In the proof above, we used the easy bound $p_{succ}\left(PGM\right) \leq p_{succ}\left(OPT\right)$, since $OPT$ is optimal by definition. This theorem also shows that $p_{succ}\left(PGM\right) \geq p_{succ}\left(OPT\right)^2$, which is not trivial and gives its name to the PGM.
    \end{subparag}
    
    \begin{subparag}{Remark 2}
        Note that, as mentioned in the proof above, this is indeed a valid POVM. We can for instance check the normalisation:
        \[\sum_{x} M_x = \rho^{-1/2} \sum_{x} \rho_x \rho^{-1/2} = \rho^{-1/2} \rho \rho^{-1/2} = I.\]
    \end{subparag}
\end{parag}

\end{document}
