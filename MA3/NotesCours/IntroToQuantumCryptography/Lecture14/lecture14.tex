% !TeX program = lualatex
% Using VimTeX, you need to reload the plugin (\lx) after having saved the document in order to use LuaLaTeX (thanks to the line above)

\documentclass[a4paper]{article}

% Expanded on 2025-11-04 at 11:18:07.

\usepackage{../../style}

\title{Intro to quantum cryptography}
\author{Joachim Favre}
\date{Mardi 04 novembre 2025}

\begin{document}
\maketitle

\lecture{14}{2025-11-04}{The protocol, finally}{
\begin{itemize}[left=0pt]
    \item Explanation of BB84.
    \item Explanation of the IID assumption we make for the analysis of BB84.
    \item Explanation of the Chernoff bound, and intuitive application to prove the correctness.
    \item Explanation of a protocol that gives more freedom to Eve to simplify the analysis.
\end{itemize}

}

\begin{parag}{QKD protocol: BB84}
    We are now ready to state our QKD protocol.

    Let $n$ be an integer, and $N = 4n$ be the number of ``words''. The protocol goes as follows: 
    \begin{enumerate}
        \item Alice samples $x \leftarrow_U \left\{0, 1\right\}^N$ and $\theta \leftarrow_U \left\{0, 1\right\}^N$. She then sends $\ket{x}_{\theta} = \ket{x_1}_{\theta_1} \cdots \ket{x_N}_{\theta_N}$ to Bob. This is the only usage of a quantum channel, all the rest is done using the CAC.
        \item Bob samples $\widetilde{\theta} \leftarrow_U \left\{0, 1\right\}^N$, and measures $\ket{x}_{\theta}$ in the basis $\widetilde{\theta}$ to get some $\widetilde{x}$.
        \item Bob tells Alice he is done.
        \item They both communicate $\theta$ and $\widetilde{\theta}$ to each other.
        \item Alice lets $S = \left\{j: \theta_j = \widetilde{\theta}_j\right\}$ to be the set of their common measures She also lets $T$ to be a random subset of $S$ to be the testing set, by taking each element of $S$ independently with probability $\frac{1}{2}$. She sends both $T$ and $S$ to Bob.
        \item They exchange $x_T = \left\{x_i \suchthat i \in T\right\}$ and $\widetilde{x}_T$.
        \item They abort if one of the following things happen.
            \begin{itemize}
                \item Leaving $\delta = \frac{1}{\left|T\right|} \left|\left\{j \in T : x_j \neq \widetilde{x}_j\right\}\right|$, if $\delta > \delta_{max}$, abort: all bits in $T$ (and in $S$) should match.
                \item If $\left|S\right| < 0.9\cdot \frac{N}{2}$, abort: half of the bits should match.
                \item If $\left|T\right| \notin \left[0.9\cdot \frac{N}{4}, 1.1\cdot \frac{N}{4}\right]$, then abort: this is also expected and will simplify the analysis.
            \end{itemize}
            \item Alice considers $x_A = x_{S \setminus T}$ and Bob considers $x_B = \widetilde{x}_{S \setminus T}$. These are almost the same and almost unknown to the adversary.
            \item Perform information reconciliation, setting the number of parity checks to be $m = h\left(1.1 \delta_{max}\right)\left|T \setminus S\right| + \log_2\left(1/\epsilon_c\right)$.
            \item Perform privacy amplification, setting the output length to be $m = \left(1 - 2 h\left(1.1 \delta_{max}\right)\right) \left|T \setminus S\right| - 2 \log_2\left(1/\epsilon_s\right)$.
    \end{enumerate}

    \begin{subparag}{Remark}
        The parameters are chosen such that $\left|S \setminus T\right| \approx n$ with high probability. In other words, Alice and Bob end up with $n$ valid bits that should approximately match and that they have not communicated.
    \end{subparag}
\end{parag}

\begin{parag}{IID assumption}
    We make the assumption that each round can be treated independently, i.e.~that $I\left(x_j \neq \widetilde{x}_j\right)$ are independent random variables for $j \in \left\{1, \ldots, N\right\}$. 

    \begin{subparag}{Remark 1}
        This assumption is not strictly necessary for the analysis and can be gotten rid of, but given that the analysis will already be quite non-trivial, it is nice to take it. Its main interest is that we are able to sue the following theorem.
    \end{subparag}

    \begin{subparag}{Remark 2}
        This assumption is valid if there is no adversary.
    \end{subparag}
\end{parag}

\begin{parag}{Theorem: Chernoff bound}
    Let $w_1, \ldots, w_t \in \left\{0, 1\right\}$ be IID Bernoulli random variables, and let $\mu = \exval\left[w_i\right]$. Then, for all $\gamma > 0$:
    \[\prob\left(\frac{1}{t} \sum_{i=1}^{t} w_i \geq \left(1 + \gamma\right) \mu\right) \leq \exp\left(- \gamma^2 t \mu / 2\right).\]
    \[\prob\left(\frac{1}{t} \sum_{i=1}^{t} w_i \leq \left(1 - \gamma\right) \mu\right) \leq \exp\left(- \gamma^2 t \mu / 2\right).\]
\end{parag}

\begin{parag}{Lemma}
    We define $z_j = I\left(x_j \neq \widetilde{x}_j\right)$ to be an indicator random variable stating that bit $j$ does not match. We moreover define $\delta_T$ and $\delta_{S \setminus T}$ to be the fraction of bits in $T$ that do not match, and similarly for $S \setminus T$:
    \[\delta_T = \frac{1}{\left|T\right|}\cdot \left|\left\{j \in T: z_j = 1\right\}\right|, \mathspace \delta_{S \setminus T} = \frac{1}{\left|S \setminus T\right|}\cdot \left|\left\{j \in S \setminus T: z_j = 1\right\}\right|.\]

    Then:
    \[\prob\left(\delta_{S \setminus T} \geq 1.02 \delta_T \right) \leq 2e^{-0.01^2 n \delta/\left(2 \cdot  1.01\right)}.\]

    In other words, $\delta_{S \setminus T} \gg \delta_{T}$ only happens with very low probability.

    \begin{subparag}{Remark}
        In other words, since we know that $\delta_T$ is small as a condition  not to abort, then $\delta_{S \setminus T}$ is also small with high probability. But then, setting $\delta' = 1.01 \delta$ and running information reconciliation on $x_A, x_B$ with parameter $\delta$, we can take a code with $m = h\left(\delta'\right) n + \log_2\left(\frac{1}{\epsilon_C}\right)$ parity checks to get a correctness error of: 
        \[\epsilon_c + 2 \exp\left(-0.01^2 n \delta / \left(2 \cdot 1.01\right)\right).\]

        This states morally the correctness of this protocol. We will prove it more formally below.
    \end{subparag}

    \begin{subparag}{Proof}
        We assume that $\left|T\right| = n$, and $\left|S \setminus T\right| = n$ for simplicity; which is not completely wrong since this will approximately hold with high probability. Moreover, we assume that the $z_1, \ldots, z_N$ are IID, as stated before.

        Let $\mu = \exval\left[z_j\right]$. Using a Chernoff bound, we have: 
        \[\prob\left(\delta_T \leq \left(1 - \gamma\right) \mu\right) \leq \exp\left(- \gamma^2 n \mu / 2\right),\]
        \[\prob\left(\delta_{S \setminus T} \geq \left(1 + \gamma\right) \mu\right) \leq \exp\left(-\gamma^2 n \mu /2\right).\]

        Hence, by the union bound: 
        \[\prob\left(\delta_{S \setminus T} \geq \frac{1 + \gamma}{1 - \gamma} \delta_T\right) \leq 2 \exp\left(- \gamma^2 n \mu / 2\right).\]
        
        For $\gamma = 0.01$, and using the fact $\mu \geq \frac{1}{1 - \gamma}$ whenever $\delta_T \leq \left(1 - gg\right) \mu$ and $\delta_{S \setminus T} \geq \left(1 + \gamma\right) \mu$, this reads:
        \[\prob\left(\delta_{S \setminus T} \geq 1.02 \delta_T \right) \leq 2e^{-0.01^2 n \delta/\left(2 \cdot  1.01\right)}.\]

        \qed
    \end{subparag}
\end{parag}

\begin{parag}{Remark}
    We now also wish to analyse secrecy. In other words, we want to find a bound on the min-entropy. To do that, we first consider a protocol that gives more freedom to Eve, which we justify by first giving an equivalent protocol.
\end{parag}

%\begin{parag}{name}
%    %later{name}
%
%    We now want to estimate $H_{min}\left(x_A \suchthat E\right) \geq kn$, where $k$ is the key rate. Note that this is conditioned on all the knowledge Eve has, including the fact that they do not abort.
%
%    However, Eve can always just keep the state $\ket{x}_{\theta}$ Alice sends at the beginning, and send something arbitrary to Bob. With some probability, then Bob will get exactly the same $\widetilde{x} = x$. There is thus a chance that an attacker gets all the knowledge. We thus instead want something of the ofrm: 
%    \[H_{min}\left(x_A \suchthat E\right) \geq kn - \log_2\left(\frac{1}{\prob\left(\text{not abort}\right)}\right).\]
%
%    %later{prob not abort is dependent on the attack, because above is described an attack that always work given that does not abort, just abort probability is very small}
%\end{parag}

\begin{parag}{Modification of the protocol: Entanglement}
    We can modify the first few steps of the protocol as follows:
    \begin{enumerate}
        \item Alice prepares $N$ EPR pairs, and sends the second qubit of each pair to Bob.
        \item When Bob received all of them, he acknowledges.
        \item Alice samples $\theta \leftarrow_U \left\{0, 1\right\}^N$ and measures her qubits to get some $x \in \left\{0, 1\right\}^N$; and Bob samples $\widetilde{\theta} \leftarrow \left\{0, 1\right\}^N$ and measures his qubits to get some $\widetilde{x} \in \left\{0, 1\right\}^N$.
        \item Continue the protocol as before from step 4, sharing $\theta$ and $\widetilde{\theta}$ and so on.
    \end{enumerate}

    This is completely equivalent. Indeed, when we measure the first qubit in the basis $\theta$ and get some outcome $x$, then the second qubit of the pair becomes $\ket{x}_{\theta}$.
\end{parag}

\begin{parag}{Modification of the protocol: Purification}
    To simplify the reasoning, we analyse another representation of the protocol, which is more general and gives strictly more power to Eve.
    \begin{enumerate}
        \item Eve prepares $\rho_{A^N B^N E}$.
        \item She sends $N$ qubits $\rho_{A^N}$ to Alice and $N$ qubits $\rho_{B^N}$ to Bob.
        \item Alice samples $\theta \leftarrow_U \left\{0, 1\right\}^N$ and measures her qubits in this basis to get some $x \in \left\{0, 1\right\}^N$; and Bob samples $\widetilde{\theta} \leftarrow \left\{0, 1\right\}^N$ and measures his qubits in this basis to get some $\widetilde{x} \in \left\{0, 1\right\}^N$.
        \item Continue the protocol as before from step 4, sharing $\theta$ and $\widetilde{\theta}$ and so on.
    \end{enumerate}
    
    \begin{subparag}{Remark 1}
        Note that the only quantum steps are the steps $1$ to $3$. The rest is completely classical.

        The quantum part looks a lot like the tripartite guessing game we analysed earlier.
    \end{subparag}

    \begin{subparag}{Remark 2}
        The IID assumption becomes that: 
        \[\rho_{A^N B^N E} = \bigotimes_{i=1}^{N} \rho_{A_i B_i E_i}.\]
    \end{subparag}
\end{parag}


\end{document}
