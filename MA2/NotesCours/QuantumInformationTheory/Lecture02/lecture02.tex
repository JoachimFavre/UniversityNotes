% !TeX program = lualatex
% Using VimTeX, you need to reload the plugin (\lx) after having saved the document in order to use LuaLaTeX (thanks to the line above)

\documentclass[a4paper]{article}

% Expanded on 2025-02-25 at 10:16:28.

\usepackage{../../style}

\title{Quantum information theory }
\author{Joachim Favre}
\date{Mardi 25 février 2025}

\begin{document}
\maketitle

\lecture{2}{2025-02-25}{Drawing this sphere probably wasn't necessary}{
\begin{itemize}[left=0pt]
    \item Proof that a 2-level system can be represented on the Bloch sphere.
    \item Proof of the linearity of the Bloch vector.
    \item Definition of entangled state.
    \item Definition of reduced state, and justification of the partial trace.
\end{itemize}
}

\subsection{State space}

\begin{parag}{Definition: Pauli matrices}
    The \important{Pauli matrices} are defined as: 
    \[I = \begin{pmatrix} 1 & 0 \\ 0 & 1 \end{pmatrix}, \mathspace \sigma_X = \begin{pmatrix} 0 & 1 \\ 1 & 0 \end{pmatrix}, \mathspace \sigma_Y = \begin{pmatrix} 0 & -i \\ i & 0 \end{pmatrix}, \mathspace \sigma_Z = \begin{pmatrix} 1 & 0 \\ 0 & -1 \end{pmatrix}.\]

    We moreover sometimes define the pseudo-vector $\bvec{\sigma} = \begin{pmatrix} \sigma_x & \sigma_Y & \sigma_Z \end{pmatrix}^T$.
    
    \begin{subparag}{Property}
        The matrices $\left(I, \sigma_X, \sigma_Y, \sigma_Z\right)$ form a basis for $2 \times 2$ matrices.
    \end{subparag}
\end{parag}

\begin{parag}{Theorem: State space of a qubit}
    Let $\rho$ be an arbitrary 2-level quantum state. It can be written using some $\bvec{r} = \begin{pmatrix} x & y & z \end{pmatrix}^T$ with $\left\|\bvec{r}\right\| \leq 1$ such that: 
    \[\rho = \frac{1}{2}\left(I + x \sigma_X + y \sigma_Y + z \sigma_Z\right) = \frac{1}{2}\left(I + \bvec{r} \dotprod \bvec{\sigma}\right).\]

    Conversely, each $\bvec{r}$ with $\left\|\bvec{r}\right\| \leq 1$ specifies a unique quantum state.
    
    \begin{subparag}{Implication}
        In other words, the state space of a 2-level quantum system is a ball of unit radius.
    \end{subparag}

    \begin{subparag}{Example}
        For instance, for $\bvec{r} = \begin{pmatrix} 0 & 0 & \pm1 \end{pmatrix}^T$, we get: 
        \[\rho_{\pm} = \frac{1}{2} \left(\begin{pmatrix} 1 & 0 \\ 0 & 1 \end{pmatrix} \pm \begin{pmatrix} 1 & 0 \\ 0 & -1 \end{pmatrix}\right).\]

        So: 
        \[\rho_+ = \begin{pmatrix} 1 & 0 \\ 0 & 0 \end{pmatrix} = \ket{0}\bra{0}, \mathspace \rho_- = \begin{pmatrix} 0 & 0 \\ 0 & 1 \end{pmatrix} = \ket{1}\bra{1}.\]
    \end{subparag}

    \begin{subparag}{Remark}
        The state space of a 2-level classical state is a line. It is the z-axis of this sphere. Equivalently: 
        \[\rho_p^{classical} = p \ket{0}\bra{0} + \left(1 -p\right)\ket{1}\bra{1} = \begin{pmatrix} p & 0 \\ 0 & 1-p \end{pmatrix}.\]

        This is thus a diagonal matrix, as we saw before. 
    \end{subparag}
    
    \begin{subparag}{Proof}
        Let $\rho$ be an arbitrary qubit state. Recall that it is Hermitian, positive semi-definite and traces to $1$. As any matrix, we can expand it in the Pauli basis: 
        \[\rho = a_0 \left(I + x \sigma_x + y \sigma_y + z \sigma_z\right).\]
        
        Since it is Hermitian, $a_0, x, y, z \in \mathbb{R}$. Moreover, since it traces to 1: 
        \[1 = \Tr\left(\rho\right) = a_0\left(2 + x\cdot 0+y\cdot 0 + z\cdot 0\right) = 2a_0 \iff a_0 = \frac{1}{2}.\]
        
        So far, this means we can write: 
        \[\rho = \frac{1}{2} \begin{pmatrix} 1 + z & x - iy \\ x+ iy & 1 - z \end{pmatrix}.\]
        
        Using the general formula for finding the eigenvalues of a $2\times2$ matrix, we find that its eigevalues are: 
        \autoeq{\lambda_{\pm} = \frac{\Tr\left(\rho\right) \pm \sqrt{\Tr\left(\rho\right)^2 - 4 \det\left(\rho\right)}}{2} = \frac{1}{2}\left(1 \pm \sqrt{x^2 + y^2 + z^2}\right) = \frac{1}{2} \left(1 \pm \left\|\bvec{r}\right\|\right),}
        for the vector $\bvec{r} = \begin{pmatrix} x & y & z \end{pmatrix}^T$. Since we know $\lambda_{-} \geq 0$, this does require that: 
        \[\left\|\bvec{r}\right\| \leq 1.\]

        \qed
    \end{subparag}
\end{parag}

\begin{parag}{Definition: Ensemble decomposition}
    Let $\sigma_a$ and $\sigma_b$, of Bloch vector representation $\bvec{r}_a$ and $\bvec{r}_b$ respectively. Moreover, let $p \in \left[0, 1\right]$ be some probability.

    We can consider the state obtained by taking $\sigma_a$ with probability $p$ and $\sigma_b$ with probability $1-p$: 
    \[\rho = p \sigma_a + \left(1-p\right) \sigma_b.\]
    
    This is called an \important{ensemble decomposition}. Its associated Bloch vector is: 
    \[\bvec{r} = p \bvec{r}_a + \left(1-p\right) \bvec{r}_b = \bvec{r}_b + p \left(\bvec{r}_a - \bvec{r}_b\right).\]
    
    \begin{subparag}{Intuition}
        Intuitively, this means that $\rho$ is on the line connecting $\sigma_a$ to $\sigma_b$, at a fraction $p$ from $\sigma_a$ and $\left(1-p\right)$ from $\sigma_b$. 
        \svghere[0.8]{ExampleEnsembleDecomposition.svg}
    \end{subparag}

    \begin{subparag}{Proof}
        We know that: 
        \[\sigma_a = \frac{1}{2}\left(I + \bvec{r}_a \dotprod \bvec{\sigma}\right), \mathspace \sigma_b = \frac{1}{2} \left(I + \bvec{r}_b \dotprod \bvec{\sigma}_b\right).\]

        Thus: 
        \[\rho = \frac{1}{2}\left(pI + \left(1-p\right)I + \left(p \bvec{r}_a + \left(1-p\right)\bvec{r}_b\right)\dotprod \bvec{\sigma}\right) = \frac{1}{2}\left(I + \bvec{r} \dotprod \bvec{\sigma}\right),\]
        with $\bvec{r} = p \bvec{r}_a + \left(1-p\right)\bvec{r}_b$.

        \qed
    \end{subparag}
\end{parag}

\begin{parag}{Definition: Pure state}
    We say that $\rho$ is a \important{pure state} if and only if there does not exist states $\sigma_a$ and $\sigma_b$, with $\sigma_a \neq \sigma_b$, and some $0 < p < 1$, such that $\rho = p \sigma_a + \left(1 - p\right) \sigma_b$.

    Any state that is not pure is moreover called \important{mixed}.

    \begin{subparag}{Implication}
        This yields that 2-level state $\rho$ is a pure state if and only if it is on the surface of the Bloch sphere, i.e. $\bvec{r} = 1$. In other words, a mixed state is such that $\left\|\bvec{r}\right\| < 1$.
    \end{subparag}
\end{parag}

\begin{parag}{Proposition}
    Let $\rho$ be some quantum state. 

    $\rho$ is a pure state if and only if there exists a $\ket{\psi}$ such that $\rho = \ket{\psi}\bra{\psi}$.

    \begin{subparag}{Proof}
        This proof is left as an exercise to the reader.
    \end{subparag}
\end{parag}

\begin{parag}{Ensemble ambiguity paradox}
    Let $\bvec{r}$ be an arbitrary vector such that $\left\|\bvec{r}\right\| \leq 1$. Consider the two following states: 
    \[\rho_+ = \frac{1}{2}\left(I + \bvec{r} \dotprod \bvec{\sigma}\right), \mathspace \rho_- = \frac{1}{2}\left(I - \bvec{r} \dotprod \bvec{\sigma}\right).\]
    
    Then, their average is just the middle of the sphere: 
    \[\rho = \frac{1}{2} \rho_+ + \frac{1}{2}\rho_- = \frac{1}{2} I.\]

    However, this result is independent of $\bvec{r}$, for instance: 
    \[\rho = \frac{1}{2} I = \frac{\ket{0}\bra{0} + \ket{1}\bra{1}}{2} = \frac{\ket{+}\bra{+} + \ket{-}\bra{-}}{2}.\]
    
    Let us for instance consider that $\ket{0}$ means the cat is alive, and $\ket{1}$ means the cat is \textit{dead}. Then, $\ket{+} = \frac{1}{\sqrt{2}}\left(\ket{0} + \ket{1}\right)$ and $\ket{-} = \frac{1}{\sqrt{2}}\left(\ket{0} - \ket{1}\right)$ are the quantum superposition of an alive and a dead cat. 
    
    What the result above tells us is that the state where we consider $\ket{0}$ and $\ket{1}$ with uniform probability, and the state where we consider $\ket{+}$ and $\ket{-}$ with uniform probability, then there is no measurement we can perform that will give us a different outcome. This is named the \important{ensemble ambiguity paradox} since we may expect those two to be distinguishable, since they are prepared differently.
\end{parag}

\begin{parag}{Proposition: State space of larger systems}
    The state space of an $n$-qubit system has dimension $d^2 - 1$, where $d = 2^n$.

    \begin{subparag}{Remark}
        This state space is highly non-trivial. Finding a good way to visualise this, even for just $n = 2$ qubits, is still an ongoing area of research.
    \end{subparag}

    \begin{subparag}{Sanity check}
        When $n = 1$, and hence $d = 2$, this proposition states that the state space has dimension $2^2 - 1 = 3$. The Bloch sphere has 3 dimensions, so this specific case checks out.
    \end{subparag}

    \begin{subparag}{Proof}
        An arbitrary $d \times d$ complex matrix contains $2 d^2$ parameters: there are $d^2$ components in the matrix, each with a real and a complex part. However, an arbitrary $d \times d$ Hermitian matrix has $2\cdot \left(\frac{d^2 - d}{2}\right) + d = d^2$ parameters, since it is determined by its upper triangular part (which contains $\frac{d^2 - d}{2}$ complex components), and its real diagonal (which contains $d$ real components). Since we moreover ask for $\Tr\left(\rho\right) = 1$, this means that $\rho$ has $d^2 - 1$ degrees of freedom. 

        \qed
    \end{subparag}
\end{parag}

\subsection{Joint systems}

\begin{parag}{Joint systems}
    Often, it is helpful to think of a higher dimensional system as composed of lower dimensional ones, relabelling $\rho$ by $\rho_{AB}$.

    If $\rho$ is $d = 2^n$ dimensional, $A$ is $d_a = 2^{n_a}$ dimensional and $B$ is $d_b = 2^{n_b}$ dimensional, then we must have $d_a d_b = d$ and $n_a + n_b = n$.

    In fact the story often goes the other way, we identify subsystems $\rho_A, \rho_B$, and then consider the joint system $\rho_{AB} = \rho_A \otimes \rho_B$, where $\otimes$ is a tensor product.
\end{parag}

\begin{parag}{Definition: Entangled state}
    Let $\rho_{AB}$ be a quantum state. We say that it is \important{entangled}, if there does not exist $\rho_{A,i}, \rho_{B,i}$ and a probability distribution $p_i \in \left[0,1\right]$ such that $\rho_{AB} = \sum_{i} p_i \rho_A^i \otimes \rho_B^i$.

    \begin{subparag}{Remark}
        Product states, i.e. states that are not entangled, can thus be written in the form:
        \[\rho_{AB} = \sum_{i} p_i \rho_{A,i} \otimes \rho_{B,i}.\]

        In other words, with probability $p_i$ we prepare $A$ in state $\rho_{A,i}$ and $B$ in state $\rho_{B_i}$. This captures classical correlation.
    \end{subparag}
    
    \begin{subparag}{Example}
        For instance, we may have the following product states: 
        \[\rho_1 = p \ket{00} \bra{00} + \left(1-p\right) \ket{11}\bra{11},\]
        \[\rho_2 = p \ket{++}\bra{++} + \left(1-p\right) \ket{--}\bra{--}.\]

        On the other hand, considering $\ket{\Phi^+} = \frac{\ket{00} + \ket{11}}{\sqrt{2}}$, the following state is entangled: 
        \[\rho_{AB} = \ket{\Phi^+} \bra{\Phi^+}.\]
    \end{subparag}
\end{parag}

\begin{parag}{Definition: Reduced state}
    Let $\rho_{AB}$ be a quantum state. 

    Its \important{reduced state} $\rho_A$ is defined to be the operator found by taking the partial trace on $B$
    \[\rho_A = \Tr_B\left(\rho_{AB}\right) = \sum_{k} \left(I_A \otimes \bra{k}_B\right) \rho_{AB} \left(I_A \otimes \ket{k}_B\right).\]

    \begin{subparag}{Intuition}
        Let's say that we have a state $\rho_{AB}$, and want to understand the properties of $A$ and $B$ separately. In other words, we want an object that quantifies all measurement and outcomes on $A$.

        More specifically, when measuring $A$, our measurements are $\widetilde{M}_{AB} = M_A \otimes I_B$, and we want $\rho_A$ such that: 
        \[\Tr\left(M_A \rho_A\right) = \Tr\left(\widetilde{M}_{AB} \rho_B\right).\]

        According to the following lemma, the partial trace is the only mathematical entity which satisfies this entity.
    \end{subparag}
\end{parag}


\begin{parag}{Lemma: Reduced state}
    Let $f: \mathcal{H}_{A} \otimes \mathcal{H}_B \mapsto \mathcal{H}_A$ be some map such that, for all $\widetilde{M}_{AB} = M_A \otimes I_B$ and all $\rho_{AB}$, the state $\rho_A = f\left(\rho_{AB}\right)$ is such that: 
    \[\Tr\left(M_A \rho_A\right) = \Tr\left(\widetilde{M}_{AB} \rho_{AB}\right).\]

    The unique mathematical entity that satisfies this property is the partial trace $f = \Tr_B$, i.e. $\rho_A = \Tr_B\left(\rho_{AB}\right)$.

    \begin{subparag}{Proof of validity 1}
        We want to prove that the partial trace does fulfil this property. We notice that: 
        \autoeq{\Tr\left(M_A \rho_A\right) = \Tr\left(M_A \sum_{k} \left(I \otimes \bra{k}\right) \rho_{AB} \left(I \otimes \ket{k}\right)\right) = \sum_{k} \Tr\left(\left(M_A \otimes \bra{k}\right) \rho_{AB} \left(I \otimes \ket{k}\right)\right).}

        Now, $\Tr\left(A\right) = \sum_{i} \bra{i} A \ket{i}$, so we can use the fact that the space of $B$ has already been traced out to simplify our result to: 
        \autoeq{\Tr\left(M_A \rho_A\right) = \sum_{k, k'} \bra{k' k} \left(M_A \otimes I\right) \rho_{AB} \ket{k' k} = \Tr\left(\left(M_A \otimes I\right) \rho_{AB}\right) = \Tr\left(\widetilde{M}_{AB} \rho_{AB}\right).}
    \end{subparag}

    \begin{subparag}{Proof of validity 2}
        Let us now make another proof that the partial trace does fulfil this property, but this time making a constructive proof. 

        We know that, writing $M_A = \sum_{i} \lambda_i \ket{\lambda_i}\bra{\lambda_i}$ and letting $\prob\left(A = \lambda_i\right)$ represent the probability of finding system $A$ to be in eigenstate $\lambda_i$, we have:
        \[\left\langle M_A \right\rangle = \sum_{i=1}^{d_A} \lambda_i \prob\left(A = \lambda_i\right) = \sum_{i=1}^{d_A} \lambda_i \sum_{j=1}^{d_B} \prob\left(A = \lambda_i, B = j\right).\]
        
        Now, we know that the probability of finding system $A$ in eigenstate $\lambda_i$ and system $B$ in state $j$ is 
        \[\prob\left(A = \lambda_i, B = j\right) = \left(\bra{\lambda_i}_A \otimes \bra{j}_B\right) \rho_{AB} \left(\ket{\lambda_i}_A \otimes \ket{j}_B\right).\]

        This finally give us that: 
        \autoeq{\left\langle M_A \right\rangle = \sum_{i=1}^{d_A} \lambda_i \sum_{j=1}^{d_B} \left(\bra{\lambda_i} \otimes \bra{j}\right) \rho_{AB} \left(\ket{\lambda_i}_A \otimes \ket{j}_B\right) = \sum_{i=1}^{d_A} \lambda_i \bra{\lambda_i}_A \underbrace{\sum_{j=1}^{d_B} \left(I_A \otimes \bra{j}_B\right) \rho_{AB} \left(I_A \otimes \ket{j}_B\right)}_{= \Tr_B\left(\rho_{AB}\right) = \rho_A} \ket{\lambda_i}_A = \sum_{i=1}^{d_A} \lambda_i \bra{\lambda_i}_A \rho_A \ket{\lambda_i}_A = \Tr\left(M_A \rho_A\right).}

        Notice that the partial trace appears naturally.
    \end{subparag}

    \begin{subparag}{Proof of uniqueness}
        A proof that the partial trace is the unique object for which this property holds can be found in Nielsen and Chuang's \textit{Quantum Computation and Quantum Information} (2010), page 107.

        \qed
    \end{subparag}
\end{parag}

\end{document}
