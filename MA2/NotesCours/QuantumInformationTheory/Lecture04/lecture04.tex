% !TeX program = lualatex
% Using VimTeX, you need to reload the plugin (\lx) after having saved the document in order to use LuaLaTeX (thanks to the line above)

\documentclass[a4paper]{article}

% Expanded on 2025-03-11 at 10:17:38.

\usepackage{../../style}

\title{QIT}
\author{Joachim Favre}
\date{Mardi 11 mars 2025}

\begin{document}
\maketitle

\lecture{4}{2025-03-11}{Everything is a quantum channel}{
\begin{itemize}[left=0pt]
    \item Definition of quantum channel.
    \item Explanation of the (Kraus) representer theorem.
    \item Many examples of channel, including the completely dephasing channel, the completely depolarising channel, the partial trace channel, state preparation and the measure-record-and-update channel.
    \item Proof of Naimark's dilation theorem.
\end{itemize}

}

\subsection{Quantum channels}
\subsubsection{Kraus representation}


\begin{parag}{Definition: Quantum channel}
    A quantum channel is a mathematical super-operator $\mathcal{E}$ (mapping operators to operators) such that:
    \begin{enumerate}
        \item \textit{(Linearity)} $\mathcal{E}\left(p \rho + \left(1 - p\right) \sigma\right) = p \mathcal{E}\left(\rho\right) + \left(1-p\right) \mathcal{E}\left(\sigma\right)$.
        \item \textit{(Outputs a genuine state)} $\Tr\left(\mathcal{E}\left(\rho\right)\right) = 1$ and $\mathcal{E}\left(\rho\right)$ is positive semi-definite.
        \item \textit{(Complete positivity)} $\left(\mathcal{E} \otimes I_B\right) \rho_{AB}$ is positive semi-definite, for any Hilbert space $\mathcal{H}_B$ and bipartite state $\rho_{AB}$.
    \end{enumerate}

    \begin{subparag}{Remark 1}
        We will see different, easier way, to express a quantum channel.
    \end{subparag}

    \begin{subparag}{Remark 2}
        Complete positivity is a very important property of quantum channels, it states that we can apply it on part of a qubit (for instance if Alice does it in her lab, leaving the part in Bob's lab untouched), while still getting a valid quantum state. 

        Any completely positive operator is also positive, but the reciprocal is not true. For instance, consider the typical example: 
        \[\mathcal{E}\left(\rho\right) = \rho^T\]

        This operator is positive. It is easy to see for a single qubit $\rho = \frac{1}{2} \left(I + \bvec{r} \dotprod \bvec{\sigma}\right)$: 
        \[\mathcal{E}\left(\rho\right) = \frac{1}{2}\left(I + \begin{pmatrix} r_x \\ r_y \\ r_z \end{pmatrix} \dotprod \begin{pmatrix} \sigma_x \\ -\sigma_y \\ \sigma_z \end{pmatrix} \right) = \frac{1}{2}\left(I + \begin{pmatrix} r_x \\ -r_y \\ r_z \end{pmatrix} \dotprod \begin{pmatrix} \sigma_x \\ \sigma_y \\ \sigma_z \end{pmatrix} \right).\]
        
        The result is thus a valid quantum state, showing $\mathcal{E}$ is positive. Now, let us prove it is not completely positive. Consider: 
        \autoeq{\left(\mathcal{E} \otimes I\right) \left(\ket{\Phi^+}\bra{\Phi^+}\right) = \left(\mathcal{E} \otimes I\right)\frac{\ket{00}\bra{00} + \ket{11}\bra{11} + \ket{00}\bra{11} + \ket{11}\bra{00}}{2} = \frac{\mathcal{E}\left(\ket{0}\bra{0}\right)\otimes\ket{0}\bra{0} + \mathcal{E}\left(\ket{1}\bra{1}\right) \otimes \ket{1}\bra{1}}{2} \fakeequal + \frac{\mathcal{E}\left(\ket{0}\bra{1}\right) \otimes \ket{0}\bra{1} + \mathcal{E}\left(\ket{1}\bra{0}\right)\otimes \ket{1}\bra{0}}{2} = \frac{\ket{00}\bra{00} + \ket{11}\bra{11} + \ket{01}\bra{10} + \ket{10}\bra{01}}{2} = \frac{\ket{00}\bra{00} + \ket{11}\bra{11} + SWAP}{2}.}

        However, this is not positive: $\ket{\Psi^-} = \frac{\ket{01} - \ket{10}}{\sqrt{2}}$ is an eigenvector of this operator with a negative eigenvalue. Hence, $\mathcal{E}$ is not completely positive and thus not a valid quantum channel.
    \end{subparag}
\end{parag}

\begin{parag}{Theorem: Representer theorem}
    Let $\mathcal{E}$ be a super-operator.

    $\mathcal{E}$ is a valid quantum channel if and only if there exists operators $\left(A_i\right)$ such that $\sum_{i} A_i^{\dagger} A_i = I$ and $\mathcal{E}\left(\rho\right) = \sum_{i} A_i \rho A_i^{\dagger}$ .

    This is named the \important{Kraus representation} of $\mathcal{E}$, and the $\left(A_i\right)$ are named \important{Kraus operators}.

    \begin{subparag}{Remark}
        This gives a way of proving that a super-operator $\mathcal{E}$ is a valid quantum channel: if we find Kraus operators, then we know it is a quantum channel. In particular, this gives us a way to prove a super-operator is completely positive.
    \end{subparag}

    \begin{subparag}{Proof $\implies$}
        We will not prove that any quantum channel admits a Kraus form in this class.
    \end{subparag}

    \begin{subparag}{Proof $\impliedby$}
        Let us prove that any super-operator written in Kraus form is a valid quantum channel. We do not prove the complete positivity property.

        \begin{itemize}[left=0pt]
            \item We first verify it outputs a valid quantum state. The trace is indeed preserved, using cyclicity of the trace and the hypothesis that $\sum_{i} A_i^{\dagger} A_i = I$: 
                \autoeq{\Tr\left(\mathcal{E}\left(\rho\right)\right) = \Tr\left(\sum_{i} A_i \rho A_i^{\dagger}\right) = \sum_{i} \Tr\left(A_i^{\dagger} A_i \rho\right) = \Tr\left(\sum_{i} A_i^{\dagger} A_i \rho\right) = \Tr\left(I \rho\right) = 1.}

            We now prove that $\sigma = \mathcal{E}\left(\rho\right)$ is positive semi-definite. To see this, we diagonalise $\rho = \sum_{\lambda} \lambda \ket{\lambda}\bra{\lambda}$, giving: 
            \autoeq{\bra{\phi} \sigma \ket{\phi} = \bra{\phi} \sum_{i} A_i \rho A_i^{\dagger} \ket{\phi} = \sum_{i, \lambda} \lambda \bra{\phi} A_i \ket{\lambda} \bra{\lambda} A_i^{\dagger} \ket{\phi} = \sum_{i, \lambda} \lambda \left|\bra{\phi} A_i \ket{\lambda}\right|^2.}
            
            We know that all eigenvalues are non-negative, $\lambda \geq 0$, since $\rho$ is a valid quantum state, so $\bra{\phi} \sigma \ket{\phi} \geq 0$, showing positive semi-definiteness.
            
            \item We now verify linearity: 
            \autoeq{\mathcal{E}\left(p \rho + \left(1-p\right)\sigma\right) = \sum_{i} A_i \left(p \rho + \left(1 - p\right)\sigma\right) A_i^{\dagger} = p \sum_{i} A_i \rho A_i^{\dagger} + \left(1-p\right) \sum_{i} A_i \sigma A_i^{\dagger} = p \mathcal{E}\left(\rho\right) + \left(1-p\right) \mathcal{E}\left(\sigma\right).}
        \end{itemize}

        \qed
    \end{subparag}
\end{parag}

\begin{parag}{Example 1}
    Let $U$ be some unitary. $\mathcal{E}\left(\rho\right) = U \rho U^{\dagger}$ is a valid Kraus representation.

    This is the usual unitary evolution of a state in quantum physics.
\end{parag}

\begin{parag}{Example 2}
    Let $\left(U_i\right)$ be a family of unitaries, and $p_i$ be a probability distribution, i.e. $0 \leq p_i \leq 1$ and $\sum_{i} p_i = 1$. Then, the following is a valid Kraus representation: 
    \[\mathcal{E}\left(\rho\right) = \sum_{i} p_i U_i \rho U_i^{\dagger}.\]

    Intuitively, this means that we apply the unitary $U_i$ with probability $p_i$. In this case, the Kraus operators are $A_i = \sqrt{p_i} U_i$, and we do have: 
    \[\sum_{i} A_i^{\dagger} A_i = \sum_{i} p_i U_i^{\dagger} U_i = \sum_{i} p_i I = I.\]
\end{parag}

\begin{parag}{Example 3: Completely dephasing channel}
    Consider the following Kraus operators: 
    \[A_0 = \ket{0}\bra{0}, \mathspace A_1 = \ket{1}\bra{1}.\]

    They are valid Kraus operators, since:
    \[\sum_{i} A_i^{\dagger} A_i = \ket{0} \braket{0}{0} \bra{0} + \ket{1} \braket{1}{1} \bra{1} = \ket{0}\bra{0} + \ket{1}\bra{1} = I.\]

    This yields that $\mathcal{E}$ is a valid quantum channel. We find that: 
    \[\mathcal{E}\left(\rho\right) = \ket{0} \bra{0} \rho \ket{0}\bra{0} + \ket{1}\bra{1} \rho \ket{1}\bra{1} = \bra{0} \rho \ket{0} \ket{0}\bra{0} + \bra{1} \rho \ket{1} \ket{1}\bra{1}.\]

    In other words: 
    \[\rho = \begin{pmatrix} \bra{0} \rho \ket{0} & \bra{0} \rho \ket{1} \\ \bra{1} \rho \ket{0} & \bra{1} \rho \ket{1} \end{pmatrix} \mapsto \mathcal{E}\left(\rho\right) = \begin{pmatrix} \bra{0} \rho \ket{0} & 0 \\ 0 & \bra{1} \rho \ket{1} \end{pmatrix}.\]

    This is named a \important{completely dephasing channel}, because it kills off diagonal components.
\end{parag}

\begin{parag}{Example 4: Completely depolarising channel}
    Let $\tau$ be some quantum state. We consider the following channel, which gets rid of the state and just outputs $\tau$: 
    \[\mathcal{E}\left(\rho\right) = \tau, \mathspace \forall \rho.\]

    This is a genuine channel, named the \important{completely depolarising channel}. We want to express it in Kraus form. First, we diagonalise $\tau = \sum_{i} \lambda_i \ket{\lambda_i}\bra{\lambda_i}$. The idea is then to exploit the fact that $1 = \Tr\left(\rho\right) = \sum_{j} \bra{j} \rho \ket{j}$, so we get consider the following family of Kraus operators $\left(A_{i, j}\right)_{i, j}$: 
    \[A_{i,j} = \sqrt{\lambda_i} \ket{\lambda_i}\bra{j}.\]
    
    Then, we do have that: 
    \autoeq{\mathcal{E}\left(\rho\right) = \sum_{i, j} A_{i, j} \rho A_{i, j}^{\dagger} = \sum_{i, j} \lambda_i  \ket{\lambda_i}\bra{j} \rho \ket{j}\bra{\lambda_i} = \sum_{i} \lambda_i  \ket{\lambda_i} \left(\sum_{j} \bra{j}\rho \ket{j}\right) \bra{\lambda_i} = \sum_{i} \lambda_i \ket{\lambda_i}\bra{\lambda_i} = \tau.}

    Moreover, they are valid Kraus operators: 
    \[\sum_{i, j} A_{ij}^{\dagger} A_{ij} = \sum_{i, j} \lambda_i \ket{j} \braket{\lambda_i}{\lambda_i} \bra{j} = \sum_{i} \lambda_i \sum_{j} \ket{j}\bra{j} = I,\]
    where we used the fact $\sum_{i} \lambda_i = \Tr\left(\rho\right) = 1$.
\end{parag}

\begin{parag}{Example 5: Partial trace}
    Consider the following quantum channel, that takes a partial trace: 
    \[\mathcal{E}\left(\rho_{AB}\right) = \Tr_B\left(\rho_{AB}\right) = \rho_A.\]
    
    Recall that we can write: 
    \[\Tr_B\left(\rho\right) = \sum_{k} \left(I \otimes \bra{k}\right) \rho \left(I \otimes \ket{k}\right).\]

    So, we read that we can take $A_k = I \otimes \bra{k}$. Note that this is not a square matrix in the general case. We can moreover check that: 
    \[\sum_{k} A_k^{\dagger} A_k = \sum_{k} \left(I \otimes \ket{k}\right)\left(I \otimes \bra{k}\right) = I \otimes \sum_{k} \ket{k}\bra{k} = I \otimes I = I.\]
\end{parag}

\begin{parag}{Example 6: State preparation}
    Consider the channel that does not take input, and outputs some state $\rho$: 
    \[\mathcal{E}\left(1\right) = \rho.\]

    Basically, $\mathcal{E}$ consists in preparing a new state $\rho$. Diagonalising $\rho = \sum_{i} \lambda_i \ket{\lambda_i}\bra{\lambda_i}$, we can take: 
    \[A_k = \sqrt{\lambda_k} \ket{\lambda_k}.\]
    
    Then, we do have that: 
    \[\sum_{k} A_k^{\dagger} A_k = \sum_{k} \lambda_k \braket{k}{k} = \sum_{k} \lambda_k = \Tr\left(\rho\right) = 1,\]
    \[\mathcal{E}\left(1\right) = \sum_{k} A_k A_k^{\dagger} = \sum_k \lambda_k \ket{\lambda_k}\bra{\lambda_k} = \rho.\]

    Again, the Kraus operators are not square matrices in this case. In fact, they are just column vectors.
\end{parag}

\begin{parag}{Definition: Isometry operator}
    An operator $U \in \mathbb{C}^{m \times n}$ is called an \important{isometry} if $U^{\dagger} U = I_m$.

    \begin{subparag}{Intuition}
        This generalises the concept of unitary operators to non-square matrices: any unitary is also an isometry. This states that the $n$ columns of $U$   are $m$-dimensional orthonormal vectors. Hence, in particular, this necessarily implies that $m \geq n$ (i.e. $U$ must be a tall matrix).

        The intuition behind this definition is that this essentially just asks for $U$ to preserve vector 2-norms:
        \[\left\|U \ket{x}\right\|_2^2 = \bra{x} U^{\dagger} U \ket{x} = \braket{x}{x} = \left\|\ket{x}\right\|_2^2.\]
    \end{subparag}

    \begin{subparag}{Remark}
        When $m = n$, then $U^{-1} = U^{\dagger}$ and hence we also have $U U^{\dagger} = I_n$. However, if $m \neq n$, then $U U^{\dagger} \neq I_n$.
    \end{subparag}
\end{parag}

\begin{parag}{Example 7: Isometric dynamics}
    Consider some isometry $U$. We then define: 
    \[\mathcal{E}\left(\rho\right) = U \rho U^{\dagger}.\]

    This is a valid channel in Kraus form. For instance, we may have $U \in \mathbb{C}^{4 \times 2}$ defined such that: 
    \[U \ket{0} = \frac{\ket{0} + \ket{2}}{\sqrt{2}}, \mathspace U \ket{1} = \frac{\ket{1} + \ket{3}}{\sqrt{2}}.\]

    Note that it is indeed a map that preserves scalar products:
    \[\bra{0} U^{\dagger} U \ket{0} = \bra{1} U^{\dagger} U \ket{1} = 1, \mathspace \bra{0} U^{\dagger} U \ket{1} = \bra{1} U^{\dagger} U \ket{0} = 0.\]

    However, since it maps a 2D space to a 4D space, it is not invertible.
\end{parag}

\begin{parag}{Example 8: Measure-record-and-update channel}
    Let $B_k$ be Kraus operators. We consider the following very important channel, named the \important{measure-record-and-update channel}:
    \[\mathcal{E}\left(\rho\right) = \sum_{k} \ket{k}\bra{k} \otimes B_k \rho B_k^{\dagger}.\]

    The idea is that $\rho$ is the system being measured, $\ket{k}\bra{k}$ is a classical registrer that captures the measurement outcome, and $B_k \rho B_k^{\dagger}$ gives the state update rule. 

    We will prove the following:
    \begin{enumerate}[left=0pt]
        \item Its Kraus operators are $A_k = \ket{k} \otimes B_k$.
        \item It corresponds to measuring the POVM $\left(M_k\right)$, where $M_k = B_k^{\dagger} B_k$.
        \item Conditional on measuring outcome $\ket{x}\bra{x}$, that happens with probability $\Tr\left(M_x \rho\right)$, the state collapses to $\ket{x}\bra{x} \otimes B_x \rho B_x^{\dagger}$.
    \end{enumerate}

    \begin{subparag}{Kraus operators}
        By hypothesis, the $B_k$ are Kraus operators. Hence, $A_k = \ket{k} \otimes B_k$ are also Kraus operators:
        \[\sum_{k} A_k^{\dagger} A_k = \sum_{k} \braket{k}{k} B_k^{\dagger} B_k = \sum_{k} B_k^{\dagger} B_k = I.\]
    \end{subparag}

    \begin{subparag}{Correspondance to POVM}
        The output state on the classical register is: 
        \autoeq{\Tr_S\left(\mathcal{E}\left(\rho\right)\right) = \Tr_S\left(\sum_{k} \ket{k}\bra{k} \otimes B_k \rho B_k\right) = \sum_{k} \ket{k}\bra{k} \Tr\left(B_k \rho B_k^{\dagger}\right) = \sum_{k} \Tr\left(M_k \rho\right) \ket{k}\bra{k},}
        where $M_k = B_k^{\dagger} B_k$. This looks like the POVM formula, that $p_k = \Tr\left(M_k \rho\right)$ tells us the probability to measure outcome $k$: 
        \[\Tr_S\left(\mathcal{E}\left(\rho\right)\right) = \sum_{k} p_k \ket{k} \bra{k}.\]
        
        Moreover, we can prove that these $\left(M_k\right)$ are indeed legit POVM, using the fact the $B_k$ are Kraus operators: 
        \begin{itemize}[left=0pt]
            \item \textit{(Normalisation)} $\sum_{k} M_k = \sum_{k} B_k^{\dagger} B_k = I,$
            \item \textit{(Positivity)} $\bra{\phi} M_k \ket{\phi} = \bra{\phi} B_k^{\dagger} B_k \ket{\phi} = \left\|B_k \ket{\phi}\right\|_2^2 \geq 0$, for all $\ket{\phi}.$
        \end{itemize}
    \end{subparag}

    \begin{subparag}{Update rule}
        If we instead trace out the classical register, we have: 
        \autoeq{\Tr_x\left(\mathcal{E}\left(\rho\right)\right) = \Tr_x\left(\sum_{k} \ket{k}\bra{k} \otimes B_k \rho B_k^{\dagger}\right) = \sum_{k} \Tr\left(\ket{k}\bra{k}\right) B_k \rho B_k^{\dagger} = \sum_{k} B_k \rho B_k^{\dagger}.}
        
        In other words, if we look at the state of a system conditional on obtaining outcome $\ket{x}\bra{x}$, we have: 
        \[\rho \mapsto \mathcal{E}\left(\rho\right) \over{\mapsto}{get $x$} \ket{x}\bra{x} \otimes B_x \rho B_x^{\dagger},\]
        with probability $\Tr\left(M_x \rho\right) = \Tr\left(B_x^{\dagger} B_x \rho\right)$. Note that $B_x \rho B_x^{\dagger}$ is not normalised, the normalised output is instead $\rho_x = \frac{B_x \rho B_x^{\dagger}}{\Tr\left(M_x \rho\right)}$.
    \end{subparag}

    \begin{subparag}{Remark}
        It is possible to introduce POVM measurements by starting from this channel. However, note that it is not necessary, since we manage developing this theory very well without an update rule.

        In fact, we can consider $B_k' = U B_k$ for some unitary $U$. This gives us some valid Kraus operators associated to the same POVM, since $M_k' = \left(B_k'\right)^{\dagger} B_k' = B_k^{\dagger} U^{\dagger} U B_k = B_k^{\dagger} B_k = M_k$. However, these have a different update rules to the state. This shows that the update rule is not unique given a POVM.
    \end{subparag}
\end{parag}

\begin{parag}{Remark}
    All these examples show that basically any operation we can take on a state---preparation, evolution, partial trace, measurement, and so on---can be seen as a channel.
\end{parag}

\begin{parag}{Naimark's dilation theorem}
    Let $\left\{M_i\right\}_{i=1}^N$ be a POVM acting on a Hilbert space of dimension $d_A$.

    Then, there exists a projective measurement $\left\{\Pi_i\right\}_{i=1}^N$ acting on a space of dimension $d_A'$, and an isometry $V: \mathcal{H}_A \mapsto \mathcal{H}_A'$ such that, for all $i$: 
    \[M_i = V^{\dagger} \Pi_i V.\]
    
    \begin{subparag}{Intuition}
        This theorem intuitively states that any POVM can be understood as some projective measurement (the usual Quantum physics I measurements) on a larger system.

        This is another instance of the church of the larger Hilbert space. This can moreover be used to realise POVMs physically.
    \end{subparag}

    \begin{subparag}{Proof}
        The idea is to use the measure-record-and-update channel: it suffices to introduce classical registers that records the outcome, and then perform a projective measurement on that register. 

        More formally, let $B_k = \sqrt{M_k}$ be the operator such that $B_k^{\dagger} B_k = M_k$. This operator always exists since $M_k$ is positive semi-definite. Indeed, let $M_k = \sum_{i} \lambda_i \ket{\lambda_i}\bra{\lambda_i}$ be its eigen-decomposition. Then, we can simply take $B_k = \sum_{i} \sqrt{\lambda_i} \ket{\lambda_i}\bra{\lambda_i}$, since $\lambda_i \geq 0$ for all $i$ by positive semi-definiteness.

        Then, we can use these to construct:
        \[V = \sum_{k} \ket{k} \otimes B_k, \mathspace \Pi_k = \ket{k}\bra{k} \otimes I.\]

        Then, we do have: 
        \autoeq{V^{\dagger} \Pi_i V = \sum_{j, k} \left(\bra{k} \otimes B_k^{\dagger}\right) \left(\ket{i}\bra{i} \otimes I\right) \left(\ket{j} \otimes B_j\right) = \sum_{j, k} \braket{k}{i} \braket{i}{j} B_k^{\dagger} B_j = B_i^{\dagger} B_i = M_i.}

        \qed
    \end{subparag}
\end{parag}



\end{document}
