% !TeX program = lualatex
% Using VimTeX, you need to reload the plugin (\lx) after having saved the document in order to use Kuala Tex (thanks to the line above)

\documentclass[a4paper]{article}

% Expanded on 2025-02-26 at 13:18:34.

\usepackage{../../style}

\title{sublinear algorithms}
\author{Joachim Favre}
\date{Mercredi 26 février 2025}

\begin{document}
\maketitle

\lecture{2}{2025-02-26}{Chirac jumps over those}{
    \begin{itemize}[left=0pt]
        \item \textit{(Look for ``Chirac turnstile'' on the internet.)}
        \item Boosting Morris algorithm to achieve arbitrary precision at an arbitrary success probability.
        \item Definition of turnstile streams.
        \item Explanation of the distinct element problem, and of its decision problem counterpart.
    \end{itemize}
    
}

\begin{parag}{Lemma}
    Let $X_n$ be the random variable representing the counter after the $n$\Th event occurred. Then: 
    \[\exval\left(2^{2 X_n}\right) = \frac{3}{2} n^2 + \frac{3}{2} n + 1.\]

    \begin{subparag}{Proof}
        This has been proven in the first exercise series.
    \end{subparag}

    \begin{subparag}{Personal remark}
        As I mentioned in the previous lemma, I presented an alternate proof method that makes this type of proofs more intuitive, in my opinion. I think it is worth showcasing what I meant here. Be careful, spoilers of the first exercise series ahead!

        We use the law of total expectation: 
        \autoeq{\exval\left(2^{2 X_k}\right) = \exval\left(\exval\left(2^{2 X_k} \suchthat X_{k-1}\right)\right) = \exval\left(2^{2 \left(X_{k-1} + 1\right)} 2^{-X_{k-1}} + 2^{2 X_{k-1}} \left(1 - 2^{-X_{k-1}}\right)\right) = \exval\left(2^{X_{k-1} + 2} + 2^{2 X_{k-1}} - 2^{X_{k-1}}\right) = 3\exval\left(2^{X_{k-1}}\right) + \exval\left(2^{2X_{k-1}}\right) = 3k + \exval\left(2^{2X_{k-1}}\right),}
        where we used our lemma, i.e. the fact $\exval\left(2^{X_{n}}\right) = n + 1$. Chaining those equalities, we find: 
        \autoeq{\exval\left(2^{2 X_n}\right) = \exval\left(2^{2 X_{n-1}}\right) + 3n = \exval\left(2^{2 X_{n-2}}\right) + 3n + 3\left(n-1\right) = \ldots = \exval\left(2^{2 X_0}\right) + \sum_{k=1}^{n} 3k.}

        We know that $\exval\left(2^{2 X_0}\right) = 2^{2\cdot 0} = 1$ and that $\sum_{k=1}^{n} k = \frac{1}{2} k\left(k+1\right)$, hence: 
        \[\exval\left(2^{2X_n}\right) = 1 + 3 \sum_{k=1}^{n} k = 1 + \frac{3}{2}n\left(n+1\right) = \frac{3}{2} n^2 + \frac{3}{2} n + 1.\]
         
        \qed
    \end{subparag}
\end{parag}

\begin{parag}{Lemma}
    Let $X_n$ be the random variable representing the counter after the $n$\Th event occurred, and let $\hat{n} = 2^{X_n} - 1$ be the estimator. Then:
    \[\Var\left(\hat{n}\right) \leq \frac{1}{2}n^2.\]

    \begin{subparag}{Proof}
        We know that $\Var\left(X + c\right) = \Var\left(X\right)$ for any random variable $X$ and constant $c$. So: 
        \autoeq{\Var\left(\hat{n}\right) = \Var\left(\hat{n} + 1\right) = \Var\left(2^{X_n}\right) = \exval\left(2^{2X_n}\right) - \exval\left(2^{X_n}\right)^2 = \left(\frac{3}{2} n^2 + \frac{3}{2} n + 1\right) - \left(n+1\right)^2 \leq \frac{1}{2} n^2.}
        
        \qed
    \end{subparag}

    \begin{subparag}{Implication}
        This allows us to know how good of an estimator $\hat{n}$ is. By Chebyshev's inequality: 
        \[\prob\left(\left|\hat{n} - n\right| \geq \epsilon n\right) \leq \frac{\Var\left(\hat{n}\right)}{n^2 \epsilon^2} \leq \frac{n^2/2}{n^2 \epsilon^2} = \frac{1}{2 \epsilon^2}.\]

        The thing is, if $\epsilon$ is small, then $\frac{1}{2 \epsilon^2}$ is large. This is interesting for large $\epsilon$, but we want it to also hold for smaller $\epsilon$. We thus want to reduce the variance. To do so, we can average estimators, yielding the following algorithm.
    \end{subparag}
\end{parag}

\begin{parag}{Morris+ algorithm}
    We maintain $t$ copies of Morris' algorithm, which output $X_n^1, \ldots, X_n^t$. We then output: 
    \[\hat{n} = \frac{1}{t} \sum_{j=1}^{t} \left(2^{X_n^j} - 1\right)\]

    We call this algorithm the Morris+ algorithm. When $t \geq \frac{1}{2 \epsilon^2 \delta}$, this is an $\left(\epsilon, \delta\right)$-approximation algorithm to the counting problem, i.e.:
    \[\prob\left(\left|\hat{n} - n\right| \geq \epsilon n\right) \leq \delta.\]
    
    Moreover, it uses memory $O\left(t \log\left(\log\left(n\right)\right)\right)$.
    
    \begin{subparag}{Proof of validity}
        The expectation of the output is unchanged: 
        \[\exval\left(\hat{n}\right) = \exval\left(\frac{1}{t} \sum_{j=1}^{t} \left(2^{X_n^j} - 1\right)\right) = \frac{1}{t} \sum_{j=1}^{t} \exval\left(2^{X_n^j} - 1\right) = \frac{1}{t} \sum_{j=1}^{t} n = n.\]

        However, the variance is decreased, since the estimator are independent by construction: 
        \autoeq{\Var\left(\hat{n}\right) = \Var\left(\frac{1}{t} \sum_{j=1}^{t} 2^{X_n^j} - 1\right) = \frac{1}{t^2} \sum_{j=1}^{t} \Var\left(2^{X_n} - 1\right) = \frac{\Var\left(2^{X_n^j} - 1\right)}{t} \leq \frac{n^2}{2t}.}
        
        Chebyshev's inequality therefore tells us that, when $t \geq \frac{1}{2 \epsilon^2 \delta}$: 
        \[\prob\left(\left|\hat{n} - n\right| \geq \epsilon n\right) \leq \frac{\Var\left(\hat{n}\right)}{\epsilon^2 n^2} \leq \frac{\frac{n^2}{2t}}{\epsilon^2 n^2} = \frac{1}{2 \epsilon^2 t} \leq \delta,\]

        \qed
    \end{subparag}

    \begin{subparag}{Space complexity}
        Since we maintain $t$ copies of Morris' algorithm, this does yields a space complexity of $O\left(t \log\left(\log\left(n\right)\right)\right) = O\left(\frac{1}{\epsilon^2 \delta} \log\left(\log\left(n\right)\right)\right)$. The $\frac{1}{\epsilon^2}$ is fine, but the $\frac{1}{\delta}$ is a big problem: we want $\delta$ to be really small to make sure our counter is valid. We consider the following idea to improve it, called the median trick.
    \end{subparag}
\end{parag}

\begin{parag}{Morris++}
    We take $t \in \Theta\left(\log\left(\frac{1}{\delta}\right)\right)$ independent copies of Morris+ with bound $\epsilon_j = \epsilon$ and failure probability $\delta_j \leq \frac{1}{3}$, and output the median of their results. 

    Since we need a space $O\left(\frac{1}{\epsilon^2} \log\left(\log\left(n\right)\right)\right)$ per copy, and we have $t\in \Theta\left(\log\left(\frac{1}{\delta}\right)\right)$ copies, this takes space $O\left(\frac{1}{\epsilon^2} \log\left(\log\left(n\right)\right) \log\left(\frac{1}{\delta}\right)\right)$.

    \begin{subparag}{Intuition}
        The idea is that at most $\frac{1}{3}$ of the estimates are outside $\left[n- \epsilon n, n + \epsilon n\right]$ on expectation. For the median to be less than $n - \epsilon$, it means that at least $\frac{1}{2}$ of the estimates were less than $n - \epsilon$, which is very unlikely. Similarly, it is very unlikely that the median is more than $n + \epsilon$.
    \end{subparag}

    \begin{subparag}{Proof}
        Let $Y_i$ be the indicator of the $i$\Th estimator being correct:
        \[Y_i = \begin{systemofequations} 0, & \text{if $i$\Th estimate fails} \\ 1, & \text{otherwise}.\end{systemofequations}\]

        Our goal is to show that, if $t \geq C \log\left(\frac{1}{\delta}\right)$ for some constant $C > 0$, the probability that the median fails is small, i.e. the probability that at least $\frac{t}{2}$ estimates fails is small: 
        \[\prob\left(\sum_{i=1}^{t} Y_i \leq \frac{t}{2}\right) \leq \delta.\]

        We leave $\mu = \sum_{i} \exval\left(Y_i\right)$. We notice that $\exval\left(Y_i\right) = 1 - \delta_i \geq \frac{2}{3}$, so $\mu \geq \frac{2}{3} t$. Similarly, $\exval\left(Y_i\right) \leq 1$ since they are Bernoulli random variables, and hence $\mu \leq t$. We can massage our result to get a formulation closer to the Chernhoff bounds: 
        \autoeq{\sum_{i=1}^{t} Y_i \leq \frac{t}{2} \implies \sum_{i} Y_i - \mu  \leq \frac{t}{2} - \mu \leq \frac{1}{2} t - \frac{2}{3} t = -\frac{1}{6} t \implies \left|\sum_{i} Y_i - \mu\right| \geq \frac{1}{6}t \geq \frac{1}{6} \mu,}
        where we used the fact $x \mapsto \left|x\right|$ is a decreasing function for negative $x$. Since $\sum_{i} Y_i \leq \frac{t}{2}$ necessarily implies that $\left|\sum_{i} Y_i - \mu\right| \geq \frac{1}{6} \mu$, this shows that:
        \[\prob\left(\sum_{i} Y_i \leq \frac{t}{2}\right) \leq \prob\left(\left|\sum_{i} Y_i - \mu\right| \geq \frac{1}{6} \mu\right).\]

        Now, applying the Chernoff bounds, with $\hat{\delta} = \frac{1}{6}$: 
        \autoeq{\prob\left(\sum_{i=1}^{n} \leq \frac{t}{2}\right) \leq \prob\left(\left|\sum_{i} Y_i - \mu\right| \geq \frac{1}{6} \mu\right) \leq 2 \exp\left(-\frac{\hat{\delta}^2 \mu}{3}\right) \leq 2 \exp\left(-\frac{\left(\frac{1}{6}\right)^2 \frac{2}{3} t}{3}\right) \leq \delta,}
        as long as $t \geq C \log\left(\frac{1}{\delta}\right)$, where $C > 0$ is a large enough constant. This is exactly what we wished for.
        
        \qed
    \end{subparag}

    \begin{subparag}{Personal remark}
        To summarise, we first improved the bound $\epsilon$ arbitrarily and improved the success probability to a constant $1 - \delta = \frac{2}{3}$, by taking an average. We then applied the median trick to improve the success probability $1 - \delta$ arbitrarily. This is a very general trick, that we will apply multiple times during this class.

        Note that to improve the bound $\epsilon$, we cannot use the median: the median of IID random variables does not necessarily converge to their expected value. However, as soon as we have a satisfying bound and a constant success probability which is at least $\frac{1}{2}$, we can make the convergence faster using the median. As we have seen with Morris+, we can also improve the success probability with an average, although the convergence is exponentially slower.
    \end{subparag}
\end{parag}

\section{Distinct elements}
\subsection{Definitions}

\begin{parag}{Definition: Turnstile stream}
    We define a \important{turnstile stream} to be a stream where items can arrive and depart.  \later{the domain of $\sigma$, what are $m$ and $n$??}

    \begin{subparag}{Example}
        For instance, we may have:
        \[\sigma = \left\langle 1, 2, 5, -2, 3001, 1, 2, -3001, 5, 5 \right\rangle.\]
        
        A negative value means that an element departed. Note that there might be multiplicites: an element might arrive twice and depart once, so the stream ends up with a single copy of this element.
    \end{subparag}

    \begin{subparag}{Remark}
        In the following lectures, we will refer to ``turnstile streams'' as ``streams''.
    \end{subparag}
\end{parag}

\begin{parag}{Definition: Frequency vector}
    Let $\sigma$ be some stream.

    We define its frequency vector $x \in \mathbb{R}^n$ such that, for all $i \in \left\{1, \ldots, n\right\}$, $x_i$ is the number of times items $i$ arrived so far.

    \begin{subparag}{Properties}
        Note that $\left\|x\right\|_1 = \sum_{i} \left|x_i\right|$ is the number of items at the end of the stream. Similarly, $\left\|x\right\|_0$ is the number of distinct elements.
    \end{subparag}
\end{parag}

\begin{parag}{Turnstile distinct elements problem}
    We are given a turnstile stream $\sigma$ of $m$ data items. Data items are integers in $\left\{1, \ldots, n\right\}$.

    We want to know how many distinct elements are left, at the end. More precisely, we want to find a single pass streaming algorithm that outputs a randomised estimator $\widetilde{k}$ of $\left\|x\right\|_0$:
    \[\prob\left(\left(1- \epsilon\right) \left\|x\right\|_0 \leq \widetilde{k} \leq \left(1+\epsilon\right) \left\|x\right\|_0\right) \geq 1 - \delta.\]

    \begin{subparag}{Intuition}
         We can suppose that $m = n^{O\left(1\right)}$.
    \end{subparag}

    \begin{subparag}{Remark}
        The trivial benchmarks are $O\left(m \log\left(n\right)\right)$ (storing the whole stream) and $O\left(n \log\left(m\right)\right)$ (storing $x$). Our algorithm should thus be better than that.

        In fact, we will achieve a 2-approximation algorithm. We will then boost it to a $\left(1+\epsilon\right)$-approximation algorithm with probability $1 - \delta$ of success in space $\lang{poly}\left(\frac{1}{\epsilon}, \ln\left(\frac{1}{\delta}\right), \ln\left(n\right)\right)$ in third exercise series.
    \end{subparag}
\end{parag}

\begin{parag}{Theorem}
    Any deterministic algorithm that approximates the number of distinct elements in a stream with error at most $\epsilon = 0.01$ must use $\Omega\left(n\right)$ space.

    \begin{subparag}{Proof}
        This will be proven in the second exercise series.
    \end{subparag}

    \begin{subparag}{Implication}
        This theorem means that randomness is necessary for our algorithm.
    \end{subparag}
\end{parag}

\begin{parag}{Distinct element decision problem}
    In order to solve our problem, we first consider a decision version.

    We are given a stream $\sigma$, and some threshold $t$. Moreover, let $x$ be the frequency vector of $\sigma$.

    The task is to output \lang{YES} if $\left\|x\right\|_0 \geq 2t$, and \lang{NO} if $\left\|x\right\|_0 \leq t$.

    \begin{subparag}{Remark}
        We do not care what is outputted if $t < \left\|x_0\right\| < 2t$.
    \end{subparag}
\end{parag}

\subsection{Naive algorithm}
\begin{parag}{Algorithm}
    We consider an algorithm to solve the distinct element decision problem.
    \begin{enumerate}[left=0pt]
        \item Select a subset $S \subseteq \left\{1, \ldots, n\right\}$ by including each $i \in \left\{1, \ldots, n\right\}$ independently with probability $\frac{1}{t}$.
        \item While receiving the stream, compute $C = \sum_{i \in S} x_i$. In other words, when some $i = a_j$ arrives or departs, we add $\text{sign}\left(i\right)$ to $C$ if $i \in S$.
        \item If $C > 0$ output \lang{YES}. Otherwise, i.e. if $C = 0$, then output \lang{NO}.
    \end{enumerate}

    \begin{subparag}{Remark}
        For now, we suppose that $S$ does not cost anything to be stored or to be generated. This allows to better understand the intuition behind this algorithm. We will later refine this hypothesis.
    \end{subparag}
    
    \begin{subparag}{Intuition}
        Intuitively, if we have very few distinct elements in the stream, they are likely not to be in $S$, and thus it is likely $C = 0$. If otherwise there are many elements, it is likely $C > 0$.
    \end{subparag}
\end{parag}

\end{document}
