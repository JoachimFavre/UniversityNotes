% !TeX program = lualatex
% Using VimTeX, you need to reload the plugin (\lx) after having saved the document in order to use LuaLaTeX (thanks to the line above)

\documentclass[a4paper]{article}

% Expanded on 2025-05-20 at 08:26:27.

\usepackage{../../style}

\title{CQP}
\author{Joachim Favre}
\date{Mardi 20 mai 2025}

\begin{document}
\maketitle

\lecture{13}{2025-05-20}{Quantum computing}{
\begin{itemize}[left=0pt]
    \item Explanation of the computation model for quantum computers.
    \item Explanation of how to compute time evolution of quantum states on a quantum computer.
    \item Explanation of how to compute energy on a quantum computer.
\end{itemize}

}

\section{Digital quantum computing}

\subsection{Model}

\begin{parag}{Idea}
    Let's now suppose that we have access to a quantum computer. We wonder how this can help us compute properties of our quantum systems.
\end{parag}

\begin{parag}{Model}
    Just like for classical computing, we do not use the physical model behind it, we add an abstraction layer without really caring how it's physically implemented. 

    Hence, we consider qubits: 
    \[\ket{0} = \ket{\uparrow} = \begin{pmatrix} 1 \\ 0 \end{pmatrix}, \mathspace \ket{1} = \ket{\downarrow} = \begin{pmatrix} 0 \\ 1 \end{pmatrix}.\]
    
    A quantum circuit manipulates $N$ qubits. We will moreover only consider unitary operators $\hat{U} \ket{\psi}$, which we refer to gates, and state how measurement is represented.
\end{parag}

\begin{parag}{Single-qubit gates}
    We consider the following single-qubit quantum gates.
    \begin{itemize}
        \item The Pauli gates:
        \[\hat{X} = \begin{pmatrix} 0 & 1 \\ 1 & 0 \end{pmatrix}, \mathspace \hat{Z} = \begin{pmatrix} 1 & 0 \\ 0 & -1 \end{pmatrix}.\]
        \item The Hadamard gate:
        \[\hat{H} = \frac{1}{\sqrt{2}} \begin{pmatrix} 1 & 1 \\ 1 & -1 \end{pmatrix}.\]
        It must not be mistaken with a Hamiltonian, which we will now note $\hat{\mathcal{H}}$. This gate is moreover important because it rotates from the $z$ to the $x$ basis. 
        \item The $\hat{T}$  gate: 
    \[\hat{T} = \begin{pmatrix} 1 & 0 \\ 0 & \exp\left(i\frac{\pi}{4}\right) \end{pmatrix}.\]
    \end{itemize}
\end{parag}

\begin{parag}{Lemma}
    Any single-qubit unitary $\hat{U}$ for a single qubit can be written as a product of $\hat{X}$, $\hat{Z}$, $\hat{H}$ and $\hat{T}$.
\end{parag}

\begin{parag}{Two-qubit gates}
    We consider the following two-qubit quantum gates.

    \begin{itemize}
        \item A controlled-unitary gate, for any $U$: 
            \begin{center}
            \hfill
            \begin{minipage}{0.55\textwidth}
                \[\ket{0}\bra{0} \otimes I + \ket{1}\bra{1} \otimes U = \begin{pmatrix} I &  \\  & U \end{pmatrix} ;\]
            \end{minipage}
            \hfill
            \begin{minipage}{0.10\textwidth}
                \begin{center}
                \begin{quantikz}
                    & \ctrl{1} & \\
                    & \gate{U} & 
                \end{quantikz}
                \end{center}
            \end{minipage}
            \hfill
            \phantom{a}
            \end{center}

        \item The controlled-not gate, which is a controlled unitary with $U = X$:
            \begin{center}
            \hfill
            \begin{minipage}{0.55\textwidth}
                \[\ket{0}\bra{0} \otimes I + \ket{1}\bra{1} \otimes U = \begin{pmatrix} 1 &  &  &  \\  & 1 &  &  \\  &  & 0 & 1 \\  &  & 1 & 0 \end{pmatrix} ;\]
            \end{minipage}
            \hfill
            \begin{minipage}{0.10\textwidth}
                \begin{center}
                \begin{quantikz}
                    & \ctrl{1} & \\
                    & \targ{} & 
                \end{quantikz}
                \end{center}
            \end{minipage}
            \hfill
            \phantom{a}
            \end{center}
            
        \item The SWAP gate:
            \begin{center}
            \hfill
            \begin{minipage}{0.45\textwidth}
                \[SWAP = \begin{pmatrix} 1 &  &  &  \\  & 0 & 1 &  \\  & 1 & 0 &  \\  &  &  & 1 \end{pmatrix}.\]
            \end{minipage}
            \hfill
            \begin{minipage}{0.40\textwidth}
                \begin{center}
                \begin{quantikz}
                    & \swap{1} & \\
                    & \targX{} & 
                \end{quantikz}
                =
                \begin{quantikz}
                    & \ctrl{1} & \targ{} & \ctrl{1} & \\
                    & \targ{} & \ctrl{-1} & \targ{} &
                \end{quantikz}
                \end{center}
            \end{minipage}
            \hfill
            \phantom{a}
            \end{center}

            The idea is that it is such that it swaps the two qubits:
            \[SWAP \begin{pmatrix} c_{00} \\ c_{01} \\ c_{10} \\ c_{11} \end{pmatrix} = \begin{pmatrix} c_{00} \\ c_{10} \\ c_{01} \\ c_{11} \end{pmatrix}.\]
    \end{itemize}
\end{parag}

\begin{parag}{Lemma}
    The $CNOT$, $\hat{T}$ and $\hat{H}$ gates are enough to approximate any arbitrary unitary up to precision $\epsilon$.
\end{parag}

\begin{parag}{Measurement}
    Given a circuit $U$, we can then measure the state using the following symbol:
    \begin{center}
    \begin{quantikz}
        \lstick{\ket{0}} & \gate[2]{U} & \meter{}  \\
        \lstick{\ket{0}} &             &           
    \end{quantikz}
    \end{center}

    The measure is always done in the $Z$-basis. Then, the probability to measure some value is:
    \[\prob\left(S_1 = \pm 1\right) = \sum_{S_2 \in \left\{\pm 1\right\}} \left|\bra{S_1, S_2} \hat{U} \ket{00}\right|^2.\]

    \begin{subparag}{Remark 1}
        Note that we measure $+1$ when the state is $\ket{0}$, and $-1$ when the sate is $\ket{1}$, since we measure spin.
    \end{subparag}

    \begin{subparag}{Remark 2}
        To measure in another basis, we can apply a unitary before the measurement. For instance, to measure in the $X$-basis, we can apply a Hadamard gate, and then measure in the $Z$-basis.
        \begin{center}
        \begin{quantikz}
            \lstick{\ket{0}} & \gate[2]{U} & \gate{H} & \meter{}  \\
            \lstick{\ket{0}} &             &          &          
        \end{quantikz}
        \end{center}
    \end{subparag}
\end{parag}

\begin{parag}{Example: Bell states}
    As an example, let us consider the following circuit:
    \begin{center}
    \begin{quantikz}
        \lstick{\ket{x}} & \gate{H} & \ctrl{1} & \\
        \lstick{\ket{y}} &          & \targ{}  &
    \end{quantikz}
    \end{center}

    Then, we have: 
    \[\ket{\beta_{x, y}} = CNOT \left(\hat{H} \otimes \hat{I}\right) \ket{xy} = CNOT \left(\frac{\ket{0} + \left(-1\right)^x \ket{1}}{\sqrt{2}} \otimes \ket{y}\right) = \frac{\ket{0x} + \left(-1\right)^x \ket{1 \bar{y}}}{\sqrt{2}},\]
    where $\bar{y} = \text{not}\left(y\right)$. We can check that this represents the four Bell states. For instance, when $x = y = 0$:
    \[\ket{\beta_{00}} = \frac{\ket{00} + \ket{11}}{\sqrt{2}}.\]
\end{parag}

\subsection{Quantum dynamics}

\begin{parag}{Goal}
    Let us first focus on simulating quantum dynamics.

    \begin{subparag}{Remark}
        Recall that the best we could do to simulate spins exactly was to store the whole wave-function, which takes exponential space and time. We will present an exponential speedup using a quantum computer.
    \end{subparag}
\end{parag}

\begin{parag}{Definition: Pauli rotation}
    Given a Pauli operator $P$ and some angle $\theta$, we define the following gate:
    \[\hat{R}_P^N\left(\theta\right) = \exp\left(-i \frac{\theta}{2} \hat{P}\right).\]

    \begin{subparag}{Remark}
        It is possible to show that it suffices to be able to implement $\hat{R}_Z$ gates. For instance: 
        \[R_{ZZ}\left(\theta\right) = \exp\left(-i \frac{\theta}{2} \hat{Z}_{1} \hat{Z}_2\right) = CNOT \left(I \otimes R_Z\left(\theta\right)\right) CNOT.\]
    \end{subparag}
\end{parag}


\begin{parag}{Example: Simulation of quantum dynamics of spin}
    Let us consider the Hamiltonian for the transverse-field Ising model:
    \[\hat{\mathcal{H}} = -\Gamma \sum_{i} \hat{X}_i + \sum_{\left\langle \ell, m \right\rangle} J_{\ell m} \hat{Z}_{\ell} \hat{Z}_m.\]
    
    Then, we have: 
    \[\ket{\psi\left(t\right)} = \exp\left(-i \hat{\mathcal{H}} t\right) \ket{0}^{\otimes n}.\]

    Let us introduce a discretisation, $\Delta_t$ and $p$ such that $p \Delta_t = t$. Since the Hamiltonian has a bounded eigenspectrum, we can compute a Trotter decomposition. A single step of the first order troterised time evolution looks like:
    \[e^{-i \Delta_t \hat{\mathcal{H}}} = \underbrace{e^{i \Gamma \hat{X}_1} \cdots e^{i \Gamma \hat{X}_N \Delta_t}}_{e^{-i \hat{H}_1 \Delta_t}} \underbrace{e^{-iJ_{12} \hat{Z}_1 \hat{Z}_2 \Delta_t} e^{-i J_{23} \hat{Z}_2 \hat{Z}_3 \Delta_t} \cdots}_{e^{-i\hat{H}_0 \Delta_t}}.\]
    
    Using the Pauli rotation gates and the fact the $R_{ZZ}$ gate be written using $CNOT$ and $R_Z$, the circuit for a single Trotter step thus looks like:
    \begin{center}
    \begin{quantikz}
        \lstick{\ket{0}} & \gate{R_X\left(-2 \Delta_t \Gamma\right)} & \ctrl{1} & & \ctrl{1} & \rstick{$\cdots$}\\
        \lstick{\ket{0}} & \gate{R_X\left(-2 \Delta_t \Gamma\right)} & \targ{}  & \gate{R_Z\left(2 \Delta_t J_{1 2}\right)} & \targ{}  & \rstick{$\cdots$}\\
        \lstick{\ket{0}} & \gate{R_X\left(-2 \Delta_t \Gamma\right)} &  & & &  \rstick{$\cdots$}
    \end{quantikz}
    \end{center}
    

    \begin{subparag}{Complexity}
        Let's analyse the complexity of the circuit. To do so, let $\alpha$ be such that $N \alpha$ is the total number of bonds in the topology. Note that, typically, $\alpha$ does not depend on $N$ (which is for instance the case if each qubit is only connected to its nearest neighbours).

        We consider a single Trotter step first. We have $N$ $\hat{R}_X$ gates, $2 \alpha N$ $CNOT$ gates and $N \alpha$ $\hat{R}_Z$ gates. Hence, the total number of gates is of order $N$.

        Since we have $p = \frac{t}{\Delta_t}$ Trotter steps, the total circuit has complexity of order: 
        \[O\left(N p\right) = O\left(N \frac{t}{\Delta_t}\right)\]

        Finally, as we took the first-order Trotter decomposition, the error is simply $\Delta_t^2 p$, since it accumulates every trotter step. Note that we could have had a smaller error if we had consider the second-order Trotter decomposition.
    \end{subparag}
\end{parag}

\begin{parag}{General time evolution}
    The approach we used above works in the very general case to find $\ket{\psi\left(t\right)}$.

    However, it requires that the initial state $\ket{\psi\left(0\right)}$ can be prepared. For instance, if this is a product state, this is fine. However, if this is the ground state of an arbitrary Hamiltonian, then it's probably not fine.

    Indeed, the main strategy we used to find the ground state of an arbitrary Hamiltonian was imaginary time evolution, which is just a mathematical curiosity, not a physical property. This is interesting because it shows that, even if we have access to a quantum computer, it does not mean that we can solve every quantum problem in polynomial time.
\end{parag}

\subsection{Energy spectrum}

\begin{parag}{Goal}
    We now analyse how to compute the energy spectrum of an arbitrary Hamiltonian, which will also prove itself to be very hard.
\end{parag}

\begin{parag}{Algorithm: Quantum phase estimation}
    Let $\mathcal{\mathcal{H}}$ be an arbitrary Hamiltonian. Suppose that we prepared a state which is an exact eigenstate of $\mathcal{H}$: 
    \[\ket{\psi} = \ket{E_n}, \mathspace \hat{\mathcal{H}} \ket{E_n} = E_n \ket{E_n}.\]

    We now consider the following circuit, leaving $\hat{U}\left(t\right) = \exp\left(-i \hat{\mathcal{H}} t\right)$.
    \begin{center}
    \begin{quantikz}[slice all] 
        \lstick{\ket{0}} & \gate{H} & \ctrl{1} & \gate{H} & \meter{} \\
        \lstick{\ket{\psi}} & & \gate{U} & & 
    \end{quantikz}
    \end{center}

    Then, leaving $\phi = E_n t$, the probability to measure $+1$ on the first qubit is: 
    \[\prob_1\left(1\right) = \cos\left(\frac{\phi}{2}\right)^2.\]
    
    \begin{subparag}{Implication}
       Let's say that we do manage to construct the state $\ket{E_n}$. We know how to construct $\hat{U}\left(t\right)$ thanks to the previous analysis. Then, we are able to run this experiment many times to evaluate the probability $\prob_1\left(1\right)$ experimentally, from which we can extract $\phi = E_n t$ and hence the energy $E_n$.

       However, being able to prepare $\ket{E_n}$ is hard.
    \end{subparag}

    \begin{subparag}{Remark}
        Time evolution is given by:
        \[\ket{\psi\left(t\right)} = e^{-i \hat{\mathcal{H}} t} \ket{\psi} = e^{-i E_n t} \ket{\psi} = e^{-i \phi} \ket{\psi}.\]

        Hence, this algorithm allows us to compute the phase of some state, which may be surprising. Note however that, here, this is not a global phase thanks to the ancilla qubit $\ket{0}$, this is just a local phase. 
    \end{subparag}

    \begin{subparag}{Proof}
        After the first Hadamard, we directly have:
        \[\ket{\psi_1} = \frac{\ket{0} + \ket{1}}{\sqrt{2}} \otimes \ket{\psi} = \frac{\ket{0} \ket{\psi} + \ket{1}\ket{\psi}}{\sqrt{2}}.\]
        
        After the controlled-unitary, we then have, using the fact $\ket{\psi}$ is an eigenvector of $\hat{\mathcal{H}}$: 
        \[\ket{\psi_2} = \frac{\ket{0}\ket{\psi} + e^{-i \phi} \ket{1} \ket{\psi}}{\sqrt{2}}.\]

        This is promising, but we still cannot measure $\phi$ in the $Z$-basis. However, after the second Hadamard: 
        \[\ket{\psi_3} = \frac{\left(1 + e^{-i\phi}\right)\ket{0}\ket{\psi} + \left(1 - e^{-i\phi}\right)\ket{1}\ket{\psi}}{2}.\]
        
        So, the probability to measure $0$ on the first qubit is: 
        \autoeq{\prob_1\left(1\right) = \frac{1}{4} \left|1 + e^{-i \phi}\right|^2 = \frac{1}{4} \left[1 + 1 + e^{i\phi} + e^{-i\phi}\right] = \frac{1}{4}\left[2 + 2 \cos\left(\phi\right)\right] = \cos\left(\frac{\phi}{2}\right)^2.}

        \qed
    \end{subparag}
\end{parag}

\begin{parag}{Algorithm: Exact energy}
    Let's now suppose that $\ket{\psi}$ is arbitrary. We consider its decomposition in the eigenstates of $\hat{\mathcal{H}}$:
    \[\ket{\psi} = \sum_{\ell=0}^{2^N - 1} c_{\ell} \ket{E_{\ell}}.\]

    We consider the exact same circuit, still leaving $U\left(t_j\right) = \exp\left(-i t_j \hat{\mathcal{H}}\right)$:
    \begin{center}
    \begin{quantikz} 
        \lstick{\ket{0}} & \gate{H} & \ctrl{1} & \gate{H} & \meter{} \\
        \lstick{\ket{\psi}} & & \gate{U\left(t_j\right)} & & 
    \end{quantikz}
    \end{center}

    Then, the probability to measure $1$ on the first qubit is:
    \[\prob_1\left(1, t_j\right) = \sum_{\ell} \left|c_{\ell}\right|^2 \cos\left(\frac{E_{\ell}t_j}{2}\right)^2.\]

    \begin{subparag}{Consequence}
        We can run this algorithm for $p$ different values of $t_j \in \left\{0, \Delta_t, 2\Delta_t, \ldots, T\right\}$. For each of these times, we evaluate $\prob_1\left(1, t_j\right)$. We can then do a discrete Fourier transform to be able to reconstruct the frequencies and hence the energies.

        Now, to be able to do that, we need to be able to detect energy differences of order:
        \[\delta E = \frac{2\pi}{p \Delta_t} = \frac{2\pi}{t}.\]
        
        To get energy of a low resolution, we need to run this for a long time.

        Moreover, the Fourier transform will give us frequencies, but with amplitude $\left|c_{\ell}\right|^2$. So, we moreover also need $\left|c_{\ell}\right|^2 \in \lang{poly}\left(\frac{1}{N}\right)$ since, otherwise, we need exponential precision and hence exponential runtime. This is the crucial limitation: in general, we are not able to make a state with this constraint. In particular, with a random state, the coefficients are exponentially small. Note that this was not a problem for the classical case, where it can be exponentially small (but non-zero), since we had exponentially fast converging algorithms. 
    \end{subparag}

    \begin{subparag}{Proof}
        We will not make the full derivation here, since it is completely similar to the previous one. However, essentially, using linearity and the previous result, we have:
        \[\ket{\psi_3\left(t_j\right)} = \frac{1}{2} \sum_{\ell} c_{\ell} \left[\left(1 + e^{-i E_{\ell} t_j}\right)\ket{0}\ket{E_{\ell}} + \left(1 - e^{-iE_{\ell}t_j}\right)\ket{1}\ket{E_{\ell}}\right].\]

        This then gives:
        \[\prob_1\left(1, t_j\right) = \sum_{\ell} \left|c_{\ell}\right|^2 \frac{\left|1 + e^{-i E_{\ell} t_j}\right|^2}{4} = \sum_{\ell} \left|c_{\ell}\right|^2 \cos\left(\frac{E_{\ell}t_j}{2}\right)^2.\]
        
        \qed
    \end{subparag}
\end{parag}

\end{document}
