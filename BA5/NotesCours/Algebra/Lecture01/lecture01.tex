% !TeX program = lualatex
% Using VimTeX, you need to reload the plugin (\lx) after having saved the document in order to use LuaLaTeX (thanks to the line above)

\documentclass[a4paper]{article}

% Expanded on 2023-09-25 at 15:20:03.

\usepackage{../../style}

\title{Algebra}
\author{Joachim Favre}
\date{Lundi 25 septembre 2023}

\begin{document}
\maketitle

\lecture{1}{2023-09-25}{Pretty colours}{
\begin{itemize}[left=0pt]
    \item Explanation of the simple induction principle, the well ordering principle and the strong induction principle; and proof that they are equivalent.
    \item Definition of divisibility, prime and composite numbers.
    \item Proof of the Euclidean division theorem.
    \item Definition of greatest common divisor.
    \item Proof of the Euclidean algorithm, and of Bézout's construction.
    \item Definition of Euler's totient function.
\end{itemize}

}

\section{Introduction}

\subsection{Axioms of natural numbers}

\begin{parag}{Definition: Natural numbers}
    The set of \important{natural numbers} is the set of non-negative numbers: 
    \[\mathbb{N} = \left\{0, 1, 2, \ldots\right\}\]

    \begin{subparag}{Remark}
        When we want to exclude the zero, we can write: 
        \[\mathbb{Z}_+ = \left\{1, 2, \ldots\right\}\]
    \end{subparag}
\end{parag}

\begin{parag}{Axiom: Induction principle}
    Let $S \subset \mathbb{N}$ be a set such that $0 \in S$ and $n \in S \implies n+1 \in S$.

    Then, $S = \mathbb{N}$.
\end{parag}

\begin{parag}{Axiom: Well ordering principle}
    Let $S \subset \mathbb{N}$ be non-empty.

    Then, $S$ has a least element.
\end{parag}

\begin{parag}{Axiom: Strong induction principle}
    Let $S \subset \mathbb{N}$ be a set such that $0 \in S$ and $\left\{0, \ldots, n\right\} \subset S \implies n + 1 \in S$.

    Then, $S = \mathbb{N}$.
\end{parag}

\begin{parag}{Theorem: Equivalence}
    We have the following implications: 
    \[\text{induction} \implies \text{strong induction} \implies \text{well ordering} \implies \text{induction}\]
    
    In other words, all three axioms are equivalent. We can take any of the propositions as an axiom, and the other two will come from that.

    \begin{subparag}{Proof 1}
        We want to show that the induction principle implies the strong induction principle. In other words, we suppose the induction principle holds and we want to prove the strong induction principle.

        Let $S \subset \mathbb{N}$ be an arbitrary set such that $0 \in S$ and $\left\{0, \ldots, n\right\} \subset S \implies n + 1 \in S$. Also, let $P\left(n\right)$ be the proposition $\left\{0, \ldots, n\right\} \subseteq S$. 

        We know that $P\left(0\right)$ is true since $0 \in S$.

        Now, let's assume that $P\left(k\right)$ is true. This means that $\left\{0, \ldots, k\right\} \subset S$. By our supposition on $S$, this implies that $k+1 \in S$. Putting both together, we get that $\left\{0, \ldots, k+1\right\} \subset S$ and thus that $P\left(k+1\right)$ is true. However, we have proven that $P\left(k\right) \implies P\left(k+1\right)$. This implies by regular induction that $P\left(n\right)$ is true for all $n \in \mathbb{N}$, and thus that strong induction indeed holds. 
    \end{subparag}
    
    \begin{subparag}{Proof 2}
        The proof that the strong induction principle implies the well ordering principle will be done in the first exercise series.
    \end{subparag}

    \begin{subparag}{Proof 3}
        We want to show that the well ordering principle implies the induction principle.

        Let $S \subset \mathbb{N}$ be an arbitrary set such that $0 \in S$ and $n \in S \implies n + 1 \in S$; and let $S' = \mathbb{N} \setminus S$.

        We suppose for contradiction that $S' \neq \o$. By the well ordering principle, this implies that there exists a least element $k \in S'$.

        We know that $0 \not \in S'$ because $0 \in S$ by hypothesis. This means that $m = k-1 \in \mathbb{N}$. We know $m \not \in S'$ since $k$ was the least element; and thus $m \in S$. Moreover, we know that $m \in S \implies m+1 \in S$ and thus $k \in S$. We have shown that $k \in S$ and $k \in S'$ even though $S \cap S' = \o$. This is our contradiction, showing that $S' = \o$ and thus $S = \mathbb{N}$. We have thus indeed shown the induction principle.

        \qed
    \end{subparag}
    
    \begin{subparag}{Remark}
        This last proof uses a structure named ``the minimal criminal''. We pick an element which should be minimal with a certain property, but we construct a smaller one with the same property.
    \end{subparag}
\end{parag}

\begin{parag}{Definition: Division}
    Let $a, b \in \mathbb{Z}$, such that $a \neq 0$. We define that $a$ \important{divides} $b$, written $a \divides b$ or $b \in a\mathbb{Z}$ (or $b = a \mathbb{Z}$ \textit{(though this notation is awful))}, when $\exists c \in \mathbb{Z}$ such that: 
    \[b = ac\]
\end{parag}

\begin{parag}{Definition: Prime number}
    Let $n \in \mathbb{Z}_+ = \left\{1, 2, \ldots\right\}$.

    If $p > 1$ and its only divisors are $1$ and $p$, then it is said to be a \important{prime}. Otherwise, it is said to be a \important{composite}.
\end{parag}

\begin{parag}{Theorem}
    Any natural number $n > 1$ has a prime divisor.

    \begin{subparag}{Proof}
        Let $S \subset \mathbb{N}_{n \geq 2}$ be the set of numbers strictly greater than 1 which have no prime divisor. 

        Let's suppose for contradiction that $S \neq \o$. Then, there exists a least element $k \in S$. We notice that $k$ cannot be a prime since, otherwise, $k \divides k$ would be a prime divisor. Since $k$ is composite, we can write $k = ab$ for $1 < a, b < k$. Since $a < k$ and $k$ was the least element, we know that $a \not \in S$. Thus, $a$ must have a prime divisor $p$, meaning that we can write $a = pt$ for some $t \in \mathbb{N}$. This yields that: 
        \[k = ab = ptb\]
        
        However, this means that $k$ has a prime divisor $p$, which is our contradiction. This implies that $S = \o$, which concludes our proof.

        \qed
    \end{subparag}
\end{parag}

\begin{parag}{Theorem}
    Any integer $n > 1$ is a product of primes.

    \begin{subparag}{Proof}
        The proof is left as an exercise to the reader, since it is very similar to the previous one.
    \end{subparag}
\end{parag}

\begin{parag}{Theorem}
    Let $n \in \mathbb{N}$ be such that $n > 1$. 

    The prime factorisation of $n$ is unique.

    \begin{subparag}{Proof}
        Let $n$ be the smallest number be the smallest integer greater than 1 with two different factorisations: 
        \[n = p_1 \cdots p_k = q_1 \cdots q_m\]
        
        For all $\left(i, j\right)$, we have $p_i \neq q_j$ since, otherwise, we could simplify it on both side and get another smaller integer $n$ with the same property. 

        Let's suppose without loss of generality that $q_1 > p_1$ and let:
        \[t = \left(q_1 - p_1\right)q_2 \cdots q_m > 0\]

        Then: 
        \autoeq{t = \overbrace{q_1 \cdots q_m}^{= n} - p_1 q_2 \cdots q_m = n - p_1 q_2 \cdots q_m = p_1 \cdots p_k - p_1 q_2 \cdots q_m = p_1 \left(p_2 \cdots p_k - q_2 \cdots q_m\right)}
        
        However, this implies that $p_1 \divides t$. Thus, this implies that $t \geq p_1 > 1$. Now, we saw that $t = \left(q_1 - p_1\right)q_2 \cdots q_m$ and $p_1 \neq q_j$ for all $j$, so $p_1 \divides t$ implies that $p_1$ must necessarily divide $\left(q_1 - p_1\right)$. In other words, there exists a $s \in \mathbb{N}$ such that: 
        \[q_1 - p_1 = sp_1 \iff q_1 = \left(s+1\right)p_1 \iff p_1 \divides q_1\]

        This is our contradiction, a prime cannot divide a different prime.

        \qed
    \end{subparag}
\end{parag}

\subsection{Basic properties of integer arithmetic}

\begin{parag}{Theorem: Euclidean division}
    Let $n \in \mathbb{Z}$ and $d \in \mathbb{Z}_+$. There exists two integers $q, r \in \mathbb{Z}$ such that $n = qd + r$ and $0 \leq r < d$. Moreover, these $q, r$ are unique.

    The $q$ is named the \important{quotient}, and the $r$ is named the \important{remainder}.

    \begin{subparag}{Proof of existence}
        Let $n \in \mathbb{Z}$ and $d \in \mathbb{Z}_+$.

        We consider the following set: 
        \[S = \left\{n - k d | k \in \mathbb{Z}\right\} \cap\mathbb{N} = \left\{n - kd | k \in \mathbb{Z} \text{ such that } n \geq kd\right\} \subset \mathbb{N}\]
        
        Intuitively, this represents the set of non-negative numbers with the same rest modulo $d$ as $n$.

        We know that $S \neq \o$. Indeed, if $n \geq 0$, we can pick $k = 0$ to get $n \in S$. Otherwise, if $n < 0$, we can take $k = \left|n\right| + 1$, which is such that $kd > \left|n\right|$ and thus $n + kd \in S$.

        By the well-ordering principle, this means that there exists a smallest element $r \in S$, i.e.: 
        \[r = n - kd \iff n = kd + r\]
        
        Now, let's suppose for contradiction that $r \geq d$. This means that: 
        \[n - \left(k+1\right)d = n - kd - d = r - d \geq 0\]
        
        However, this is a contradiction since $r$ was the least element in $S$, and we constructed a new one smaller. We have thus indeed shown the existence of $r, k$ such that $n = kd + r$ and $0 \leq r < d$.
    \end{subparag}

    \begin{subparag}{Proof of unicity}
        Let's suppose for contradiction that we can write: 
        \[n = r_1 + q_1 d = r_2 + q_2 d\]
        
        If $q_1 = q_2$, then necessarily we have $r_1 = r_2$ and they are not unique. Thus, let's suppose without loss of generality that $q_1 > q_2$. This implies that: 
        \[r_2 = \underbrace{\left(q_1 - q_2\right)}_{> 0}d + \underbrace{r_1}_{\geq 0} \geq d\]
        
        However, this is a contradiction to the definition of $r_2$, which should be such that $0 \leq r_2 < d$.

        \qed
    \end{subparag}

    \begin{subparag}{Remark}
        The integers $\left(q, r\right)$ can be found by doing an integer long division \textit{(division en colonne} in French).
    \end{subparag}
\end{parag}

\begin{parag}{Definition: GCD}
    Let $a, b \in \mathbb{Z}$.

    Their \important{greatest common divisor}, written $\gcd\left(a, b\right)$, is the greatest positive integer $c$ such that $c \divides a$ and $c \divides b$.
\end{parag}

\begin{parag}{Theorem: Euclidean algorithm}
    Let $n, q \in \mathbb{Z}$ and $d \in \mathbb{Z}_+$ such that $n = dq + r$ and $0 \leq r < d$.

    Then, $\gcd\left(n, d\right) = \gcd\left(d, r\right)$.

    \begin{subparag}{Proof}
        Let $c_1$ be an arbitrary divisor of both $n$ and $d$. Since $n = qd + r$, this means that $c_1$ also divides $r = n - qd$.

        Let $c_2$ be an arbitrary divisor of both $d$ and $r$. Since $n = qd + r$, this means that $c_2$ also divides $n$.

        However, this implies that the set of common divisors of both $n$ and $d$ is the same as the set of common divisors of $d$ and $r$. This indeed yields that: 
        \[\gcd\left(n, d\right) = \gcd\left(d, r\right)\]
        
        \qed
    \end{subparag}

    \begin{subparag}{Euclidean algorithm}
        This lemma is a proof that the Euclidean algorithm works. 

        Let $d_1 = q_1 d_2 + d_3$. We know that $\gcd\left(d_1, d_2\right) = \gcd\left(d_2, d_3\right)$. Thus, we can divide $d_1$ by $d_2$ to get $d_3$, then divide $d_2$ by $d_3$, and so on until we get $d_k = 0$. We then know that: 
        \[\gcd\left(d_1, d_2\right) = d_{k-1}\]
        
        This always terminate since the sequence of the $d_i$ is strictly decreasing.
    \end{subparag}

    \begin{subparag}{Example}
        Let us consider: 
        \[\gcd\left({\color{MyGreen}492}, {\color{red}126}\right)\]

        We make the following divisions, computed using long division: 
        \[{\color{MyGreen}492} = {\color{red}126}\cdot 3 + {\color{blue}114}, \mathspace {\color{red}126} = {\color{blue}114}\cdot 1 + {\color{orange}12}, \mathspace {\color{blue}114} = {\color{orange}12}\cdot 9 + {\color{MyPurple}6}, \mathspace {\color{orange}12} = {\color{MyPurple}6}\cdot 2 + 0\]
        
        Since we got a rest of 0 at the end, it yields that the GCD was the rest of the division right before, \textcolor{MyPurple}{6}: 
        \[\gcd\left({\color{MyGreen}492}, {\color{red}126}\right) = {\color{MyPurple}6}\]
        
    \end{subparag}
    
\end{parag}

\begin{parag}{Corollary 1}
    Let $a, b \in \mathbb{Z}_+$. Then, there exists $x, y \in \mathbb{Z}$ such that: 
    \[\gcd\left(a, b\right) = ax + by\]
    
    \begin{subparag}{Proof}
        This can be proven by using an algorithm. The idea is to first use the Euclidean algorithm, and then rewind back to our numbers.

        Let's make a proof by example. In the previous example, we found that: 
        \[\gcd\left({\color{MyGreen}492}, {\color{red}126}\right) = {\color{MyPurple}6}\]

        The intermediate steps of the Euclidean algorithm were:
        \[{\color{MyGreen}492} = {\color{red}126}\cdot 3 + {\color{blue}114}, \mathspace {\color{red}126} = {\color{blue}114}\cdot 1 + {\color{orange}12}, \mathspace {\color{blue}114} = {\color{orange}12}\cdot 9 + {\color{MyPurple}6}, \mathspace {\color{orange}12} = {\color{MyPurple}6}\cdot 2 + 0\]

        We start with the second-to-last equality: 
        \[\gcd\left(492, 126\right) = {\color{MyPurple}6} = {\color{blue}114} - {\color{orange}12}\cdot 9\]
        
        We then use  the third-to-last equality to see that ${\color{orange}12} = {\color{red}126} - {\color{blue}114}\cdot 1$, so: 
        \[{\color{MyPurple}6} = {\color{blue}114} - \left({\color{red}126} - {\color{blue}114}\cdot 1\right)\cdot 9 = {\color{blue}114}\cdot 10 - {\color{red}126}\cdot 9\]
        
        Finally, the first equality yields that ${\color{blue}114} = {\color{MyGreen}492} - {\color{red}126}\cdot 3$, and thus:
        \[{\color{MyPurple}6} = \left({\color{MyGreen}492} - 3\cdot {\color{red}126}\right)\cdot 10 - {\color{red}126}\cdot 9 = {\color{MyGreen} 492}\cdot 10 - {\color{red}126}\cdot 39\]

        To sum up, we found that: 
        \[{\color{MyPurple}6} = \gcd\left({\color{MyGreen}492}, {\color{red}126}\right) = {\color{MyGreen} 492}\cdot 10 - {\color{red}126}\cdot 39\]

        This is the end of our algorithm.
    \end{subparag}
\end{parag}

\begin{parag}{Corollary 2}
    Let $a, b \in \mathbb{Z}_+$ and $d = \gcd\left(a, b\right)$.

    The equation $ax + by = c \in \mathbb{Z}$ has a solution $x, y \in \mathbb{Z}$ if and only if $c \in d \mathbb{Z}$ (meaning $d \divides c$).

    \begin{subparag}{Proof $\implies$}
        Let $c \in \mathbb{Z}$ and $x, y \in \mathbb{Z}$ be solutions to $ax + by = c$.

        We notice that, by definition of the greatest common divisor, $d \divides a$ and $d \divides a$. In particular, this means that $d \divides ax + by$, and thus this necessarily means that $d \divides c$.
    \end{subparag}

    \begin{subparag}{Proof $\impliedby$}
        Let $c = kd$ for some $k \in \mathbb{Z}$. 

        By our first corollary, we know we can find $x, y \in \mathbb{Z}$ such that: 
        \[ax + by = \gcd\left(a, b\right) = d\]

        Multiplying both sides by $k$, this implies that: 
        \[akx + bky = kd \iff a \widetilde{x} + b \widetilde{y} = c\]
        for $\widetilde{x} = kx$ and $\widetilde{y} = ky$.
        
        We have indeed found a solution to $ax + by = c$.

        \qed
    \end{subparag}
\end{parag}

\begin{parag}{Definition: Relatively prime}
    Let $a, b \in \mathbb{Z}_+$. 

    We say that they are \important{relatively prime} if: 
    \[\gcd\left(a, b\right) = 1\]
\end{parag}


\begin{parag}{Bézout's theorem}
    Let $a, b \in \mathbb{Z}_+$.

    They are relatively prime if and only if there exists $x, y \in \mathbb{Z}$ such that: 
    \[ax + by = 1\]
    
    \begin{subparag}{Proof}
        This is a special case of the second corollary.
    \end{subparag}
\end{parag}

\begin{parag}{Definition: Euler's totient function}
    Let $n \in \mathbb{Z}_+$.

    We define $\phi\left(n\right)$ to be the number of positive integers $k$, such that $1 \leq k \leq n$ and: 
    \[\gcd\left(n, k\right) = 1\]
    
    \begin{subparag}{Example}
        For instance, since 1 is only coprime with 1: 
        \[\phi\left(1\right) = 1\]
        
        Also, since 3 is coprime with both 1 and 2: 
        \[\phi\left(3\right) = 2\]
   \end{subparag}
\end{parag}

\begin{parag}{Property}
    Let $p$ be a prime number. Then: 
    \[\phi\left(p\right) = p-1\]

    \begin{subparag}{Proof}
        Any of the $p-1$ numbers strictly smaller than $p$ is coprime with $p$, so this gives our result.

        \qed
    \end{subparag}
\end{parag}

\begin{parag}{Property}
    Let $p$ and $q$ be different prime numbers. Then: 
    \[\phi\left(pq\right) = \left(p-1\right)\left(q-1\right)\]

    \begin{subparag}{Proof}
        There are $pq$ numbers from $1$ to $pq$. $p$ of them are divisible by $q$ and $q$ of them are divisble by $p$. We finally need to be careful by not double counting that $pq$ is divisible by both $p$ and $q$. Thus: 
        \[\phi\left(pq\right) = pq - p - q + 1 = \left(p-1\right)\left(q-1\right)\]

        \qed
    \end{subparag}
\end{parag}

\begin{parag}{Property}
    Let $m, n$ such that $\gcd\left(m, n\right) = 1$. 

    Then: 
    \[\phi\left(nm\right) = \phi\left(n\right)\phi\left(m\right)\]
    
    \begin{subparag}{Proof}
        The proof will be done later, using ring theory.
    \end{subparag}
\end{parag}

\end{document}

