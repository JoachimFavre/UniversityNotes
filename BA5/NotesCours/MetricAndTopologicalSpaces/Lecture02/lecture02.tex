% !TeX program = lualatex
% Using VimTeX, you need to reload the plugin (\lx) after having saved the document in order to use LuaLaTeX (thanks to the line above)

\documentclass[a4paper]{article}

% Expanded on 2023-09-26 at 10:13:19.

\usepackage{../../style}

\title{Topo}
\author{Joachim Favre}
\date{Mardi 26 septembre 2023}

\begin{document}
\maketitle

\lecture{2}{2023-09-26}{The start of the fun stuff}{
\begin{itemize}[left=0pt]
    \item Definition of topological spaces, and some examples.
    \item Definition of metric spaces.
    \item Definition of a basis for a topology, and some examples.
    \item Definition of interior, closure, boundary and dense set.
    \item Proof that the intersection between the closure of a set and the closure of its complement yields its boundary.
\end{itemize}

}

\section{Topological space}

\subsection{Definitions}


\begin{parag}{Definition: Topology}
    Given a set $X$, a \important{topology} on $X$ is a collection $\tau$ of subsets of $X$ such that:
    \begin{enumerate}
        \item $\o \in \tau$ and $X \in \tau$
        \item Given a family $\left\{A_{\alpha}\right\}$ in $\tau$, we have: 
        \[\bigcup_{\alpha}^{} A_{\alpha} \in \tau\]
        \item If $U, V \in \tau$, then $U \cap V \in \tau$.
    \end{enumerate}

    We then say a set $X$ endowed with a topology $\tau$, shortened $\left(X, \tau\right)$, is a \important{topological space}. We call the elements of $\tau$ \important{open}.

    \begin{subparag}{Remark}
        The third property shows that any finite intersection is in $\tau$, whereas the second one states that any infinite union is in $\tau$.
    \end{subparag}

    \begin{subparag}{Personal remark: Intuition}
        Intuitively, this expands the notion of open subsets for any set $X$, even for non-metric spaces.

        This first says that $\o$ and $X$ are open. Second, any infinite union of open set is also open. Third, any finite intersection of open set is open.
    \end{subparag}
\end{parag}

\begin{parag}{Definition: Closed}
    If $\left(X, \tau\right)$ is a toplogical space and $F \subset X$ is of the form $F = X \setminus U$ for some $U \in \tau$, then $F$ is called \important{closed}.

    \begin{subparag}{Intuition}
        Informally, the complement of an element of $\tau$, which we called open, is named closed.
    \end{subparag}

    \begin{subparag}{Properties}
        We can translate the three properties of open set to closed ones:
        \begin{enumerate}
            \item $\o$ and $X$ are closed. In particular, this means that some sets can be both open and closed.
            \item The intersection of any infinite collection of closed set is closed.
            \item The union of any finite collection of closed set is closed.
        \end{enumerate}
    \end{subparag}
\end{parag}

\begin{parag}{Example 1}
    Let $R = \mathbb{C}\left[x_1, \ldots, x_n\right]$ be the ring of $n$\Th degree polynomial with complex coefficients. We define the following mapping of an element of $R$ to a subset of $\mathbb{C}^n$, for any $F \subset R$: 
    \[D\left(F\right) = \left\{x = \left(x_1, \ldots, x_n\right) \in \mathbb{C}^n | f\left(x\right) \neq 0 \text{ for some } f \in F\right\}\]

    We want to show that $\left(\mathbb{C}^n, \left\{D\left(F\right)\right\}\right)$ is a toplogical space. In fact, $\tau = \left\{D\left(F\right)\right\}$ is named the Zariski topology.
    
    \begin{subparag}{Property 1}
        Using a small abuse of notation, we notice that: 
        \[D\left(0\right) = \left\{x \in \mathbb{C}^n | 0\left(x\right) \neq 0\right\} = \o\]
        
        Also: 
        \[D\left(1\right) = \left\{x \in \mathbb{C}^n | 1\left(x\right) \neq 0\right\} = \mathbb{C}^n\]
    \end{subparag}
    
    \begin{subparag}{Property 2}
        We pick a family $\left\{F_{\alpha}\right\}$ (which can be infinite, even uncountable) in $R$. We then have: 
        \[\bigcup_{\alpha}^{} D\left(F_{\alpha}\right) = D\left(\bigcup_{\alpha}^{} F_{\alpha}\right)\]
    \end{subparag}
    
    \begin{subparag}{Property 3}
        Let $F, G \subset R$. Then: 
        \[D\left(F\right) \cap D\left(G\right) = D\left(FG\right)\]
        where: 
        \[FG = \left\{fg | f \in F \text{ and } g \in G\right\}\]
    \end{subparag}
\end{parag}


\begin{parag}{Example 2}
    Let $X$ be any set. Then, $\tau = \mathcal{P}\left(X\right)$ is a topology, named the discrete topology.

    This will be proven in the second series.
\end{parag}

\begin{parag}{Example 3}
    Let $X$ be any set. Then, $\tau = \left\{\o, X\right\}$ is a topology, named the indiscrete topology.
\end{parag}

\begin{parag}{Example 4}
    Let $X$ be any set. Then, the following is a topology, named the co-finite topology: 
    \[\tau = \left\{A \subset X | X \setminus A \text{ is finite, or } A = \o\right\}\]
\end{parag}

\begin{parag}{Definition: Metric space}
   Let $M$ be a set and $d: M \times M \mapsto \mathbb{R}$ be some function.

   $\left(M, d\right)$ is a \important{metric space} of distance $d$, if it is such that, for any $x,y,z \in M$:
   \begin{itemize}
       \item (Reflexivity) $d\left(x, y\right) = 0 \iff x = y$
       \item (Symmetry) $d\left(x, y\right) = d\left(y, x\right)$
       \item (Triangle inequality) $d\left(x, y\right) + d\left(y, z\right) \geq d\left(x, z\right)$
   \end{itemize}
\end{parag}


\begin{parag}{Coherence of the definition}
    Let $\left(M, d\right)$ be a metric space. We declare $U \subset M$ to be open (i.e., in the topology) if and only if for any $p \in U$ we can find a $\delta > 0$ such that we can find an open ball of radius $\delta$ entirely inside $U$: 
    \[B\left(p, \delta\right) = \left\{q \in M | d\left(p, q\right) < \delta\right\} \subset U\]

    We note the topology $\tau^d$.

    \begin{subparag}{Observations 1}
        We notice that open balls are open.
    \end{subparag}
    
    \begin{subparag}{Observation 2}
        Given $p \in M$, $\left\{p\right\}$ is closed.
    \end{subparag}
    
    \begin{subparag}{Remark}
        Let $d$ be the discrete metric:
        \begin{functionbypart}{d\left(x, y\right)}
            0, & \text{if } x = y \\
            1, & \text{otherwise}
        \end{functionbypart}
        
        Then $\tau^d$ is the discrete topology, $\tau^d = \mathcal{P}\left(X\right)$.
    \end{subparag}
\end{parag}

\begin{parag}{Definition: Basis}
    Let $X$ be a set.

    A collection $\mathcal{B}$ of subsets of $X$ is called a \important{basis} for a topology on $X$ if and only:
    \begin{enumerate}
        \item For any $x \in X$, we can find $B \in \mathcal{B}$ such that $x \in B$.
        \item For any $B_1, B_2 \in \mathcal{B}$ and any $x \in B_1 \cap B_2$, then $\exists B \in \mathcal{B}$ such that $x \in B \subset B_1 \cap B_2$.
    \end{enumerate}

    \begin{subparag}{Remark}
        Even though this is a basis for a topology, there is no topology in the definition. However, every time we make such a basis, it generates a topology, as explained in the following proposition.
    \end{subparag}

    \begin{subparag}{Personal remark: Intuition}
        This definition is justified by the following proposition. However, we can informally translate the two properties as:
        \begin{enumerate}
            \item We must be able to unionise elements of the basis to get $X$. Equivalently, unionising all elements of the basis must yield $X$.
            \item We must be able to unionise elements to make the intersection of any two other elements.
        \end{enumerate}
    \end{subparag}
\end{parag}

\begin{parag}{Proposition}
    Let $X$ be a set and $\mathcal{B}$ be a basis for a topology on $X$.

    Then, the following is a topology (generated by $\mathcal{B}$):
    \[\tau^{\mathcal{B}} = \left\{U \subset X | x \in U \implies \exists B \in \mathcal{B} \text{ such that } x \in B \subset U\right\}\]

    We say that $\mathcal{B}$ is a basis for $\tau^{\mathcal{B}}$.

    \begin{subparag}{Personal remark: Intuition}
        This says that the elements of $\tau^\mathcal{B}$ consists in a union of elements of the base, in all the possible ways. We get all the properties of a topology thanks to the properties of the base; intuitively:
        \begin{enumerate}
            \item $X \in \tau^{\mathcal{B}}$ since, by definition of the base, we must be able to unionise its elements to get $X$. $\o \in \tau^{\mathcal{B}}$ because the union of no set is the empty set (more formally, the definition of $\tau^\mathcal{B}$ accepts the empty set).
            \item Since we already have all the possible unions of our basis inside the topology, we still have that the union of any number of elements from the topology is inside the topology.
            \item The definition of the basis tells us that we can unionise elements from the basis to get the intersection of any two basis element. Thus, when intersecting two elements, we still get an element of the topology.
        \end{enumerate}
    \end{subparag}

    \begin{subparag}{Personal remark: Example}
        Let me take an example from Wikipedia. The set of open intervals in $\mathbb{R}$ is a basis for the metric topology of $\mathbb{R}$. In other words, we can get any open set from this topology by unionising open intervals. This definitely justifies the term ``basis''.
    \end{subparag}
\end{parag}

\begin{parag}{Example 1}
    Let $\left(X, \tau\right)$ be a topological space. We can take $\mathcal{B}$ to be a collection of subsets of $X$ such that any element of $\tau$ is a union of sets in $\mathcal{B}$.

    Then, $\tau^{\mathcal{B}} = \tau$.

    \begin{subparag}{Personal remark}
        This justifies the intuition above.
    \end{subparag}
\end{parag}

\begin{parag}{Example 2}
    Let $\left(M, d\right)$ be a metric space and $\tau^d$ be the metric topology. We also let $\mathcal{B}$ to be the collection of all open balls.

    Then, $\tau^{\mathcal{B}} = \tau^d$.
\end{parag}

\begin{parag}{Example 3}
    Let $\mathbb{C}^n$ with the Zuriski topology $\tau^z$. Also, let $\mathcal{B}$ to be the collection of all $D\left(f\right)$ for all $f \in \mathcal{\mathbb{C}}\left[x_1, \ldots, x_n\right]$.

    Then, $\tau^{\mathcal{B}} = \tau^z$
\end{parag}

\subsection{Intrinsic properties of topologies}

\begin{parag}{Definition: Interior point}
    Let $\left(X, \tau\right)$ be a topological space, and $A \subset X$.

    Given $x \in A$, we say that $x$ is an \important{interior point} of $A$ if and only if there exists some $U \in \tau$ such that $x \in U \subset A$.
\end{parag}

\begin{parag}{Proposition: Interior}
    Let $\left(X, \tau\right)$ be a topological space and $A \subset X$.

    The three following sets are equal:
    \begin{itemize}
        \item $\displaystyle \left\{x \in X | x \text{ is an interior point of }  A\right\}$
        \item The union of all opens (of $\tau$) contained in $A$.
        \item The largest open contained in $A$.
    \end{itemize}
    
    We call this set the \important{interior} of $A$, written $\text{int}\left(A\right)$.
\end{parag}

\begin{parag}{Proposition}
    Let $\left(X, \tau\right)$ be a topological space, and $A \subset X$.

    A set is open if and only if it is equal to its interior: 
    \[A \in \tau \iff A = \text{int}\left(A\right)\]
\end{parag}

\begin{parag}{Definition: Closure point}
    Let $\left(X, \tau\right)$ be a topological space, and $A \subset X$.

    We say that $x \in X$ is a \important{closure point} of $A$ if and only if for any open $U$ containing $x$ ($U \in \tau$ such that $x \in U$) we have $U \cap A \neq \o$. 
\end{parag}

\begin{parag}{Proposition: Closure}
    Let $\left(X, \tau\right)$ be a topological space, and $A \subset X$.

    The three following sets are equal:
    \begin{itemize}
        \item $\displaystyle \left\{x \in X | x \text{ is a closure point of } A\right\}$
        \item The intersection of all closed sets containing $A$.
        \item The smallest closed set containing $A$.
    \end{itemize}
    
    We call this set the \important{closure} of $A$, written $\text{cl}\left(A\right)$ or $\bar{A}$.
\end{parag}

\begin{parag}{Observation}
    Let $\left(X, \tau\right)$ be a topological space, and $A \subset X$.
    
    Then: 
    \[\text{int}\left(A\right) \subset A \subset \text{cl}\left(A\right)\]
\end{parag}

\begin{parag}{Definition: Boundary}
    Let $\left(X, \tau\right)$ be a topological space, and $A \subset X$.

    The \important{boundary} of $A$, written $\partial A$, is the set: 
    \[\partial A = \text{cl}\left(A\right) \setminus \text{int}\left(A\right)\]

    \begin{subparag}{Remark}
        We notice that we always have: 
        \[\partial A \subset \text{cl}\left(A\right)\]
    \end{subparag}

    \begin{subparag}{Property}
        It is possible to show that:
        \[\text{cl}\left(A\right) = A \cup \partial A\]
    \end{subparag}
\end{parag}

\begin{parag}{Definition: Dense set}
    Let $\left(X, \tau\right)$ be a topological space, and $A \subset X$.

    $A$ is said to be \important{dense} in $X$ if and only if for every $U \in \tau \setminus \o$, then $U \cap A \neq \o$.
\end{parag}

\begin{parag}{Proposition}
    Let $\left(X, \tau\right)$ be a topological space, and $A \subset X$. 

    $A$ is dense if and only if $\text{cl}\left(A\right) = X$.
\end{parag}

\begin{parag}{Proposition}
    Let $\left(X, \tau\right)$ be a topological space, and $A \subset X$.

    $x \in \partial A$ if and only if, for any $U \in \tau$ such that $x \in U$, it is a closure point of both $A$ and $X \setminus A$: 
    \[U \cap A \neq \o \text{ and } U \cap \left(X \setminus A\right) \neq \o\]

    \begin{subparag}{Personal remark}
        This implies that: 
        \[\partial A = \text{cl}\left(A\right) \cap \text{cl}\left(X \setminus A\right)\]
    \end{subparag}

    \begin{subparag}{Remark}
        Equivalently, $x \not \in \partial A$ if and only if, there exists a $U \in \tau$ for which $x \in U$, such that: 
        \[U \cap A = \o \text{ or } U \cap \left(X \setminus A\right) = \o\]

        We will use this formulation for the proof.
    \end{subparag}

    \begin{subparag}{Proof $\implies$}
        Let $x \not \in \partial A$. Since $\partial A = \text{cl}\left(A\right) \setminus \text{int}\left(A\right)$ by definition, it either means that $x \in \text{int}\left(A\right)$ or $x \in X \setminus \text{cl}\left(A\right)$.

        First, let's suppose that $x \in \text{int}\left(A\right)$. Then, to show that it is not a closure point of $X \setminus A$, we can take $U = \text{int}\left(A\right) \subset A$. This is indeed such that $x \in U$, and: 
        \[U \cap \left(X \setminus A\right) = \o\]
        
        Let's now suppose that $x \in X \setminus \text{cl}\left(A\right)$. To show that it is not a closure point of $X$, we can take $U = X \setminus \text{cl}\left(A\right) \subset X \setminus A$ which is indeed such that $x \in U$ and: 
        \[U \cap A = \o\]
    \end{subparag}

    \begin{subparag}{Proof $\impliedby$}
        Let $x \in X$. We suppose that there exists a $U \in \tau$ such that $x \in U$ and: 
        \[U \cap A = \o \text{ or } U \cap \left(X \setminus A\right) = \o\]
        
        First, let's suppose that $U \cap A = \o$. This yields that $A \subset X \setminus U$. Since $X \setminus U$ is moreover closed, this implies that $\text{cl}\left(A\right) \subset X \setminus U$. However, $x \not \in X \setminus U$, which means that:
        \[x \not \in \text{cl}\left(A\right) \subset \text{cl}\left(A\right) \setminus \text{int}\left(A\right) = \partial A\]

        Let's now suppose that $U \cap \left(X \setminus A\right) = \o$. This means that $U \subset A$. Since $U$ is moreover open, this implies that $U \subset \text{int}\left(A\right)$. Since $x \in U$, this means that $x \in \text{int}\left(A\right)$, yielding:
        \[x \not \in \text{cl}\left(A\right) \setminus \text{int}\left(A\right) = \partial A\]

        \qed
    \end{subparag}
\end{parag}


\end{document}
