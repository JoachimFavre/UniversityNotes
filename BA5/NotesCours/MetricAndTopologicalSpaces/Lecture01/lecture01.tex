% !TeX program = lualatex
% Using VimTeX, you need to reload the plugin (\lx) after having saved the document in order to use LuaLaTeX (thanks to the line above)

\documentclass[a4paper]{article}

% Expanded on 2023-09-19 at 10:11:53.

\usepackage{../../style}

\title{Metric and topological spaces}
\author{Joachim Favre}
\date{Mardi 19 septembre 2023}

\begin{document}
\maketitle

\lecture{1}{2023-09-19}{Recalls}{
\begin{itemize}[left=0pt]
    \item Recall of important theorems on $\mathbb{R}$.
    \item Recall of the definition of bijectivity.
    \item Explanation and example of the Schröder-Bernstein theorem.
\end{itemize}

}

\section{Recalls}

\subsection{Important theorems on $\mathbb{R}$}

\begin{parag}{Bolzano-Weierstrass theorem for $\mathbb{R}$}
    Every bounded sequence in $\mathbb{R}$ admits a converging subsequence.
\end{parag}

\begin{parag}{Heine-Borel theorem for $\mathbb{R}$}
    Let $A \subset \mathbb{R}$. The following are equivalent:
    \begin{itemize}
        \item $A$ is compact, meaning it is closed and bounded.
        \item Every open cover (a cover using only open sets) of $A$ admits a finite subcover.
    \end{itemize}
\end{parag}

\begin{parag}{Extreme value theorem (EVT)}
    Let $f: \left[a, b\right] \mapsto \mathbb{R}$ be a function.

    If $f$ is continuous, then there exists $c, d \in \left[a, b\right]$ such that:  
    \[f\left(c\right) \leq f\left(x\right) \leq f\left(d\right), \mathspace \forall x \in \left[a, b\right]\]
\end{parag}

\begin{parag}{Intermediate value theorem (IVT)}
    Let $f: \left[a, b\right] \mapsto \mathbb{R}$ be a function, and let $y$ be such that $f\left(a\right) \leq y \leq f\left(b\right)$ (assuming without loss of generality that $f\left(a\right) \leq f\left(b\right)$).

    If $f$ is continuous, then there exists a $c \in \left[a, b\right]$ such that: 
    \[f\left(c\right) = y\]
\end{parag}

\subsection{Functions and set}

\begin{parag}{Definition: Injectivity}
    Let $A, B$ be sets and $f: A \mapsto B$ be a function. 

    $f$ is said to be \important{injective} if and only if, for any $a, b \in A$: 
    \[f\left(a\right) = f\left(b\right) \implies a = b\]
\end{parag}

\begin{parag}{Definition: Injectivity}
    Let $A, B$ be sets and $f: A \mapsto B$ be a function. 

    $f$ is said to be \important{surjective} if and only if, for any $b \in A$, there exists a $a \in A$ such that: 
    \[f\left(a\right) = b\]
\end{parag}

\begin{parag}{Definition: Bijectivity}
    Let $A, B$ be sets and $f: A \mapsto B$ be a function. 

    $f$ is said to be \important{bijective} if and only if it is both injective and surjective.
\end{parag}

\begin{parag}{Schröder-Bernstein Theorem}
    Let $A, B$ be sets.

    If there exists an injection from $A$ to $B$ and one from $B$ to $A$, then there exists a bijection between $A$ and $B$.
\end{parag}

\begin{parag}{Proposition}
    The relation $A \sim B$ stating the existence of a bijection $f: A \mapsto B$ is an equivalence relation.

    \begin{subparag}{Proof}
        This proof is left as an exercise to the reader.
    \end{subparag}
\end{parag}

\begin{parag}{Example}
    We wonder if we can find a bijection between $\mathbb{R}^2$ and $\mathbb{R}$. In other words, we want to show the equivalence relation $\mathbb{R} \sim \mathbb{R}^2$. We will first need two lemmas.
\end{parag}

\begin{parag}{Lemma 1}
    We have the following equivalence relation: 
    \[\mathbb{R} \sim \left(0, 1\right)\]

    \begin{subparag}{Proof}
        We can use the following function $f: \mathbb{R} \mapsto \left(0,1\right)$: 
        \[f\left(x\right) = \frac{1}{\pi}\arctan\left(x\right) + \frac{1}{2}\]

        This is indeed bijective, by the properties of the arctan function.

        \qed
    \end{subparag}
\end{parag}

\begin{parag}{Lemma 2}
    We have the following equivalence relation: 
    \[\left(0, 1\right) \times \left(0, 1\right) \sim \left(0, 1\right)\]
    
    \begin{subparag}{Proof}
        We want to use the Schröder-Bernstein theorem. We thus make two injections, one from both direction.

        The first injection is easy: 
        \[\begin{split}
        f: \left(0, 1\right) &\longmapsto \left(0, 1\right) \times \left(0, 1\right) \\
        x &\longmapsto \left(x, 0.5\right)
        \end{split}\]
        
        The second one is a bit more complicated. The idea is to use the decimal expansion of our numbers:
        \[\begin{split}
        f: \left(0, 1\right)\times\left(0,1\right) &\longmapsto \left(0, 1\right) \\
        \left(x, y\right) = \left(\sum_{n = 1}^{\infty} x_n 10^{-n}, \sum_{n=1}^{\infty} y_n 10^{-n}\right) &\longmapsto \sum_{n = 1}^{\infty} \left(x_n 10^{-2n} + y_n 10^{-2n - 1}\right)
        \end{split}\]

        In other words, we map $\left(x, y\right)$ to $0.x_1 y_1 x_2 y_2 \ldots$. This is injective, but not bijective since, for instance, $0.5$ is not in the image.
        
        The last thing that we need to care about for our second injection is the fact that some numbers have two decimal expansions: 
        \[0.5 = 0.4\bar{9}\]
        
        We can remove this ambiguity by always choosing the same (it does not matter which).

        \qed
    \end{subparag}
\end{parag}

\begin{parag}{Theorem}
    We have the following equivalence: 
    \[\mathbb{R} \times \mathbb{R} \sim \mathbb{R}\]

    \begin{subparag}{Proof}
        This proof directly comes from the two previous lemma and the fact the fact that the existence of a bijection is an equivalence relation.
    \end{subparag}
\end{parag}

\end{document}
