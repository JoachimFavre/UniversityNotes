% !TeX program = lualatex
% Using VimTeX, you need to reload the plugin (\lx) after having saved the document in order to use LuaLaTeX (thanks to the line above)

\documentclass[a4paper]{article}

% Expanded on 2024-12-10 at 13:17:01.

\usepackage{../../style}

\title{Algo}
\author{Joachim Favre}
\date{Mardi 10 décembre 2024}

\begin{document}
\maketitle

\lecture{23}{2024-12-10}{Back to greed}{
    \begin{itemize}[left=0pt]
        \item Explanation of the monotone submodular function maximisation problem.
        \item Explanation and proof of the greedy algorithm to solve the monotone submodular function maximisation problem.
        \item Explanation of the submodular function maximisation problem.
        \item Explanation and proof of the double-greedy algorithm to solve the submodular function maximisation problem.
    \end{itemize}
    
}

\subsection{Maximalisation}

\begin{parag}{Definition: Monotone submodular function}
    Let $f: 2^N \mapsto \mathbb{R}$ be a set function.

    $f$ is said to be \important{monotone} if, for all $A, B \subseteq N$ so that $A \subseteq B$. 
    \[f\left(A\right) \leq f\left(B\right).\]

    \begin{subparag}{Remark}
        Equivalently, $f$ is monotone if and only if, for all set $B \subseteq N$ and element $u \in N$, then: 
        \[f\left(u \suchthat B\right) \geq 0.\]
    \end{subparag}
\end{parag}

\begin{parag}{Definition: Normalised set function}
    Let $f: 2^N \mapsto \mathbb{R}$ be a set function.

    It is said to be \important{normalised} if $f\left(\o\right) = 0$.

    \begin{subparag}{Remark}
        For the rest of this section, we will consider that our set functions are normalised.

        Note that this can always be done without loss of generality since, given a set function $f'\left(x\right)$, we can normalise it by considering $f\left(x\right) = f'\left(x\right) - f'\left(\o\right)$; and optimising the first function is equivalent to optimising the second one.
    \end{subparag}
\end{parag}

\begin{parag}{Monotone submodular function maximisation}
    Let $f: 2^N \mapsto \mathbb{R}$ be a normalised monotone submodular function, and $k \in \mathbb{N}$.  

    We want to maximise it subject to the cardinality constraint $k$, i.e. we want to find a $S \subseteq N$ such that $\left|S\right| \leq k$ and $f\left(S\right)$ is maximised: 
    \[S = \argmax_{\substack{T \subseteq N \\ \left|T\right| \leq k}} f\left(T\right).\]

    \begin{subparag}{Remark}
        Maximising submodular functions is an \lang{NP}-hard problem, since it generalises the max-cut problem. So, we will consider approximation algorithms.
    \end{subparag}
\end{parag}


\begin{parag}{Wrong greedy algorithm}
    We want to make a greedy algorithm.

    We may intuitively wish to do the following algorithm:
    \begin{enumerate}
        \item Order $e_1, \ldots, e_n$ so that $f\left(e_1\right) \geq f\left(e_{i+1}\right)$
        \item Output $\left\{e_1, \ldots, e_k\right\}$.
    \end{enumerate}

    However, the issue here is that we only consider how much each element brings, when adding them to the empty set $\o$. We should instead consider how much it increases the objective function, at each iteration. This leads to the following argument.
\end{parag}

\begin{parag}{Greedy algorithm}
    We consider the following greedy algorithm for maximising monotone submodular functions $f$.
    \begin{enumerate}
        \item $S \leftarrow \o$.
        \item For $i = 1, \ldots, k$, take $u_i = \argmax_{u \in N \setminus S} f\left(u \suchthat S\right)$, and increase $S \leftarrow S + u_i$.
        \item Output $S$.
    \end{enumerate}
    
    \begin{subparag}{Example}
        For instance, let us consider the following sets $T_1 = \left\{1, 2, 3, 4\right\}$, $T_2 = \left\{1, 2, 5\right\}$, and $T_3 = \left\{3, 4, 6\right\}$:
        \svghere[0.6]{SubmodularFunctionMaximisationGreedyExample.svg}

        We consider the submodular function $f$ that maps each $S \subseteq \left\{1, 2, 3\right\}$ to $\left|\bigcup_{i \in S} T_i\right|$. Finally, we suppose that $k = 2$.

        The greedy algorithm first picks the set with largest marginal value, $T_1$, with a marginal value of $4$. Then, $T_2$ and $T_3$ have the same marginal value of $1$, so it adds one of them. It thus outputs $5$.

        However, the optimal solution is taking the sets $T_2$ and $T_3$, yielding a value of $6$. This shows indeed that this is only an approximation algorithm.
    \end{subparag}
\end{parag}

\begin{parag}{Lemma}
    Let $f: 2^N \mapsto \mathbb{R}$ be a submodular function, and $S, T \subseteq N$.

    Then: 
    \[\sum_{e \in T \setminus S} f\left(e \suchthat S\right) \geq f\left(S \cup T\right) - f\left(S\right) = f\left(T \suchthat S\right).\]
    
    \begin{subparag}{Intuition}
        For $S = \o$, this theorem reads, when $f$ is normalised: 
        \[\sum_{e \in T} f\left(e\right) \geq f\left(T\right).\]
        
        However, the intuitive definition of a submodular function is that we have a diminishing return, i.e. indeed that the sum of the individual returns should be less than the total return. 
    \end{subparag}

    \begin{subparag}{Proof}
        We make a proof similar to the one we made when proving an equivalent definition of submodular functions.

        We order elements of $T \setminus S$ as $e_1, e_2, \ldots, e_{\left|T \setminus S\right|}$, and we let $T_i = \left\{e_1, e_2, \ldots, e_i\right\}$. We notice that $T_0 = \o$ and $T_{\left|T \setminus S\right|} = T \setminus S$. 

        Using submodularity and collapsing the telescopic series: 
        \autoeq{\sum_{e \in T \setminus S} f\left(e \suchthat S\right) = \sum_{i=1}^{\left|T \setminus S\right|} f\left(e_i \suchthat S\right) \geq \sum_{i=1}^{\left|T \setminus S\right|} f\left(e_i \suchthat S \cup T_{i-1}\right) = \sum_{i=1}^{\left|T \setminus S\right|} \left[f\left(S \cup T_i\right) - f\left(S \cup T_{i-1}\right)\right] = f\left(S \cup T_{\left|T \setminus S\right|}\right) - f\left(S \cup T_0\right) = f\left(S \cup T\right) - f\left(S\right).}
        
        \qed
    \end{subparag}
\end{parag}

\begin{parag}{Theorem}
    Let $f: 2^N \mapsto \mathbb{R}$ be a normalised monotone submodular function, and $k \in \mathbb{N}$. Also, let $S$ be the output of the greedy algorithm on $\left(f, k\right)$, and $O$ be an optimal solution (of cardinality at most $k$).

    Then: 
    \[f\left(S\right) \geq \left(1 - \left(1 - \frac{1}{k}\right)^k\right) f\left(O\right).\]
    
    \begin{subparag}{Implication}
        In particular, this means that, for all $k$: 
        \[f\left(S\right) \geq \left(1 - \frac{1}{e}\right) f\left(O\right) \geq 0.632 f\left(O\right).\]

        The greedy algorithm is thus a $\left(1- \frac{1}{e}\right)$-approximation algorithm, which can be shown to be tight.
    \end{subparag}

    \begin{subparag}{Proof}
        Let $u_i$ be the $i$\Th element selected by the greedy algorithm, and $S_i = \left\{u_1, \ldots, u_i\right\}$. We first show a recursive relation between $f\left(S_i\right)$ and $f\left(S_{i-1}\right)$, which we then use to show our inequality by induction.

        By definition of the greedy algorithm: 
        \[f\left(u_i \suchthat S_{i-1}\right) \geq f\left(u \suchthat S_{i-1}\right), \mathspace \forall u \in N \setminus S_i.\]

        We can make a stronger claim since any $u \in S_i$ that was already picked would not add any value to the function. Indeed, for all $u \in S_i$, $f\left(u \suchthat S_i\right) = 0$, and $f\left(u_i \suchthat S_{i-1}\right) \geq 0$ since $f$ is monotone. Thus:
        \[f\left(u_i \suchthat S_{i-1}\right) \geq f\left(u \suchthat S_{i-1}\right), \mathspace \forall u \in N.\]

        In particular, this holds for all $u \in O \setminus S_{i-1}$. We add up this inequality for all $u \in O \setminus S_{i-1}$, giving: 
        \[\left|O \setminus S_{i-1}\right| f\left(u_i \suchthat S_{i-1}\right) \geq \sum_{u \in O \setminus S_{i-1}} f\left(u \suchthat S_{i-1}\right).\]
        
        We can now use our lemma to simplify this to:
        \[\left|O \setminus S_{i-1}\right|\cdot f\left(u_i \suchthat S_{i-1}\right) \geq f\left(O \suchthat S_{i-1}\right) = f\left(O \cup S_{i-1}\right) - f\left(S_{i-1}\right).\]

        By monotonicity of $f$, we know $f\left(O \cup S_{i-1}\right) \geq f\left(O\right)$. So, using the fact $\left|O\right| \leq k$ by hypothesis: 
        \autoeq{\left|O \setminus S_{i-1}\right|\cdot f\left(u_i \suchthat S_{i-1}\right) \geq f\left(O\right) - f\left(S_{i-1}\right) \iff f\left(u_i \suchthat S_{i-1}\right) \geq \frac{f\left(O\right) - f\left(S_{i-1}\right)}{\left|O \setminus S_{i-1}\right|} \geq \frac{f\left(O\right) - f\left(S_{i-1}\right)}{k}.}

        This is our recurrence relation. We want to use it to show by induction that, for all $i \leq k$: 
        \[f\left(S_i\right) \geq \left[1 - \left(1 - \frac{1}{k}\right)^i\right]f\left(O\right).\]

        The case where $i = 0$ is easy, since $f$ is normalised, so $f\left(S_0\right) = f\left(\o\right) = 0$, and hence both sides are equal to $0$. For the inductive case, we have, using the inductive hypothesis: 
        \autoeq{f\left(S_i\right) = f\left(u_i \suchthat S_{i-1}\right) + f\left(S_{i-1}\right) \geq \frac{f\left(O\right) - f\left(S_{i-1}\right)}{k} + f\left(S_{i-1}\right) = \frac{f\left(O\right)}{k} + \left(1 - \frac{1}{k}\right)f\left(S_{i-1}\right) \geq \frac{f\left(O\right)}{k} + \left(1 - \frac{1}{k}\right)\left[1 - \left(1 - \frac{1}{k}\right)^{i-1}\right]f\left(O\right) = \frac{f\left(O\right)}{k} + \left(1 - \frac{1}{k}\right)f\left(O\right) - \left(1 - \frac{1}{k}\right)^{i}f\left(O\right) = \left[1 - \left(1 - \frac{1}{k}\right)^i\right]f\left(O\right),}
        finishing the proof.

        \qed
    \end{subparag}

    \begin{subparag}{Remark}
        There are other algorithms that relax the hypotheses. For instance, the continuous greedy algorithm and the non-oblivious local search algorithms allow to give a $\left(1 - \frac{1}{e}\right)$-approximation for maximising a monotone submodular function subject to a matroid constraint; i.e. it drops the cardinality constraint, replacing it by a matroid constraint.
    \end{subparag}
\end{parag}

\begin{parag}{Submodular function maximisation}
    Let $f : 2^N \mapsto \mathbb{R}$ be a normalised submodular function.

    We want to maximise it, i.e. we want to find a $S \subseteq N$ such that $f\left(S\right)$ is maximised:
    \[S = \argmax_{T \subseteq N} f\left(T\right).\] 

    \begin{subparag}{Remark}
        This is almost the same problem as the one we were considering before, except that we now drop the hypothesis that $f$ is monotone, and the cardinality constraint $k$.

        The cardinality constraint was important when $f$ was monotone because, for a monotone function, the maximum is always $f\left(N\right)$. For a non-monotone function, a smaller set may give a bigger value.
    \end{subparag}

    \begin{subparag}{Observation}
        For this problem, the greedy algorithm fails terribly.

        Consider the set $N = \left\{u_1, \ldots, u_n, v\right\}$, with the submodular function given by: 
        \[f\left(S\right) = \begin{systemofequations} 2, & \text{if } v \in S,\\ \left|S\right|, & \text{otherwise.} \end{systemofequations}\]

        Note that this $f$ is not monotone, $\left\{u_1, u_2, u_3\right\} \subseteq \left\{u_1, u_2, u_3, v\right\}$, but: 
        \[f\left(\left\{u_1, u_2, u_3\right\}\right) = 3 > f\left(\left\{u_1, u_2, u_3, v\right\}\right) = 2.\]

        The greedy algorithm will start by picking $S_1 = \left\{v\right\}$, since it is the one that gives the best instantaneous return. However, then, there is nothing else it can add to $S_i$ to increase the return, it will not be able to increase the objective function at any other iteration, outputting $2$. On the other hand, the best solution is $k$, with for instance $S_k = \left\{u_1, \ldots, u_k\right\}$.
    \end{subparag}
\end{parag}

\begin{parag}{Double greedy algorithm}
    Let $f: 2^N \mapsto \mathbb{R}$ be a normalised submodular function.

    The double greedy algorithm goes as follows:
    \begin{enumerate}
        \item Order elements $u_1, \ldots, u_n$ arbitrarily.
        \item Start with $X_0 = \o$ and $Y_0 = N$.
        \item For each $i = 1, \ldots, n$: 
            \begin{enumerate}
                \item We note: 
                \[a_i = f\left(X_{i-1} + u_i\right) - f\left(X_{i-1}\right) = f\left(u_i \suchthat X_{i-1}\right),\] 
                \[b_i = f\left(Y_{i-1} - u_i\right) - f\left(Y_{i-1}\right).\]
                \item If $a_i \geq b_i$, we add $u_i$ to both $X_i$ and $Y_i$. In other words, we define $X_i = X_{i-1} + u_i$, and $Y_i = Y_{i-1} + u_i = Y_{i-1}$.
                \item  If $a_i < b_i$, we remove $u_i$ form both set. In other words, we define $X_i = X_{i-1} - u_i = X_{i-1}$ and $Y_i = Y_{i-1} - u_i$.
            \end{enumerate}
        \item Return $X_n$ (which equals $Y_n$).
    \end{enumerate}

    \begin{subparag}{Intuition}
        Intuitively, there are two ways to implement greedy.

        On one hand, we can start with $S = \o$, and then repeatedly add elements $u$ that maximises $f\left(S + u\right) - f\left(S\right)$. On the other hand, we can start with $S = N$, and repeatedly remove elements $u \in S$ that maximises $f\left(S - u\right) - f\left(S\right)$.

        The idea of this double greedy algorithm is to merge the two greedy algorithms.
    \end{subparag}

    \begin{subparag}{Remark}
        A good way to visualise this algorithm is using inclusion vectors, i.e. a vector $x \in \left\{0, 1\right\}^N$ so that $x_i = 1 \iff i \in S$.

        The algorithm starts with: 
        \[X_0 = \left(0, 0, \ldots, 0\right), \mathspace Y_0 = \left(1, 1, \ldots, 1\right).\]
        
        Then, at iteration $i$, the algorithm have made inclusion decisions $d_1, \ldots, d_i \in \left\{0, 1\right\}$, and we will have: 
        \[X_i = \left(d_1, d_2, \ldots, d_i, 0, \ldots, 0\right), \mathspace Y_i = \left(d_1, d_2, \ldots, d_i, 1, \ldots, 1\right).\]
    \end{subparag}
\end{parag}

\begin{parag}{Lemma 1}
    Let $i$ be arbitrary.

    At iteration $i$, the double greedy algorithm is such that: 
    \[a_i + b_i \geq 0.\]

    \begin{subparag}{Remark}
        Note however that we may have $a_i < 0$ or $b_i < 0$, since $f$ is not monotone.
    \end{subparag}

    \begin{subparag}{Proof}
        By construction, at iteration $i-1$, we have: 
        \[X_{i-1} = \left(d_1, \ldots, d_{i-1}, 0, \ldots, 0\right), \mathspace Y_{i-1} = \left(d_1, \ldots, d_{i-1}, 1, \ldots, 1\right).\]
        
        Thus, $X_{i-1} \subseteq Y_{i-1}$, and $u_i \in Y_{i-1}$. We also notice that $X_{i-1} + u_i$ is $X_{i-1}$ with a $1$ at the $i$\Th bit, and $Y_{i-1} - u_i$ is $Y_i$ with a 0 at the $i$\Th bit: 
        \[X_{i-1} + u_i = \left(d_1, \ldots, d_{i-1}, 1, 0, \ldots, 0\right),\]
        \[Y_{i-1} - u_i = \left(d_1, \ldots, d_{i-1}, 0, 1, \ldots, 1\right).\]
        
        Their intersection and their union are thus: 
        \[\left(X_{i-1} + u_i\right) \cap \left(Y_{i-1} - u_i\right) = X_{i-1},\]
        \[\left(X_{i-1} + u_i\right) \cup \left(Y_{i-1} - u_i\right) = Y_{i-1}.\]
        
        But then, by submodularity, we get: 
        \autoeq[s]{f\left(X_{i-1} + u_i\right) + f\left(Y_{i-1} - u_i\right) \geq f\left(\left(X_{i-1} + u_i\right) \cap \left(Y_{i-1} - u_i\right)\right) + f\left(\left(X_{i-1} + u_i\right) \cup \left(Y_{i-1} - u_i\right)\right) = f\left(X_{i-1}\right) + f\left(Y_{i-1}\right).}
        
        This yields the result: 
        \[a_i + b_i = f\left(X_{i-1} + u_i\right) - f\left(X_{i-1}\right) + f\left(Y_{i-1} - u_i\right) - f\left(Y_{i-1}\right) \geq 0.\]

        \qed
    \end{subparag}
\end{parag}

\begin{parag}{Lemma 2}
    Let $i$ be arbitrary.

    Let $OPT$ be an optimal set, and $OPT_i = \left(OPT \cup X_i\right) \cap Y_i$.

    At iteration $i$, the double greedy algorithm is such that:
    \[\left[f\left(X_i\right) + f\left(Y_i\right)\right] - \left[f\left(X_{i-1}\right) + f\left(Y_{i-1}\right)\right] \geq f\left(OPT_{i-1}\right) - f\left(OPT_i\right).\]
    
    \begin{subparag}{Intuition}
        We will use this lemma for the following theorem, to show that the double greedy algorithm is a $\frac{1}{3}$-approximation algorithm.

        The idea is that we can use this to make an elevator argument. The algorithm starts with $f\left(X_0\right) = 0$ and $f\left(OPT_0\right) = f\left(OPT\right)$ is optimal. Then, we can use this lemma to bound how much $OPT_{i}$ decreases every time we increase $X_i$ every iteration; until the algorithm gets to $X_n = OPT_{n}$.
    \end{subparag}

    \begin{subparag}{Proof}
        Note that we have:
        \[X_i = \left(d_1, \ldots, d_i, 0, \ldots, 0\right),\] 
        \[Y_i = \left(d_1, \ldots, d_i, 1, \ldots, 1\right),\]
        \[OPT_i = \left(d_1, \ldots, d_{i}, o_{i+1}, \ldots, o_n\right).\]

        We suppose that $a_i \geq b_i$ (the case where $a_i < b_i$ is similar). By definition of the double greedy algorithm, we have: 
        \[X_i = X_{i-1} + u_i, \mathspace Y_{i} = Y_{i-1}.\]

        Since $a_i + b_i \geq 0$ by our previous lemma, and since we supposed $a_i \geq b_i$, this means $a_i \geq 0$. Recall that our goal is to show:
        \[\underbrace{\left[f\left(X_i\right) + f\left(Y_i\right)\right] - \left[f\left(X_{i-1}\right) + f\left(Y_{i-1}\right)\right]}_{=LHS} \geq \underbrace{f\left(OPT_{i-1}\right) - f\left(OPT_i\right)}_{= RHS}.\]

        But then, the left hand-side $LHS$ is simply:
        \[LHS = f\left(X_{i-1} + u_i\right) - f\left(X_{i-1}\right) = f\left(u_i \suchthat X_{i-1}\right) = a_i \geq 0.\]

        Our goal is to show that the right-hand side $RHS$ is at most $a_i$, or at most $0$.

        Notice $OPT_i = OPT_{i-1} + u_i$ by definition. We thus consider two cases.
        \begin{itemize}[left=0pt]
            \item Suppose first that $u_i \in OPT$. This means $OPT_i = OPT_{i-1}$, and hence the right hand side is just $0$: 
            \[RHS = f\left(OPT_{i-1}\right) - f\left(OPT_i\right) = 0.\]

            This does imply that $LHS \geq RHS$.

            \item Suppose now that $u_i \not\in OPT$. This yields:
            \[OPT_{i-1} = \left(d_1, \ldots, d_{i-1}, 0, o_{i+1}, \ldots, o_n\right),\] 
            \[OPT_{i} = \left(d_1, \ldots, d_{i-1}, 1, o_{i+1}, \ldots, o_n\right).\]
            
            We thus have $OPT_{i-1} \subseteq Y_{i-1} - u_i$. But then, the right hand side becomes, by submodularity: 
            \autoeq{RHS = f\left(OPT_{i-1}\right) - f\left(OPT_i\right) = -f\left(u_i \suchthat OPT_{i-1}\right) \leq -f\left(u_i \suchthat Y_{i-1} - u_i\right) = f\left(Y_{i-1} - u_i\right) - f\left(Y_{i-1}\right) = b_i.}
            
            However, $a_i \geq b_i$, showing $LHS = a_i \geq b_i \geq RHS$.
        \end{itemize}

        In both cases, we have $LHS \geq RHS$, showing our result.
        
        \qed
    \end{subparag}
\end{parag}

\begin{parag}{Theorem}
    Let $f: 2^N \mapsto \mathbb{R}$ be a normalised submodular function, and $OPT$ be its maximum value.

    The double greedy algorithm returns a solution of value at least $\frac{1}{3} OPT$.

    \begin{subparag}{Remark}
        It is possible to improve this algorithm to get $\frac{1}{2} OPT$, by adding a random choice between the two branches.

        Let us explain this randomised algorithm intuitively. Recall that $a_i + b_i \geq 0$, so $a_i \geq 0$ or $b_i \geq 0$. When $b_i < 0$, we still pick the branch $a_i \geq b_i$ (i.e. the branch that adds $u_i$ to both $X_{i-1}$ and $Y_{i-1}$). Similarly, when $a_i < 0$, we still take the branch $a_i < b_i$ (i.e. the branch that removes $u_i$ from both $X_{i-1}$ and $Y_{i-1}$). However, when both $a_i \geq 0$ and $b_i \geq 0$, then we consider the first branch with probability $\frac{a_i}{a_i + b_i}$ and the second one with probability $\frac{b_i}{a_i + b_i}$. 
    \end{subparag}

    \begin{subparag}{Proof}
        Let $OPT$ be the optimal set, i.e. $OPT = \argmax_{S \subseteq N} f\left(S\right)$. Just like our lemma, we define: 
        \[OPT_i = \left(OPT \cup X_i\right) \cap Y_i.\]

        We again have: 
        \[X_i = \left(d_1, \ldots, d_i, 0, \ldots, 0\right),\] 
        \[Y_i = \left(d_1, \ldots, d_i, 1, \ldots, 1\right),\]
        \[OPT_i = \left(d_1, \ldots, d_{i}, o_{i+1}, \ldots, o_n\right).\]

        In particular, this means $OPT_0 = OPT$ and $OPT_n = X_n = Y_n$.

        Our second lemma states that:
        \[\left[f\left(X_i\right) + f\left(Y_i\right)\right] - \left[f\left(X_{i-1}\right) + f\left(Y_{i-1}\right)\right] \geq f\left(OPT_{i-1}\right) - f\left(OPT_i\right).\]

        Adding these equations for each $i \in \left\{1, \ldots, n\right\}$:
        \autoeq[s]{\sum_{i=1}^{n} \left[\left(f\left(X_i\right) + f\left(Y_i\right)\right) - \left(f\left(X_{i-1}\right) - f\left(Y_{i-1}\right)\right)\right] \geq \sum_{i=1}^{n} \left[f\left(OPT_{i-1}\right) - f\left(OPT_i\right)\right].}
        
        We recognise that they are telescoping series so, using the fact $OPT_0 = OPT$ and $OPT_n = X_n = Y_n$, and that $f\left(\o\right) = 0$ since it is normalised:
        \autoeq{f\left(X_n\right) + f\left(Y_n\right) - f\left(X_0\right) - f\left(Y_0\right) \geq f\left(OPT_0\right) - f\left(OPT_n\right) \iff f\left(X_n\right) + f\left(Y_n\right) \geq f\left(OPT_0\right) - f\left(OPT_n\right) \iff 3f\left(X_n\right) \geq f\left(OPT\right),}
        giving our result.

        \qed

    \end{subparag}
\end{parag}


\end{document}
