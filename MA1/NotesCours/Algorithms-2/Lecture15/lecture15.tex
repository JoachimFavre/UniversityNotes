% !TeX program = lualatex
% Using VimTeX, you need to reload the plugin (\lx) after having saved the document in order to use LuaLaTeX (thanks to the line above)

\documentclass[a4paper]{article}

% Expanded on 2024-11-12 at 14:15:16.

\usepackage{../../style}

\title{Algo-2}
\author{Joachim Favre}
\date{Mardi 12 novembre 2024}

\begin{document}
\maketitle

\lecture{15}{2024-11-12}{Steak haché}{
\begin{itemize}[left=0pt]
    \item Explanation and proof of the Chenhoff bounds.
    \item Application to the polling example.
    \item Introduction to hashing.
    \item Definition of 2-universal hash family.
\end{itemize}

}

\begin{parag}{Theorem: Chernhoff bounds}
    Let $p_1, \ldots, p_n \in \left[0, 1\right]$, and let $X_1, \ldots, X_n$ be $n$ independent Bernoulli random variables, so that: 
    \[X_i = \begin{systemofequations} 1, & \text{with probability $p_i$}, \\ 0, & \text{with probability $1 - p_i$} \end{systemofequations}.\]

    Moreover, let $X = \sum_{i=1}^{n} X_i$, and $\mu = \exval\left(X\right) = \sum_{i=1}^{n} p_i$.

    Then, for all $\delta > 0$: 
    \[\prob\left(X \geq \left(1 + \delta\right) \mu\right) \leq \exp\left(-\frac{\delta^2}{2 + \delta} \mu\right), \mathspace \prob\left(X \leq \left(1 - \delta\right) \mu\right) \leq \exp\left(- \frac{\delta^2 \mu}{2}\right).\]

    The first inequality tells us a probability for the upper tail, whereas the second one tells us something for the lower tail. 
    
    \begin{subparag}{Remark}
        Note that the two inequalities are necessary; for the upper tail, we have a linear dependence on $\delta$, whereas for the lower tail, we have a quadratic one.

        However, if we restrict $\delta \in \left]0, 1\right] $, we can bound both tails at once: 
        \[\prob\left(\left|X - \mu\right| \geq \delta \mu\right) \leq 2 \exp\left(-\frac{\delta^2 \mu}{3}\right).\]
    \end{subparag}

    \begin{subparag}{Proof 1}
        We consider the upper tail first. We notice that, applying Markov's bound, for $a = \left(1 + \delta\right) \mu > 0$ and all $s > 0$: 
        \[\prob\left(X \geq a\right) = \prob\left(sX \geq sa\right) = \prob\left(e^{sX} \geq e^{sa}\right) \leq \frac{\exval\left(e^{s X}\right)}{e^{sa}}.\]

        However, using the independence of our random variables: 
        \[\exval\left(e^{sX}\right) = \exval\left(e^{s \sum_{i} X_i}\right) = \exval\left(\prod_{i} e^{s X_i}\right) = \prod_{i} \exval\left(e^{s X_i}\right).\]
        
        Now, using the fact that our random variables are Bernoulli, for all $i$ we have, using the fact that $1 + x \leq \exp\left(x\right)$ for all $x \in \mathbb{R}$: 
        \[\exval\left(e^{s X_i}\right) = p_i e^{s\cdot 1} + \left(1 - p_i\right)e^{s\cdot 0} = 1 + p_i \left(e^s - 1\right) \leq \exp\left(p_i\left(e^s - 1\right)\right).\]
        
        This means that, by definition of $\mu$:
        \[\exval\left(e^{sX}\right) = \prod_{i} \exval\left(e^{s X_i}\right) \leq \exp\left(\sum_{i} p_i \left(e^s - 1\right)\right) = \exp\left(\mu \left(e^s - 1\right)\right).\]
        
        We can finally feed it back to Markov's inequality: 
        \[\prob\left(X \geq a\right) \leq \frac{\exp\left(e^{s X}\right)}{e^{sa}} \leq \frac{\exp\left(\mu\left(e^s - 1\right)\right)}{e^{sa}} \leq \frac{\exp\left(\mu\left(e^s - 1\right)\right)}{e^{s \left(1 + \delta\right) \mu}},\]
        where we used the fact $a = \left(1 + \delta\right)\mu$, since we are considering the upper tail. We can use calculus tools to notice that this function of $s$ is minimised for $s = \ln\left(1 + \delta\right)$, giving: 
        \[\prob\left(X \geq a\right) \leq \left(\frac{\exp\left(\delta\right)}{\left(1 + \delta\right)^{\left(1 + \delta\right)}}\right)^{\mu}.\]

        The result then comes from the fact $\frac{2 \delta}{2 + \delta} \leq \ln\left(1 + \delta\right)$.
    \end{subparag}

    \begin{subparag}{Proof 2}
        For the lower tail, it is very similar. For $a = \left(1 - \delta\right)\mu > 0$ and all $s < 0$ (this time, $s$ is negative): 
        \autoeq{\prob\left(X \leq a\right) = \prob\left(sX \geq sa\right) = \prob\left(e^{s X} \geq e^{sa}\right) \leq \frac{\exp\left(e^{s X}\right)}{e^{sa}} \leq \frac{\exp\left(\mu\left(e^s - 1\right)\right)}{e^{s \left(1 - \delta\right)\mu}}.}

        The result then comes by letting $s = -\ln\left(1 - \delta\right)$.
    \end{subparag}
\end{parag}

\begin{parag}{Example: Polling}
    Let us consider again the following random variable: 
    \[X_i = \begin{systemofequations} 1, & \text{with probability $p$}, \\ 0, & \text{with probability $1-p$.} \end{systemofequations}\]

    Our goal is still to see how well $\sum_{i} X_i$ is an approximation to $\exval\left(D\right) = p$. Recall that, leaving $\bar{X}_n = \frac{1}{n} \sum_{i=1}^{n} X_i$, Chebyshev's bound gave us that, for $n \geq \frac{1}{4 \epsilon^2 \delta} \in \Omega\left(\frac{1}{\epsilon^2 \delta}\right)$, then: 
    \[\prob\left(\left|\bar{X}_n - p\right| \geq \epsilon\right) \leq \delta.\]

    We want to see the guarantee the Cherhhoff bounds give us. Taking $X = n\cdot \bar{X}_n = \sum_{i=1}^{n} X_i$ to be a sum of Bernoulli random variables, we have: 
    \autoeq{\prob\left(\left|\frac{1}{n}X - p\right| \geq \epsilon\right) = \prob\left(\left|X - np\right| \geq n\epsilon\right) = \prob\left(X - np \geq n\epsilon\right) + \prob\left(np - X \geq n\epsilon\right) = \prob\left(X \geq np + \epsilon n\right) + \prob\left(X \leq np - \epsilon n\right) = \prob\left(X \geq \left(1 + \frac{\epsilon}{p}\right)np \right) + \prob\left(X \leq \left(1 - \frac{\epsilon}{p}\right)pn\right) \leq \exp\left(\frac{-\left(\frac{\epsilon}{p}\right)^2 np}{2 + \frac{\epsilon}{p}}\right) + \exp\left(\frac{-\left(\frac{\epsilon}{p}\right)^2 np}{2}\right) \in e^{-\Omega\left(\epsilon^2 n\right)}.}

    Asking for this probability to be less than $\delta$ thus asks for $n \in \Omega\left(\frac{1}{\epsilon^2} \ln\left(\frac{1}{\delta}\right)\right)$. This is a much better bound than Chebyshev for the success probability $\delta$; it allows us to ask for a success probability that is exponentially close to $1$.
\end{parag}

\begin{parag}{Hoeffding's bound}
    Let $a, b \in \mathbb{R}$, and let $X_1, \ldots, X_n$ be independent random variables, such that $a < X_i < b$ for all $i$. Moreover, let $X = \sum_{i=1}^{n} X_i$ and $\mu = \exval\left(X\right)$.

    Then, for all $\delta > 0$: 
    \[\prob\left(X \geq \left(1 + \delta\right) \mu\right) \leq \exp\left(-\frac{2 \delta^2 \mu^2}{n\left(b - a\right)^2}\right), \mathspace \prob\left(X \leq \left(1 - \delta\right) \mu\right) \leq \exp\left(-\frac{\delta^2 \mu^2}{n\left(b-a\right)^2}\right).\]
    
    Again, the first inequality bounds the upper tail of the distribution, whereas the second one bounds the upper tail.

    \begin{subparag}{Remark}
        This result is quite powerful, since it only asks for the random variables to be bounded, without any constraint on their distribution.
    \end{subparag}
\end{parag}

\subsection{Hashing}
\subsubsection{Hash tables}

\begin{parag}{Motivational example}
    Let's say that we have a huge universe $U$, which can be all $1000 \times 1000$ pictures for instance, and let us consider that we have a dataset $S$ with $\left|S\right|$ data items. 

    We want to make a database, in which we can insert, delete, and query whether it contains an element. An idea to do so it to make an array of $\left|U\right|$ bits, where there is a 1 if it is contained, and 0 otherwise. This is really efficient, but takes a huge amount of memory. 

    With hashing, we can get a data structure with $O\left(\left|S\right|\right)$ space complexity, and expected $O\left(1\right)$ insert, delete and query, called a hash table.
\end{parag}

\begin{parag}{Data structure: Hash table}
    Let $U$ be some universe, and $h: U \mapsto \left\{1, \ldots, N\right\}$ be some function (which we call ``hash function'').

    A \important{hash table} is a table $T$ of size $N$ where, for every element $i \in S$, we store $i$ in a linked list at $T\left[h\left(i\right)\right]$.

    \begin{subparag}{Remark 1}
         For this to work, we need the linked lists to be generally small. Indeed, all insert, delete and query procedures run in time $O\left(L_x\right)$, where $L_x$ is the length of the linked list $T\left[h\left(x\right)\right]$, for $x \in S$.
    \end{subparag}

    \begin{subparag}{Remark 2}
        Note that, if we store the elements of $S$ in the hash table, we want $N \in O\left(\left|S\right|\right)$. We will want to design hash functions that allow to have a small $N$ for efficient space complexity, while preserving a small $L_x$ for efficient time complexity.
    \end{subparag}

    \begin{subparag}{Remark 3}
        We will consider randomness on hash functions in the rest of this class. Note however that the randomness is chosen when the data structure is instantiated; i.e. the hash function used by this data structure will always be the same throughout its runtime.
    \end{subparag}
\end{parag}


\begin{parag}{Example 1}
    A way to construct a hash function $h$ is: 
    \[\begin{split}
    h: U &\longmapsto \left\{1, 2, \ldots, N\right\} \\
    x &\longmapsto x \Mod N.
    \end{split}\]
    
    This is a bad hash function when all the elements of our dataset $S$ have the same residue modulo $N$. A way to fix this is to consider families of hash function $\mathcal{H}$ that are good on average, from which we take a hash function uniformly at random.
\end{parag}

\begin{parag}{Example 2: Uniform hashing}
    We can consider the following hash family: 
    \[\mathcal{H} = \left\{\text{all functions $U \mapsto \left\{1, \ldots, N\right\}$}\right\}.\]

    Note that $\mathcal{H}$ is huge $\left|\mathcal{H}\right| = N^{\left|U\right|}$. We can sample a hash function uniformly at random from this set by taking an element from $\left\{1, \ldots, N\right\}$ uniformly at random for each $U$. Note that this is not really implementable efficiently, it takes size $\log_2\left(\left|\mathcal{H}\right|\right) = \left|U\right|\log_2\left(N\right)$ (we can't even do this lazily: fixing only the values of the function we have already seen require us to know if we have already seen the value using some data structure, and thus this is a chicken-and-egg problem), but this is more or less the best hash family we can do and will thus serve as a comparison point for the following hash functions.
\end{parag}

\begin{parag}{Definition: 2-universal hash family}
    Let $\mathcal{H}$ be a family of hash functions $h: U \mapsto \left\{1, \ldots, N\right\}$.

    We say $\mathcal{H}$ is a \important{2-universal hash family} (or just a universal hash family) if, for all $x, y \in U$ where $x \neq y$: 
    \[\prob_{h \followsdistr \mathcal{H}}\left(h\left(x\right) = h\left(y\right)\right) \leq \frac{1}{N}.\]

    \begin{subparag}{Intuition}
        Intuitively, this asks for a low collision probability for two distinct keys.
    \end{subparag}

    \begin{subparag}{Remark}
        The uniform hashing family is 2-universal, it is such that $\prob_{h \followsdistr \mathcal{H}}\left(h\left(x\right) = h\left(y\right)\right) = \frac{1}{N}$ for $x \neq y$.

        Our goal will however be to construct a 2-universal $\mathcal{H}$ with $O\left(\log\left(\left|U\right|\right)\right)$ elements.
    \end{subparag}
\end{parag}

\begin{parag}{Theorem}
    Let $\mathcal{H}$ be a 2-universal hash family. Also, let $S \subseteq U$ be an arbitrary set, and $x \in S$ be arbitrary. We consider $L_x$ to be the random variable representing the length of the linked list $T\left[h\left(x\right)\right]$ (its randomness comes from the choice of $h \in \mathcal{H}$, chosen at the start of the program, but fixed throughout the execution).

    Then: 
    \[\exval\left(L_x\right) \leq 1 + \frac{\left|S\right| - 1}{N}.\]

    \begin{subparag}{Remark}
        We notice that if $N \geq \left|S\right|$, we get $O\left(1\right)$ expected runtime.
    \end{subparag}
    
    \begin{subparag}{Proof}
        Let us consider the following indicator variable:
        \[I_y = \begin{systemofequations} 1, & \text{if $h\left(y\right) = h\left(x\right)$}, \\ 0, & \text{otherwise,} \end{systemofequations} \mathspace \forall y \in U.\]

        We notice that: 
        \[L_x = \sum_{y \in S} I_y.\]

        But then, by linearity of expectations: 
        \autoeq{\exval\left(L_x\right) = \exval\left(\sum_{y \in S} I_y\right) = \sum_{y \in S} \exval\left(I_y\right) = \sum_{y \in S} \prob\left(I_y = 1\right) = 1 + \sum_{\substack{y \in S \\ y \neq x}} \prob\left(h\left(x\right) = h\left(y\right)\right).}

        We can now use the fact that $\mathcal{H}$ is a 2-universal hash family, to get that: 
        \[\exval\left(L_x\right) \leq 1 + \sum_{\substack{y \in S \\ y \neq x}} \frac{1}{N} = 1 + \frac{\left|S\right| - 1}{N}\]
        
        
        \qed
    \end{subparag}
\end{parag}



\end{document}
