% !TeX program = lualatex
% Using VimTeX, you need to reload the plugin (\lx) after having saved the document in order to use LuaLaTeX (thanks to the line above)

\documentclass[a4paper]{article}

% Expanded on 2024-12-02 at 13:15:29.

\usepackage{../../style}

\title{Algo 2}
\author{Joachim Favre}
\date{Lundi 02 décembre 2024}

\begin{document}
\maketitle

\lecture{20}{2024-12-02}{Nearest neighbourhood search}{
\begin{itemize}[left=0pt]
    \item Explanation of the nearest neighbourhood search problem.
    \item Explanation of the approximate nearest neighbourhood search problem.
    \item Definition of locality sensitive hash families.
    \item Explanation and proof on how to use a locality sensitive hash family to build an approximate nearest neighbourhood datastructure.
\end{itemize}

}

\subsubsection{Locality sensitive hashing and nearest neighbourhood search}

\begin{parag}{Nearest neighbourhood search problem}
    We have $n$ documents $p_1, \ldots, p_n \in \mathbb{R}^d$, that are embedded in a $d$-dimensional space so that similar documents are close in the space (like for large language models). 

    Given a query $q \in \mathbb{R}^d$, the goal is to find a document $p_i$ that minimises the distance between $p_i$ and $q$: 
    \[\text{dist}\left(p_i, q\right) \leq \text{dist}\left(p_j, q\right), \mathspace \forall j.\]
    
    \begin{subparag}{Remark}
        We can for instance take the following distances:
        \begin{itemize}
            \item \textit{(Euclidean norm)} $\text{dist}\left(p, q\right) = \left\|p - q\right\|_2$;
            \item \textit{(Squared Euclidean norm)} $\text{dist}\left(p, q\right) = \left\|p - q\right\|_2^2$;
            \item \textit{(Cosine similarity)} $\text{dist}\left(p, q\right) = p\dotprod q$ assuming $\left\|p\right\| = \left\|q\right\| = 1$;
            \item \textit{($L_1$ distance)} $\displaystyle \text{dist}\left(p, q\right) = \left\|p - q\right\|_1 = \sum_{i=1}^{d} \left|p_i - q_i\right|.$
            \item \textit{(Hamming distance)} $\displaystyle \text{dist}\left(p, q\right) = \left\|p - q\right\|_1 = \left|\left\{i \suchthat p_i \neq q_i\right\}\right|$, where $p, q \in \mathbb{F}_2^d$.
        \end{itemize}
        
        We will focus on the last distance in this course.
    \end{subparag}
\end{parag}

\begin{parag}{Naive approach}
    Naively, we can compare $q$ with all $p_1, \ldots, p_n$ and return the best. Each comparison takes time $\Theta\left(d\right)$, so the whole running time is $\Theta\left(nd\right)$.

    This is however not great if we have many queries to do.
\end{parag}

\begin{parag}{One-dimensional approach}
    Let us suppose that $d = 1$ as a start. If we preprocess the database by sorting it, we can do a binary search in time $O\left(\log\left(n\right)\right)$. This is much better.

    However, binary search does not generalise well for higher dimensions, because of the curse of dimensionality.

    \begin{subparag}{Remark}
        Solving the problem as we stated is non-trivial. We will thus instead relax it.
    \end{subparag}
\end{parag}

\begin{parag}{Approximate nearest neighbour search problem}
    We relax our problem to the \important{approximate nearest neighbour search} (ANNS).

    Our goal is to build a data-structure $\text{ANNS}\left(c, r\right)$ with parameters $c \geq 1$ and $r \geq 0$ that has the following guarantees on a query $q$:
    \begin{itemize}
        \item If there exists some $p_i$ such that $\text{dist}\left(p_i, q\right) \leq r$, then return a $p_j$ so that $\text{dist}\left(p_j, q\right) \leq c\cdot r$.
        \item Otherwise, there is no guarantee on the output.
    \end{itemize}

    \begin{subparag}{Example}
        We may for instance consider the following case, where we input $q$. There exists a $p_i$ that has distance $r$ from $q$, so a potential output is $p_j$.
        \svghere[0.5]{ApproximateNearestNeighbourSearchExample.svg}
    \end{subparag}

    \begin{subparag}{Remark}
        This $r$ may be annoying, so we want to consider how to circumvent this issue.
        
        If $r \to \infty$, there will always be a point respecting the conditions of the first guarantee, but this would no longer give any guarantee on $p_j$. Let us thus consider a better way to get rid of $r$.

        We first scale our distances so that no distance is bigger than $1$ (constructing our dataset in such a way that they are all normalised for the Euclidean distance, or using cosine-distance for instance). We define the smallest possible distance between two points to be $\delta$, called the aspect ratio of the dataset (more generally, $\delta$ is the margin of error at which we accept two documents to be considered equal). We can then construct a $\text{ANNS}\left(r, c\right)$ data-structures for $r \in \left\{\delta, \left(1 + \epsilon\right) \delta, \left(1 + \epsilon\right)^2 \delta, \ldots\right\}$. We run all of them in parallel, and return the best result.
    \end{subparag}
\end{parag}

\begin{parag}{Observation}
    Let us suppose that we have a hash family $\mathcal{H}$ such that: 
    \[\prob_{h \followsdistr \mathcal{H}}\left[h\left(x\right) = h\left(y\right)\right] = \begin{systemofequations} 1, & \text{if } \text{dist}\left(x, y\right) \leq r, \\ 0, & \text{if $\text{dist}\left(x, y\right) > cr$.} \end{systemofequations}\]

    We can then solve $\text{ANNS}\left(c, r\right)$ by using a simple hash table. It may have an insane number of collisions if all elements are at a distance at most $r$, but this is not an issue for our problem since we can just output the first element of the linked list our query hashes to.

    However, this type of hash function is not really doable, so we relax it to the following definition.
\end{parag}

\begin{parag}{Definition: Locality sensitive hashing function}
    Let $\mathcal{H}$ be a hash family.

    $\mathcal{H}$ is said to be an \important{$\left(r, cr, p_1, p_2\right)$-locality sensitive hash (LSH) family} if, for all $x ,y \in \mathbb{R}^d$:
    \begin{itemize}
        \item If $\text{dist}\left(x, y\right) \leq r$, then $\displaystyle \prob_{h \followsdistr \mathcal{H}}\left(h\left(x\right) = h\left(y\right)\right) \geq p_1.$
        \item If $\text{dist}\left(x, y\right) \geq cr$, then $\displaystyle \prob_{h \followsdistr \mathcal{H}}\left(h\left(x\right) = h\left(y\right)\right) \leq p_2.$
    \end{itemize}
    
    We moreover let $\alpha$ to be such that $p_1 = p_2^{\alpha}$, i.e. $\alpha = \log_{p_2}\left(p_1\right) = \frac{\ln\left(p_1\right)}{\ln\left(p_2\right)}$. We will see that $\alpha$ is a critical parameter, time complexity will depend on this value. It represents how much bigger $p_1$ is to $p_2$.

    \begin{subparag}{Intuition}
        Intuitively, this is allows us to ask $\mathcal{H}$ to be of the form:
        \[\prob_{h \followsdistr \mathcal{H}}\left[h\left(x\right) = h\left(y\right)\right] = \begin{systemofequations} \text{high}, & \text{if $\text{dist}\left(x, y\right) \leq r$} \\ \text{low}, & \text{if $\text{dist}\left(x, y\right) \geq cr$.} \end{systemofequations}\]

        This is thus indeed a relaxation of the constraint on hash families considered in the previous paragraph.
    \end{subparag}
\end{parag}

\begin{parag}{Example}
    We consider a hashing family for the universe $U = \left\{0, 1\right\}^d$, and Hamming distance: 
    \[\text{dist}\left(x, y\right) = \left|\left\{i \suchthat x_i \neq y_i\right\}\right| = \left\|x - y\right\|_1.\]
    
    We define $\mathcal{H} = \left\{h_1, \ldots, h_d\right\}$ where $h_i\left(x\right) = x_i$ just outputs the $i$\Th coordinate of the vector. Let $x, y \in \left\{0, 1\right\}^d$. We have two cases to consider to see if this is a good LSH family:
    \begin{itemize}[left=0pt]
        \item If $\text{dist}\left(x, y\right) \leq r$, this means by definition of Hamming distance that at least $d - r$ components of $x$ and $y$ are the same, so: 
        \[\prob_{h_i \followsdistr \mathcal{H}}\left(h_i\left(x\right) = h_i\left(y\right)\right) = \prob_{i \in \left\{1, \ldots, n\right\}}\left(x_i = y_i\right) \geq \frac{d-r}{d} = 1 - \frac{r}{d} \approx \exp\left(-\frac{r}{d}\right).\]
        \item If $\text{dist}\left(x, y\right) > cr$, then at most $d - cr$ components of $x$ and $y$ are the same, and thus: 
        \[\prob_{h_i \followsdistr \mathcal{H}}\left(h_i\left(x\right) = h_i\left(y\right)\right) \leq 1 - \frac{cr}{d} \approx \exp\left(-\frac{cr}{d}\right).\]
    \end{itemize}

    So, this is a LHS, with: 
    \[\alpha = \frac{\ln\left(p_1\right)}{\ln\left(p_2\right)} \approx \frac{-r/d}{-cr/d} = \frac{1}{c}.\]
   \end{parag}

\begin{parag}{Algorithm}
    We construct an $\text{ANNS}\left(c, r\right)$ data structure from a $\text{LSH}\left(r, c, p_1, p_2\right)$ family.

    To construct it, we do:
    \begin{enumerate}
        \item Let $\ell \in C\cdot n^{\alpha} \ln\left(n\right)$ for some $C > 0$ and $k = \frac{\ln\left(1/n\right)}{\ln\left(p_2\right)} = \frac{\ln\left(n\right)}{\ln\left(1 / p_2\right)}$.
        \item Choose $k\cdot \ell$ hash functions $h_{11}, h_{12}, \ldots, h_{1k}, h_{21}, \ldots \ldots, h_{\ell k}$ independently from the LSH family. 
        \item Construct $\ell$ hash tables, one for each hash function $f_i = \left[h_{i1}, \ldots, h_{ik}\right]$.
        \item Hash each data point $p_j$ in each table.
    \end{enumerate}

    Then, on a query $q$, we do the following for each $i \in \left\{1, \ldots, \ell\right\}$ until one returns:
    \begin{enumerate}
        \item Compute $f_i\left(q\right)$.
        \item Loop over all points $p$ that hashes to the same value, $f_i\left(p\right) = f_i\left(q\right)$. If $p$ is such that $\text{dist}\left(p, q\right) \leq c\cdot r$, return $p$.
    \end{enumerate}

    When there exists some $p_i$ so that $\text{dist}\left(p_i, q\right) \leq r$, this algorithm returns a valid $p_j$ with probability at least $1 - \frac{1}{n^C}$. It runs in expected time $O\left(\ell\cdot \left(k + d\right)\right) = O\left(n^{\alpha} \ln\left(n\right) \left(\frac{\ln\left(n\right)}{\ln\left(1/p_2\right)} + d\right)\right)$. Its space is $O\left(n \ell k\right) = O\left(n^{1 + \alpha} \frac{\ln\left(n\right)^2}{\ln\left(1/p_2\right)}\right)$. 

    \begin{subparag}{Intuition}
        The idea to make some ``super-hash functions'' $f_i = \left[h_{i1}, \ldots, h_{ik}\right]$ that combine many hash functions is here to decrease the number of bad collisions from $p_2$ to $p_2^k = \frac{1}{n}$. Then, the idea to run multiple hash tables at the same time is here to increase the success probability.

        Those two ideas are important.
    \end{subparag}

    \begin{subparag}{Proof}
        We build this datastructure constructively, from a simple hash table using a single LSH function.

        \begin{itemize}[left=0pt]
            \item First, we want to decrease the number of bad collisions, i.e. the number of points $p_i, p_j$ that hash to the same value $h\left(p_i\right) = h\left(p_j\right)$ even though $\text{dist}\left(p_i, p_j\right) \geq cr$. Currently, there are $p_2\cdot n$ such collisions on expectation, which is very bad. 

            So, we instead consider a single table with a ``super-hash function'': 
            \[f\left(p\right) = \left[h_1\left(p\right), h_2\left(p\right), \ldots, h_k\left(p\right)\right],\]
            where each $h_i \in \mathcal{H}$ are taken independently.

            For $p, q$ with $\text{dist}\left(p, q\right) > cr$, we have, by independence of the hash functions: 
            \[\prob\left(h\left(p\right) = h\left(q\right)\right) \leq p_2^k.\]
           
            We can thus select $k$ so that $p_2^k = \frac{1}{n}$. Intuitively, we just made a LSH function $f$ so that $\widetilde{p}_2 \approx 0$. We now want to boost $\widetilde{p}_1$ to $1$.
            
            \item More formally, if $\text{dist}\left(p, q\right) \leq r$, we notice that, by independence:
            \[\prob\left(h\left(p\right) = h\left(q\right)\right) \geq p_1^k = p_2^{\alpha k}.\]

            We want to boost this value to increase the success probability, by considering $\ell = C \ln\left(n\right)n^{\alpha} $ independent ``super-hash functions'': 
            \[f_i = \left[h_{i1}, h_{i2}, \ldots, h_{ik}\right], \mathspace i \in \left\{1, \ldots, \ell\right\}.\]

            We construct a hash table for each $f_1, \ldots, f_{\ell}$. This yields that, for $p, q$ so that $\text{dist}\left(p, q\right) \leq r$, the success probability of the whole datastructure is: 
            \autoeq{\prob\left(\exists i \suchthat f_i\left(p\right) = f_i\left(q\right)\right) = 1 - \prob\left(\forall i \in \left\{1, \ldots, \ell\right\} \suchthat f_i\left(p\right) \neq f_i\left(q\right)\right) \geq 1 - \left(1 - p_1^k\right)^{\ell} \geq 1 - e^{- p_1^k \ell} \geq 1 - \exp\left(-\frac{1}{n^{\alpha}} \ell\right) \geq 1 - \frac{1}{n^c},}
            if $\ell = C \ln\left(n\right) n^{\alpha}$.
        \end{itemize}

        We have just proven the success probability. We finally just need to consider its time and space complexity.

        The expected running time on a query consists in, for each of the $\ell$ table, compute first the $k$ hash functions, and then compute the distance (in time $O\left(d\right)$) with all the $c$ elements that hash to the same value. Since the probability of a bad collision is at most $p_2^k = \frac{1}{n}$ by the first step, and since we have $n$ elements, each table has only $c = 1$ bad collision on expectation. This yields a time complexity of: 
        \[O\left(\ell \left(k + dc\right)\right) = O\left(\ell \left(k + d\right)\right) = O\left(n^{\alpha} \ln\left(n\right)\left(\frac{\ln\left(n\right)}{\ln\left(1/p_2\right)} + d\right)\right)\]

        For the space complexity, we have $\ell$ hash tables, each containing $n$ elements of size $k$, giving $\Theta\left(n \ell k\right) = \Theta\left(n^{1 + \alpha} \ln\left(n\right) \frac{\ln\left(n\right)}{\ln\left(1/p_2\right)}\right)$.

        \qed
    \end{subparag}
\end{parag}

\end{document}
