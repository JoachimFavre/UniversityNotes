% !TeX program = lualatex
% Using VimTeX, you need to reload the plugin (\lx) after having saved the document in order to use LuaLaTeX (thanks to the line above)

\documentclass[a4paper]{article}

% Expanded on 2024-11-18 at 13:17:50.

\usepackage{../../style}

\title{Alga 2}
\author{Joachim Favre}
\date{Lundi 18 novembre 2024}

\begin{document}
\maketitle

\lecture{16}{2024-11-18}{I love algebra}{
\begin{itemize}[left=0pt]
    \item Construction of a 2-universal hash function.
    \item Explanation of the two-hashing scheme for hash tables.
    \item Definition of $k$-wise independence.
    \item Construction of a $k$-wise independent hash function.
    \item Explanation of load balancing.
\end{itemize}
}

\begin{parag}{Bertrand's postulate}
    Let $n > 1$ be an integer.

    Then, there exists a primer number $p$ so that: 
    \[n \leq p < 2n.\]
\end{parag}

\begin{parag}{Definition}
    Let $p$ be a prime number between $\left|U\right|$ and $2\left|U\right|$. For $a, b \in \mathbb{F}_p$ but $a \neq 0$, we define: 
    \[f_{a, b}\left(x\right) = \left(ax + b\right) \Mod p, \mathspace h_{a, b}\left(x\right) = f_{a, b}\left(x\right) \Mod N.\]
    
   Then, we define the following hash family:  
   \[\mathcal{H} = \left\{h_{a,b} \suchthat a, b \in \mathbb{F}_p, a \neq 0\right\}.\]

   \begin{subparag}{Remark}
       We notice that $\left|\mathcal{H}\right| = p\left(p-1\right) \approx p^2$. So, a hash function takes memory $\log_2\left(\left|\mathcal{H}\right|\right) = O\left(2 \log_2\left(p\right)\right) = O\left(\log_2\left(\left|U\right|\right)\right)$. This is indeed storable.
   \end{subparag}

   \begin{subparag}{Intuition}
       Intuitively, $f_{a, b}\left(x\right)$ first reorders elements of U, by adding space between consecutive elements for instance. Then, $h_{a, b}\left(x\right)$ compresses the space onto $\left\{0, \ldots, N-1\right\}$.
   \end{subparag}
\end{parag}

\begin{parag}{Lemma}
    Let $p$ be a prime number, and let $y, s, t \in \mathbb{F}_p$ be arbitrary.

    For any $x \in \mathbb{F}_p \setminus \left\{y\right\}$, the following system has exactly one solution $a, b \in \mathbb{F}_p^2$, modulo $p$: 
    \[\begin{systemofequations} ax + b = s, \\ ay + b = t. \end{systemofequations}\]

    \begin{subparag}{Proof}
        This system is of the form: 
        \[\begin{pmatrix} x & 1 \\ y & 1 \end{pmatrix} \begin{pmatrix} a \\ b \end{pmatrix}  = \begin{pmatrix} s \\ t \end{pmatrix}.\]

        Solving it over our field $\mathbb{F}_p$, we find that: 
        \[a = \left(x-y\right)^{-1} \left(s - t\right), \mathspace b = s - ax.\]

        This is indeed unique, when $x \neq y$.

        \qed
    \end{subparag}
\end{parag}

\begin{parag}{Theorem}
        $\mathcal{H}$ is 2-universal.

        \begin{subparag}{Remark}
            This hash family could thus be used for hash tables.
        \end{subparag}

        \begin{subparag}{Proof}
            Let $x \neq y$. Our previous lemma yields that, for any $s, t$: 
            \[\prob_{a, b}\left(f_{a, b}\left(x\right) = s \land f_{a, b}\left(y\right) = t\right) = \begin{systemofequations} 0, & \text{if $s = t$,} \\ \frac{1}{p\left(p-1\right)}, & \text{if $s \neq t$.} \end{systemofequations}\]

            Indeed, our lemma tells us that there exists a unique $\left(a, b\right)$ that solves both $f_{a, b}\left(x\right) = s$ and $f_{a, b}\left(y\right) = t$. When $s = t$, we will always have $a = 0$, which is not valid by definition of our Hash function. When $s \neq t$, the $\left(a, b\right)$ pair will always be valid (since $a \neq 0$). However, we pick $\left(a, b\right)$ uniformly at random in a set of $\left(p-1\right)\cdot p$ possibilities ($\left(p-1\right)$ possibilities for $a$ and $p$ for $b$), giving this result.

            So, for all $x, y \in U$ so that $x \neq y$:
            \autoeq[s]{\prob_{h \followsdistr \mathcal{H}}\left(h\left(x\right) = h\left(y\right)\right) = \sum_{\substack{s, t \in \mathbb{F}_p \\ s = t \Mod N}} \prob_{a, b}\left(f_{a, b}\left(x\right) = s \land f_{a, b}\left(y\right) = t\right) = \sum_{s, t \in \mathbb{F}_p} I\left(s = t \Mod N\right) \prob\left(f_{a, b}\left(x\right) = s \land f_{a, b}\left(y\right) = t\right) = \frac{1}{p\left(p-1\right)} \underbrace{\sum_{\substack{s, t \in \mathbb{F}_p \\ s \neq t}} I\left(s = t \Mod N\right)}_{= S}.}

            Our goal is to show that the sum $S$ is so that $S \leq \frac{p\left(p-1\right)}{N}$, since it would indeed show the 2-universal property, i.e. that $\prob\left(h\left(x\right) = h\left(y\right)\right) \leq \frac{1}{N}$. Given a fixed $s$, note that there are at most $\left\lceil \frac{p-1}{N} \right\rceil $ values in $\left\{0, \ldots, p-1\right\}$ that have the same rest modulo $N$ as $s$. For instance, considering the following diagram where $p = 11$ and $N = 3$, there are at most one element with any given rest modulo $N$ per block. Also, since almost each block has size $N = 3$, there are $\left\lceil \frac{p-1}{N} \right\rceil = \left\lceil \frac{10}{3} \right\rceil = 4$ blocks. This shows there are at most $\left\lceil \frac{p-1}{N} \right\rceil $ elements with any given rest modulo $N$.
            \begin{center}
            \renewcommand{\arraystretch}{1.75}
            \begin{tabular}{r|ccc|ccc|ccc|cc|}
                \cline{2-12}
                $x$ & $0$ & $1$ & $2$ & $3$ & $4$ & $5$ & $6$ & $7$ & $8$ & $9$ & $10$ \\
                $x \Mod 3$ & $0$ & $1$ & $2$ & $0$ & $1$ & $2$ & $0$ & $1$ & $2$ & $0$ & $1$ \\
                \cline{2-12}
            \end{tabular}
            \renewcommand{\arraystretch}{1}
            \end{center}
            

            Mathematically, this translates to:
            \[\sum_{\substack{t \in \mathbb{F}_p \\ t \neq s}} I\left(s = t \Mod N\right) \leq \left\lceil \frac{p-1}{N} \right\rceil - 1 \leq \frac{p-1}{N}.\]

            This allows us to finish the proof: 
            \autoeq{\prob_{h \followsdistr \mathcal{H}}\left(h\left(x\right) = h\left(y\right)\right) = \frac{1}{p\left(p-1\right)} \sum_{s \in \mathbb{F}_p} \sum_{\substack{t \in \mathbb{F}_p \\ t \neq s}} I\left(s = t \Mod N\right) \leq \frac{1}{N\cdot p} \sum_{s \in \mathbb{F}_p} = \frac{1}{N}.}

            \qed
    \end{subparag}
\end{parag}

\subsubsection{Collision free hashing}

\begin{parag}{Goal}
    As explained before, a hash table uses a linked list to handle collisions. We want to consider another hash table scheme.
\end{parag}

\begin{parag}{Proposition: Expected number of collisions}
    Let $S$ be some dataset we want to store in a hash table, $\mathcal{H}$ be a 2-universal hash family, and let $C$ be a random variable that counts the number of collisions.

    Then: 
    \[\exval\left(C\right) \leq \binom{\left|S\right|}{2} \cdot \frac{1}{N}.\]

    If moreover $N \geq \left|S\right|^2$, then: 
    \[\exval\left(C\right) \leq \frac{1}{2}.\]

    \begin{subparag}{Proof}
        We have using the fact that $\mathcal{H}$ is a 2-universal hash family: 
        \[\exval\left(C\right) = \sum_{\substack{x, y \in S \\ x \neq y}} \prob_{h \followsdistr \mathcal{H}}\left(h\left(x\right) = h\left(y\right)\right) \leq \frac{1}{N} \sum_{\substack{x, y \in S \\ x \neq y}} = \binom{\left|S\right|}{2} \frac{1}{N}.\]

        The second inequality directly comes from the fact that $\binom{\left|S\right|}{2} \cdot \frac{1}{N} \leq \frac{1}{2}$ when $N \geq \left|S\right|^2$.

        \qed
    \end{subparag}

    \begin{subparag}{Remark}
        This states that, if we have a hash table of quadratic size, then we can expect to have no collision; which would thus not require any linked list.

        However, quadratic memory is not great, which leads to the following scheme.
    \end{subparag}
\end{parag}

\begin{parag}{Collision-free two-hashing scheme}
    We consider a hash table with hash function $h$ where, instead to use a linked list to handle collision, we use a quadratic size hash table In other words, for each $i \in \left\{1, \ldots, N\right\}$, there is a hash table with hash function $h_i$, which uses enough space so that it does not have to use linked lists.

    This scheme takes memory $O\left(\left|S\right|\right)$.

    \begin{subparag}{Proof}
        For $i \in \left\{1, \ldots, N\right\}$, let $s_i$ be the number of elements of $S$ hashing into the $i$\Th slot: 
        \[s_i = \left|\left\{x \in S \suchthat h\left(x\right) = i\right\}\right|.\]

        Then, there exists a collision free hash function $h_i$ for $\left\{x \in S \suchthat h\left(x\right) = i\right\}$ that requires $s_i^2$ slots, i.e. quadratically many. We can for instance find it by sampling it randomly from our hash family, as the proposition above states.

        The expected total size is the sum of the second-layer hash tables:
        \[\exval\left(\sum_{i} s_i^2\right) = \exval\left(\sum_{i} s_i\left(s_i - 1\right)\right) + \exval\left(\sum_{i} s_i\right).\]

        Note that $\frac{s_i\left(s_i - 1\right)}{2}$ is the number of pairs in the $i$\Th second-level table. By construction, elements of this second-level table hash to the same value under the first-level hash function $h$. But then, $\sum_{i} \frac{s_i \left(s_i - 1\right)}{2}$ is the total number of collisions in the whole table, the $C$ from the propositions above. This yields that, using our proposition: 
        \[\exval\left(\sum_{i} s_i\left(s_i - 1\right)\right) = 2 \exval\left(\sum_{i} \frac{s_i\left(s_i - 1\right)}{2}\right) = 2 \exval\left(C\right) \leq 2\cdot \binom{\left|S\right|}{2} \cdot \frac{1}{N}.\]

        Moreover, the sum of elements of the second level tables, $\sum_{i} s_i$, is the total number of elements in the whole table, i.e. $\sum_{i} s_i = \left|S\right|$. Putting both together, the expected size is, if $N \geq \left|S\right|$: 
        \[\exval\left(\sum_{i} s_i^2\right) = 2 \binom{S}{2}\cdot \frac{1}{N} + \left|S\right| \leq 2\left|S\right|.\]

        \qed
    \end{subparag}
\end{parag}

\subsubsection{$k$-wise independence}

\begin{parag}{Definition: Pairwise independence}
    Let $\mathcal{H}$ be a family of hash functions.

    $\mathcal{H}$ is said to be \important{pairwise independent} if for all $x, y \in \mathcal{U}$ so that $x \neq y$, and for all $s, t \in \left\{0, \ldots, N-1\right\}$: 
    \[\prob_{h \followsdistr \mathcal{H}}\left(h\left(x\right) = s \land h\left(y\right) = t\right) = \frac{1}{N^2}.\]

    \begin{subparag}{Intuition}
        In other words, the joint distribution $\left(h\left(x\right), h\left(y\right)\right)$ is uniform.
    \end{subparag}
\end{parag}

\begin{parag}{Proposition}
    Let $\mathcal{H}$ be a pairwise independent hash family. 

    Then, it is 2-universal.

    \begin{subparag}{Proof}
        The proof is direct:
        \[\prob\left(h\left(x\right) = h\left(y\right)\right) = \sum_{s = 0}^{N-1} \prob\left(h\left(x\right) = s \land h\left(y\right) = s\right) = \sum_{s = 0}^{N - 1} \frac{1}{N^2} = \frac{1}{N}.\]

        \qed
    \end{subparag}
\end{parag}

\begin{parag}{Definition: Onewise independence}
    Let $\mathcal{H}$ be a family of hash functions.

    $\mathcal{H}$ is said to be \important{onewise independent} if for all $x \in \mathcal{U}$ and for all $s \in \left\{0, \ldots, N-1\right\}$: 
    \[\prob_{h \followsdistr \mathcal{H}}\left(h\left(x\right) = s\right) = \frac{1}{N}.\]

    \begin{subparag}{Remark}
        Pairwise independence implies onewise independence.
    \end{subparag}
\end{parag}

\begin{parag}{Definition: $k$-wise independence}
    Let $\mathcal{H}$ be a family of hash functions.

    $\mathcal{H}$ is said to be \important{$k$-wise independent} if the joint distribution of $\left(h\left(x_1\right), \ldots, h\left(x_k\right)\right)$ is uniform for all $x_1, \ldots, x_k$.
\end{parag}

\begin{parag}{Proposition}
    Let $k \in \mathbb{N}$ be some integer. Also, let $p$ be a prime between $\left|U\right|$ and $2\left|U\right|$. For $a_0, \ldots, a_{k-1} \in \mathbb{F}_p$, we define: 
    \[f_{a_0, \ldots, a_{k-1}}\left(x\right) = a_{k-1}x^{k-1} + \ldots + a_1 x + a_0, \mathspace h_{a_0, \ldots, a_{k-1}}\left(x\right) = f_{a_0, \ldots, a_{k-1}}\left(x\right) \Mod N.\]
   
    Finally, we define the following hash family: 
    \[\mathcal{H} = \left\{h_{a_0, \ldots, a_{k-1}} \suchthat a_0, \ldots, a_{k-1} \in \mathbb{F}_p\right\}.\]
    
    This hash family is $k$-wise independent.

    \begin{subparag}{Remark}
        This can also be stored efficiently, in $O\left(k\ln\left(\left|U\right|\right)\right)$ memory.
    \end{subparag}

    \begin{subparag}{Intuition}
        This is very similar to our definition above for a 2-universal hash family, generalised to polynomials of arbitrary degree, except that we also allow allow all coefficients to be $0$.
    \end{subparag}
\end{parag}

\subsubsection{Load balancing}

\begin{parag}{Load balancing problem}
    We have $n$ keys hashing into $n$ slots. We want to analyse what type of load balancing we can hope for, i.e., with high probability, what the maximum number of keys hashed to the same number of slots.

    \begin{subparag}{Terminology}
        We will refer to the keys as ``balls'', and the slots as ``bins''. The question is then, putting each ball to a random bin, what is the largest number of balls we can expect to have in a single bin. 
    \end{subparag}

    \begin{subparag}{Example}
        This is for instance useful to split users trying to log in to a website, onto some web servers.
    \end{subparag}
\end{parag}

\begin{parag}{Proposition}
    We have that:
    \[\prob\left(\text{there exists a bin that gets at least $\Omega\left(\frac{\ln\left(n\right)}{\ln\left(\ln\left(n\right)\right)}\right)$ balls}\right) \leq \frac{1}{n}.\]

    In other words, with high probability, no bin will receive more than $\Omega\left(\frac{\ln\left(n\right)}{\ln\left(\ln\left(n\right)\right)}\right)$ balls.

    \begin{subparag}{Proof}
        Let $i, k$  be arbitrary. We want to compute the probability that the $i$\Th bin gets at least $k$ balls. Since there are $n$ balls in total, there are $\binom{n}{k}$ ways to put $k$ balls in the bin; each having probability $\frac{1}{n^k}$ to appear. The probability that bin $i$ gets $k$ balls is the union of all these events so, by the union bound, it is less than or equal to the sum of their probabilities:
        \[\prob\left(\text{bin $i$ gets at least $k$ balls}\right) \leq \prob\left(\text{bin $i$ gets $k$ balls}\right) \leq \binom{n}{k} \frac{1}{n^k}.\]

        We can simplify this to:
        \autoeq{\prob\left(\text{bin $i$ gets at least $k$ balls}\right) \leq \binom{n}{k} \frac{1}{n^k} = \frac{n\left(n-1\right) \cdots \left(n - k + 1\right)}{k!} \cdot \frac{1}{n^k} = \frac{1}{k!}\cdot \frac{n}{n}\cdot \frac{n-1}{n}\cdots \frac{n-k+1}{n} \leq \frac{1}{k!}.}
        
        Now, by Stirling's formula: 
        \[k! \sim \sqrt{2\pi k} \left(\frac{k}{e}\right)^k.\]

        So, if $k \in \Omega\left(\frac{\ln\left(n\right)}{\ln\left(\ln\left(n\right)\right)}\right)$, then $\frac{1}{k!} \leq \frac{1}{n^2}$. The union bound thus implies that:
        \autoeq[s]{\prob\left(\text{there exists a $i$ so that bin $i$ gets at least $k$ balls}\right) \leq n \prob\left(\text{bin $i$ gets at least $k$ balls}\right) = n\cdot \frac{1}{k!} \leq n\cdot \frac{1}{n^2} = \frac{1}{n}.}
        
        \qed
    \end{subparag}

    \begin{subparag}{Remark}
        It can be shown that, actually, with high probability, there is a bin that receives at least $\Omega\left(\frac{\ln\left(n\right)}{\ln\left(\ln\left(n\right)\right)}\right)$ ball, i.e. that this result is tight.

        However, we may consider a slightly different scheme, which improves this result.
    \end{subparag}
\end{parag}

\begin{parag}{Proposition}
    Let's suppose that each ball samples two bins $b_1 = h_1\left(x\right), b_2 = h_2\left(x\right)$ at random, and goes to the least loaded bin.

    Then, the maximum bin load is $O\left(\ln\left(\ln\left(n\right)\right)\right)$.

    This is named the power of two choices.

    \begin{subparag}{Remark}
        Going from two to three choices does not change the asymptotic behaviour.
    \end{subparag}
\end{parag}

\end{document}
