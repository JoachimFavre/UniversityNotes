% !TeX program = lualatex
% Using VimTeX, you need to reload the plugin (\lx) after having saved the document in order to use LuaLaTeX (thanks to the line above)

\documentclass[a4paper]{article}

% Expanded on 2024-12-17 at 13:18:16.

\usepackage{../../style}

\title{Algorithms 2}
\author{Joachim Favre}
\date{Mardi 17 décembre 2024}

\begin{document}
\maketitle

\lecture{25}{2024-12-17}{Smile because it happened}{
    \begin{itemize}[left=0pt]
        \item \textit{Don't cry because it's over, smile because it happened.}
        \item Proof of the convergence speed of the mixing of random walks.
        \item Definition of conductance.
        \item Explanation of the spectral graph partitioning algorithm.
        \item Definition of the Laplacian matrix of a graph.
        \item Explanation and proof of Cheeger's inequalities.
    \end{itemize}
}

\begin{parag}{Lemma}
    Let $G = \left(V, E\right)$ be a $d$-regular graph, $M = \frac{1}{d} A$ be its random walk matrix, $\epsilon > 0$ and $C \geq 3$. 

    If $k \geq \frac{C}{2\epsilon}\ln\left(n\right)$ and $\max\left(\lambda_2, \left|\lambda_n\right|\right) \leq 1 - \epsilon$, then for all starting vertex $s \in V$: 
    \[\left\|M^k p - \begin{pmatrix} \frac{1}{n} \\ \vdots \\ \frac{1}{n} \end{pmatrix} \right\|_2^2 \in o\left(\frac{1}{n^2}\right),\]
    where $p = e_s$ only has a $1$ at index $s$, and zeros everywhere else.
    
    We call the $\epsilon$ the \important{spectral gap.}

    \begin{subparag}{Intuition}
        This lemma shows that, under some hypotheses, no matter where we start from, the distribution for the random walk will tend to a uniform distribution. In other words, walking long enough, we will forget where we started from.

        Let us moreover justify intuitively the hypothesis that $\max\left(\lambda_2, \left|\lambda_n\right|\right) \leq 1 - \epsilon$. First, $\lambda_2 \approx 1$ is a problem, because it means that there are multiple connected components. But then, we will always stay in the component we started in, never reaching the fully uniform distribution. Second, $\lambda_n \approx -1$ is a problem as well, since it would mean the graph has a bipartite component. Starting in a bipartition at iteration $0$, we will be in this bipartition at every even iteration, and in the other bipartition at every odd iteration. We thus never forget completely where we started from either.
    \end{subparag}

    \begin{subparag}{Proof}
        Let $v_1, v_2, \ldots, v_n$ orthonormal eigenvectors of $M$, associated to eigenvalues $\lambda_1 \geq \lambda_2 \geq \ldots \geq \lambda_n$. We know we can express $p$ in the $\left(v_i\right)$ basis: 
        \[p = \sum_{i=1}^{n} \alpha_i, v_i, \mathspace \alpha_i = \left\langle p, v_i \right\rangle.\]

        However, note that we can always pick $v_1 = \begin{pmatrix} \frac{1}{\sqrt{n}} & \cdots & \frac{1}{\sqrt{n}} \end{pmatrix}^T $, meaning that: 
        \[\alpha_1 = \left\langle p, v_1 \right\rangle = \frac{1}{\sqrt{n}}.\]
        
        Thus: 
        \[p = \frac{1}{\sqrt{n}} \begin{pmatrix} \frac{1}{\sqrt{n}} \\ \vdots \\ \frac{1}{\sqrt{n}} \end{pmatrix} + \sum_{i \geq 2} \alpha_i v_i = \begin{pmatrix} \frac{1}{n} \\ \vdots \\ \frac{1}{n} \end{pmatrix} + \sum_{i \geq 2} \alpha_i v_i.\]
        
        Now, notice that, using the fact the $v_i$ are eigenvectors of $M$, and that $v_1$ has eigenvalue $1$: 
        \[M^k p = M^k \left(\begin{pmatrix} \frac{1}{n} \\ \vdots \\ \frac{1}{n} \end{pmatrix} + \sum_{i \geq  2} \alpha_i v_i\right) = \begin{pmatrix} \frac{1}{n} \\ \vdots \\ \frac{1}{n} \end{pmatrix} + \sum_{i \geq 2} \lambda_i^k \alpha_i v_i.\] 

        We are now able to compute our $\ell_2$ norm: 
        \[x = \left\|M^k p - \begin{pmatrix} \frac{1}{n} \\ \vdots \\ \frac{1}{n} \end{pmatrix} \right\|_2^2 = \left\|\sum_{i\geq 2} \lambda_i^k \alpha_i v_i\right\|^2.\] 

        However, we know that the $\ell_2$ norm is such that, for orthonormal vectors $a, b$ and scalars $\alpha, \beta$:
        \autoeq{\left\|\alpha a + \beta b\right\|^2 = \left(\alpha a + \beta b\right) \dotprod \left(\alpha a + \beta b\right) = \alpha^2 \left(a \dotprod a\right) + 2 \alpha \beta \left(a \dotprod b\right) + \beta^2 \left(b \dotprod b\right) = \alpha^2 + \beta^2.}

        So, since our eigenvectors are orthonormal by construction, and using the fact that $\left|\lambda_i\right| < 1 - \epsilon$ for all $i$: 
        \[x = \sum_{i \geq 2} \left(\lambda_i^k \alpha_i\right)^2 = \sum_{i \geq 2} \lambda_i^{2k} \alpha_i^2 \leq \left(1 - \epsilon\right)^{2k} \sum_{i \geq 2} \alpha_i^2.\]
        
        However, notice that: 
        \[\sum_{i \geq 2} \alpha_i^2  \leq \sum_{i \geq 1} \alpha_i^2 = \left\|p\right\| = 1.\]
        
        Therefore, using the fact $k \geq \frac{C}{2\epsilon} \ln\left(n\right)$:
        \[x \leq \left(1 - \epsilon\right)^{2k} \leq \left(1 - \epsilon\right)^{\frac{C}{\epsilon} \ln\left(n\right)} = \left(\left(1 - \epsilon\right)^{\frac{1}{\epsilon}}\right)^{C \ln\left(n\right)}.\]
        
        This finally simplifies to our result, using the fact $C \geq 3$: 
        \[x \leq \frac{1}{e^{C \ln\left(n\right)}} = n^{-C} \leq \frac{1}{n^3} \in o\left(\frac{1}{n^2}\right)\]
        
        \qed
    \end{subparag}
\end{parag}

\subsection{Graph connectivity}

\begin{parag}{Intuition: Cheeger's inequalities}
    Our lemma on random walks tell us intuitively that the following affirmations should hold:
    \begin{enumerate}
        \item If $\lambda_2$ is small (i.e. far from $1$), then a random walk mixes quickly, and so there are no ``small'' cuts in $G$.
        \item If $\lambda_2$ is close to $1$, then a random walk takes a lot of time to mix, so then there exits a ``small'' cut, which we want to be able to find efficiently.
    \end{enumerate}
    
    \begin{subparag}{Remark}
        Small cut are typically not what we want in practice, since a small cut may just be $S = \left\{v\right\}$ where $v$ is connected to a single other vertex. We will thus consider ``sparse'' cuts instead.

        Let us thus formalise the definitions used for this theorem, before proving some lemmas, and then coming back to it formally.
    \end{subparag}
\end{parag}

\begin{parag}{Definition: Conductance}
    Let $G = \left(V, E\right)$ be a $d$-regular graph, with $\left|V\right| = n$, and let $S \subsetneq V$ where $S \neq \o$.

    We define the \important{conductance of the cut} $\left(S, V \setminus S\right)$ as: 
    \[h\left(S\right) = \frac{\left|\delta\left(S\right)\right|}{d \min\left\{\left|S\right|, \left|V \setminus S\right|\right\}},\]
    where $\delta\left(S\right)$ is the set of edges crossing $\left(S, V \setminus S\right)$.

    The \important{conductance} of $G$ is then defined to be: 
    \[h\left(G\right) = \min_{\substack{S \subsetneq V \\ S \neq \o}} h\left(S\right).\]

    \begin{subparag}{Property}
        Note that for any $S \subsetneq V$ with $S \neq \o$: 
        \[h\left(S\right) = h\left(V \setminus S\right).\]
    \end{subparag}
    
    \begin{subparag}{Intuition}
        Note that $\left|\delta\left(S\right)\right|$ is, by definition, the number of edges crossing the cut. Moreover, supposing that $\left|S\right| < \left|V \setminus S\right|$ without loss of generality (we can simply switch $S$ and $V \setminus S$ otherwise), $d \min\left\{\left|S\right|, \left|V \setminus S\right|\right\} = d \left|S\right|$ is an estimation of the number of edges incident to vertices $v \in S$. Note however that this is not completely correct. If all those edges cross the cut, then each vertex $v \in S$ has $d$ edges, and so we do have a total of $d \left|S\right|$ edges incident on $S$. However, if none of those edges cross the cut, this strategy double-counts all of them, there is instead $\frac{1}{2} d\left|S\right|$ edges incident on vertices in $S$. 

        All this means is that: 
        \[h\left(S\right) = c\cdot \text{percentage of edges incident on $S$ that cross the cut},\]
        for some $c \in \left[\frac{1}{2}, 1\right]$.
        
        So, if $h\left(S\right) = 0.1$, then roughly $\SI{10}{\%}$ to $\SI{20}{\%}$ of edges indicent on $S$ leave $S$; and $\SI{80}{\%}$ to $\SI{90}{\%}$ stay in $S$. This is notably interesting to find communities in the graph.
    \end{subparag}

    \begin{subparag}{Remark}
        A cut with small conductance corresponds to the ``sparse'' we were mentioning before.
    \end{subparag}
\end{parag} 

\begin{parag}{Spectral graph partitioning algorithm}
    Let $G = \left(V, E\right)$ be a $d$-regular graph, of random walk matrix $M$, and eigenvalues $\lambda_1 \geq \lambda_2 \geq \ldots \geq \lambda_n$.

    We make an algorithm, called \important{spectral partitioning}, that finds a cut of conductance $h\left(G\right) = \sqrt{2\left(1 - \lambda_2\right)}$:
    \begin{enumerate}
        \item Sort vertices in non decreasing order of $v_2$: 
        \[v_2\left(1\right) \leq v_2\left(2\right) \leq \ldots \leq v_2\left(n\right).\]
        \item Let $i$ be such that $h\left(\left\{1, \ldots, i\right\}\right)$ is minimised.
        \item Output $S = \left\{1, \ldots, i\right\}$.
    \end{enumerate}

    \begin{subparag}{Intuition}
        Intuitively, we put all vertices on a horizontal line in the order given by $v_2$, and try all vertical cuts. This can be implemented very quickly, in time $O\left(n \ln\left(n\right)\right)$. Indeed, step (2) can be computed in time $O\left(n\right)$, because we do not need to look at al edges in the graph to compute $h\left(\left\{1, \ldots, i\right\}\right)$, we can reuse work for $h\left(\left\{1, \ldots, i-1\right\}\right)$.
    \end{subparag}

    \begin{subparag}{Remark}
        We will not prove that this algorithm does output a cut of conductance $h\left(G\right) = \sqrt{2\left(1 - \lambda_2\right)}$ in this class.
    \end{subparag}
\end{parag}

\begin{parag}{Definition: Rayleigh quotient}
    Let $x \neq 0$ be some vector and $M$ be some matrix.

    We call the following a \important{Rayleigh quotient}: 
    \[\frac{x^T M x}{x^T x}.\]
\end{parag}

\begin{parag}{Lemma 1}
    Let $G = \left(V, E\right)$ be a $d$-regular graph, of random walk matrix $M$, with eigenvalues $\lambda_1 \geq \lambda_2 \geq \ldots \geq \lambda_n$.

    Then, we have: 
    \[\lambda_1 = \max_{x \neq 0} \frac{x^T M x}{x^T x}.\]
    
    \begin{subparag}{Proof}
        Let $x \neq 0$ be an arbitrary vector. We can write $x$ in the eigenbasis $\left(v_i\right)$ associated to the eigenvalues:  
        \[x = \sum_{i} \alpha_i v_i \implies \frac{x^T M x}{x^T x} = \frac{\sum_{i} \lambda_i \alpha_i^2}{\sum_{i} \alpha_i^2}.\]

        This is however a weighted average of the eigenvalues. This is maximal when all the weight is put on the largest eigenvalue, $\lambda_1$.
        
        \qed
    \end{subparag}
\end{parag}

\begin{parag}{Lemma 2}
    Let $G = \left(V, E\right)$ be a $d$-regular graph, of random walk matrix $M$, with eigenvalues $\lambda_1 \geq \lambda_2 \geq \ldots \geq \lambda_n$ associated to eigenvectors $v_1, \ldots, v_n$.
    
    Then, we have: 
    \[\lambda_2 = \max_{x \perp v_1} \frac{x^T M x}{x^T x}.\]
\end{parag}

\begin{parag}{Definition: Laplacian matrix}
    Let $G = \left(V, E\right)$ be a $d$-regular graph, of random walk matrix $M$.

    We define $L = I - M$ to be the \important{Laplacian} of $G$, where $I$ is the $n \times n$ identity matrix.
\end{parag}

\begin{parag}{Property}
    Let $G = \left(V, E\right)$ be a $d$-regular graph, of Laplacian $L$.

    Then, for all $x \in \mathbb{R}^n \setminus \left\{0\right\}$: 
    \[x^T L x = \frac{1}{d} \sum_{\left\{i, j\right\} \in E} \left(x\left(i\right) - x\left(j\right)\right)^2.\]
    
    \begin{subparag}{Proof}
        This will be proven in the twelfth exercise series.
    \end{subparag}
\end{parag}

\begin{parag}{Lemma}
    Let $G = \left(V, E\right)$ be a $d$-regular graph, of random walk matrix $M$, with eigenvalues $\lambda_1 \geq \lambda_2 \geq \ldots \geq \lambda_n$, associated to normalised eigenvectors $v_1, \ldots, v_n$.

    Then, noting $\bvec{1}$ to be the all ones vector: 
    \[1 - \lambda_2 = \min_{x \perp \bvec{1}} \dfrac{1}{d}\cdot \dfrac{\sum_{\left\{i, j\right\} \in E} \left(x\left(i\right) - x\left(j\right)\right)^2}{x^T x}.\]

    This bound is moreover tight for $x = v_2$, i.e.:
    \[1 - \lambda_2 = \frac{1}{d}\cdot \frac{1}{d} \sum_{\left\{i, j\right\} \in E} \left(v_2\left(i\right) - v_2\left(j\right)\right)^2.\]

    \begin{subparag}{Intuition}
        Putting both affirmations together, this theorem states that:
        \[1 - \lambda_2 = \frac{1}{d} \sum_{i, j \in E} \left(v_2\left(i\right) - v_2\left(j\right)\right)^2 = \min_{x \perp \bvec{1}} \dfrac{1}{d}\cdot \dfrac{\sum_{\left\{i, j\right\} \in E} \left(x\left(i\right) - x\left(j\right)\right)^2}{x^T x}.\]

        In other words, $v_2$ assigns a value to each vertex, in such a way that the average stretch $\left(v_2\left(i\right) - v_2\left(j\right)\right)^2$ is minimised, while being prevented to set all vertices to the same value by being orthogonal to $\bvec{1}$. However, something that minimises the average stretch is a form of min-cut.  This gives more intuition on why the spectral graph partitioning algorithm works.

        This lemma is moreover the main intuition on why the following theorem should hold, by a similar reasoning.
    \end{subparag}

    \begin{subparag}{Personal remark}
        We can use a generalisation of this lemma to draw graphs. Consider for instance the following hexagonal topology with a small hole:
        \svghere[0.45]{HexagonalLatticeWithHole_Grid.svg} 

        We can take two arbitrary eigenvectors $v_k$ and $v_{\ell}$, and draw each vertex $i \in V$ at position $\left(v_k\left(i\right), v_{\ell}\left(j\right)\right)$. For $\left(k, \ell\right) = \left(2, 3\right)$ and $\left(k, \ell\right) = \left(n-1, n\right)$, we respectively have:
        \begin{center}
        \begin{minipage}{0.45\textwidth}
        \svghere{HexagonalLatticeWithHole_SmallEigs.svg}
        \end{minipage}
        \hfill
        \begin{minipage}{0.45\textwidth}
        \svghere{HexagonalLatticeWithHole_LargeEigs.svg}
        \end{minipage}
        \end{center}
        
        This makes sense: as stated by the lemma, $v_2$ is the vector orthogonal to $v_1 = \bvec{1}$ that minimises the average stretch $\left(v_2\left(i\right) - v_2\left(j\right)\right)^2$. Similarly $v_3$ is the vector orthogonal to both $v_1$ and $v_2$ that minimises the average stretch. On the other hand, $v_{n-1}$ is the vector that minimises the average stretch while being orthogonal to $v_1, \ldots, v_{n-2}$. In other words, it is one of the vector that maximises the average stretch. Similarly, $v_n$ is the vector that maximises the average stretch.

        So, on the first drawing, the average stretch is very small in both the $x$ and $y$ direction, giving this pretty result. On the second drawing, however, the average stretch is very large in both directions. This explains why connected vertices end up close to one another in the first picture, but far from each other in the second.

        Note that drawing with $\left(k, \ell\right) = \left(2, 3\right)$ is something used in practice to visualise graphs. This is for instance the technique used by the Python package \texttt{networkx} for its \texttt{draw\_spectral} function.
    \end{subparag}
    
    \begin{subparag}{Proof}
        We can always take $v_1 = \bvec{1}$. So, by one of our lemmas: 
        \autoeq{1 - \lambda_2 = 1 - \max_{x \perp \bvec{1}} \frac{x^T M x}{x^T x} = \min_{x \perp \bvec{1}} \left(1 - \frac{x^T M x}{x^T x}\right)  = \min_{x \perp \bvec{1}} \left(\frac{x^T x - x^T M x}{x^T x}\right) = \min_{x \perp \bvec{1}} \left(\frac{x^T \left(I - M\right)x}{x^T x}\right).}

        We recognise the Laplacian of $G$ so, using its property:
        \[1 - \lambda_2 = \min_{x \perp \bvec{1}} \frac{x^T L x}{x^T x} = \min_{x \perp \bvec{1}} \frac{1}{d}\cdot \frac{\sum_{\left\{i, j\right\} \in E} \left(x\left(i\right) - x\left(j\right)\right)^2}{x^T x}.\]

        We know that $1 - \lambda_2 = 1 - v_2^T M v_2$. We can then use exactly the same equalities to show that the bound is tight for $x = v_2$.

        \qed
    \end{subparag}
\end{parag}
 
\begin{parag}{Theorem: Cheeger's inequalities}
    Let $G = \left(V, E\right)$ be a $d$-regular graph, of random walk matrix $M$, with eigenvalues $\lambda_1 \geq \lambda_2 \geq \ldots \geq \lambda_n$.

    Then:
    \[\frac{1 - \lambda_2}{2} \leq h\left(G\right) \leq \sqrt{2 \left(1 - \lambda_2\right)}.\]

    \begin{subparag}{Intuition}
        As we wanted intuitively, when $\lambda_2 \ll 1$ is small, then $h\left(G\right)$ is large. There is thus no ``sparse'' cut in $G$. Also, when $\lambda_2 \approx 1$, then $h\left(G\right) \approx 0$, and so there exists very ``sparse'' cuts, that can be found efficiently using the spectral graph partitioning algorithm.

        Note that this theorem shows the result of the spectral partitioning algorithm is quite interesting.
    \end{subparag}
   
    \begin{subparag}{Remark 1}
        Note that this is not exactly the same $\epsilon$ as the one we had for mixing time for random walks, but we may still refer to $\epsilon = 1 - \lambda_2$ as a spectral gap.
    \end{subparag}

    \begin{subparag}{Remark 2}
        Both inequalities are tight. We will not see this affirmation formally in this class, however, intuitively:
        \begin{itemize}[left=0pt]
            \item The first one is reached for the hypercube, i.e. $V = \left\{0, 1\right\}^d$ and $E = \left\{\left\{x, x + e_i\right\} \suchthat x \in V, i \in \left\{1, \ldots, d\right\}\right\}$, for $d = \log_2\left(n\right)$.
            \item The second one is reached on the cycle on $n$ vertices.

                It is possible to show that $h\left(G\right) = \frac{2}{n}$, by splitting it in the middle: two edges cross the cut, and $\left|S\right| = \left|V \setminus S\right| = \frac{n}{2}$. It is moreover also possible to show that $\lambda_2 \in 1 - O\left(\frac{1}{n^2}\right)$. We intuitively do have $h\left(G\right) \in O\left(\frac{1}{n}\right)$ and $\sqrt{2\left(1 - \lambda_2\right)} \in O\left(\frac{1}{n}\right)$.
        \end{itemize}
    \end{subparag}

    \begin{subparag}{Proof (1)}
        We want to show the first inequality:
        \[1 - \lambda_2 \leq 2 h\left(G\right)\]

        Let $S \subseteq V$ be arbitrary. We assume without loss of generality that $\left|S\right| \leq \frac{1}{2} \left|V\right|$ since $h\left(S\right) = h\left(V \setminus S\right)$ (and we could thus replace $S \leftarrow V \setminus S$ if $\left|S\right| > \frac{1}{2} \left|V\right|$). We consider $y = \bvec{1}_S - \frac{\left|S\right|}{\left|V\right|} \bvec{1}$, where $\bvec{1}_S$ is the indicator of $S$. In other words, $y$ is defined by:
        \[y\left(i\right) = \begin{systemofequations} 1 - \frac{\left|S\right|}{V}, & \text{if $i \in S$,} \\ -\frac{\left|S\right|}{\left|V\right|}, & \text{otherwise}. \end{systemofequations}\]
        
        Let $\delta\left(S\right) \subseteq E$ be the set of edges crossing the cut. We notice that for all edge $\left\{i, j\right\} \in \delta\left(S\right)$ crossing the cut, then $\left(y\left(i\right) - y\left(j\right)\right)^2 = 1$. On the other hand, for any edge $\left\{i, j\right\} \not\in \delta\left(S\right)$ staying in $S$ or staying in $V \setminus S$, then $\left(y\left(i\right) - y\left(j\right)\right)^2 = 0$. This yields, by our lemma: 
        \[1 - \lambda_2 \leq \frac{\sum_{\left\{i, j\right\} \in E} \left(y\left(i\right) - y\left(j\right)\right)^2}{d\cdot y^T y} = \frac{\left|\delta\left(S\right)\right|}{d\cdot y^T y}.\]

        We can moreover compute the squared norm of $y$:
        \autoeq{y^T y = \sum_{i} y\left(i\right)^2 = \sum_{i \in S} \left(1 - \frac{\left|S\right|}{\left|V\right|}\right)^2 + \sum_{i \in V \setminus S} \left(\frac{\left|S\right|}{\left|V\right|}\right)^2 = \left|S\right| \left(1 - \frac{\left|S\right|}{\left|V\right|}\right)^2 + \left(\left|V\right| - \left|S\right|\right)\left(\frac{\left|S\right|}{\left|V\right|}\right)^2,}

        We can use algebra to simplify this to:
        \[y^T y = \frac{\left|S\right|\cdot \left|V \setminus S\right|}{\left|V\right|} \geq \frac{\left|S\right|}{2}\]
        where we used that $\left|V \setminus S\right| \geq \frac{1}{2} \left|V\right|$, since we asked for $\left|S\right| \leq \frac{1}{2} \left|V\right|$.
        
        But then, since $\left|S\right| \leq \frac{1}{2}\left|V\right| \leq \left|V \setminus S\right|$, we recognise the definition of conductance: 
        \[1 - \lambda_2 \leq \frac{\left|\delta\left(S\right)\right|}{d\cdot y^Ty} \leq \frac{2\cdot  \left|\delta\left(S\right)\right|}{d\cdot \left|S\right|} = 2\cdot \frac{\left|\delta\left(S\right)\right|}{d\cdot \min\left\{\left|S\right|, \left|V \setminus S\right|\right\}} = 2 h\left(S\right).\]

        This is exactly the inequality we wished to prove, since $S$ was arbitrary.
    \end{subparag}

    \begin{subparag}{Proof (2)}
        The second inequality comes from the spectral partitioning algorithm: it is always possible to find a cut $S$ such that $h\left(S\right) \leq \sqrt{2\left(1 - \lambda_2\right)}$. Since moreover $h\left(G\right) \leq h\left(S\right)$ by definition, this gives our result.

        \qed
    \end{subparag}
\end{parag}

\end{document}
