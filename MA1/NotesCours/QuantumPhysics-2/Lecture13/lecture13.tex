% !TeX program = lualatex
% Using VimTeX, you need to reload the plugin (\lx) after having saved the document in order to use LuaLaTeX (thanks to the line above)

\documentclass[a4paper]{article}

% Expanded on 2024-12-16 at 11:50:30.

\usepackage{../../style}

\title{QP2}
\author{Joachim Favre}
\date{Lundi 16 décembre 2024}

\begin{document}
\maketitle

\lecture{13}{2024-12-16}{Ah yes, $\mathfrak{so}\left(3\right) = SO\left(3\right)$}{
\begin{itemize}[left=0pt]
    \item Construction of irreps for $\mathfrak{so}\left(3\right)$.
    \item Explanation and examples of the Clebsch-Gordan decomposition.
\end{itemize}

}

\subsection{Addition of angular momentum}

\begin{parag}{Personal remark}
    This lecture was not given by the Professor, we had to read the lecture notes instead. This is my best attempt to explain how I understood this subject, notably compiling some questions I asked to TAs. There are probably some inaccuracies, do not hesitate to point them out to me.

    \begin{subparag}{Acknowledgement}
        A big thanks to Nathan Brunet, Jonas Daverio, Gabriel Pescia and Reyhaneh Saem for their answers and their patience on my questions.
    \end{subparag}
\end{parag}

\begin{parag}{Definition: Angular momentum}
    Let $\mathcal{H}$ be a quantum system, with position operator $\hat{r} = \left(\hat{x}, \hat{y}, \hat{z}\right)$ and momentum operator $\hat{p} = \left(\hat{p}_x, \hat{p}_y, \hat{p}_z\right)$.

    We define its angular momentum operators as:
    \[\begin{pmatrix} \hat{J}_x \\ \hat{J}_y \\ \hat{J}_z \end{pmatrix} = \begin{pmatrix} \hat{y} \hat{p}_z - \hat{z} \hat{p}_y \\ \hat{z} \hat{p}_x - \hat{x} \hat{p}_z \\ \hat{x} \hat{p}_y - \hat{y} \hat{p}_x \end{pmatrix}.\]

    \begin{subparag}{Intuition}
        Recall that, in classical mechanics, we defined angular momentum as:
        \[\bvec{L} = \bvec{x} \times \bvec{p} = \begin{pmatrix} y p_z - z p_y \\ z p_x - x p_z \\ x p_y - y p_x\end{pmatrix}.\]

        We thus simply promote everything to an operator for quantum systems.
    \end{subparag}
\end{parag}

\begin{parag}{Theorem}
    Let $\mathcal{H}$ be a quantum system, of angular momentum operators $J_x, J_y, J_z$.

    Then, $i J_x, i J_y, i J_z$ represent a basis for a representation of $\mathfrak{so}\left(3\right)$.

    \begin{subparag}{Proof}
        It is possible to prove, like we did in Quantum physics I, that: 
        \[\left[J_i, J_x\right] = i \hbar \sum_{k} \epsilon_{ijk} J_k.\]

        Leaving $\hbar = 1$ for simplicity, this shows that their structure constants are the same as the lie algebra corresponding to the orthogonal group $SO\left(3\right)$. Since lie algebras can be identified by their structure constants, this shows our result.

        \qed
    \end{subparag}
\end{parag}

\begin{parag}{Theorem}
    Let $j \in \left\{0, \frac{1}{2}, 1, \ldots\right\}$ be arbitrary.

    We can construct a representation of $\mathfrak{so}\left(3\right)$, which we note $j$, given by some eigenvectors $\ket{j m}$ of $J_z$. 

    To do so, we define the following ladder operators: 
    \[J_- = J_x - i J_y, \mathspace J_+ = J_x + i J_y.\]
    
    They are such that:
    \begin{itemize}
        \item $J_z \ket{j, m} = m$.
        \item $\left[J_z, J_{\pm}\right] = \pm J_{\pm}.$
        \item $\left[J_+, J_-\right] = 2J_z$.
        \item $J_+ \ket{j, m} = \sqrt{j\left(j+1\right) - m\left(m+1\right)} \ket{j, m+1}$.
        \item $J_- \ket{j, m} = \sqrt{j\left(j+1\right) - m\left(m-1\right)} \ket{j, m-1}.$
    \end{itemize}
    
    All these properties allow to compute $\bra{j, m'} J_z \ket{j, m} = m \delta_{m, m'}$, and the coordinates of $J_x = \frac{1}{2}\left(J_+ + J_-\right)$ and $J_y = \frac{1}{2i}\left(J_+ - J_y\right)$. These $J_x, J_y, J_z$ form a basis for a representation of $\mathfrak{so}\left(3\right)$.

    \begin{subparag}{Example}
        For instance, the $j= 0$ representation is the trivial irrep: 
        \[J_x = J_y = J_z = 0.\]

        Moreover, the $j = \frac{1}{2}$ irrep is, in the $\ket{\frac{1}{2}, \frac{1}{2}}, \ket{\frac{1}{2}, -\frac{1}{2}}$ basis: 
        \[J_x = \frac{1}{2} \begin{pmatrix} 1 & 0 \\ 0 & 1 \end{pmatrix}, \mathspace J_y = \frac{1}{2}\begin{pmatrix} 0 & -i \\ i & 0 \end{pmatrix}, \mathspace J_z = \frac{1}{2} \begin{pmatrix} 1 & 0 \\ 0 & -1 \end{pmatrix}.\]

        We recognise the Pauli matrices. As a last example, the $j = 1$ irrep is, in the $\ket{1, 1}, \ket{1, 0}, \ket{1, -1}$ basis: 
        \[J_x = \frac{1}{\sqrt{2}} \begin{pmatrix} 0 & 1 & 0 \\ 1 & 0 & 1 \\ 0 & 1 & 0 \end{pmatrix}, \mathspace J_y = \frac{1}{\sqrt{2}} \begin{pmatrix} 0 & -i & 0 \\ i & 0 & -i \\ 0 & i & 0 \end{pmatrix},\]
        \[J_z = \begin{pmatrix} 1 & 0 & 0 \\ 0 & 0 & 0 \\ 0 & 0 & -1 \end{pmatrix}.\]
    \end{subparag}

    \begin{subparag}{Proof}
        We make a constructive proof.

        We want to construct a representation of $\mathfrak{so}\left(3\right)$, by diagonalising $J_z$. So, let $\ket{m}$ be an eigenstate associated to eigenvalue $m$: 
        \[J_z \ket{m} = m \ket{m}.\]

        Since $\left[J_x, J_z\right] \neq 0$ and $\left[J_y, J_z\right] \neq 0$, this shows that we cannot block diagonalise them at the same time, and hence the $\ket{m}$ are not eigenstates of $J_x$ and $J_y$. Let us instead introduce some helper operators: 
        \[J_+ = J_x + i J_y, \mathspace J_- = J_x - i J_y.\]
        
        Notice that they are such that, using the commutation properties of the angular momentum operators: 
        \[\left[J_z, J_{\pm}\right] = \left[J_z, J_x \pm i J_y\right] = iJ_y \pm i\cdot \left(-i J_x\right) = i J_y \pm J_x = \pm J_{\pm},\]
        \[\left[J_+, J_-\right] = \left[J_x + i J_y, J_x - i J_y\right] = 2i \left[J_y, J_x\right] = 2i\left(-i J_z\right) = 2 J_z.\]

        Moreover, we notice that $J_{\pm}^{\dagger} = J_{\mp}$ since $J_x$ and $J_y$ are Hermitian. 

        We can thus evaluate: 
        \autoeq{J_z J_{\pm} \ket{m} = J_{\pm} J_z \ket{m} + \left[J_z, J_{\pm}\right]\ket{m} = m J_{\pm} \ket{m} \pm J_\pm \ket{m} = \left(m \pm 1\right) J_{\pm} \ket{m}.}

        In other words, $J_+\ket{m}$ is an eigenket of $J_z$, with eigenvalue $m + 1$; hence $J_+ \ket{m} = c_{m+1}\ket{m+1}$ for some constant $c_{m+1}$. Similarly, $J_- \ket{m} = b_{m-1} \ket{m - 1}$. This justifies the notation $J_+$ and $J_-$: they allow to move up or down on the ladder of eigenstates $\ldots, \ket{m-2}, \ket{m-1}, \ket{m}, \ket{m+1}, \ldots$.

        Moreover, we can link $c_m$ and $b_m$ by using the fact that $J_+ = J_-^{\dagger}$:
        \autoeq{b_m = \bra{m} J_- \ket{m + 1} = \bra{m+1} J_{-}^{\dagger} \ket{m}^* = \bra{m+1} J_+ \ket{m}^* = c_{m+1}^*.}

        This shows that $J_+ \ket{m} = c_{m+1} \ket{m+1}$ and $J_- \ket{m} = c_m^* \ket{m-1}$. We want to evaluate those values $c_m$ explicitly, by finding a recurrence relation. We notice that: 
        \[J_+ J_- \ket{m} =  c_m^* J_+ \ket{m-1} = \left|c_m\right|^2 \ket{m},\]
        \[J_- J_+ \ket{m} = c_{m+1} J_- \ket{m+1} = \left|c_{m+1}\right| \ket{m}.\]

        This allows us to find a recurrence relation for our coefficients, using the fact $\left[J_+, J_-\right] = 2J_z$ as found above: 
        \autoeq{\left|c_m\right|^2 - \left|c_{m+1}\right|^2 = \bra{m}\left(J_+ J_- - J_- J_+\right) \ket{m} = \bra{m} \left[J_+, J_-\right] \ket{m} = \bra{m} 2J_z \ket{m} = 2m.}

        We thus only need an initial condition in order to be able to solve our recurrence, i.e. we need to the value of some $c_{m_0}$. To do so, we notice that we aim to make a finite-dimensional representation of $\mathfrak{so}\left(3\right)$, so we cannot just have infinitely many such eigenvectors in our ladder. This means that there is some lower bound $\ell$ and upper bound $j$ so that all our states are $\ket{\ell}, \ket{\ell + 1}, \ldots, \ket{j-1}, \ket{j}$. In particular, this asks for: 
        \[J_+\ket{j} = 0 \implies c_{j+1} = 0, \mathspace J_- \ket{\ell} = 0 \implies c_{\ell} = 0.\]

        This allows us to find the initial condition for our recurrence relation: 
        \autoeq{0 = \bra{j} J_- J_+ \ket{j} = \bra{j} J_+ J_- \ket{j} + \bra{j} \left[J_+, J_-\right] \ket{j} = \left|c_j\right|^2 + 2\bra{j} J_z \ket{j} = \left|c_j\right|^2 + 2j.}

        Therefore, $\left|c_j\right|^2 = 2j$. However this, together with the recurrence $\left|c_m\right|^2  = \left|c_{m+1}\right|^2 + 2m$ we found earlier, can be solved to find, for all $s \geq 0$: 
        \autoeq{\left|c_{j-s}\right|^2 = \left|c_{j-s + 1}\right|^2 + 2\left(j-s\right) = \left|c_{j-s + 2}\right|^2 + 2\left(j-s + 1\right) + 2\left(j-s\right) = \ldots = \left|c_j\right|^2 + 2 \sum_{k=j-s}^{j} k.}

        Being careful on the fact that the sum may range between negative and positive numbers (we may have $j - s < 0$), we can finally simplify this to: 
        \[\left|c_{j-s}\right|^2 = \left(s+1\right)\left(2j - s\right).\]
        
        In particular, we notice that this goes to zero if and only if $s = 2j$, i.e. $c_{-j} = 0$. This shows that $J_- \ket{-j} = c_{-j} = 0$. This means that the ladder terminates at this point, and hence that we had $\ell = -j$. Moreover, we know that we can apply $2j$ times the operator $J_-$ on $\ket{j}$ to get to $\ket{-j}$, so $2j$ must be an integer. This thus forces that $j$ is a half integer, $j \in \left\{0, \frac{1}{2}, 1, \frac{3}{2}, \ldots\right\}$. Since this is a fundamental parameter in our construction, we may write $\ket{m} \equiv \ket{j, m}$.
        
        To finish with, we can express $\left|c_m\right|$:
        \autoeq{\left|c_{j-s}\right|^2 = \left(s+1\right)\left(2j - s\right) \implies \left|c_m\right|^2 = \left(j-m+1\right)\left(2j - j + m\right) = \left(j-m+1\right)\left(j + m\right) \implies \left|c_m\right| = \sqrt{\left(j-m+1\right)\left(j+m\right)}.}
        
        We have a choice on the phase of the $c_m$, which makes sense since global phases do not matter in quantum physics. By simplicity, we can thus simply write: 
        \[c_m = \sqrt{\left(j-m+1\right)\left(j+m\right)} = \sqrt{j\left(j+1\right) - m\left(m-1\right)}.\]

        \qed
    \end{subparag}
\end{parag}

\begin{parag}{Theorem}
    Let $j \in \left\{0, \frac{1}{2}, 1, \ldots\right\}$ be arbitrary.

    The representation $j$ of $\mathfrak{so}\left(3\right)$ is an irrep.

    \begin{subparag}{Remark 1}
        We did not define formally the notion of irreducible representation of lie algebras. Intuitively, this is just like for groups, a representation is reducible if all its elements can be block diagonalised. In particular, it suffices to block diagonalise a basis, $J_x, J_y, J_z$ in our case.
    \end{subparag}

    \begin{subparag}{Remark 2}
        Note that before in the course we were considering finite groups, and hence there was a finite number of irreps. Now that we are considering a continuous group, it is not surprising that we have infinitely many irreps.
    \end{subparag}
\end{parag}

\begin{parag}{Clebsch-Gordan algorithm}
    We consider the following algorithm, which allows to decompose the representation $j \otimes j'$ as a direct sum of irreps.

    The idea is to decompose the $\ket{J, M}$ basis (associated to the angular momentum of the larger system $J_k = J_k^{\left(1\right)} \otimes I + I \otimes J_k^{\left(2\right)}$) into the basis $\ket{j_1, m_1} \otimes \ket{j_2, m_2}$ (associated to the angular momentum of the two sub-systems). To do so, we can use an inductive process with the following three rules:
    \begin{enumerate}
        \item Start with $\ket{J = j_1 + j_2, M = j_1 + j_2} = \ket{j_1, j_1} \otimes \ket{j_2, j_2}$.
        \item Given $\ket{J = j, M = m}$, find $\ket{J = j, M = m - 1}$ using the fact:
        \autoeq[s]{\ket{J = j, M = m-1} \propto J_- \ket{J = j, M = m} = \left(J_-^{\left(1\right)} \otimes I\right) \ket{J = j, M = m} + \left(I \otimes J_-^{\left(2\right)}\right) \ket{J = j, M = m}.}
        \item Given $\ket{J = j+1, M = j}, \ldots, \ket{J = j_1 + j_2, M = j}$, find $\ket{J = j, M = j}$ using the fact it is orthogonal to all $\ket{J = j+1, M = j}, \ldots, \ket{J = j_1 + j_2, M = j}$ and that it uses the same basis ket (i.e. it can be expressed in the basis $\left\{\ket{j_1, m_1 = k} \otimes \ket{j_1, m_2 = j - k} \suchthat k \right\}$).
    \end{enumerate}

    We can moreover make the process faster, by using the following symmetry: 
    \autoeq[s]{\left(\bra{j_1, -m_1} \otimes \bra{j_2, -m_2}\right)\ket{J=j, M=-m} = \left(-1\right)^{j_1 + j_2 - j} \left(\left(\bra{j_1, m_1} \otimes \bra{j_2, m_2}\right)\right)\ket{J=j, M=m}.}

    In other words, we get $\ket{J=j, M=-m}$ from $\ket{J=j, M=m}$ by replacing each $\ket{j_1, m_1} \otimes \ket{j_2, m_2}$ in its basis expansion by $\left(-1\right)^{j_1 + j_2 - j} \ket{j_1, m_1} \otimes \ket{j_2, m_2}.$

    \begin{subparag}{Remark}
        Methodically, we may always start with $J = j_1 + j_2$ and $M = j_1 + j_2$. We can then find $\ket{J = j_1 + j_2, M}$ for each $M = j_1 + j_2, \ldots, -j_1 - j_2$. We can then decrease the value of $J$ by one, and start again; doing this until $J = j_1 - j_2$.
    \end{subparag}

    \begin{subparag}{Intuition}
        The idea is that we know $J_x^{\left(1\right)}, J_y^{\left(1\right)}, J_z^{\left(1\right)}$ form a basis for the $j_1$ irrep of $\mathfrak{so}\left(3\right)$, and similarly for $J_x^{\left(2\right)}, J_y^{\left(2\right)}, J_z^{\left(2\right)}$. We know that this implies $J_k^{\left(1\right)} \otimes J_k^{\left(2\right)}$ for $k \in \left\{x, y, z\right\}$ is also the basis for a representation of $\mathfrak{so}\left(3\right)$ (although not an irrep). 

        However, physically, $\ket{j_1, m_1} \otimes \ket{j_2, m_2}$ is a basis that shows the angular momentum of two subsystems. We can expect that there is some states $\ket{J, M}$ that should represent the total angular momentum of the full system. Since we want angular momentums to add, this means that the larger system has basis $J_k = J_k^{\left(1\right)} \otimes I + I \otimes J_k^{\left(2\right)}$. Indeed, for instance, we do have $J_z \ket{j_1, m_1} \otimes \ket{j_2, m_2} = m_1 + m_2$.

        This construction thus allows to find how to change between bases $\ket{j_1, m_1} \otimes \ket{j_2, m_2}$ and $\ket{J, M}$. To do so, we use the three rules explained above iteratively. Let us explain the intuition behind them.
        \begin{enumerate}
            \item  The first one states that the only way for the full system to reach total $z$-angular momentum $M = j_1 + j_2$ is that both sub-systems have a maximal $z$-angular momentum, $m_1 = j_1$ and $m_2 = j_2$. 
            \item The second rule simply uses the ladder operator on the full system, since we know how it will act on the subsystems.
            \item The third rule uses orthogonality, and the fact that $\ket{J = j, M = m}$ uses the exact same basis as $\ket{J = j', M = m}$, $\left\{\ket{j_1, m_1 = k} \otimes \ket{j_2, m_2 = m-k} \suchthat k\right\}$. The latter fact can be understood intuitively by stating that, for the whole system to have $z$-angular momentum $M$, then the subsystems must be such that $m_1 + m_2 = M$.
        \end{enumerate}
        

        Finally, we range $J \in \left\{\left|j_1 - j_2\right|, \ldots, j_1 + j_2\right\}$. Indeed, the total angular momentum $J$ is maximal when the angular momentum of both subsystems go in the same direction, in which case it is $j_1 + j_2$; and it is minimal when the angular momentum of both subsystems go in the opposite direction, in which case it is $\left|j_1 - j_2\right|$. Note that this intuition relies heavily on the classical intuition, there is no real concept of ``direction'' for angular momentum in quantum.
    \end{subparag}
\end{parag}

\begin{parag}{Example 1}
    We have: 
    \[\frac{1}{2} \otimes \frac{1}{2} = 1 \oplus 0.\]

    \begin{subparag}{Proof}
        \begin{itemize}[left=0pt]
            \item We start by finding all $\ket{J, M}$ for $J = 1$. We know:
            \[\ket{J = 1, M = 1} = \ket{j_1 = \frac{1}{2}, m_1 = \frac{1}{2}} \otimes \ket{j_1 = \frac{1}{2}, m_1 = \frac{1}{2}}.\]

            For the sake of simplicity, we will write this equivalently as: 
            \[\ket{1, 1} = \ket{\frac{1}{2}, \frac{1}{2}} \otimes \ket{\frac{1}{2}, \frac{1}{2}}.\]
        

            Since we know $J_- \ket{1, 1} = \sqrt{2} \ket{1, 0}$ and $J_-^{\left(1\right)} \ket{\frac{1}{2}, \frac{1}{2}} = \ket{\frac{1}{2}, -\frac{1}{2}}$:
            \autoeq{\ket{1, 0} = \frac{1}{\sqrt{2}} J_- \ket{1, 1} = \frac{\left(J_-^{\left(1\right)} \otimes I\right) \ket{1, 1} + \left(I \otimes J_-^{\left(2\right)}\right) \ket{1, 1}}{\sqrt{2}} = \frac{J_-^{\left(1\right)} \ket{\frac{1}{2}, \frac{1}{2}} \otimes \ket{\frac{1}{2}, \frac{1}{2}} + \ket{\frac{1}{2}, \frac{1}{2}} \otimes J_-^{\left(2\right)} \ket{\frac{1}{2}, \frac{1}{2}}}{\sqrt{2}} = \frac{\ket{\frac{1}{2}, -\frac{1}{2}} \otimes \ket{\frac{1}{2}, \frac{1}{2}} + \ket{\frac{1}{2}, \frac{1}{2}} \otimes \ket{\frac{1}{2}, -\frac{1}{2}}}{\sqrt{2}}.}
            
            Finding $\ket{1, -1}$ is again very similar, since $J_- \ket{1, 0} = \sqrt{2} \ket{1, -1}$ and $J_-^{\left(1\right)} \ket{\frac{1}{2}, -\frac{1}{2}} = 0$: 
            \autoeq[s]{\ket{1, -1} = \frac{1}{\sqrt{2}} J_- \ket{1, 0} = J_- \frac{\ket{\frac{1}{2}, -\frac{1}{2}} \otimes \ket{\frac{1}{2}, \frac{1}{2}} + \ket{\frac{1}{2}, \frac{1}{2}} \otimes \ket{\frac{1}{2}, -\frac{1}{2}}}{2} = \frac{0 \otimes \ket{\frac{1}{2}, \frac{1}{2}} + \ket{\frac{1}{2}, -\frac{1}{2}} \otimes \ket{\frac{1}{2}, -\frac{1}{2}} + \ket{\frac{1}{2}, -\frac{1}{2}} \otimes \ket{\frac{1}{2}, -\frac{1}{2}} + \ket{\frac{1}{2}, \frac{1}{2}} \otimes 0}{2} = \ket{\frac{1}{2}, -\frac{1}{2}} \otimes \ket{\frac{1}{2}, -\frac{1}{2}}.}
             
             \item We now look for all $\ket{J, M}$ for $J = 0$. We know $\ket{J = 0, M = 0}$ is orthogonal to $\ket{J = 1, M = 0}$. Therefore, it must be: 
             \[\ket{J=0, M =0} = \frac{\ket{\frac{1}{2}, -\frac{1}{2}} \otimes \ket{\frac{1}{2}, \frac{1}{2}} - \ket{\frac{1}{2}, \frac{1}{2}} \otimes \ket{\frac{1}{2}, -\frac{1}{2}}}{\sqrt{2}}.\]
        \end{itemize}

        As a quick sanity check, one can verify that all those vectors are indeed normalised.
    \end{subparag}
\end{parag}

\begin{parag}{Example 2}
    We have:
    \[1 \otimes 1 = 2 \oplus 1 \oplus 0.\]

    \begin{subparag}{Proof}
        \begin{itemize}
            \item We start with $\ket{J, M}$ for $J = j_1 + j_2 = 2$. We know: 
            \[\ket{J = 2, M = 2} = \ket{j_1 = 1, m_1 = 1} \otimes \ket{j_2 = 1, m_2 = 1}.\]

            Let us simplify the notation even more than before, since we know that $j_1 = 1$ and $j_2 = 1$ will always hold. We thus write this in the form: 
            \[\ket{2, 2} = \ket{1} \otimes \ket{1}.\]

            \textit{(We may even write this as $\ket{2, 2} = \ket{1, 1}$ if we really feel like we want to use horrendous notations as good physicists.)} Then, doing something completely similar to the previous example: 
            \[\ket{2, 1} = \frac{\ket{1} \otimes \ket{0} + \ket{0} \otimes \ket{1}}{\sqrt{2}}.\]
            
            We can apply this again, using the fact $J_- \ket{J=2, M=1} = \sqrt{6}\ket{J=2, M=0}$: 
            \autoeq{\ket{2, 0} = \frac{1}{\sqrt{6}}J_- \ket{2, 1} = \frac{\sqrt{2}\ket{-1} \otimes \ket{1} + \sqrt{2} \ket{0} \otimes \ket{0} + \sqrt{2} \ket{-1} \otimes \ket{1} + \sqrt{2} \ket{0} \otimes \ket{0}}{\sqrt{6}\cdot \sqrt{2}} = \frac{\ket{-1} \otimes \ket{1} + 2 \ket{0} \otimes \ket{0} + \ket{-1} \otimes \ket{1}}{\sqrt{6}}.}
            
            We can now exploit symmetry, to find that: 
            \[\ket{2, -1} = \frac{\ket{-1} \otimes \ket{0} + \ket{0} \otimes \ket{-1}}{\sqrt{2}},\]
            \[\ket{2, -2} = \ket{-1} \otimes \ket{-1}.\]
             
            \item We now consider $J = 1$.  We know that $\ket{1, 1}$ must be orthogonal with $\ket{2,1}$, so we can take: 
            \[\ket{1, 1} = \frac{\ket{0} \otimes \ket{1} - \ket{1} \otimes \ket{0}}{\sqrt{2}}.\]

            We then find: 
            \[\ket{1, 0} = \frac{\ket{-1} \otimes \ket{1} - \ket{1} \otimes \ket{-1}}{\sqrt{2}},\]
            \[\ket{1, -1} = \frac{\ket{-1} \otimes \ket{0} - \ket{0} \otimes \ket{-1}}{\sqrt{2}}.\]
            
            \item We finally consider $J = 0$. We know $\ket{0, 0}$ must be orthogonal to both $\ket{2, 0}$ and $\ket{1, 0}$. It is possible to guess the answer, or solve a linear system, to find that: 
            \[\ket{0, 0} = \frac{\ket{-1} \otimes \ket{1} - \ket{0} \otimes \ket{0} + \ket{1} \otimes \ket{-1}}{\sqrt{3}}.\]
        \end{itemize}
    \end{subparag}
\end{parag}

\begin{parag}{$SO\left(3\right)$ representation}
    Note that, all we have done so far only applies to representations of $\mathfrak{so}\left(3\right)$. Indeed, $SO\left(3\right)$ is not simply connected, so we cannot just get a representation of $SO\left(3\right)$ by exponentiating the representations of $\mathfrak{so}\left(3\right)$.

    However, physically, this actually works. Indeed, exponentiating a representation of $SO\left(3\right)$, we get something called a ``projective representation'', where we allow global phases i.e. $R\left(g_1\right)\cdot R\left(g_2\right) = e^{i\phi} R\left(g_1 * g_2\right)$. Since global phases do not matter in quantum physics, everything we did here actually also allows us to find representations $SO\left(3\right)$.

    In practice, since there is this strong link between $\mathfrak{so}\left(3\right)$ and $SO\left(3\right)$, people may use the symbol $SO\left(3\right)$ when speaking of either. This is why, in exercise series, $\ket{j, m}$ is called a representation of $SO\left(3\right)$ (when, formally, it is really a representation of $\mathfrak{so}\left(3\right)$).

    \begin{subparag}{Personal remark}
        This actually took me so long to understand. A big thanks to Jonas Daverio for explaining me this.
        \imagehere[0.6]{BetterNotationsWouldHaveSavedMeASoMuchTimeThisIsDeserved.png}
    \end{subparag}
\end{parag}

\end{document}
