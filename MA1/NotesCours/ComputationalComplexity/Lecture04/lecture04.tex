% !TeX program = lualatex
% Using VimTeX, you need to reload the plugin (\lx) after having saved the document in order to use LuaLaTeX (thanks to the line above)

\documentclass[a4paper]{article}

% Expanded on 2024-10-03 at 15:29:09.

\usepackage{../../style}

\title{Comp comp}
\author{Joachim Favre}
\date{Jeudi 03 octobre 2024}

\begin{document}
\maketitle

\lecture{4}{2024-10-03}{This course makes us relativise}{
\begin{itemize}[left=0pt]
    \item Explanation of the diagonalisation argument, and application to the time hierarchy theorem.
    \item Definition of oracle Turing machine.
    \item Proof that the $\lang{P}$ versus $\lang{NP}$ problem does not relativise.
\end{itemize}

}

\subsection{Time hierarchy}

\begin{parag}{Diagonalisation}
    There exists a language that is not decided by any Turing machine.

    \begin{subparag}{Remark}
        The important thing about this theorem is the diagonalisation argument, which is the only known method to prove limitation of Turing machines.
    \end{subparag}

    \begin{subparag}{Proof}
        Order all Turing machines $M_1, M_2, \ldots$, and consider the language of Turing matrices that reject their own encoding: 
        \[\lang{DIAG} = \left\{\left\langle M \right\rangle \suchthat M\left(\left\langle M \right\rangle\right) = 0\right\}.\]

        We suppose that all Turing machines halt on all inputs, for simplicity of the proof. The argument can be generalised to any Turing machine without too much issue.

        Let us try to construct a Turing machine $N$ that decides \lang{DIAG} towards contradiction. If $M_j\left(\left\langle M_j \right\rangle\right) = 1$, then $N\left(\left\langle M_j \right\rangle\right) = 0$. On the other hand, if $M_j\left(\left\langle M_j \right\rangle\right) = 0$, then $N\left(\left\langle M_j \right\rangle\right) = 1$. In other words, considering the following table of the outputs of Turing machines $M_1, M_2, \ldots$ on inputs $\left\langle M_1 \right\rangle, \left\langle M_2 \right\rangle, \ldots$, we constructed a Turing machine $N$ so that, on input $\left\langle M_j \right\rangle$, it outputs a different value than $M_j$.
        \imagehere[0.6]{DiagonalisationArgument.png}

        This means that $N$ is not any element of the list. However, this is not possible since we considered all Turing machines.

        Another, more formal argument, is to suppose for contradiction that there there exists some $N$ that decides \lang{DIAG}. Since we listed all Turing machines, it must be in the list: $N = M_i$. We notice the following propositions are equivalent:
        \begin{itemize}
            \item $M_i\left(\left\langle M_i \right\rangle\right) = 1$.
            \item Since $M_i$ decides \lang{DIAG}, $\left\langle M_i \right\rangle \in \lang{DIAG}$.
            \item By definition of \lang{DIAG}, $M_i\left(\left\langle M_i \right\rangle\right) = 0$.
        \end{itemize}

        This is however a direct contradiction. 

        \qed
    \end{subparag}
\end{parag}

\begin{parag}{Definition: Universal Turing machine}
    There exists a Turing Machine $U$ such that, for any Turing machine $M$ and input $x$: 
    \[U\left(\left\langle M, x \right\rangle\right) = M\left(x\right).\]

    It moreover runs in $c_M \cdot t_M\left(n\right)^2$ steps for some constant $c_M$ depending on $M$.

    \begin{subparag}{Intuition}
        In other words, $U$ is capable of simulating any Turing machine.
    \end{subparag}

    \begin{subparag}{Remark}
        The runtime can be optimised. The best known yet is $c_M \cdot t_M\left(n\right) \log\left(t_M\left(n\right)\right)$.
    \end{subparag}
\end{parag}

\begin{parag}{Time-hierarchy theorem}
    Let $f, g: \mathbb{N} \mapsto \mathbb{N}$ be two functions.

    If $f\left(n\right) \log\left(f\left(n\right)\right) \in o\left(g\left(n\right)\right)$, then:
    \[\lang{TIME}\left(f\left(n\right)\right) \subsetneq \lang{TIME}\left(g\left(n\right)\right).\]

    \begin{subparag}{Intuition}
        Intuitively, if $g$ grows sufficiently faster than $f$, then there are some problems that are $O\left(g\left(n\right)\right)$ but not $O\left(f\left(n\right)\right)$.
    \end{subparag}

    \begin{subparag}{Remark}
        It is possible to show a similar hierarchy for \lang{NTIME}.
    \end{subparag}

    \begin{subparag}{Proof}
        We do the proof for the special case $\lang{TIME}\left(n\right) \subsetneq \lang{TIME}\left(n^3\right)$, the general case is completely analogous. We make a diagonalisation argument, similar to the one that shows \lang{DIAG} is undecidable.

        The idea of this proof is that knowing whether a Turing machine halts with some specific output in a cubic number of step cannot be solved with a linear number of steps.

        We consider the following language: 
        \[D = \left\{\left\langle M, 1^k \right\rangle \suchthat U\left(M, \left\langle M, 1^k \right\rangle\right) = 0 \text{ in $k^3$ steps}\right\},\]
        where $U$ is a universal Turing machine and $1^k$ is $k$ written in unary, i.e. a $1$ repeated $k$ times.

        Note that we only need the universal Turing machine to be able to give a constraint on the number of steps. We cannot say ``run $M$ for $k^3$ steps'': there needs to be a simulation that we can stop, and adding it incurs some slowdown.

        By definition of $D$, $D \in \lang{TIME}\left(n^3\right)$: we can always just run $U$ on the input, which will always take at most $k^3 \in O\left(n^3\right)$ time.

        Now, let's suppose towards contradiction that $D \in \lang{TIME}\left(n\right)$, i.e. that there exists some $M$ that decides $D$ in time $t_M\left(n\right) = c\cdot n$ for some constant $c$. We notice that the following propositions are equivalent:
        \begin{itemize}
            \item $M\left(\left\langle M, 1^k \right\rangle\right) = 1$.
            \item $\left\langle M, 1^k \right\rangle \in D$, since $M$ decides $D$.
            \item $U\left(M, \left\langle M, 1^k \right\rangle\right) = 0$ in $k^3$ steps, by definition of $D$.
            \item Since $M$ has a linear runtime and $U$ only introduces a quadratic penalty, on input $M, \left\langle M, 1^k \right\rangle$, $U$ finishes in time: 
            \[c_m t_M\left(n\right)^2 = c_m c^2 \left(\left|M\right| + k\right)^2 \in \Theta\left(k^2\right).\]
            
            However, this yields that, for a sufficiently large $k$: 
            \[M\left(\left\langle M, 1^k \right\rangle\right) = U\left(M, \left\langle M, 1^k \right\rangle\right) = 0.\]
        \end{itemize}

        The first and last proposition are in direct contradiction.

        \qed
    \end{subparag}
\end{parag}

\subsection{Oracle Turing machine}

\begin{parag}{Definition: Oracle Turing machine}
    Let $A \subseteq \left\{0, 1\right\}^*$ be an arbitrary language (potentially even undecidable). 

    An \important{oracle Turing machine} (OTM) $M^A$ is a Turing machine with a new instruction: in a single step, it can answer queries of the form $x \over{\in}{?} A$ for an arbitrary $x \in \left\{0, 1\right\}^*$.

    \begin{subparag}{Intuition}
        The idea is that we set some problem $A$ to be easy, and we want to see what impact it has on the complexity world.
    \end{subparag}
\end{parag}

\begin{parag}{Definition: Universal OTM}
    There exists a universal oracle Turing machine $U^{\bullet}$ so that, for all language $A$: 
    \[U^A\left(\left\langle M^{\bullet}, x \right\rangle\right) = M^A\left(x\right).\]
    
    It moreover runs in  $c_M t_M\left(n\right)^2$ steps.

    \begin{subparag}{Intuition}
        In other words, this oracle Turing machine can simulate any oracle Turing machine, with any oracle.
    \end{subparag}
\end{parag}

\begin{parag}{Definition: Oracle complexity class}
    The $\lang{P}^A$ complexity class is defined by: 
    \[\lang{P}^A = \left\{L \suchthat L \text{ is decided by a poly-time oracle Turing machine $M^A$}\right\}.\]

    Since \lang{SAT} is \lang{NP}-complete, we moreover note $\lang{P}^{\lang{NP}} = \lang{P}^{\lang{SAT}}$.
    
    The $\lang{NP}^A$ complexity class is defined by: 
    \[\lang{NP}^A = \left\{L \suchthat L \text{ is decided by a poly-time oracle NTM $M^A$}\right\}.\]

    \begin{subparag}{Examples}
        For instance, we trivially have: 
        \[\lang{P}^\o = \lang{P}.\]

        Similarly, for any $A \in \lang{P}$: 
        \[\lang{P}^A = \lang{P}.\]

        More interestingly, if we look at problems in \lang{NP}: 
        \[\lang{NP} \cup \lang{coNP} \subseteq \lang{P}^{\lang{NP}}.\]

        It is unknown if they are equal or not. This will be formalised with the polynomial hierarchy.
    \end{subparag}
\end{parag}

\begin{parag}{Theorem: Turing reduction}
    Let $A, B$ be two languages.

    If $A \leq_P B$, then $A \in \lang{P}^B$.

    \begin{subparag}{Remark}
        Sometimes, we write $A \in \lang{P}^B$ as $A \leq_p^T B$, and we call this polytime Turing reduction, or Cook reduction.

        Intuitively, the reductions we have seen only allow a single call to the oracle: we must convert the problem $A$ to the problem $B$. On the other hand, if $A \in \lang{P}^{B}$, we can use a polynomial number of calls to the oral for $B$ in the reduction.
    \end{subparag}

    \begin{subparag}{Proof}
        This proof is considered trivial and left as an exercise to the reader. \wink
    \end{subparag}
\end{parag}

\begin{parag}{Definition: \lang{EXP}}
    The \lang{EXP} complexity class is defined by: 
    \[\lang{EXP} = \bigcup_{k=1}^{\infty} \lang{TIME}\left(2^k\right).\]
\end{parag}
    
\begin{parag}{Theorem: \lang{EXP}-complete}
    There exists some language which is \lang{EXP}-complete (under poly-time reductions).

    \begin{subparag}{Proof}
        This will be shown in the fourth exercise series.
    \end{subparag}
\end{parag}

\begin{parag}{Theorem: \lang{P} vs \lang{NP} does not relativise}
    The \lang{P} versus \lang{NP} problem does not relativise. In other words, There exists languages $A$ and $B$ such that: 
    \[\lang{P}^A = \lang{NP}^A, \mathspace \lang{P}^B \neq \lang{NP}^B.\]

    \begin{subparag}{Personal remark}
        I use the terminology ``does not relativise'' I found in the reference book \textit{(Computational complexity: a modern approach}, Sanjeev Arora and Boaz Barak). The Professor instead used the terminology that ``\lang{P} versus \lang{NP} relativises both way''. 
    \end{subparag}

    \begin{subparag}{Intuition}
        In other words, if we have a proof that only considers Turing machines as black boxes that manage to prove that $\lang{P} = \lang{NP}$, then it must be wrong. Indeed, we could use the exact same proof by just adding oracles (that are simply one more instruction for Turing machines), which would imply $\lang{P}^B = \lang{NP}^B$ contradicting this theorem.

        Diagonalisation is a proof technique that considers those Turing machines as black boxes. It can therefore not be used to solve the $\lang{P}$ versus $\lang{NP}$ problem. However, this is the only proof technique we know that can separate complexity classes.
    \end{subparag}

    \begin{subparag}{Proof $=$}
        Let $A$ be an \lang{EXP}-complete problem.

        Our goal is to show that: 
        \[\lang{EXP} \subseteq \lang{P}^A \subseteq \lang{NP}^A \subseteq \lang{EXP}.\]

        The fact that $\lang{EXP} \subseteq \lang{P}^A$ is easy: any language $L \in \lang{EXP}$ can be solved by a single query to the oracle. We thus want to show $\lang{NP}^A \subseteq \lang{EXP}$. For any $L \in \lang{NP}^A$, we can try all exponential number of possibilities for certificate in exponential time, and solve all query access to $A$ in exponential time as well, showing $L \in \lang{EXP}$.

        The chain of inclusions therefore necessarily imply that $\lang{P}^A = \lang{NP}^A$.
    \end{subparag}

    \begin{subparag}{Proof $\neq$}
        For any given $B$, we can consider the following language: 
        \[L_B = \left\{1^n \suchthat \exists x \in \left\{0, 1\right\}^n, x \in B\right\}.\]

        We notice that, still for any $B$, we have $L_B \in \lang{NP}^B$. Indeed, we can directly use $x$ as the witness, and then use the oracle to know whether $x \in B$. Our goal will therefore be to construct $B$ in such a way that $L_B \not \in \lang{P}^B$.

        Let $M_1^B, M_2^B, \ldots$ be an ordering of all poly-time Turing machines. We will consider each machine $M_i^B$, one at a time, and update $B$ in such a way that $M_i^B$ fails to decide $L_B$ on some input $1^n$. This will indeed give our result. 

        So, let $M_i^B$ be an arbitrary machine. We consider a $n$ large enough that is not used for any machine before in the list. Note that, since $M_i^B$ is poly-time, it can only make a polynomial number of queries to its oracle $B$ on the input $1^n$. We choose $B \cap \left\{0, 1\right\}^n$ so that all those queries do not belong to $B$. Since there are exponentially many values in $\left\{0, 1\right\}^n$, this moreover necessarily mean that $M_i^B$ could not have considered all the values, and we can still choose to include them in $B$ or not. We do this adversarially, just to make $M_i^B$ wrong: if it outputs yes, we include all of them, otherwise we include none of them (i.e. leaving $B \cap \left\{0, 1\right\}^n = \o$).

        Since we consider a different $n$ for each machine, this yields that $M_i^B$ cannot decide $L_B$, for all $i$. But then, there does not exist any poly-time oracle decider for $L_B$, showing $L_B \not \in \lang{P}$.

        \qed
    \end{subparag}
\end{parag}

\begin{parag}{Theorem: Time hierarchy relativises}
    The $\lang{TIME}\left(n\right)$ vs $\lang{TIME}\left(n^3\right)$ relativises. In other words, for all language $A$, we have: 
    \[\lang{TIME}^A\left(n\right) \subsetneq \lang{TIME}^A\left(n^3\right).\]
    
    \begin{subparag}{Proof}
        The argument we used to prove $\lang{TIME}\left(n\right) \subsetneq \lang{TIME}\left(n^3\right)$ was solely based on diagonalisation, which is a relativising argument. Therefore, this result must also relativise.

        In other words, we could use the exact same proof we used for non-oracle Turing machines but with oracle Turing machines, it would still work.

        \qed
    \end{subparag}
\end{parag}

\end{document}
