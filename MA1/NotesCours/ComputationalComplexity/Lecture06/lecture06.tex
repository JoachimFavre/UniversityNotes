% !TeX program = lualatex
% Using VimTeX, you need to reload the plugin (\lx) after having saved the document in order to use LuaLaTeX (thanks to the line above)

\documentclass[a4paper]{article}

% Expanded on 2024-10-17 at 15:17:51.

\usepackage{../../style}

\title{Compcomp}
\author{Joachim Favre}
\date{Jeudi 17 octobre 2024}

\begin{document}
\maketitle

\lecture{6}{2024-10-17}{Pretty diagram that takes way too much time}{
\begin{itemize}[left=0pt]
    \item Definition of the polynomial hierarchy complexity classes.
    \item Proof of inclusions of those complexity classes.
    \item Proof that if a level of the polynomial hierarchy collapses, then the whole polynomial hierarchy collapses.
    \item Proof of the Karp-Lipton theorem.
\end{itemize}

}

\subsection{Polynomial hierarchy}

\begin{parag}{Observation}
    So far, we have defined: 
    \[A \in \lang{NP}: x \in A \iff \exists y, M\left(x, y\right) = 1,\]
    \[B \in \lang{PSPACE}: x \in B \iff Q_1 y_1, Q_2 y_2, \ldots, Q_m y_m, M\left(x, y\right) = 1,\]
    where $M$ are poly-time Turing machines and $m = n^{O\left(1\right)}$.

    For \lang{NP}, we have a single quantifier $\exists$. For \lang{PSPACE}, we have polynomially many. It is thus interesting to fill in the gaps by introducing some hierarchy.
\end{parag}

\begin{parag}{Definition: $\Sigma_i$\lang{P} complexity class}
    We define the \important{$\Sigma_i \lang{P}$} complexity class so that $L \in \Sigma_i \lang{P}$ if and only if there exists a poly-time Turing machine $M\left(x, y\right)$ so that:
    \[x \in L \iff \exists y_1, \forall y_2, \exists y_3, \ldots, Q y_i, M\left(x, y\right) = 1,\]
    where $Q = \exists$ if $i$ is odd and $Q = \forall$ if $i$ is even, and where all $y_i$ have polynomial length in $\left|x\right|$.

    \begin{subparag}{Remark}
        Since we allow the $y_i$ to have polynomial size, note that we can always merge consecutive $\exists$ or consecutive $\forall$ quantifiers: 
        \[\exists a, \exists b \equiv \exists \left(a, b\right) \equiv \exists c, \mathspace \forall a, \forall b \equiv \forall \left(a, b\right) \equiv \forall c.\]

        So, the $i$ defines the number of transitions from one quantifier to another we are allowed to make, not the number of quantifiers.
    \end{subparag}

    \begin{subparag}{Mnemonic}
        $\Sigma$ looks a bit like the $\exists$ quantifier, so $\Sigma \lang{P}$ starts with an $\exists$ quantifier.
    \end{subparag}
\end{parag}

\begin{parag}{Definition: $\Pi_i \lang{P}$ complexity class}
    We define the \important{$\Pi_i\lang{P}$} complexity class so that:
    \[\Pi_i \lang{P} = \lang{co}\Sigma_i\lang{P}\]

    In other words, $L \in \Pi_{i} \lang{P}$ if and only if: 
    \[\forall y_i, \exists y_2, \forall y_3, \ldots, Q y_i, M\left(x, y\right) = 1,\]
    where $Q = \forall$ if $i$ is odd and $Q = \exists$ if $i$ is even, and where all $y_i$ have polynomial length in $\left|x\right|$.
\end{parag}

\begin{parag}{Theorem}
    We have: 
    \[\Sigma_i \lang{P} \subseteq \Sigma_{i+1} \lang{P}, \mathspace \Sigma_i \lang{P} \subseteq \Pi_{i+1} \lang{P}.\]

    \begin{subparag}{Proof}
        The first inclusion comes directly from the definition. The second one comes by simply adding an unused $\forall$ quantifier at the beginning of the chain.

        \qed
    \end{subparag}
\end{parag}

\begin{parag}{Proposition}
    We have: 
    \[\lang{P} = \Sigma_0 \lang{P} = \Pi_0 \lang{P},\] 
    \[\Sigma_1 \lang{P} = \lang{NP}, \mathspace \Pi_1 \lang{P} = \lang{coNP}\]
\end{parag}

\begin{parag}{Definition: Polynomial hierarchy}
    We define the \important{polynomial hierarchy} complexity class, written \lang{PH}, as: 
    \[\lang{PH} = \bigcup_{i=0}^{\infty} \Sigma_i \lang{P} = \bigcup_{i=0}^{\infty}  \Pi_i \lang{P}.\]

    \begin{subparag}{Remark}
        Since $\lang{PH}$ always has a constant number of quantifiers but \lang{PSPACE} can have polynomially many, we have: 
        \[\lang{PH} \subseteq \lang{PSPACE}.\]
    \end{subparag}
\end{parag}

\begin{parag}{Example 1}
    We consider the language of circuits for which there are no smaller circuit that compute the same function: 
    \[\lang{MinCircuit} = \left\{\left\langle C \right\rangle \suchthat \text{$C$ is a circuit of minimal size}\right\}.\]

    We notice that $\left\langle C \right\rangle \in \lang{MinCircuit}$ if and only if all circuit of smaller size compute another function, i.e. 
    \[\forall C': \left|C'\right| < \left|C\right|, \exists x: C'\left(x\right) \neq C\left(x\right) \equiv \forall C', \exists x, \left(\left|C'\right| < \left|C\right|\right) \to \left(C'\left(x\right) \neq C\left(x\right)\right).\]

    Note that this is indeed valid since $C'$ and $x$ have polynomial size, and $\left|C'\right| < \left|C\right|$ and $C'\left(x\right) \neq C\left(x\right)$ both take polynomial time to check. 

    We can thus see that $\lang{MinCircuit} \in \Pi_2 \lang{P}$. It is not believed to be lower in the hierarchy.
\end{parag}

\begin{parag}{Example 2}
    We consider the following language: 
    \[\lang{MaxIS} = \left\{\left\langle G, k \right\rangle \suchthat \text{the largest independent set in $G$ has size exactly $k$}\right\}.\]

    We showed that $\lang{MaxIS} \in \lang{P}^{\lang{NP}}$ in the fourth exercise series. Let us moreover try to find where it is in the hierarchy.

    We notice that $\left\langle G, k \right\rangle \in \lang{MaxIS}$ if and only if there exists some $S \subseteq V$ which is an independent set so that $\left|S\right| = k$, and for any $S' \subseteq V$, $\left|S'\right| > k$ implies $S'$ is not an independent set. To simplify our notations, we write $\mathbb{S}$ to be the set of independent sets in $G$ (i.e. $S \in \mathbb{S}$ means that $S$ is an independent set):
    \[\begin{split}
        & \left(\exists S \subseteq V, S \in \mathbb{S} \land \left|S\right| = k\right) \land \left(\forall S', \left|S'\right| > k \to S' \not\in \mathbb{S}\right) \\
        & \equiv \exists S \subseteq V, \forall S' \subseteq V, S \in \mathbb{S} \land \left|S\right| = k \land \left(\left|S'\right| > k \to S' \not \in \mathbb{S}\right).
    \end{split}\]

    So, $\lang{MaxIS} \in \Sigma_2 \lang{P}$. However, the $\exists$ and the $\forall$ quantifiers are independent. So, we can swap them without any issue:
    \[\begin{split}
        & \left(\exists S \subseteq V, S \in \mathbb{S} \land \left|S\right| = k\right) \land \left(\forall S', \left|S'\right| > k \to S' \not\in \mathbb{S}\right) \\
        & \equiv \forall S' \subseteq V, \exists S \subseteq V, \left(\left|S'\right| > k \to S' \not \in \mathbb{S}\right) \land S \in \mathbb{S} \land \left|S\right| = k.
    \end{split}\]

    This also shows that $\lang{MaxIS} \in \Pi_2 \lang{P}$.

    To sum up, we have seen that $\lang{MaxIS} \in \lang{P}^{\lang{NP}}$ and $\lang{MaxIS} \in \Sigma_2 \lang{P} \cap \Pi_2 \lang{P}$. This is no accident, as the following theorem shows.
\end{parag}

\begin{parag}{Definition: $\Delta_i \lang{P}$ complexity class}
    We define the \important{$\Delta_i \lang{P}$} complexity class by: 
    \[\Delta_i \lang{P} = \lang{P}^{\Sigma_i \lang{\lang{P}}}.\]

    \begin{subparag}{Remark}
        In particular: 
        \[\Delta_1 \lang{P} = \lang{P}^{\lang{NP}}.\]
    \end{subparag}
\end{parag}

\begin{parag}{Theorem}
    We have, for all $i$:
    \[\Delta_i \lang{P} \subseteq \Sigma_{i+1} \lang{P} \cap \Pi_{i+1} \lang{P}.\]

    \begin{subparag}{Remark}
        In particular, this means that: 
        \[\lang{P}^{\lang{NP}} = \Delta_1 \lang{P} \subseteq \Sigma_2 \lang{P} \cap \Pi_2 \lang{P}.\]
    \end{subparag}
   
    \begin{subparag}{Proof}
        We will show the case $i = 1$, i.e. we want to show that:
        \[\lang{P}^{\lang{NP}} \subseteq \Sigma_2 \lang{P} \cap \Pi_2 \lang{P}.\]

        The other cases can be done using a similar idea.

        Let $L \in \lang{P}^{\lang{SAT}}$ be arbitrary. Let $M^{\lang{SAT}}$ be an oracle poly-time Turing machine that decides $L$, which exists by definition. On input $x$, it will have some configuration tree where we have branching due to the answer of the \lang{SAT} oracle. In other words, $M^{\lang{SAT}}$ will start running and, at some point, it will do a query to its \lang{SAT} oracle. If the oracle answers $\phi_1 \in \lang{SAT}$, the machine will do something; if however $\phi_1 \not\in\lang{SAT}$, the machine will do something potentially different. Every time the machine queries the oracle, this creates more branching in the configuration tree.

        Now, $x$ is accepted if and only if the only valid path in this tree is an accepting path. A path with answers $a = \left(a_1, a_2, \ldots, a_k\right)$ is valid if $a_i = 0$ if and only if $\phi_i \not \in \lang{SAT}$, and $a_i = 1$ if and only if $\phi_i \in \lang{SAT}$. To simplify the notation, we consider the sets of indices where the oracle outputs ``Yes'' and the ones where it outputs ``No'': 
        \[Y\left(a\right) = \left\{i \suchthat a_i = 1\right\}, \mathspace N\left(a\right) = \left\{i \suchthat a_i = 0\right\}.\]
        
        Let $N\left(x, a\right)$ to be the Turing machine that runs $M^{\lang{SAT}}\left(x\right)$ but by answering the $i$\Th call to the oracle by $a_i$. By our reasoning, $x$ is accepted by $M^{\lang{SAT}}\left(x\right)$ if and only if $N\left(x, a\right)$ accepts and the $a$ we feed it is valid:
        \[\begin{split}
            & \exists a, \left(\forall i \in Y\left(a\right), \phi_i \in \lang{SAT}\right) \land \left(\forall j \in N\left(a\right), \phi_j \not\in\lang{SAT}\right) \land N\left(x, a\right) \\
            & \equiv \exists a, \left(\forall i \in Y\left(a\right), \exists y_i, \phi\left(y_i\right) = 1\right) \land \left(\forall j \in N\left(a\right), \forall n_j, \phi\left(n_j\right) = 0\right) \land N\left(x, a\right) \\
            & \equiv \exists a, \exists y, \forall n, \forall i \in Y\left(a\right), \forall j \in N\left(a\right), \phi_i\left(y_i\right) = 1 \land \phi_j\left(n_j\right) = 0 \land N\left(x, a\right).
        \end{split}\]

        This does show that $L \in \Sigma_2 \lang{P}$. We have thus shown that $\lang{P}^{\lang{SAT}} \subseteq \Sigma_2 \lang{P}$. Now, considering any arbitrary $L \in \lang{P}^{\lang{SAT}}$ again, we know that $\bar{L} \in \lang{P}^{\lang{SAT}}$, since we can just invert the output of a decider for $L$. So, by what we have found above, this means that: 
        \[\bar{L} \in P^{\lang{SAT}} \subseteq \Sigma_2 \lang{P} \iff L \in \lang{co} \Sigma_2 \lang{P} = \Pi_2 \lang{P}.\]
        
        We have thus shown that $\lang{P}^{\lang{SAT}} \subseteq \Pi_2 \lang{P}$ as well, finishing the proof.

        \qed
    \end{subparag}
\end{parag}

\begin{parag}{Diagram}
    We have found the following chain of inclusions for the polynomial hierarchy:
    \svghere{PolynomialHierarchyInclusions.svg}

    It is a popular hypothesis that \lang{PH} is infinite, i.e., for all $i$:
    \[\lang{PH} \neq \Sigma_i \lang{P}.\]

    This is formalised thanks to the following proposition.
\end{parag}

\begin{parag}{Theorem}
    Let $i$ be arbitrary. If $\Sigma_i \lang{P} = \Pi_i \lang{P}$, then:
    \[\lang{PH} = \Sigma_i \lang{P}.\]

    In other words, if there exists a $i$ such that $\Sigma_i \lang{P} = \Pi_i \lang{P}$, then \lang{PH} collapses.

    \begin{subparag}{Implication}
        Note that, if $\lang{P} = \lang{NP}$, then $\lang{NP} = \lang{coNP}$; as shown in the first exercise series. So, if $\lang{P} = \lang{NP}$, then the polynomial hierarchy collapses. This does not seem intuitively to be true.
    \end{subparag}
    
    \begin{subparag}{Remark}
        As mentioned earlier, we believe unlikely that there exists a $i$ such that $\Sigma_i \lang{P} = \Pi_i \lang{P}$.
    \end{subparag}

    \begin{subparag}{Proof}
        We make the proof for $i = 1$, other cases are similar. We therefore suppose by hypothesis that $\lang{NP} = \lang{coNP}$.

        Let $L \in \lang{PH}$ be arbitrary. By definition of $L \in \lang{PH}$, there exists a $k$ such that $L \in \Sigma_k \lang{P}$. We can consider $k$ to be odd, the case where it is even is completely symmetrical. By definition of $\Sigma_k \lang{P}$, we have: 
        \[x \in L \iff \exists y_1, \forall y_2, \ldots, \forall y_{k-1}, \exists y_k, M\left(x, y\right) = 1.\]

        Intuitively, we use our hypothesis to switch the last quantifier (from $\exists$ to $\forall$ here). This will then allow us to merge those quantifiers, which we can do iteratively.

        More formally, let us consider the following language:
        \[L' = \left\{\left(x, y_1, \ldots, y_{k-1}\right) \suchthat \exists y_k, M\left(x, y\right) = 1\right\}.\]

        We notice that $L' \in \lang{NP}$ by definition. However, since $\lang{NP} = \lang{coNP}$ by hypothesis, this means $L' \in \lang{coNP}$ as well. In other words, there is some other Turing machine $M'$ allowing us to express $L'$ as:
        \[L' = \left\{\left(x, y_1, \ldots, y_{k-1}\right) \suchthat \forall y_k, M'\left(x, y\right) = 1\right\}.\]

        This therefore tells us that we can switch the last quantifier, using $M'$ instead of $M$:
        \[x \in L \iff \exists y_1, \forall y_2, \ldots, \forall y_{k-1}, \forall y_k, M'\left(x, y\right) = 1.\]

        However, we can merge the two last quantifiers: $\forall y_{k-1}, \forall y_k \equiv \forall y_{k-1}'$, showing $L \in \Sigma_{k-1} \lang{P}$. We can iterate this argument, until $L \in \lang{NP}$.

        \qed
    \end{subparag}
\end{parag}

\begin{parag}{Theorem}
    There exists an oracle $A$ such that $\lang{PH}^{A}$ is infinite.

    \begin{subparag}{Implication}
        Just like we saw before, this means that no diagonalisation argument can collapse the polynomial hierarchy.
    \end{subparag}
\end{parag}

\begin{parag}{Theorem}
    If \lang{PH} is infinite, \lang{PH} has no complete problem.
    
    \begin{subparag}{Proof}
        Intuitively, if there was some complete problem, it would have to lie in some level. But then, the whole hierarchy would collapse to this level.

        This theorem will be proven formally in the sixth exercise series.
    \end{subparag}
\end{parag}

\begin{parag}{Karp-Lipton theorem}
    We have that: 
    \[\lang{NP} \subseteq \lang{P/poly} \implies \Sigma_2 \lang{P} = \Pi_2 \lang{P}.\]
    
    \begin{subparag}{Implication}
        Since it is believed $\lang{PH}$ is infinite, so this would mean that $\lang{NP} \not\subseteq \lang{P/poly}$. This is a very interesting result, since $\lang{P/poly}$ is a very powerful complexity class.
    \end{subparag}

    \begin{subparag}{Proof}
        We suppose by hypothesis that $\lang{NP} \subseteq \lang{P / poly}$, i.e. $\lang{SAT} \in \lang{P / poly}$. We start by analysing what this hypothesis means. By definition of $\lang{P/poly}$, there exists polynomial-size circuits $\left\{C_n\right\}_{n \in \mathbb{N}}$ such that: 
        \[\phi \in \lang{SAT} \iff C_n\left(\phi\right) = 1.\]

        As we have seen in the first exercise series, we can use iteratively the knowledge whether a CNF is satisfiable to get a satisfying assignment. So, we can actually ask that the polynomial-size circuit are such that: 
        \[\phi \in \lang{SAT} \iff C_n\left(\phi\right) \text{ outputs a satisfying assignment to $\phi$}.\]
        

        Since $C_n\left(\phi\right)$ is a satisfiable assignment to $\phi$, this yields: 
        \[\phi \in \lang{SAT} \iff \phi\left(C_n\left(\phi\right)\right) = 1.\]

        Intuitively, this means that, to show a CNF is satisfiable, we no longer need a satisfying assignment as a certificate; we need a circuit. The big advantage is that this circuit is independent of the CNF, unlike the certificate which could be different for each CNF. This is the big idea of this proof.

        We now want to show $\Sigma_2 \lang{P} = \Pi_2 \lang{P}$. To do so, we start by showing $\Pi_2 \lang{P} \subseteq \Sigma_2 \lang{P}$. So, let $L \in \Pi_2 \lang{P}$ be arbitrary. We know by definition that there exists some Turing machine $M\left(x, y, z\right)$ so that: 
        \[x \in L \iff \forall y, \exists z, M\left(x, y, z\right) = 1.\]

        We know we can convert $M\left(x, y, z\right)$ to a CNF $\phi\left(x, y, z\right)$ using Cook-Levin constructions. Then, for any $x, y$, we consider $\phi_{x, y}\left(z\right) = \phi\left(x, y, z\right)$ to be a boolean function with hard-wired $x$ and $y$, but parameter $z$. We can put everything together to find that $x \in L$ if and only if:
        \autoeq[s]{\forall y, \exists z, \left(M\left(x, y, z\right) = 1\right) \equiv \forall y, \exists z, \left(\phi_{x, y}\left(z\right) = 1\right) \equiv \forall y, \exists C_n, \left(\phi_{x, y}\left(C_n\left(\phi_{x, y}\right)\right) = 1\right) \equiv \exists C_n, \forall y, \left(\phi_{x, y}\left(C_n\left(\phi_{x, y}\right)\right) = 1\right),}
        where we were allowed to commute $\forall y$ and $\exists C_n$ because they are independent.

        We have thus shown that $L \in \Sigma_2\lang{P}$, i.e. $\Pi_2 \lang{P} \subseteq \Sigma_2 \lang{P}$. Now, to show that $\Sigma_2 \lang{P} \subseteq \Pi_2 \lang{P}$, let $L \in \Sigma_2 \lang{P}$ be arbitrary. We know by definition and what we just found that: 
        \[\bar{L} \in \Pi_2 \lang{P} \subseteq \Sigma_2 \lang{P} \implies \bar{L} \in \Sigma_2 \lang{P} \implies L \in \Pi_2 \lang{P},\]
        finishing the proof.
        
        \qed
    \end{subparag}
\end{parag}

\end{document}
