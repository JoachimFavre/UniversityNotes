\documentclass[a4paper]{article}

% Expanded on 2021-11-08 at 13:10:18.

\usepackage{../../style}

\title{Algèbre linéaire}
\author{Joachim Favre}
\date{Lundi 08 novembre 2021}

\begin{document}
\maketitle

\lecture{14}{2021-11-08}{Changements de base et applications linéaires}{
    \begin{itemize}[left=0pt]
        \item Explication des 4 sous-espaces intéressants de toute matrice.
        \item Extension du ``grand théorème'' (celui avec (maintenant) 18 proposition équivalentes pour dire qu'une matrice est inversible).
        \item Explication de la méthode pour faire un changement de base, en passant par la base canonique, ou avec une matrice de passage quelconque qui amène d'une base $\mathcal{B}$ à une base $\mathcal{C}$.
        \item Extension des applications linéaires à n'importe quel espace vectoriel.
    \end{itemize}

}

\parag{Sous-espaces intéressants}{
    On commence à réaliser qu'il y a \important{quatre sous-espaces} intéressants pour une matrice $A \in \mathbb{R}^{m \times n}$:
    \begin{itemize}
        \item $\im A^T$ et $\ker A$, sous-espaces de $\mathbb{R}^{n}$, et tels que:
            \[\dim\im A^T + \dim\ker A = n\]

        \item $\im A$ et $\ker A^T$, sous-espaces de $\mathbb{R}^{m}$, et tels que:
            \[\dim\im A + \dim\ker A^T = m\]
    \end{itemize}

    Pour les équations ci-dessus, on utilise le théorème du rang, qui nous dit $\rang A + \dim\ker A = n$, avec $\rang A = \dim \im A = \dim \im A^T = \rang A^T$.

    On reviendra là-dessus plus tard.
}

\parag{Vrai ou faux}{
    On a déterminé que les opérations élémentaires sur les lignes n'affectent pas l'espace des lignes:
    \[\lgn A = \lgn M = \lgn\left(EA\right)\]
    où $M$ est une forme échelonnée de $A$, $M = EA$.

    Nous nous demandons donc si les affirmations suivantes sont vraies ou fausses:
    \begin{enumerate}
        \item $\im A = \im M$
        \item $\ker A = \ker M$
    \end{enumerate}

    \subparag{Solution au point (1)}{
        C'est faux. Si on a :
        \[A = \begin{bmatrix} 1 \\ 1 \end{bmatrix} \implies M = \begin{bmatrix} 1 \\ 0 \end{bmatrix}  \]

        Alors, clairement,
        \[\im A \neq \im M\]

    }

    \subparag{Solution au point (2)}{
        C'est vrai. On peut faire la suite d'implication (qui vont dans les deux sens) suivantes:
        \begin{multiequation}
                 & x \in \ker A \\
            \iff & A \bvec{x} = \bvec{0} \\
            \iff & EA \bvec{x} = E \bvec{0} \\
            \iff & M \bvec{x} = \bvec{0} \\
            \iff & \bvec{x} \in \ker M
        \end{multiequation}

        Il est important de noter qu'on utilise le fait que $E$ soit inversible pour affirmer que:
        \[A \bvec{x} = \bvec{0} \iff EA \bvec{x} = E \bvec{0}\]

        Sinon, on aurait bien une implication ``$\!\!{\implies}\!\!$'', mais pas un biconditionnel ``$\!\!{\iff}\!\!$''.
    }
}

\parag{Résumé}{
    Nous pouvons résumer ce que nous avons fait à l'aide des théorèmes suivants.

    \subparag{Théorème 1}{
        Si $A$ et $B$ sont deux matrices équivalentes selon les lignes, alors leur lignes engendrent le même espace. Si $B$ est échelonnée, alors les lignes non-nulles de $B$ forment une base de cet espace.
    }

    \subparag{Théorème 2}{
        Soit $A$ une matrice $m \times n$. L'espace engendré par les colonnes de $A$ et l'espace engendré par les lignes de cette matrice ont la même dimension. Cette dimension commune, le rang de $A$, est égale au nombre de positons pivots de $A$ et vérifie la relation :
        \[\rang A + \dim\ker A = n \iff \rang A + \dim\ker A^T = m\]

    }

    \subparag{Théorème 3 (suite du grand théorème)}{
        Soit $A$ une matrice $n \times n$. Les propriétés suivantes sont équivalentes :

        \begin{enumerate}[left=0pt]
            \item $A$ est inversible.
            \item $A$ est équivalente selon les lignes à la matrice unité $n \times n$, $I_n$.
            \item $A$ admet $n$ positions de pivot.
            \item L'équation $A \bvec{x} = \bvec{0}$ admet la solution triviale pour unique solution.
            \item Les colonnes de $A$ sont linéairement indépendantes.
            \item L'application $\bvec{x} \mapsto A \bvec{x}$ est injective.
            \item Pour tout vecteur $\bvec{b} \in \mathbb{R}^{n}$, l'équation $A \bvec{x} = \bvec{b}$ admet au moins une solution (on aurait aussi pu dire ``une solution unique'', ou au plus une solution).
            \item Les colonnes de $A$ engendrent $\mathbb{R}^{n}$.
            \item L'application linéaire $x \mapsto A \bvec{x}$ est surjective.
            \item Il existe une matrice $C$ de taille $n \times n$ telle que $CA = I_n$.
            \item Il existe une matrice $D$ de taille $n \times n$ telle que $AD = I_n$.
            \item $A^T$ est inversible.
            \item Les colonnes de $A$ forment une base de $\mathbb{R}^{n}$.
            \item $\im A = \mathbb{R}^{n}$
            \item $\dim\im A = n$
            \item $\rang A = n$
            \item $\ker A = \left\{\bvec{0}\right\}$
            \item $\dim\ker A = 0$
        \end{enumerate}

    }
}

\subsection{Changements de base}
\parag{Introduction}{
    Plus tôt, on a considéré la matrice de passage $P_{\mathcal{B}}$ qui permet de ``passer'' des coordonnées de $\bvec{x} \in \mathbb{R}^{n}$ dans une base $\mathcal{B} = \left(\bvec{b}_1, \ldots, \bvec{b}_n\right)$ de $\mathbb{R}^{n}$ à $\bvec{x}$ lui même, c'est-à-dire, aux coordonnées de $\bvec{x}$ dans la base canonique:
    \[\bvec{x} = P_{\mathcal{B}} \left[\bvec{x}\right]_{\mathcal{B}} \mathspace \text{avec } P_{\mathcal{B}} = \begin{bmatrix}  &  &  \\ \bvec{b}_1 & \ldots & \bvec{b}_n \\  &  &  \end{bmatrix} \]

    Nous allons généraliser ce concept de deux façons:
    \begin{enumerate}
        \item On va passe d'une base $\mathcal{B}$ à une base $\mathcal{C}$ quelconque ().
        \item On va travailler dans un espace (ou sous-espace) quelconque.
    \end{enumerate}
}

\parag{But}{
    Soit $V$ un espace vectoriel, et soient $\mathcal{B} = \left(\bvec{b}_1, \ldots, \bvec{b}_n\right)$ et $\mathcal{C} = \left(\bvec{c}_1, \ldots, \bvec{c}_n\right)$ deux bases de $V$ (on sait qu'elles ont le même nombre d'éléments, par le concept de dimension).

    Considérons un vecteur $\bvec{x} \in V$. Si on a les coordonnées de $\bvec{x}$ dans la base $\mathcal{B}$, notées $\left[\bvec{x}\right]_{\mathcal{B}}$, comment peut-on trouver les coordonnées de $\bvec{x}$ dans la base $\mathcal{C}$, notées $\left[\bvec{x}\right]_{\mathcal{C}}$ ?
}

\parag{Exemple}{
    \imagehere{ExplicationChangementBase.png}

    En regardant les deux images ci-dessus, on remarque que:
    \[\bvec{x} = 3\bvec{b}_1 + 1\bvec{b}_2 = 6\bvec{c}_1 + 4\bvec{c}_2\]

    En d'autres mots:
    \[\left[\bvec{x}\right]_{\mathcal{B}} = \begin{bmatrix} 3 \\ 1 \end{bmatrix} \mathspace \text{ et } \mathspace \left[\bvec{x}\right]_{\mathcal{C}} = \begin{bmatrix} 6 \\ 4 \end{bmatrix} \]

    Le point $\bvec{x}$ est le même dans les deux images, c'est la manière de le décrire qui est différente.

    On veut trouver ce qui lie $\left[\bvec{x}\right]_{\mathcal{B}}$ et $\left[\bvec{x}\right]_{\mathcal{C}}$. On se rend compte que:
    \[\left[\bvec{x}\right]_{\mathcal{C}} = \left[3\bvec{b}_1 + \bvec{b}_2\right]_{\mathcal{C}}\]

    Puisque l'application coordonnées est linéaire :
    \[\left[3\bvec{b}_1 + \bvec{b}_2\right]_{\mathcal{C}} = 3\left[\bvec{b}_1\right]_{\mathcal{C}} + \left[\bvec{b}_2\right]_{\mathcal{C}} = \begin{bmatrix}  &  \\ \left[\bvec{b}_1\right]_{\mathcal{C}} & \left[\bvec{b}_2\right]_{\mathcal{C}} \\  &  \end{bmatrix} \begin{bmatrix} 3 \\ 1 \end{bmatrix} =  \begin{bmatrix}  &  \\ \left[\bvec{b}_1\right]_{\mathcal{C}} & \left[\bvec{b}_2\right]_{\mathcal{C}} \\  &  \end{bmatrix} \left[\bvec{x}\right]_{\mathcal{B}}\]

    Si on reçoit (magiquement) l'information que $\bvec{b}_1 = 4\bvec{c}_1 + \bvec{c}_2$ et $\bvec{b}_2 = -6\bvec{c}_1 + \bvec{c}_2$, alors:
    \[\left[\bvec{b}_1\right]_{\mathcal{C}} = \begin{bmatrix} 4 \\ 1 \end{bmatrix} \mathspace \text{ et } \mathspace \left[\bvec{b}_2\right]_{\mathcal{C}} = \begin{bmatrix} -6 \\ 1 \end{bmatrix} \]

    On peut donc affirmer pour tout $\bvec{x} \in V$ que:
    \[\left[\bvec{x}\right]_{\mathcal{C}} = \begin{bmatrix} 4 & -6 \\ 1 & 1 \end{bmatrix} \left[\bvec{x}\right]_{\mathcal{B}}\]

    En particulier, avec $\bvec{x}$ tel que
    \[\left[\bvec{x}\right]_{\mathcal{B}} = \begin{bmatrix} 3 \\ 1 \end{bmatrix} \]

    Alors, on obtient:
    \[\left[\bvec{x}\right]_{\mathcal{C}} = \begin{bmatrix} 4 & -6 \\ 1 & 1 \end{bmatrix} \begin{bmatrix} 3 \\ 1 \end{bmatrix} = 3\begin{bmatrix} 4 \\ 1 \end{bmatrix} + \begin{bmatrix} -6 \\ 1 \end{bmatrix} = \begin{bmatrix} 6 \\ 4 \end{bmatrix}\]

    Ce qui est bien ce qu'on avait trouvé graphiquement.
}

\parag{Généralisation}{
    De manière générale, soient $\mathcal{B} = \left(\bvec{b}_1, \ldots, \bvec{b}_n\right)$ et $\mathcal{C} = \left(\bvec{c}_1, \ldots, \bvec{c}_n\right)$ deux bases d'un espace vectoriel $V$. Étant donné un vecteur $\bvec{x} \in V$, soit
    \[\left[\bvec{x}\right]_{\mathcal{B}} = \begin{bmatrix} a_1 \\ \vdots \\ a_n \end{bmatrix} \]
    son vecteur de coordonnées dans la base $\mathcal{B}$.

    Autrement dit, $\bvec{x} = a_1 \bvec{b}_1 + \ldots + a_n  \bvec{b}_n$. Nous voulons maintenant trouver $\left[\bvec{x}\right]_{\mathcal{C}}$. Pour faire cela, on peut utiliser la linéarité de l'application coordonnées :
    \[\left[\bvec{x}\right]_{\mathcal{C}} = \left[a_1 \bvec{b}_1 + \ldots + a_n \bvec{b}_n\right]_{\mathcal{C}} = a_1 \left[\bvec{b}_1\right]_{\mathcal{C}} = a\left[\bvec{b}_1\right]_{\mathcal{C}} + \ldots + a_n\left[\bvec{b}_n\right]_{\mathcal{C}}\]

    Qu'on peut transformer en produit matrice-vecteur:
    \[\left[\bvec{x}\right]_{\mathcal{C}} = \begin{bmatrix}  &  &  \\ \left[\bvec{b}_1\right]_{\mathcal{C}} & \ldots & \left[\bvec{b}_n\right]_{\mathcal{C}} \\  &  &  \end{bmatrix} \begin{bmatrix} a_1 \\ \vdots \\ a_n \end{bmatrix} = \begin{bmatrix}  &  &  \\ \left[\bvec{b}_1\right]_{\mathcal{C}} & \ldots & \left[\bvec{b}_n\right]_{\mathcal{C}} \\  &  &  \end{bmatrix} \left[\bvec{x}\right]_{\mathcal{B}}\]


    Cette matrice permet de passer des coordonnées $\mathcal{B}$ vers $\mathcal{C}$. Elle est de taille $n \times n$ et on la note $\matricepassage{C}{B}$. On l'appelle la matrice de passage, ou la matrice de changement de base de $\mathcal{B}$ à $\mathcal{C}$. Cette matrice est unique car on a construit cette matrice, et à aucun moment on n'a dû faire de choix.

    On remarque que notre résultat peut être appliqué pour n'importe quel espace vectoriel, et non pas uniquement $\mathbb{R}^2$.
}

\parag{Théorème}{
    Soit $\mathcal{B} = \left(\bvec{b}_1, \ldots, \bvec{b}_n\right)$ et $\mathcal{C} = \left(\bvec{c}_1, \ldots, \bvec{c}_n\right)$ deux bases d'un espace vectoriel $V$. Il existe une matrice $n \times n$ unique, notée $\matricepassage{C}{B}$, telle que pour tout vecteur $\bvec{x} \in V$:
    \[\left[\bvec{x}\right]_{\mathcal{C}} = \matricepassage{C}{B}\left[\bvec{x}\right]_{\mathcal{B}}\]

    Les colonnes de $\matricepassage{C}{B}$ sont les colonnes de composantes, dans la base $\mathcal{C}$, des vecteurs de la base $\mathcal{B}$. Autrement dit:
    \[\matricepassage{C}{B} = \begin{bmatrix}  &  &  \\ \left[\bvec{b}_1\right]_{\mathcal{C}} & \ldots & \left[\bvec{b}_n\right]_{\mathcal{C}} \\  &  &  \end{bmatrix} \]
}

\parag{Inverse de la matrice de passage}{
    La matrice suivante passe de $\mathcal{B}$ à $\mathcal{C}$:
    \[\matricepassage{C}{B} = \begin{bmatrix}  &  &  \\ \left[\bvec{b}_1\right]_{\mathcal{C}} & \ldots & \left[\bvec{b}_n\right]_{\mathcal{C}} \\  &  &  \end{bmatrix} \]

    De plus, la matrice suivante passe de $\mathcal{C}$ à $\mathcal{B}$:
    \[\matricepassage{B}{C} = \begin{bmatrix}  &  &  \\ \left[\bvec{c}_1\right]_{\mathcal{B}} & \ldots & \left[\bvec{c}_n\right]_{\mathcal{B}} \\  &  &  \end{bmatrix} \]

    On remarque que leur produit donne la matrice identité, puisque cela correspond à transformer les coordonnées de $\bvec{x}$ de la base $\mathcal{B}$ vers la base $\mathcal{C}$, puis de la matrice $\mathcal{C}$ vers la base $\mathcal{B}$ ; donc, on n'a pas bougé:
    \[\matricepassage{B}{C}\matricepassage{C}{B} = I_n\]

    On peut donc en déduire que:
    \[\left(\matricepassage{C}{B}\right)^{-1} = \matricepassage{B}{C} \iff \left(\matricepassage{B}{C}\right)^{-1} = \matricepassage{C}{B} \]
}

\parag{Exemple}{
    Plus tôt, on avait un exemple dans $V = \mathbb{R}^2$ avec :
    \[\bvec{b}_1 = 4\bvec{c}_1 + \bvec{c}_2 \mathspace \text{ et } \bvec{b}_2 = -6 \bvec{c}_1 + \bvec{c}_2\]

    Donc:
    \[\left[\bvec{b}_1\right]_{\mathcal{C}} = \begin{bmatrix} 4 \\ 1 \end{bmatrix} \mathspace \text{ et } \mathspace \left[\bvec{b}_2\right]_{\mathcal{C}} = \begin{bmatrix} -6 \\ 1 \end{bmatrix} \]

    Ainsi, on en déduit:
    \[\left[\bvec{x}\right]_{\mathcal{C}} = \begin{bmatrix} 4 & -6 \\ 1 & 1 \end{bmatrix} \left[\bvec{x}\right]_{\mathcal{B}} = \matricepassage{C}{B} \left[\bvec{x}\right]_{\mathcal{B}} \]

    On peut alors calculer:
    \[\begin{bmatrix}  &  \\ \left[\bvec{c}_1\right]_{\mathcal{B}} & \left[\bvec{c}_2\right]_{\mathcal{B}} \\  &  \end{bmatrix} = \matricepassage{B}{C} = \left(\matricepassage{C}{B}\right)^{-1} = \begin{bmatrix} 4 & -6 \\ 1 & 1 \end{bmatrix}^{-1} = \begin{bmatrix} 1/10 & 3 / 5 \\ -1 / 10 & 2 /  5 \end{bmatrix}  \]

    On en déduit donc:
    \[\left[\bvec{c}_1\right]_{\mathcal{B}} = \begin{bmatrix} 1 / 10 \\ -1 / 10 \end{bmatrix} \mathspace \text{ et } \mathspace \left[\bvec{c}_2\right]_{\mathcal{B}} = \begin{bmatrix} 3 / 5 \\ 1 / 5 \end{bmatrix} \]

    De plus:
    \[\bvec{c}_1 = \frac{1}{10}\bvec{b}_1 - \frac{1}{10}\bvec{b}_2 \mathspace \text{ et } \mathspace \bvec{c}_2 = \frac{3}{5}\bvec{b}_1 + \frac{2}{5}\bvec{b}_2\]
}

\subsection{Applications linéaires d'espaces quelconques}
\parag{Retour aux applications}{
    Nous voulons exprimer une application linéaire par rapport à deux bases.

    Soit $T : V \mapsto W$ une application linéaire d'un espace vectoriel $V$ vers un espace vectoriel $W$. Dans le cas particulier $T : \mathbb{R}^n \mapsto \mathbb{R}^m$, on a vu comment exprimer $T$ au moyen d'une matrice $A \in \mathbb{R}^{m \times n}$, $T\left(\bvec{x}\right) = A \bvec{x}$. On peut faire quelque chose de similaire dans des espaces abstraits si on a une base de $V$ et une base de $W$.

    Soit $\mathcal{V} = \left(\bvec{v}_1, \ldots, \bvec{v}_n\right)$ une base de $V$ (donc $\dim V = n$), et soit $\mathcal{W} = \left(\bvec{w}_1, \ldots, \bvec{w}_m\right)$ (donc $\dim W = m$).

    Tout $\bvec{x} \in V$ s'écrit de manière unique dans la base $\mathcal{V}$:
    \[\bvec{x} = c_1 \bvec{v}_1 + \ldots + c_n \bvec{v}_n \mathspace \text{avec } \left[\bvec{x}\right]_{\mathcal{V}} = \begin{bmatrix} c_1 \\ \vdots \\ c_n \end{bmatrix} \]

    On se demande maintenant que vaut $T\left(\bvec{x}\right)$. On utilise la linéarité de $T$:
    \[T\left(\bvec{x}\right) = T\left(c_1 \bvec{v}_1 + \ldots + c_n \bvec{v}_n\right) = c_1 T\left(\bvec{v}_1\right) + \ldots + c_n  T\left(\bvec{v}_n\right)\]

    Chacun des vecteurs $T\left(\bvec{v}_1\right), \ldots, T\left(\bvec{v}_n\right)$ appartient à $W$. Donc, on peut les exprimer dans la base $\mathcal{W}$ :
    \[\left[T\left(\bvec{v}_1\right)\right]_{\mathcal{W}}, \ldots, \left[T\left(\bvec{v}_n\right)\right]_{\mathcal{W}}\]

    Ce sont $n$ vecteurs-colonnes dans $\mathbb{R}^{m}$. Par linéarité de l'application coordonnées, on trouve:
    \[\left[T\left(\bvec{x}\right)\right]_{\mathcal{W}} = \left[c_1 T\left(\bvec{v}_1\right) + \ldots + c_n T\left(\bvec{v}_n\right)\right]_{\mathcal{W}} = c_1 \left[T\left(\bvec{v}_1\right)\right]_{\mathcal{W}} + \ldots + c_n\left[T\left(\bvec{v}_n\right)\right]_{\mathcal{W}}\]

    On peut donc l'écrire sous la forme d'un produit matrice-vecteur:
    \[\begin{bmatrix}  &  &  \\ \left[T\left(\bvec{v}_1\right)\right]_{\mathcal{W}} & \ldots & \left[T\left(\bvec{v}_n\right)\right]_{\mathcal{W}} \\  &  &  \end{bmatrix} \underbrace{\begin{bmatrix} c_1 \\ \vdots \\ c_n \end{bmatrix}}_{\left[\bvec{x}\right]_{\mathcal{V}}} \]

    \subparag{Note personnelle}{
        La différence entre $V$, la matrice, et $\mathcal{V}$, la base, me fait penser à ce XKCD:

        \imagehere{XKCDPolice.png}
        
        \begin{center}
            \url{https://xkcd.com/2309/}
        \end{center}
        
    }
    
}

\parag{Théorème}{
    Soit $T: V \mapsto W$ une application linéaire et soient $\mathcal{V}$ et $\mathcal{W}$ des bases de $V$ et $W$, respectivement. Il existe une matrice unique $A$ telle que, pour tout $\bvec{x} \in V$:
    \[\left[T\left(\bvec{x}\right)\right]_{\mathcal{W}} = A \left[\bvec{x}\right]_{\mathcal{V}}\]

    Cette matrice est donnée par :
    \[A = \begin{bmatrix}  &  &  \\ \left[T\left(\bvec{v}_1\right)\right]_{\mathcal{W}} & \ldots & \left[T\left(\bvec{v}_n\right)\right]_{\mathcal{W}} \\  &  &  \end{bmatrix} \]
    où $\bvec{v}_1, \ldots, \bvec{v}_n$ sont les vecteurs de la base $\mathcal{V}$.

}

\end{document}
