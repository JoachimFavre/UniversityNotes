\documentclass[a4paper]{article}

% Expanded on 2021-12-01 at 15:14:59.

\usepackage{../../style}

\title{AICC 1}
\author{Joachim Favre}
\date{Mercredi 01 décembre 2021}

\begin{document}
\maketitle

\lecture{22}{2021-12-01}{Closed forms of recurrence relations}{
\begin{itemize}[left=0pt]
    \item Explanation on how to solve linear homogeneous recurrence relation of degree $k$ with constant coefficients, and definition of its characteristic equation.
    \item Definition of generating functions.
    \item Explanation on how to solve recurrence relations using generating functions.
\end{itemize}
   
}

\subsection{Solving recurrence relations}
\subsubsection{Linear homogeneous recurrence relations with constant coefficients}
\parag{Definition}{
    A \important{linear homogeneous recurrence relation of degree $k$ with constant coefficients} is a recurrence relation of the form 
    \[a_n = c_1 a_{n-1} + c_2 a_{n-2} + \ldots + c_k a_{n-k}\]
    where $c_1, \ldots, c_k$ are real numbers and $c_k \neq 0$.

    \subparag{Observation}{
        By strong induction, a sequence satisfying such a recurrence relation is uniquely determined by the recurrence relation and the $k$ initial conditions $a_0 = C_1, \ldots a_k = C_k$.
    }

    \subparag{Terminology}{
        \begin{itemize}[left=0pt]
            \item It is \important{linear} since the right-hand side is a sum of the previous term of the sequence each multiplied by a function of $n$.
            \item It is \important{homogeneous} because no terms occur that are not multiple of the $a_j$s (there is no constant).
            \item It has \important{constant coefficients} $c_1, \ldots, c_k$.
            \item The \important{degree} is $k$ because $a_n$ is expressed in terms of the previous $k$ terms of the sequence (the term ``degree'' will take all its sense with the characteristic equation).
        \end{itemize}
    }
    
    \subparag{Examples}{
        \begin{itemize}[left=0pt]
            \item $P_n = 1.11 P_{n-1}$ is a linear homogeneous recurrence relation of degree one.
            \item $f_n = f_{n-1} + f_{n-2}$ is a linear homogeneous recurrence relation of degree two.
            \item $a_n = a_{n-1} + a_{n-2}^2$ is not linear.
            \item $H_n = 2H_{n-1} + 1$ is not homogeneous.
            \item $B_n = nB_{n-1}$ does not have constant coefficients.
        \end{itemize}
    }
}

\parag{Characteristic equation}{
    Let's say we have the relation $a_n = c_1 a_{n_1} + \ldots + c_k a_{n-k}$. Then, we can guess that the solution is $a_n = r^{n}$, $r \in \mathbb{R}^*$, and see what happens: 
    \[r^{n} = c_1 r^{n-1} + c_2 r^{n-2} + \ldots + c_k r^{n-k} \implies r^{n} - c_1 r^{n-1} - c_2 r^{n-2} - \ldots - c_k r^{n-k} = 0\]

    Dividing by $r^{n-k}$, since it is non-zero (unless $a_n = 0$ for all $n$): 
    \[r^k - c_1 r^{k-1} - c_2 r^{k-2} - \ldots c_k = 0\]

    We call this the characteristic equation, and we can see that the sequence $\left\{r^{n}\right\}$ is a solution if and only if $r$ is a solution to the characteristic equation. 

    The solutions to the characteristic equations are called the \important{characteristics roots} of the recurrence relation. The roots can be used to give a \important{closed formula} for the recurrence relation.
}

\parag{Theorem}{
    Let $c_1$ and $c_2$ be real numbers. Moreover, let's suppose that $r^2 - c_1r - c_2$ has two \textit{distinct} roots $r_1$ and $r_2$.

    The sequence $\left\{a_n\right\}$ is a solution to the recurrence relation $a_n = c_1 a_{n-1} + c_2 a_{n-2}$ if and only if: 
    \[a_n = \alpha_1 r_1^{n} + \alpha_2 r_2^{n}, \mathspace n = 0, 1, 2, \ldots\]
    where $\alpha_1$ and $\alpha_2$ are constants.

    \subparag{Proof}{
        We want to show that $\alpha_1 r_1^n + \alpha_2 r_2^n$ is a solution of $a_n = c_1 a_{n-1} + c_2 a_{n-2}$: 
        \[\alpha_1 r_1^{n} + \alpha_2 r_2^{n} = c_1 \left(\alpha r_1^{n-1} + \alpha_2 r_2^{n-2}\right) + c_2\left(\alpha_2 r_1^{n-2} + \alpha_2 r_2^{n-2}\right)\]
        
        Putting everything on the same side and grouping things, we get something which we want to prove that it is equal to 0:
        \[\left(\alpha_1 r_1^{n} + c_1 \alpha_1 r_1^{n-1} - c_2 \alpha_1 r_1^{n-2}\right) + \left(\alpha_2 r_2^{n} - c_1 \alpha_2 r_2^{n-1} - c_2 a_2 r_2^{n-2}\right)\]
        
        Thus, factoring: 
        \[\alpha_1 r_1^{n-2} \underbrace{\left(r_1^2 - c_1 r_1 - c_2\right)}_{=0} + \alpha_2 r_2^{n-2}\underbrace{\left(r_2^2 - c_1 r_2 - c_2\right)}_{=0} = 0\]
        as required.
    }

    \subparag{Note}{
        $\alpha_1$ and $\alpha_2$ depend on the initial conditions.
    }
    
}

\parag{Example 1}{
    Let's say we want to find the solution to the recurrence relation $a_n = a_{n-1} + 2 a_{n-2}$, with $a_0 = 2$ and $a_1 = 7$.

    Let's get the characteristic equation by substituting $a_n = r^{n+2}$: 
    \[r^{n+2} = r^{n+1} + 2r^{n} \implies r^2 - r - 2 = 0 \implies \left(r + 1\right)\left(r - 2\right) = 0\]
    
    Our roots are thus $r = -1$ and $r = 2$. Therefore, we know that solutions are of the form $a_n = \alpha_1 2^n + \alpha_2 \left(-1\right)^n$. We know that $a_0 = 2$ and $a_1 = 7$, thus: 
    \begin{systemofequations}{}
    &\ 2 = a_0 = \alpha_1 + \alpha_2 \\
    &\ 7 = a_1 = \alpha_1 \cdot 2 + \alpha_2 \left(-1\right) = 2\alpha_1 - \alpha_2
    \end{systemofequations}

    This is a system of linear equations, which we can solve (using regular linear algebra methods), and which will give us: 
    \[\alpha_1 = 3, \mathspace \alpha_2 = -1\]
    
    As a result, we get our closed form: 
    \[a_n = 3\cdot 2^n  - \left(-1\right)^{n}\]
}

\parag{Example 2}{
    The sequence of Fibonacci numbers satisfies the recurrence relation $f_n = f_{n-1} + f_{n-2}$, with the initial conditions $f_0 = 0$ and $f_1 = 1$. 

    This sequence has the following characteristic equation: 
    \[r^2 - r - 1 = 0 \implies r_1 = \frac{1 + \sqrt{5}}{2} = \phi, r_2 = \frac{1 - \sqrt{5}}{2} = \bar{\phi}\]
    
    Substituting the initial conditions we get: 
    
    \begin{systemofequations}{}
    &\ 0 = f_0 = \alpha_1 + \alpha_2 \implies \alpha_1 = -\alpha_2 \\
    &\ 1 = f_1 = \alpha_1 \left(\frac{1 + \sqrt{5}}{2}\right) + \alpha_2 \left(\frac{1 - \sqrt{5}}{2}\right)
    \end{systemofequations}

    Solving this system, we get: 
    \[\alpha_1 = \frac{1}{\sqrt{5}}, \mathspace \alpha_2 = -\frac{1}{\sqrt{5}}\]

    Finally, we can put everything together: 
    \[f_n = \frac{1}{\sqrt{5}} \left(\frac{1 + \sqrt{5}}{2}\right)^n - \frac{1}{\sqrt{5}} \left(\frac{1 - \sqrt{5}}{2}\right)^n = \frac{\phi^n - \bar{\phi}^n}{\sqrt{5}}\]
}

\parag{Theorem (degree 2 with repeated roots)}{
    Let $c_1$ and $c_2$ be real numbers with $c_2 \neq 0$. Suppose that $r^2 - c_1 r - c_2 = 0$ has one repeated root $r_0$.

    The sequence $\left\{a_n\right\}$ is a solution to the recurrence relation $a_n = c_1 a_{n-1} + c_2 a_{n-2}$ if and only if: 
    \[a_n = \alpha_1 r_0^{n} + \alpha_2 n r_0^{n} \]
    where $\alpha_1$ and $\alpha_2$ are constants.
}

\parag{Example}{
    Let's say we want to find the solution to the recurrence relation $a_n = 6 a_{n-1} - 9a_{n-2}$ with $a_0 = 1$ and $a_1 = 6$.

    Its characteristic equation is given by: 
    \[r^2 - 6r + 9 = \left(r - 3\right)^2 \implies r_0 = 3\]

    Thus, our solution is of the form: 
    \[a_n = \alpha_1 3^n + \alpha_2 n 3^{n}\]
    
    Using our initial conditions, we get: 
    \begin{systemofequations}{}
    &\ 1 = a_0 = \alpha_1 + \alpha_2 \cdot 0 = \alpha_1 \\
    &\ 6 = a_1 = \alpha_1 \cdot 3 + \alpha_2 \cdot 1 \cdot 3^1 = 3\alpha_1 + 3\alpha_2
    \end{systemofequations}

    Solving it, we get: 
    \[\alpha_1 = 1, \mathspace \alpha_2 = 1\]

    Thus: 
    \[a_n = 3^n + n3^n\]
}

\parag{Theorem}{
    Let $c_1, \ldots, c_k$ be real numbers. Let's suppose that the characteristic equation $r^k - c_1 r^{k-1} - \ldots - c_k = 0$ has $k$ distinct roots $r_1, \ldots, r_k$.

    Then, a sequence $\left\{a_n\right\}$ is a solution of the recurrence relation $a_n = c_1 a_{n-1} + \ldots + c_k a_{n-k}$ if and only if $a_n$ is of the form: 
    \[a_n = \alpha_1 r_1^n + \alpha_2 r_2^n + \ldots + \alpha_k r_k^n, \mathspace n = 0, 1, 2, \ldots\]
    where $\alpha_1, \ldots, \alpha_k$ are constants.
}

\parag{Theorem}{
    Let $c_1, \ldots, c_k$ be real numbers. Let's suppose that the characteristic equation $r^k - c_1 r^{k-1} - \ldots - c_k = 0$ has $t$ distinct roots $r_1, \ldots, r_t$, with multiplicities $m_1, \ldots, m_t$ respectively.

    Then, a sequence $\left\{a_n\right\}$ is a solution of the recurrence relation $a_n = c_1 a_{n-1} + \ldots + c_k a_{n-k}$ if and only if $a_n$ is of the form:

    \begin{multiequality}
        a_n =\ & \left(\alpha_{1, 0} + a_{1,1}n + \ldots + \alpha_{1,m_1 - 1}n^{m_1 - 1}\right)r_1^n \\
        & + \left(\alpha_{2, 0} + a_{2,1}n + \ldots + \alpha_{2,m_2 - 1}n^{m_2 - 1}\right)r_2^n \\
        & + \ldots \\
        & + \left(\alpha_{t, 0} + a_{t,1}n + \ldots + \alpha_{t,m_t - 1}n^{m_t - 1}\right)r_t^n
    \end{multiequality}
    

    \subparag{Personal note}{
        This formula is not hard to learn when you understood it. We are basically generalising what we saw with degree 2.

        However, you can still note that using matrices simplifies the notation (this is more some ``pretty'' mathematical stuff than a useful notation): 
        \[a_n = \begin{bmatrix} r_1^n \\ \vdots \\ r_t^n \end{bmatrix} \dotprod \left(\begin{bmatrix} \alpha_1 & \ldots & \alpha_{1, m-1} \\ \vdots &  & \vdots \\ \alpha_{t, 0} & \ldots & \alpha_{t, m-1} \end{bmatrix} \begin{bmatrix} 1 \\ n \\ \vdots \\ n^{m-1} \end{bmatrix}\right)\]
        where $m = \max\left(m_1, \ldots, m_t\right)$ and $\alpha_{p, q}$ is 0 if $q \geq m_p$ (we had to increase the number of coefficients because a matrix has the same number of coefficients on each row, but it's like setting all the unnecessary ones to 0).

        
    }
    
}

\parag{Exemple}{
    Let's say we want to solve the recurrence relation $a_n = -3a_{n-1} - 3a_{n-2} - a_{n-3}$, with $a_0 = 1$, $a_1 = -2$ and $a_2 = -1$.

    Its characteristic equation is given by: 
    \[r^3 + 3r^2 + 3r + 1 = 0 \iff \left(r + 1\right)^3 = 0 \implies r_0 = -1\]
    
    So, $a_n$ is of the form: 
    \[\alpha_1 \left(-1\right)^n + \alpha_2 n \left(-1\right)^n + \alpha_3 n^2 \left(-1\right)^n\]

    With our initial conditions, we get: 
    \begin{systemofequations}{}
    &\ 1 = a_0 = \alpha_1 \\
    &\ -2 = a_1 = \alpha_1 \left(-1\right)^1 + \alpha_2 \cdot 1 \cdot \left(-1\right)^1 + \alpha_3 \cdot 1^2 \left(-1\right) = -\alpha_1 - \alpha_2 - \alpha_3 \\
    &\ -1 = a_2 = \alpha_1 \left(-1\right)^2 + \alpha_2 \cdot 2 \cdot \left(-1\right)^2 + \alpha_3 \cdot 2^2 \cdot \left(-1\right)^2 = \alpha_1 + 2\alpha_2 + 4\alpha_3^2
    \end{systemofequations}

    Solving the system, we get: 
    \[\alpha_1 = 1, \mathspace \alpha_2 = 3, \mathspace \alpha_3 = -2\]
    
    So, the closed form of our sequence is given by: 
    \[a_n = \left(-1\right)^n + 3n\left(-1\right)^n - 2n^2 \left(-1\right)^n\]
}

\subsubsection{Generating functions}
\parag{Definition}{
    The \important{generating function} for the infinite sequence $a_0, \ldots, a_k, \ldots$ of real numbers is the infinite series 
    \[G\left(x\right) = a_0 + a_1 x + \ldots + a_k x^k + \ldots = \sum_{k=0}^{\infty} a_k x^k\]

    \subparag{Examples}{
        The sequence $\left\{a_k\right\}$ with $a_k = k+1$ has the generating function: 
        \[\sum_{k=0}^{\infty} \left(k+1\right)x^k\]
        
        Similarly, if $a_k = 2^k$, then the generating function is given by: 
        \[\sum_{k=0}^{\infty} 2^k x^k\]
    }
}

\parag{Remarkable generating functions}{
    We know the closed form of the following generating functions: 

    \begin{center}
    \begin{tabular}{|c|c|}
        \hline
        \fullbf{$a_k$} & \fullbf{Its generating function} \\
        \hline
        $\displaystyle C\left(n, k\right)$ & $\displaystyle G\left(x\right) = \left(1 + x\right)^n = \sum_{k=0}^{\infty} \binom{n}{k} x^k$ \\
        \hline
        $\displaystyle a^n$ & $\displaystyle G\left(x\right) = \frac{1}{1 - ax} = \sum_{k=0}^{\infty} a^k x^k$ \\
        \hline
        $\displaystyle C\left(n + k - 1, k\right)$ & $\displaystyle \displaystyle G\left(x\right) = \frac{1}{\left(1 - x\right)^n} = \sum_{k=0}^{\infty} \binom{n+k-1}{k} x^k$ \\
        \hline
    \end{tabular}
    \end{center}

    We'll generalise binomial coefficients in the next lesson, we only need to know that $C\left(n, r\right) = 0$ when $r \geq n$.
    
}

\parag{Using generating functions}{
    Let's say we want to solve the recurrence relation $a_k = 3a_{k-1}$ with initial condition $a_0 = 2$, using generating functions.

    We have: 
    \[G\left(x\right) = \sum_{k=0}^{\infty} a_k x^k = a_0 + \sum_{k=1}^{\infty} a_k x^k = a_0 + \sum_{k=1}^{\infty} 3a_{k-1}x^k =a_0 + 3x\sum_{k=1}^{\infty} a_{k-1}x^{k-1}\]
    
    Changing the index, taking $k-1 \to k$, we get:
    \[G\left(x\right) = a_0 + 3x \sum_{k=0}^{\infty} a_k x^k = a_0 + 3xG\left(x\right)\]

    So, we can solve the equation: 
    \[G\left(x\right) = a_0 + 3xG\left(x\right) \iff G\left(x\right) = \frac{2}{1 - 3x} = 2\cdot \sum_{k=0}^{\infty} 3^k x^k = \sum_{k=0}^{\infty} 2\cdot 3^k x^k\]
    
    We can see that, in the generating function, we have $a_k = 2\cdot 3^k$. We could have solved this much simpler by ``seeing it'', or by using the method previously explained, but the point is that, now, we know how it works.
}

\parag{Solving Hanoi Tower}{
    When we wanted to solve our Hanoi Tower problem, we had found $H_n = 2H_{n-1} + 1$ and $H_0 = 0$. This is not homogeneous, so let's use generating functions: 
    \[G\left(x\right) = \sum_{k=0}^{\infty} H_k x^k = \underbrace{H_0}_{=0} + \sum_{k=1}^{\infty} H_{k} x^k = \sum_{k=1}^{\infty} \left(2 H_{k - 1} + 1\right)x^k \]

    Let us split this function and then do the regular ``factoring out-change of index trick'': 
    \[G\left(x\right) = \sum_{k=1}^{\infty} 2H_{k-1} x^k + \sum_{k=1}^{\infty} x^k = 2x \sum_{k=1}^{\infty} H_{k-1} x^{k-1} + x \sum_{k=1}^{\infty} x^{k-1} = 2x \sum_{k=0}^{\infty} H_k x^k + x\sum_{k = 0}^{\infty} x^k\]

    We can now recognise our generating function, and one of the remarkable ones: 
    \[G\left(x\right) = 2x G\left(x\right) + x \frac{1}{1-x} \implies G\left(x\right) = \frac{x}{\left(1 - 2x\right)\left(1 - x\right)}\]

    Let us use partial fractions to split this fraction: 
    \[G\left(x\right) = \frac{x}{\left(1 - 2x\right)\left(1 - x\right)} = \frac{a}{1 - 2x} + \frac{b}{1 - x} = \frac{a\left(1 - x\right) + b\left(1 - 2x\right)}{\left(1 - 2x\right)\left(1 - x\right)}\]
    
    So, we need to solve the following equation: 
    \[x = a\left(1 -x\right) + b\left(1 - 2x\right) = a - ax + b - 2bx \iff \left(a + b\right) + x\left(-1 - a - b\right) = 0\]
    
    Thus we have the following system to solve: 
    \begin{systemofequations}{}
    &\ a + b = 0 \\
    &\ -1 - a - b = 0
    \end{systemofequations}

    Solving it, we find: 
    \[a = 1, \mathspace b = -1\]
    
    Coming back to our generating function, we have: 
    \[G\left(x\right) = \frac{a}{1 - 2x} + \frac{b}{1 - x} = \frac{1}{1-2x} - \frac{1}{1-x} = \sum_{k=0}^{\infty} 2^k x^k - \sum_{k=0}^{\infty} x^k = \sum_{k=0}^{\infty} \left(2^k -  1\right)x^k\]
    
    So, we have found $a_n = 2^n - 1$
}

\parag{General method}{
    \begin{itemize}[left=0pt]
        \item Define the generating function $G\left(x\right)$.
        \item Use the recurrence relation to derive an alternative expression for $G\left(x\right)$ (using the ``factoring out-change of index'' trick).
        \item Solve the equation for $G\left(x\right)$. 
        \item If it is a fraction of polynomial, use partial fractions to split it as a sum of terms of the form $\frac{c_i}{x - r_i}$ (it is always possible if the degree of the numerator is less than the one of the denominator). 
        \item Determine the power expansion of $G\left(x\right)$, using remarkable generating functions or Taylor series (we have never used Taylor series and we will probably never use them, but I put them here because it would be a valid method if one forgot the remarkable generating functions).
    \end{itemize}
}

\end{document}
