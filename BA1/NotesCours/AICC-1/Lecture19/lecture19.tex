\documentclass[a4paper]{article}

% Expanded on 2021-11-23 at 08:27:37.

\usepackage{../../style}

\title{AICC}
\author{Joachim Favre}
\date{Mardi 23 novembre 2021}

\begin{document}
\maketitle

\lecture{19}{2021-11-23}{Euclid is the best, and we learn to count}{
\begin{itemize}[left=0pt]
    \item Definition of GCD and LCM.
    \item Explanation of the Euclidean algorithm, and of the link between the GCD and the LCM of two numbers.
    \item Explanation of some facts about number theory, which will not be at the exam.
    \item Explanation of the product, addition and soustraction rules. 
    \item Definition of permutations and combinations.
\end{itemize}
}

\subsection{GCD and LCM}
\parag{Definition of GCD}{
    Let $a$ and $b$ be integers, not both zero. The largest integer $d$ such that $d \divides a$ and $d \divides b$ is called the \important{greatest common divisor} (GCD) of $a$ and $b$. It is denoted by $\gcd\left(a, b\right)$.
}

\parag{Example}{
    We have: 
    \[\gcd\left(24, 36\right) = 12, \mathspace \gcd\left(17, 22\right) = 1\]
}

\parag{Definition of relatively prime}{
    The integers $a$ and $b$ are \important{relatively prime} if $\gcd\left(a, b\right) = 1$.

    The integers $a_1, \ldots, a_n$ are \important{pairwise relatively prime} if $\gcd\left(a_i, a_j\right) = 1$ for $1 \leq i < j \leq n$.
}

\parag{Example}{
    We can verify that $10, 17$ and $21$ are pairwise relatively prime: 
    \[\gcd\left(10, 17\right) = 1, \mathspace \gcd\left(10, 21\right) = 1, \mathspace \gcd\left(17, 21\right) = 1\]
    
    However, 10, 19 and 24 are not pairwise relatively prime since: 
    \[\gcd\left(10, 24\right) = 2 \neq 1\]
}

\parag{Finding the GCD}{
    Let's write $a$ and $b$ as their prime factorisation: 
    \[a = p_1^{a_1}\cdot p_2^{a_2}\cdot\ldots\cdot p_n^{a_n}, \mathspace b = p_1^{b_1}\cdot p_2^{b_2}\cdot\ldots\cdot p_n^{b_n}\]
    where each exponent is a nonnegative integer. We can compute the GCD the following way:
    \[\gcd\left(a, b\right) = p_1^{\min\left(a_1, b_1\right)} p_2^{\min\left(a_2, b_2\right)} \cdot\ldots\cdot p_n^{\min\left(a_n, b_n\right)}\]
    
    However, finding the prime factorisation of a number is a very complicated problem. It is not polynomial (and this is the basis of many cryptography algorithm).
}

\parag{Definition of LCM}{
    The \important{least common multiple} (LCM) of two positive integers $a$ and $b$ is the smallest positive integer that is divisible by both $a$ and $b$. It is denoted by $\lcm\left(a, b\right)$.

    We can also find it using prime factorisations: 
    \[\lcm\left(a, b\right) = p_1^{\max\left(a_1, b_1\right)} \cdot p_2^{\max\left(a_2, b_2\right)} \cdot \ldots \cdot p_n^{\max\left(a_n, b_n\right)}\]
}

\parag{Example}{
    Let's find the prime factorisation of 120 and 500: 
    \[120 = 2^3 \cdot 3 \cdot 5 = 2^3 \cdot 3^1 \cdot 5^1, \mathspace 500 = 2^2 \cdot 5^3 = 2^2 \cdot 3^0 \cdot 5^3\]
    
    Then: 
    \[\gcd\left(120, 500\right) = 2^2 \cdot 3^0 \cdot 5^1 = 4\cdot 5 = 20\]
    \[\lcm\left(120, 500\right) = 2^3 \cdot 3^1 \cdot 5^3 = 8\cdot 3\cdot 125 = 3000\]
}


\parag{Theorem}{
    Let $a$ and $b$ be positive integers. Then: 
    \[a\cdot b = \gcd\left(a, b\right) \cdot \lcm\left(a, b\right)\]
    
    \subparag{Proof}{
        This proof relies on the fact that: 
        \[a + b = \max\left(a, b\right) + \min\left(a, b\right)\]
        
        Thus: 
        \begin{multiequality}
         & \gcd\left(a, b\right)\lcm\left(a,b\right)  \\
         =\ & \left(p_1^{\min\left(a_1, b_1\right)}\cdot\ldots\cdot p_n^{\min\left(a_n, b_n\right)}\right)\left(p_1^{\max\left(a_1, b_1\right)}\cdot\ldots\cdot p_n^{\max\left(a_n, b_n\right)}\right)  \\
         =\ & p_1^{\min\left(a_1, b_1\right)} p_1^{\max\left(a_1, b_1\right)} \cdot \ldots \cdot p_n^{\min\left(a_n, b_n\right)}p_n^{\max\left(a_n, b_n\right)} \\
         =\ & p_1^{\min\left(a_1, b_1\right) + \max\left(a_1, b_1\right)} \cdot \ldots \cdot p_n^{\min\left(a_n, b_n\right) + \max\left(a_n, b_n\right)}  \\
         =\ & p_1^{a_1 + b_1} \cdot \ldots p_n^{a_n + b_n} \\
         =\ & p_1^{a_1} \cdot p_1^{b_1} \cdot \ldots \cdot p_n^{a_n}\cdot p_n^{a_n} \\
         =\ & p_1^{a_1} \cdot \ldots \cdot p_n^{a_n} \cdot p_1^{b_1} \cdot \ldots \cdot p_n^{b_n} \\
         =\ & a \cdot b    
        \end{multiequality}
        
    }
}

\parag{Observation}{
    If we have $a > b$ and $r = a \Mod b$, then: 
    \[\gcd\left(a, b\right) = \gcd\left(r, b\right)\]
    
    Indeed, we know that $r = a \Mod b$ is equivalent to $a = q\cdot b + r$. We want to show that: 
    \[d \divides a \text{ and } d \divides b \iff d \divides b \text{ and } d \divides r\]
    
    First, let's suppose that $d \divides a$ and $d \divides b$. Divisibility preserves addition, so 
    \[d \divides a - qb = r \implies d \divides r\]

    Now, let's suppose that if $d \divides r$ and $d \divides b$: 
    \[d \divides q \cdot b + r = a \implies d \divides a\]
    
    So, we have indeed shown that: 
    \[d \divides a \text{ and } d \divides b \iff d \divides b \text{ and } d \divides r\]
}

\parag{Example}{
    Let's say we want to compute $\gcd\left(287, 91\right)$: 
    \[\gcd\left(287, 91\right) = \gcd\left(14, 91\right) = \gcd\left(91, 14\right) = \gcd\left(7, 14\right) = \gcd\left(14, 7\right) = \gcd\left(0, 7\right) = 7\]
    
    Indeed, using long division we can see that: 
    \[287 = 3\cdot 91 + 14\]
    \[91 = 6\cdot 14 + 7\]
    \[7 = 2\cdot 7 + 0\]
    
}

\begin{filecontents*}[overwrite]{euclideanAlgorithmIterative.code}
procedure gcd_iterative(a > b: positive integers)
    x := a
    y := b
    while y != 0
        r := x mod y
        x := y
        y := r
    return x
\end{filecontents*}

\begin{filecontents*}[overwrite]{euclideanAlgorithmRecursive.code}
procedure gcd_recursive(a > b: positive integers)
    if b == 0 then return a
    else return gcd(b, a mod b)
\end{filecontents*}


\parag{Euclidean algorithm}{
    This algorithm is called the Euclidean algorithm (it was invented by Euclid). Iteratively, it is written:
    \importcode{euclideanAlgorithmIterative.code}{pseudo}
    
    Recursively, it looks much simpler:
    \importcode{euclideanAlgorithmRecursive.code}{pseudo}

    \subparag{Proof}{
        Let's say we want to compute $\gcd\left(a, b\right)$. So, we can take $r_0 = a$ and $r_1 = b$. 

        Then, we perform a division and we get $r_2 = r_0 \Mod r_1$. Clearly, $r_1 > r_2 \geq 0$.

        Similarly, taking $r_3 = r_1 \Mod r_2$, then $r_2 > r_3 \geq 0$.

        Since $r_i > r_{i+1} \geq 0$, then $r$ decreases at least by 1 at each step, so there must be a step at which it is equal to 0. Let's call this $r_{n+1} = r_{n-1} \Mod r_n$, where $r_{n+1} = 0$. Then: 
        \[\gcd\left(a, b\right) = \gcd\left(r_0, r_1\right) = \ldots = \gcd\left(r_{n-1}, r_{n}\right) = \gcd\left(r_n, 0\right) = r_n\]
    }
}

\parag{Lamé's theorem}{
    Let $a$ and $b$ be positive integers with $a \geq b$. 

    The number of divisions used by the Eucliden algorithm to find $\gcd\left(a, b\right)$ is less than or equal to five times the number of decimal digits in $b$. 

    In other words, the Euclidean algorithm has a complexity of $O\left(\log b\right)$ (careful, it is a big-$O$, not a big-$\Theta$).
}

\subsection{More facts about number theory (not in the exam)}

\parag{Distribution of primes}{
    Let's define $\pi\left(x\right)$ be the number of primes not exceeding $x$.

    The prime number theorem states that the following function approaches 1 as $x$ tends toward infinity:
    \[\frac{\pi\left(x\right)}{\mathspace \frac{x}{\ln\left(x\right)} \mathspace}.\]

    This implies that the odds that a random positive integer less than $n$ is prime are approximately of:
    \[\frac{1}{\ln\left(x\right)}.\]
}

\parag{Primes and arithmetic progressions}{
    We wonder if there are long arithmetic progressions made up entirely of primes. For example, the following one consists of five primes: 
    \[5, 11, 16, 23, 29\]

    We can also find one of ten primes: 
    \[199, 409, 619, 820, 1039, 1249, 1459, 1669, 1879, 2089\]
    
    In the 1930s, Paul Erdös conjectured that for every positive integer $n > 1$, there is an arithmetic progression of length $n$ made up entirely of primes. This was proven 70 years later, in 2006, by Ben Green and Terrence Tau.
}

\parag{Mersene primes}{
    Prime numbers of the forme $2^p - 1$, where $p$ is prime, are called Mersene primes.

    For example, the following numbers are Mersene primes. 
    \[2^2 - 1 = 3, \mathspace 2^3 - 1 = 7, \mathspace 2^5 - 1 = 37, \mathspace 2^7 - 1 = 127\]
    
    However, all numbers of this form are not primes, for example the following number isn't prime (so it isn't a Mersene prime):
    \[2^{11} - 1 = 2047 = 23\cdot 89\]

    The advantage of such primes is that there exists a very efficient way to verify whether numbers of the form $2^p - 1$ are primes. Thus, the largest primes nowadays are Mersene primes. In mid 2018, the largest known prime --- which has 25 million decimal digits --- was: 
    \[2^{82\,589\,933} - 1\]
}

\parag{Goldbach's Conjecture}{
    In 1742 Goldbach conjectured that every even integer $n > 2$ is the sum of two primes.

    It has been verified by computer for all positive integers up to $1.6 \cdot 10^{18}$, but no one has found a proof yet. 
}

\parag{The Twin Prime Conjecture}{
    We define twin primes to be a pair of primes that differ by 2. For example: 
    \[3,5 \mathspace 5,7 \mathspace 11, 13 \mathspace\]
    
    It is conjectured that there are infinitely twin primes.

    In 2018, the largest twin primes we found --- which have $388\,342$ decimal digits --- are: 
    \[2\,996\,863\,034\,895 \cdot 2^{1\,290\,000} \pm 1\]
    
}


\section{Counting}
\subsection{Basic rules}
\parag{Introduction}{
    We are asking ourselves some of the following questions:
    \begin{itemize}
        \item How many different passwords exists?
        \item How many outcomes does an experiment have?
        \item How many steps an algorithm performs?
        \item How many moves are possible in a game?
    \end{itemize}

    Counting is  the subject of the mathematical discipline of \important{combinatorics}.
}

\parag{Short history}{
    Mahavira (850 AD) provided formulas for permutations and combinations. Pascal (\nth{17} century) developed Pascal's triangle. Euler (\nth{18} century) was the first to use generating functions.
}

\parag{Product rule}{
    Let $A$ and $B$ be two tasks, with respectively $n_1$ and $n_2$ ways to do them.

    There are $n_1 \cdot n_2$ ways to do both tasks.
}

\parag{Example}{
    When rolling two d-6 (dices with 6 faces), there is a total of $6\cdot 6 = 36$ possible pairs.
}

\parag{Counting Cartesian product}{
    Let $A_1, A_2, \ldots, A_m$ be finite sets. 

    We have that:
    \[\left|A_1 \times A_2 \times \ldots \times A_n\right| = \left|A_1\right|\cdot \left|A_2\right| \cdot \ldots \cdot \left|A_n\right|\] 

    \subparag{Proof}{
        The task of choosing an element in the Cartesian product is done by choosing an element in $A_1$, an element in $A_2$ and so on until an element in $A_m$. This theorem then follows from the product rule.
 
        \qed
    }
}

\parag{Example 1}{
    We wonder how many different license plate can be made if each plate contains a sequence of 2 uppercase letters followed by 5 digits.

    Since there are 26 different letters and 10 different digits, then this number is given by: 
    \[26^2 \cdot 10^5 = 67\,600\,000\]
}

\parag{Example 2}{
    We wonder how many functions there are from a set with $m$ elements to a set with $n$ elements. For each of the $m$ elements of the input, we have $n$ possibilities in the other set. So, this number is given by: 
    \[n\cdot n\cdot \ldots \cdot n = n^m\]
}

\parag{Example 3}{
    Let's say we now want to know how many one-to-one functions there are from a set with $m$ elements to one with $n$ elements. For the first element, we have $n$ possibilities. Then, for the second one, we only have $n-1$ possibilities, since we want our function to be injective. And, so on until we have $n - \left(m - 1\right)$ possibilities for the last element. Thus, this number is given by:
    \[n\left(n-1\right)\cdot\ldots\cdot\left(n - m + 1\right) = \frac{n!}{\left(n - m\right)!}\]
}

\parag{Example 4}{
    We wonder what is the number of different subsets of a finite set $S$, $\left|P\left(S\right)\right|$.

    Let $T$ be any subset of $S$. We can make a sequence of 0s and 1s to say for each element of $S$ if it is in $T$. Since there are 2 possibilities for each element (either it is in the set or not), we find by the product rule: 
    \[\left|P\left(S\right)\right| = 2\cdot 2\cdot \ldots \cdot 2 =  2^n\]
}

\parag{Sum rule}{
    Let $A$ and $B$ be two tasks, with respectively $n_1$ and $n_2$ ways to do them.

    If none of the set of $n_1$ ways is the same as any of the set of $n_2$ ways, then there are $n_1 + n_2$ ways to do task $A$ or $B$.
}

\parag{Example}{
    Le't say that a student can choose a semester project from one of three laboratories. The three laboratories offer 5, 3 and 7 possible projects, respectively. No project is offered by several laboratories. 

    We know by the sum rule that there are $5 + 3 + 7 = 15$ ways to choose a project.
}

\parag{Sum rule in terms of set}{
    Let $A$ and $B$ be sets. If $A$ and $B$ are disjoint sets, meaning $A \cap B = \o$, then we know that:    \[\left|A \cup B\right| = \left|A\right| + \left|B\right|\]

    This is the sum rule can be phrased in term of sets.

    More generally, if we have $n$ sets $A_1, \ldots, A_n$ such that $A_i \cap A_j = \o$ for all $i, j$, then: 
    \[\left|A_1 \cup \ldots \cup A_n\right| = \left|A_1\right| + \ldots + \left|A_n\right|\]
    
}

\parag{Example 1}{
    Suppose variable names in a programming language can be either a single letter or a letter followed by a digit. We want to find the number of possible names. We have $26$ possibilities for the single letter, and $26\cdot 10$ for the letter followed by a digit. Thus, the number of variable names is given by: 
    \[26 + 26\cdot 10 = 286\]
}

\parag{Example 2}{
    Let's say that each user on a computer system has a password, which is 6 to 8 characters long, where each character is an uppercase letter or a digit. Each password must contains at least one digit. 

    By the sum rule, the number of passwords is given by: 
    \[P = P_6 + P_7 + P_8\]
    
    For $P_6$, by the product rule, we have $\left(26 + 10\right)^6$ such passwords containing 6 upper case letters or digits. However, we to remove the passwords consisting of only letter. Thus, we get that: 
    \[P_6 = \left(26 + 10\right)^6 - 26^6\]

    Doing the same reasoning for $P_7$ and $P_8$, we get: 
    \[P = P_6 + P_7 + P_8 = \left(36^6 - 26^6\right) + \left(36^7 - 26^7\right) + \left(36^8 - 26^8\right) = 2\,483\,687\,808\,960\]
}

\parag{Subtraction rule}{
    If a task can be done either in one of $n_1$ ways or in one of $n_2$ ways, then the total number of ways to do the task in $n_1 + n_2$ minus the number of ways to do the task that are common to the two different ways.

    This is also known as the \important{principle of inclusion-exclusion}:
    \[\left|A \cup B\right| = \left|A\right| + \left|B\right| - \left|A \cap B\right|\]
}

\parag{Example}{
    We wonder how many bit strings of length 8 either start with a 1 bit or end with the two bits 00.

    Let $A$ be the set of 8-bit bitstrings starting with 1, and $B$ be the set of 8-bit bistrings ending with 00. We know that $\left|A\right| = 2^7$ since we fixed the first position and we have 7 other free positions. Using the same reasoning, we find that $\left|B\right| = 2^6$.

    We still need to find the intersection. We know that $A \cap B = 2^5$ since there are 3 positions fixed (the bitstring begins with 1 and ends with 00) and 5 free positions.

    So: 
    \[\left|A \cup B\right| = \left|A\right| + \left|B\right| - \left|A \cap B\right| = 2^7 + 2^6 - 2^5 = 160\]
}

\subsection{Permutations and combinations}
\parag{Definition of permutations}{
    A \important{permutation} of a set of $n$ distinct objects is an \textit{ordered} arrangement of these objects. An ordered arrangement or $r$ elements of a set is called an \important{$r$-permutation}. We note this $P\left(n, r\right)$.
}

\parag{Example}{
    Let $S = \left\{1, 2, 3\right\}$. Then $3, 1, 2$ is an permutation of $S$. $3, 2$ is a 2-permutation of $S$.
}

\parag{Theorem (counting the number of permutations)}{
    Let $n$ be a positive integer and $r$ be an integer, where $1 \leq r \leq n$.

    The number of $r$-permutations of a set with $n$ distinct elements is given by: 
    \[P\left(n, r\right) = n\left(n-1\right)\cdot\ldots\cdot\left(n - r + 1\right)\]
    
    \subparag{Proof}{
        The first element can be chosen in $n$ ways, the second in $n-1$ ways, and so on until the last element, which there are $m - \left(r - 1\right)$ ways to choose.

        So, by the product rule: 
        \[P\left(n, r\right) = n\left(n-1\right)\ldots\left(n-r+1\right)\]

        Moreover, $P\left(n, 0\right) = 1$ since there is only one way to order zero elements.
    }

    \subparag{Corollary}{
        We have that: 
        \[P\left(n, r\right) =  n\cdot\ldots\cdot\left(n - r + 1\right) = \frac{n!}{\left(n - r\right)!}\]
        
    }
    
    
}

\parag{Example}{
    We wonder how many ways are there to select a first-prize winner, a second prize winner and a third-prize winner from 100 different different people who have entered a contest. 

    This is number is given by: 
    \[P\left(100, 3\right) = \frac{100!}{\left(100 - 3\right)!} = \frac{100!}{97!} = 970\,200\]
    
}

\parag{Definition of combinations}{
    An \important{$r$-combination} of elements of a set (which contains $n$ elements) is an \textit{unordered} selection of $r$ elements from the set. Thus, an $r$-combination is simply a subset of the set with $r$ elements. This number is denoted: 
    \[C\left(n, r\right) \text{ or } \binom{n}{r}\]
    
    We read the second one ``$n$ choose $r$''.
}

\parag{Example}{
    Let $S$ be the set $\left\{a, b, c, d\right\}$. Then, $\left\{a, c, d\right\}$ is a 3-combination from $S$. It is the same as $\left\{d, c, a\right\}$ since order does not matter.
}

\end{document}
