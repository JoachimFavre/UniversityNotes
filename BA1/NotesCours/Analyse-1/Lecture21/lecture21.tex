\documentclass[a4paper]{article}

% Expanded on 2021-12-06 at 10:18:50.

\usepackage{../../style}

\title{Analyse 1}
\author{Joachim Favre}
\date{Lundi 06 décembre 2021}

\begin{document}
\maketitle

\lecture{21}{2021-12-06}{Propriétés des DL, et rayons de convergence}{
\begin{itemize}[left=0pt]
    \item Explication des opérations algébriques sur les développements limités, et des développements limités de fonctions composées.
    \item Explication du calcul de limites par développements limités. 
    \item Définition des séries entières, et de leur domaine de convergence.
    \item Démonstration que le domaine de convergence d'une série entière est toujours un intervalle contenant $x_0$ et centré sur ce dernier.
    \item Définition du rayon de convergence.
\end{itemize}

}

\subsection{Retour aux développements limités}
\parag{Proposition: opérations algébriques sur les DL}{
    Soient $f,g : E \mapsto \mathbb{R}$ deux fonction admettant un DL autour de $x = a$: 
    \[f\left(x\right) = a_0 + a_1\left(x - a\right) + \ldots + a_n\left(x - a\right)^n + \left(x - a\right)^n \epsilon_1\left(x\right) = P_f^n\left(x\right) + \left(x - a\right)^n \epsilon_1\left(x\right)\]
    \[g\left(x\right) = b_0 + b_1\left(x - a\right) + \ldots + b_n\left(x - a\right)^n + \left(x - a\right)^n \epsilon_2\left(x\right) = P_g^n\left(x\right) + \left(x - a\right)^n \epsilon_2\left(x\right)\]
    \[\lim_{x \to a} \epsilon_1\left(x\right) = 0, \mathspace \lim_{x \to a} \epsilon_2\left(x\right) = 0\]

    Alors:
    \begin{enumerate}
        \item $\alpha f\left(x\right) + \beta g\left(x\right)$ admet un DL d'ordre $n$ autour de $x = a$ et: 
            \[P^n_{\alpha f + \beta g}\left(x\right) = \alpha P_{f}^n\left(x\right) + \beta P_g^n\left(x\right)\]
        \item $f\left(x\right)g\left(x\right)$ admet aussi un DL d'ordre $n$: 
            \[P_{f \cdot g}^n\left(x\right) = P_f^n\left(x\right) P_g^n\left(x\right)\]
            où on ne conserve que les termes d'ordre $\leq n$.
            
        \item Si $b_0 \neq 0$ et $g\left(x\right) \neq 0$, alors: 
            \[P^n_{\frac{f}{g}}\left(x\right) = \frac{P_f^n\left(x\right)}{P_{g}^n\left(x\right)}\]
        
    \end{enumerate}
}

\parag{Exemple}{
    Trouvons le DL d'ordre 4 autour de $x = 0$ de la fonction suivante: 
    \[f\left(x\right) = \frac{1}{1 - x^2}\]

    \subparag{Méthode 1}{
        Utilisons les fractions partielles et la première propriété:
        \begin{multiequality}
        \frac{1}{1 - x^2} =\ & \frac{1}{2} \frac{1}{1-x} + \frac{1}{2} \frac{1}{1+x}  \\
        =\ & \frac{1}{2}\left(1 + x + x^2 + x^4 + x^4 \epsilon_1\left(x\right)\right)  \\
         & + \frac{1}{2} \left(1 - x + x^2 - x^3 + x^4 + x^4 \epsilon_2\left(x\right)\right)  \\
        =\ & 1 + x^2 + x^4 + \epsilon\left(x\right) x^4 
        \end{multiequality}
        
    }

    \subparag{Méthode 2}{
        Nous aurions aussi pu utiliser la deuxième propriété:
        \begin{multiequality}
        \frac{1}{1 - x^2} =\ & \frac{1}{1-x} \cdot \frac{1}{1 + x} \\
        =\ & \left(1 + x + x^2 + x^3 + x^4 + \epsilon_1\left(x\right) x^4\right) \\
        & \cdot\left(1 - x + x^2  - x^3 + x^4 + \epsilon_2\left(x\right)x^4\right)  \\
        =\ & 1 + x + x^2 + x^3 + x^4 + \left(-x -x^2 -x^3 - x^4\right)  \\
        & + \left(x^2 + x^3 + x^4\right) + \left(-x^3 -x^4\right) + x^4 + \epsilon\left(x\right) x^4 \\
        =\ & 1 + x^2 + x^4 + \epsilon\left(x\right)x^4    
        \end{multiequality}
    }
    
    \subparag{Méthode 3}{
        Nous aurions pu utiliser la troisième proposition. En effet, faisons une division en colonne:

        \begin{center}
        \begin{tabular}{c@{\,}c@{\,}c@{\,}c@{\,}c@{\,}c@{\,}c@{\,}c@{\,}|l}
            1 &   &    &   &    &   &    & $\quad$ & $1 - x^2$ \\
            \hhline{~~~~~~~~|-}
            1 & $-$ & $x^2$ &   &    &   &    &   & $1 + x^2 + x^4 + \ldots$ \\
            \hhline{---~~~~~|~}
              &   & $x^2$ &   &    &   &    &   & \\
              &   & $x^2$ & $-$ & $x^4$ &   &    &   & \\
            \hhline{~~---~~~|~}
              &   &    &   & $x^4$ &   &    &   & \\
              &   &    &   & $x^4$ & $-$ & $x^6$ &   & \\
            \hhline{~~~~---~|~}
              &   &    &   &    &   & $x^6$ &   & \\
              &   &    &   &    &   & $\vdots$  &   &  
        \end{tabular}
        \end{center}

        Ainsi, en prenant les termes de degré $\leq 4$, on a bien: 
        \[\frac{1}{1 - x^2} = 1 + x^2 + x^4 + \epsilon\left(x\right) x^4\]
    }
    
    \subparag{Méthode 4}{
        Quatrièmement, nous aurions pu utiliser une autre méthode, dite des coefficients indéterminés. On cherche des coefficients tels que: 
        \[f\left(x\right) = a_0 + a_1 x + a_2 x + a_3 x^3 + a_4 x^4 + x^4 \epsilon\left(x\right) = \frac{1}{(1-x)^2}\]
        
        De là, on en déduit que: 
        \[\left(a_0 + a_1 x + a_2 x^2 + a_3 x^3 + a_4 x^4\right)\left(1 - x^2\right) = 1 \text{ jusqu'à l'ordre 4}\]
        
        En développant, on trouve le système suivant:
        \[a_0 = 1, \mathspace a_1 = 0, \mathspace a_2 - a_0 = 0, \mathspace a_3 - a_1 = 0, \mathspace a_4 - a_2 = 0\]

        En résolvant ce système, on obtient: 
        \[a_0 = 1, \mathspace a_1 = 0, \mathspace a_2 = 1, \mathspace a_3 = 0, \mathspace a_4 = 1\]
        
        On a donc bien obtenu: 
        \[f\left(x\right) = 1 + x^2 + x^4 + \epsilon\left(x\right) x^4\]
    }
}

\parag{Proposition: DL d'une fonction composée}{
    Soient $f$ et $g$, deux fonctions, où $f$ admet un DL autour de $x = a$, et $g$ autour de $y = 0$:
    \[f\left(x\right) = a_1\left(x - a\right) + \ldots + a_n\left(x - a\right)^n + \left(x - a\right)^n \epsilon_1\left(x\right)\]
    \[g\left(y\right) = g\left(0\right) + b_1 y + \ldots + b_n y^n + y^n \epsilon_2\left(y\right)\]
    
    Alors, $g \circ f$ admet un DL d'ordre $n$ autour de $x = a$, donné par: 
    \[P_{g \circ f}^n\left(x\right) = g\left(0\right) + b_1\left(P_f^n\left(x - a\right)\right) + b_2\left(P_f^n\left(x - a\right)\right)^2 + \ldots + b_n\left(P_f^n\left(x - a\right)^n\right)\]
    où on ne conserve que les termes d'ordre $\leq n$.
}

\parag{Exemple}{
    Reprenons notre exemple ci-dessus, calculons le DL de la fonction suivante autour de $x = 0$: 
    \[h\left(x\right) = \frac{1}{1 - x^2}\]
    
    Prenons $g\left(y\right) = \frac{1}{1 - y}$ et $f\left(x\right) = x^2$. On voit que $f$ est son propre développement limité autour de $x = 0$, donc:
    \[g\left(y\right) = 1 + y + y^2 + y^3 + y^4 + \epsilon_1\left(y\right) y^4, \mathspace f\left(x\right) = x^2\]

    Ainsi, on peut remplacer $y = x^2$: 
    \[h\left(x\right) = 1 + \left(x^2\right) + \left(x^2\right)^2 + \epsilon\left(x\right)x^4 = 1 + x^2 + x^4 + \epsilon\left(x\right) x^4\]
}

\subsection{Développements limités pour le calcul des limites}
\parag{Exemple}{
    Disons que nous voulons calculer la limite suivante (qui est une forme indéterminée $\frac{0}{0}$): 
    \[\lim_{x \to 0} \frac{\sin^2 x - x^2}{\left(e^x  - 1 - x\right)^2}\]
    
    On se souvient que: 
    \[\sin\left(x\right) = x - \frac{x^3}{3!} + \frac{x^5}{5!} - \ldots\]
    
    De plus, on connait déjà le développement limité de $e^x$, par sa définition: 
    \[e^x = 1 + x + \frac{x^2}{2!} + \frac{x^3}{3!} + \ldots = \sum_{k=0}^{\infty} \frac{x^k}{k!} = \sum_{k=0}^{\infty} \frac{f^{\left(k\right)}\left(0\right)}{k!} x^k\]
    
    Commençons par le dénominateur: 
    \[\left(e^x - 1 - x\right)^2 = \left(\frac{x^2}{2} + \frac{x^3}{3!} + \frac{x^4}{4!} + \ldots\right)^2 = \left(\frac{x^2}{2}\right)^2 + \epsilon_1\left(x\right) x^4 = \frac{x^4}{4} + \epsilon_1\left(x\right) x^4\]
    
    On prend le premier terme qui n'est pas 0. Nous aurons peut-être besoin d'aller plus loin, mais pas nécessairement. 

    De plus: 
    \[\sin^2\left(x\right) = \left(x - \frac{x^3}{3!} + \frac{x^5}{5!} - \ldots\right)^2 = x^2 - 2 \frac{x x^3}{3!} + \epsilon_2\left(x\right) x^4 = x^2 - \frac{1}{3}x^4 + \epsilon_2\left(x\right) x^4\]
   
    On prend un développement limité du même ordre que le dénominateur. 

    Ainsi, on obtient par les opérations algébriques sur les limites:
    \[\lim_{x \to 0} \frac{\sin^2\left(x\right) - x^2}{\left(e^x - 1 - x\right)^2} = \lim_{x \to 0} \frac{x^2 - \frac{1}{3} x^4 + \epsilon_2\left(x\right) x^4 - x^{2}}{\frac{x^4}{4} + \epsilon_1\left(x\right) x^4} = \lim_{x \to 0} \frac{\frac{-1}{3}x^4 + \epsilon_2\left(x\right)x^4}{\frac{x^4}{4} + \epsilon_1\left(x\right)x^4}\]

    Ainsi, en simplifiant les $x^4$, on obtient que c'est égal à:
    \[\lim_{x \to 0} \frac{\frac{-1}{3} + \epsilon_2\left(x\right)}{\frac{1}{4} + \epsilon_1\left(x\right)} = \frac{-4}{3}\]
    
    
    \subparag{Note 1}{
        Si on n'avait pas calculé assez de terme pour un développement limité, on se serait retrouvé avec une forme indéterminée de la forme $\frac{\epsilon_1\left(x\right)}{\epsilon_2\left(x\right)}$. Si on avait calculé trop de termes, alors ils tendraient juste vers 0, donc ce ne serait pas un problème.

        Dans les deux cas, cela nous force à faire trop de calculs, mais ce n'est pas ``grave'', on n'obtient pas la mauvaise valeur à la fin.
    }
    
    \subparag{Note 2}{
        Cette méthode est plus puissante que Bernoulli-L'Hospital. Si BL fonctionne, alors cette méthode fonctionne. Notez que, si on utilise BL pour cet exemple, il faudra dériver quatre fois.
    }
}

\parag{Exercice au lecteur}{
    Calculez la limite suivante à l'aide des développements limités: 
    \[\lim_{x \to 0} \frac{e^{\sin\left(x\right)} - e^x}{x^3}\]
    
}

\section{Séries entières}
\subsection{Rayon de convergence}
\parag{Définition}{
    On appelle \important{série entière} les expressions sous la forme: 
    \[\sum_{k=0}^{\infty} a_k\left(x - x_0\right)^k, \mathspace a_k \in \mathbb{R}\ \forall k \in \mathbb{N}\]
     
    Le \important{domaine de convergence} est défini par: 
    \[D = \left\{x \in \mathbb{R} \telque \sum_{k=0}^{\infty} a_k\left(x - x_0^k\right) \text{ converge}\right\}\]
    
    La fonction $f\left(x\right) = \sum_{k=0}^{\infty} a_k\left(x - x_0\right)^k$ avec $x\in D$ est définie par la série entière.
}

\parag{Exemple 1}{
    Par la définition de l'exponentielle: 
    \[e^x \over{=}{déf} \sum_{k=0}^{\infty} \frac{x^k}{k!}\]
    
    On peut (re-)trouver son domaine de convergence. Utilisons le critère de d'Alembert:
    \[\lim_{k \to \infty} \left|\frac{a_{k+1}}{a_k}\right| = \lim_{k \to \infty} \left|\frac{x^{k+1}}{\left(k + 1\right)!} \cdot \frac{k!}{x^k}\right| = \lim_{k \to \infty} \frac{\left|x\right|}{k + 1} = 0, \mathspace \forall x \in \mathbb{R}\]
    
    Ainsi, cette série entière converge pour tout $x$ dans $\mathbb{R}$, donc $D = \mathbb{R}$.
}

\parag{Exemple 2}{
    Prenons la série entière suivante: 
    \[\sum_{k=0}^{\infty} a_k\left(x - x_0\right)^k, \mathspace a > 0\]
    
    À nouveau, on peut utiliser le critère de d'Alembert: 
    \[\left|\frac{a^{k+1}\left(x - x_0\right)^{k+1}}{a^k \left(x - x_0\right)^k}\right| = a \left|x - x_0\right|\]

    Donc, si $\left|x - x_0\right| < \frac{1}{a}$, alors la série converge absolument. Si $\left|x - x_0\right| > \frac{1}{a}$, alors elle diverge grossièrement. On doit encore considérer les cas où $\left|x - x_0\right| = \frac{1}{a}$.

    Si $x = x_0 + \frac{1}{a}$, alors on a: 
    \[\sum_{k=0}^{\infty} a^k \frac{1}{a^k} = \sum_{k=0}^{\infty} 1 \text{, qui diverge}\]

    Si $x = x_0 - \frac{1}{a}$, alors: 
    \[\sum_{k=0}^{\infty} a^k \left(\frac{-1}{a}\right)^k = \sum_{k=0}^{\infty} \left(-1\right)^k \text{, qui diverge}\]
    
    On peut donc en déduire que: 
    \[D = \left]x_0 - \frac{1}{a}, x_0 + \frac{1}{a}\right[ \]
}

\parag{Théorème: rayon de convergence}{
    Soit la série entière $\sum_{k=0}^{\infty} a_k\left(x - x_0\right)^{k}$.

    Il existe $r$, son \important{rayon de convergence}, où $0 \leq r \leq +\infty$, tel que:
    \begin{enumerate}
        \item La série converge absolument pour tout $x$ tels que $\left|x - x_0\right| < r$.
        \item La série diverge pour tout $x$ tels que $\left|x - x_0\right| > r$
    \end{enumerate}

    \subparag{Remarque 1}{
        La convergence de la série entière en $x = x_0 \pm r$ doit être étudié séparément.
    }
    
    \subparag{Remarque 2}{
        On voit donc que $D$ est un intervalle qui contient $x_0$, et centré en $x_0$.
    }
    
    \subparag{Remarque 3}{
        Si $r \neq 0$ et $r \in \mathbb{R}_+$, alors $D$ est un des 4 intervalles suivants: 
        \[\left]x_0 - r\right[, \mathspace \left[x_0 - r\right[, \mathspace \left]x_0 - r\right], \mathspace \left[x_0 - r\right]  \]

        Si $r = 0$, alors $D = \left\{x_0\right\}$. Par exemple, la série entière $\sum_{k=0}^{\infty} k! x^k$ ne converge que si $x = 0$, donc $D = \left\{0\right\}$.

        Si $r = +\infty$, alors $D = \mathbb{R}$. Par exemple, la série entière $\sum_{k=0}^{\infty} \frac{x^k}{k!} = e^x$ converge pour tout $x$, donc $r = +\infty$ et $D = \mathbb{R}$.
    }
}

\parag{Proposition}{
    Soit $\sum_{k=0}^{\infty} a_k\left(x - x_0\right)^k$ une série entière de rayon de convergence $r$.
    \begin{enumerate}
        \item Supposons que $a_k \neq 0$ pour tout $k \in \mathbb{N}$. Si $\lim_{k \to \infty} \left|\frac{a_{k+1}}{a_k}\right| = \ell$, où $0 \leq \ell \leq +\infty$, alors $r = \frac{1}{\ell}$.

        On a donc: 
        \[r = \lim_{k \to \infty} \left|\frac{a_k}{a_{k+1}}\right|\]
        
        \item Si $\lim_{k \to \infty} \left|a_k\right|^{\frac{1}{k}} = \ell$ avec $0 \leq \ell \leq +\infty$, alors $r = \frac{1}{\ell}$.
    \end{enumerate}
    
    \subparag{Preuve du point 1}{
        Utilisions le critère de d'Alembert: 
        \[\lim_{x \to \infty} \left|\frac{a_{k+1} \left(x - x_0\right)^{k+1}}{a_k \left(x - x_0\right)^k}\right| = \lim_{k \to \infty} \underbrace{\left|x - x_0\right|}_{\text{constant}} \underbrace{\left|\frac{a_{k+1}}{a_k}\right|}_{\to \ell} = \left|x - x_0\right| \ell\]

        Donc, la série converge absolument si $\left|x - x_0\right|\ell < 1$, ce qui est équivalent à: 
        \[\left|x - x_0\right| < \frac{1}{\ell} = r\]
        
        De la même manière, la série diverge grossièrement si $\left|x - x_0\right| \ell > 1$, ce qui est équivalent à: 
        \[\left|x - x_0\right| > \frac{1}{\ell} = r\]
    }
}

\parag{Exemple}{
    Soit la série entière suivante: 
    \[\sum_{n=1}^{\infty} \frac{x^n}{n2^n} = \sum_{n=1}^{\infty} a_n x^n, \mathspace a_n = \frac{1}{n 2^n}\]
    
    Utilisons la première partie de la proposition: 
    \[\lim_{n \to \infty} \left|\frac{a_n}{a_{n+1}}\right| = \lim_{n \to \infty} \frac{1}{n 2^n} \frac{\left(n + 1\right)2^{n+1}}{1} = \lim_{n \to \infty} \frac{2\left(n+1\right)}{n} = 2\]
    
    Notre rayon de convergence est donc $r = 2$, on en déduit que $\left]-2, 2\right[ \subset D$. Nous devons encore étudier $x = \pm 2$. Commençons par $x = 2$: 
    \[\sum_{n=1}^{\infty} \frac{2^n}{n 2^n} = \sum_{n=1}^{\infty} \frac{1}{n}\]
    qui diverge puisque c'est la série harmonique. Donc, $2 \not \in D$.

    Étudions maintenant $x = -2$: 
    \[\sum_{n=1}^{\infty} \frac{\left(-2\right)^n}{n2^n} = \sum_{n=1}^{\infty} \frac{\left(-1\right)^n}{n}\]
    qui est la série harmonique alternée et qui converge par le critère de Leibniz. Donc, $2 \in D$.
    
    Ainsi, on en déduit que notre domaine de convergence est $D = \left[-2, 2\right[ $, avec un rayon de convergence $r = 2$.
}



\end{document}
