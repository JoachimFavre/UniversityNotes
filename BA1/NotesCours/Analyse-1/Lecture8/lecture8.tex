% !TeX program = lualatex

\documentclass[a4paper]{article}

% Expanded on 2021-10-18 at 10:11:39.

\usepackage{../../style}

\title{Analyse I}
\author{Joachim Favre}
\date{Lundi 18 octobre 2021}

\begin{document}
\maketitle

\lecture{8}{2021-10-18}{Suite de la suite des limites de suites}{
\begin{itemize}[left=0pt]
    \item Démonstration de la limite des racines, et de la limites des suites géométriques.
    \item Démonstration du critère de d'Alembert.
    \item Définition des limites infinies, et de ses propriétés (par exemple, la règle du gendarme).
    \item Explication des formes indéterminées.
    \item Démonstration que $\left(\cos\left(n\right)\right)$ et $\left(\sin\left(n\right)\right)$ ne convergent pas. 
    \item Démonstration du théorème de la convergence des suites monotones.
\end{itemize}

}

\begin{parag}{Limite de racine}
    Soit $a_0 = 1$, et $a_n = \sqrt[n]{a}$ quand $n \geq 1$, pour $a > 0$. Alors: 
    \[\lim_{n \to \infty} a_n = 1\]

    \begin{subparag}{Preuve}
        \important{$a = 1$:} Alors la suite est donnée par $a_n = 1$ pour tout $n$, donc 
        \[\lim_{n \to \infty} a_n = \lim_{n \to \infty} 1 = 1\]
        
        \vspace{1em}
        \important{$a > 1$:} On remarque que 
        \[x^{n} - 1 = \left(x - 1\right)\left(x^{n-1} + x^{n-2} + \ldots + x + 1\right) \ \ \forall x \in \mathbb{R}\ \forall n \in \mathbb{N}^*\]

        De là, on en déduit que, quand $x > 0$: 
        \[x - 1 = \frac{x^{n} - 1}{x^{n-1} + \ldots + x + 1}\]

        Donc, si on prend $x = \sqrt[n]{a} > 1$: 
        \[0 < \sqrt[n]{a} - 1 = \frac{a - 1}{\underbrace{\underbrace{a^{\frac{n-1}{n}}}_{>1} + \ldots + \underbrace{a^{\frac{1}{n}}}_{>1} + 1}_{>n}} < \frac{a - 1}{n}\]
        
        En d'autres mots: 
        \[0 < \sqrt[n]{a} - 1 < \frac{a - 1}{n}\]
        
        Puisque le troisième terme tend vers 0 quand $n \to \infty$, on peut utiliser le théorème des gendarmes :       
        \[\lim_{n \to \infty} \left(\sqrt[n]{a} - 1\right) = 0 \implies \lim_{n \to \infty} \sqrt[n]{a} = 1\]
        
        \vspace{1em}
        \important{$0 < a < 1$ :} Soit $b = \frac{1}{a} > 1$. Par la deuxième partie, on sait que 
        \[\lim_{n \to \infty} \sqrt[n]{b} = 1 \implies \lim_{n \to \infty} \frac{1}{\sqrt[n]{a}} = 1 \implies \lim_{n \to \infty} \sqrt[n]{a} =  \frac{1}{\lim\limits_{n \to \infty} \sqrt[n]{b}} = 1\]
        
    \end{subparag}
\end{parag}

\begin{parag}{Suite géométrique}
    Soit $a_n = a_0 r^{n}$, avec $a_0 \in \mathbb{R}$, $a_0 \neq 0$, $r \in \mathbb{R}$. Alors: 
    \begin{itemize}
        \item $\lim\limits_{n \to \infty} a_0 r^{n} = 0$, $\left|r\right| < 1$
        \item $\lim\limits_{n \to \infty} a_0 r^{n} = a_0$, $r = 1$
        \item $\left(a_n\right)$ diverge quand $\left|r\right| > 1$ ou $r = -1$.
    \end{itemize}
    
    \begin{subparag}{Preuve}
        \begin{enumerate}[left=0pt]
            \item \important{Soit $r > 1$}. Prenons $r = 1 + x$, avec $x > 0$. Donc: 
            \[r^n = \left(1 + x\right)^n = 1 + \binom{n}{1}x + \underbrace{\binom{n}{2}x^2 + \ldots + \binom{n}{n}x^n}_{>0} \geq 1 + \binom{n}{1}x = 1 + nx\]
         
            Par la propriété d'Archimède, on sait que pour tout $M > 0$, $\exists n \in \mathbb{N}$ tel que $(1 + nx) > M$. Donc: 
            \[\left|a_0 r^n\right| = \left|a_0 \left(1 + x\right)^{n}\right| \geq \left|a_0\right|\left|1 + nx\right| > \left|a_0\right|M\]


            On en déduit que la suite n'est pas bornée, et donc divergente.

            \vspace{1em}

            \item \important{Soit $0 < r < 1$}. Soit $q = \frac{1}{r} > 1$. Donc, $\forall M > 0$, on sait que $\exists n_0 \in \mathbb{N}$ tel que $q^{n} > M$ pour tout $n > n_0$.

            Soit $\epsilon > 0$. On choisit $M = \frac{\left|a_0\right|}{\epsilon} $. Ainsi: 
            \[q^{n} > \frac{\left|a_0\right|}{\epsilon} \implies \frac{1}{q^{n}} = r^{n} < \frac{\epsilon}{\left|a_0\right|} \mathspace \forall n \geq n_0\]
            
            En conséquence, 
            \[\left|a_0\right|r^{n} < \epsilon\]

            Par la définition de la limite, on a donc: 
            \[\left|a_0 r^n - 0\right| < \epsilon \implies \lim_{n \to \infty} a_0 r^{n} = 0\]
            
            \item \important{Soit $r = 1$}. On en déduit que 
            \[\lim_{n \to \infty} a_0 r^{n} = \lim_{n \to \infty} a_0 = a_0\]
            
            
        \item \important{Soit $r < 0$}. Exercice au lecteur. Astuce: si $\left|r\right| > 1$, alors par le point (1) on sait que $\left|a_0 r^{n}\right|$ n'est pas bornée. 

                Si $-1 < r < 0$, on peut prendre $q = -r$ donc $\lim_{n \to \infty} a q^{n} = 0$ par le point (2). Donc: 
                \[\lim_{n \to \infty} \underbrace{a_0 \left(-1\right)^{n}}_{\text{borné}} \underbrace{q_n}_{\to 0} = 0\]
        \end{enumerate}
    \end{subparag}
\end{parag}

\begin{parag}{Exemple}
    Soit la suite 
    \[a_n = \frac{5^{n}}{n!}\]

    Alors, on a 
    \[\lim_{n \to \infty} a_n = 0\]
    
    Pour utiliser les mots de la professeure, ``la factorielle est plus forte que l'exponentielle''.

    \begin{subparag}{Preuve}
        Soit $n > 6$. Alors on a: 
        \[a_n = \underbrace{\frac{5}{1} \cdot \frac{5}{2} \cdot \frac{5}{3} \cdot \frac{5}{4} \cdot \frac{5}{5}}_{M} \cdot \underbrace{\frac{5}{6}}_{\leq \frac{5}{6}} \cdot \underbrace{\frac{5}{7}}_{< \frac{5}{6}} \cdot \ldots \cdot \underbrace{\frac{5}{n}}_{< \frac{5}{6}} < M\left(\frac{5}{6}\right)^{n-5} = M\left(\frac{6}{5}\right)^{5} \left(\frac{5}{6}\right)^{n}\]
        
        Donc, 
        \[\lim_{n \to \infty} M\left(\frac{6}{5}\right)^5 \left(\frac{5}{6}\right)^n =  M\left(\frac{6}{5}\right)^5\lim_{n \to \infty} \left(\frac{5}{6}\right)^n = 0\]
        par le théorème des limites de suites géométriques. De plus, notre suite est bornée: 
        \[0 \leq \frac{5^{n}}{n!} \leq M\left(\frac{6}{5}\right)^{n} \left(\frac{5}{6}\right)^{n} \forall n > 6\]
        
        Puisque les deux bornes tendent vers 0 quand $n \to \infty$ alors $a_n$ tend vers 0 par le théorème des gendarmes. 
    \end{subparag}
    
    \begin{subparag}{Remarque}
        De la même manière on en déduit que 
        \[\lim_{n \to \infty} \frac{1000000^n}{n!} = 0\]
    \end{subparag}
\end{parag}

\begin{parag}{Remarques}
    \begin{enumerate}[left=0pt]
        \item Si $\lim\limits_{n \to \infty} x_n = \ell \in \mathbb{R}$, alors:
        \[\lim\limits_{n \to \infty} \left|x_n\right| = \left|\ell\right|\]
        
        \item Si $\lim\limits_{n \to \infty} \left|x_n\right| = 0$. Alors, on sait que 
        \[\lim\limits_{n \to \infty} x_n = 0\]
        
    \item En général $\lim\limits_{n \to \infty} \left|x_n\right| = \ell \neq 0$ n'implique pas que $\left(x_n\right)$ converge.
    \item Si $\left(a_n\right)$ est bornée et $\lim\limits_{n \to \infty} b_n = 0$, alors: 
    \[\lim_{n \to \infty} a_n b_n = 0\]
    
    \end{enumerate}
    
    \begin{subparag}{Démonstrations}
        \begin{enumerate}[left=0pt]
            \item La démonstration est laissée en exercice au lecteur, mais on utilise que $\left|\left|x_n\right| - \left|\ell\right|\right| \leq \left|x_n - \ell\right|$ (qu'il faut aussi démontrer, naturellement).
            \item On sait que pour tout $n \geq n_0$ alors 
        \[0 \leq \left|x_n\right| \leq \epsilon \implies -\epsilon \leq x_n \leq \epsilon \implies \lim_{n \to \infty} x_n = 0\]
    \item On peut prendre $a_n = \left(-1\right)^n$ comme contre-exemple. En effet 
        \[\lim_{n \to \infty} \left|\left(-1\right)^n\right| = \lim_{n \to \infty} 1 = 1\]
        converge, alors que $\left(a_n\right)$ diverge.
    \item Soit $\left(a_n\right)$ borné, c'est à dire que $\left|a_n\right| < M$ pour tout $n \in \mathbb{N}$. Alors on a: 
    \[0 \leq \left|a_n b_n\right| \leq M \left|b_n\right|\]
    
    Or, par la propriété (2), $\left|b_n\right|$ tend vers 0 quand $n \to \infty$. Donc, par les théorème des gendarmes, on sait que 
    \[\lim_{n \to \infty} \left|a_n b_n\right| = 0 \over{\implies}{\text{propriété 2}} \lim_{n \to \infty} a_n b_n = 0 \]
        \end{enumerate}
    \end{subparag}
\end{parag}

\begin{parag}{Théorème (Critère de d'Alembert)}
    Soit $\left(a_n\right)$ une suite telle que $a_n \neq 0$ pour tout $n \in \mathbb{N}$ et 
    \[\lim_{n \to \infty} \left|\frac{a_{n+1}}{a_n}\right| = \rho \geq 0\]
    
    Alors, si $\rho < 1$, $\lim_{n \to \infty} a_n = 0$. Sinon, si $\rho > 1$, alors $\left(a_n\right)$ diverge. Si $\rho = 1$, on ne sait rien.

    \begin{subparag}{Preuve}
        \important{$\rho < 1$:} Par hypothèse, nous avons:
        \[\lim_{n \to \infty} \left|\frac{a_{n+1}}{a_n}\right| = \rho < 1\]
        
        On sait que pour tout $\epsilon > 0$, il existe $n_0 \in \mathbb{N}$ tel que $\forall n \geq n_0$ on a:
        \[\left|\frac{a_{n+1}}{a_n}\right| \leq \rho + \epsilon\]
        
        On choisit $\epsilon$ de telle manière que $\rho + \epsilon < 1$ (on sait qu'il existe puisque $\rho < 1$). Soit $m > n_0$. On sait que 
        \[\left|\frac{a_m}{a_{n_0}}\right| = \underbrace{\left|\frac{a_m}{a_{m-1}}\right|}_{\leq \rho + \epsilon}\underbrace{\left|\frac{a_{m-2}}{a_{m-2}}\right|}_{\leq \rho + \epsilon}\ldots\underbrace{\left|\frac{a_{n_0 + 1}}{a_{n_0}}\right|}_{\leq \rho + \epsilon} \leq \left(\rho + \epsilon\right)^{m - n_0}\]
        
        Donc: 
        \[0 \leq \left|a_m\right| \leq \left(\rho + \epsilon\right)^{m - n_0} \left|a_{n_0}\right| \mathspace \forall m > n_0\]
        
        Or la borne supérieure est une suite géométrique, et elle tend vers 0 quand $n$ tend vers l'infini. Donc, par le théorème des gendarmes, on sait que 
        \[\lim_{n \to \infty} \left|a_m\right| = 0 \implies \lim_{m \to \infty} a_m = 0\]
        
        \vspace{1em}

        \important{$\rho > 1$:} On sait que $\exists n_0 \in \mathbb{N}$ tel que $\forall n \geq n_0$, alors 
        \[\left|\frac{a_{n+1}}{a_n}\right| > \rho - \epsilon\]

        À nouveau, on choisit $\epsilon$ tel que $\rho - \epsilon > 1$ (on sait qu'il existe puisque $\rho > 1$). On a donc: 
        \[\left|\frac{a_m}{a_{n_0}}\right| = \underbrace{\left|\frac{a_m}{a_{m-1}}\right|}_{\geq \rho + \epsilon}\underbrace{\left|\frac{a_{m-2}}{a_{m-2}}\right|}_{\geq \rho + \epsilon}\ldots\underbrace{\left|\frac{a_{n_0 + 1}}{a_{n_0}}\right|}_{\geq \rho + \epsilon} \geq \left(\rho - \epsilon\right)^{m - n_0}\]
        
        En d'autres mots, 
        \[\left|a_m\right| \geq \left(\rho - \epsilon\right)^{m - n_0} \left|a_{n_0}\right|\]
        
        Puisque la borne inférieure diverge, alors $\left(a_n\right)$ est forcément divergente.

        \qed
    \end{subparag}
\end{parag}

\subsection{Limites infinies}
\begin{parag}{Définition}
    On dit que $\left(a_n\right)$ tend vers $+\infty$ si $\forall A > 0$, $\exists n_0 \in \mathbb{N}$ tel que $\forall n \geq n_0$, alors $a_n \geq A$.

    On note: 
    \[\lim_{n \to \infty} a_n = \infty\]
\end{parag}

\begin{parag}{Définition}
    On dit que $\left(a_n\right)$ tend vers $-\infty$ si $\forall A > 0$, $\exists n_0 \in \mathbb{N}$ tel que $\forall n \geq n_0$, alors $a_n \leq -A$.

    On note: 
    \[\lim_{n \to \infty} a_n = -\infty\]
\end{parag}

\begin{parag}{Terminologie}
   On ne dit pas que des limites ``convergent vers l'infini'', puisqu'elles divergent. On dit qu'elles ``tendent vers l'infini''.
\end{parag}

\begin{parag}{Propriétés}
    \begin{enumerate}[left=0pt]
        \item Si $\lim\limits_{n \to \infty} a_n = \infty = \lim\limits_{n \to \infty} b_n$, alors:
        \[\lim\limits_{n \to \infty} \left(a_n + b_n\right) = \infty\]

        \item $\lim\limits_{n \to \infty} a_n = \pm \infty$ et $\left(b_n\right)$ est bornée, alors 
        \[\lim_{n \to \infty} \left(a_n \pm b_n\right) = \pm \infty\]
        \item Si $\lim\limits_{n \to \infty} b_n = \infty$ et $a_n \geq b_n$, alors 
        \[\lim_{n \to \infty} a_n = \infty\]
        
        \item Si $\lim\limits_{n \to \infty} b_n = -\infty$ et $a_n \leq b_n$, alors 
        \[\lim_{n \to \infty} a_n = -\infty\]
        
        \item Si $\left(a_n\right)$ est bornée et $\lim\limits_{n \to \infty} b_n = \pm \infty$, alors 
        \[\lim_{n \to \infty} \frac{a_n}{b_n} = 0\]
    
        \item Si $\lim\limits_{n \to \infty} \left|\frac{a_{n+1}}{a_n}\right| = \infty$ et $a_n \neq 0$ pour tout $n$. Alors, $\left(a_n\right)$ diverge.
    \end{enumerate}
    
    On appelle les propriété 3 et 4 ``la règle d'un gendarme'', par opposition à la ``règle des deux gendarmes'' pour les limites finies.

    La démonstration du point $5$ est laissée au lecteur.

    Le point 6 est une continuation au critère de d'Alembert.
\end{parag}

\begin{parag}{Formes indéterminées}
    \begin{enumerate}[left=0pt]
        \item $\infty - \infty$
        \item $0\cdot \infty$
        \item $\frac{\infty}{\infty}$
        \item $\frac{0}{0}$
        \item $1^{\infty}$
        \item $0^{0}$
        \item $\infty^{0}$
    \end{enumerate}
    
\end{parag}

\begin{parag}{Suites trigonométriques}
    $\left(\cos n\right)$ et $\left(\sin n\right)$ sont divergentes (mais bornées).

    \begin{subparag}{Démonstration}
    Supposons par l'absurde que $\exists \lim_{n \to \infty} \cos n = \ell$. L'ensemble 
    \[\left\{\cos\left(n\right), \cos\left(n+1\right), \cos\left(n+2\right), \cos\left(n+3\right), \cos\left(n+4\right)\right\}\]
    contient toujours des valeurs positives et négatives pour tout $n$, puisque $3 \leq \pi < 4$. Il existe donc un nombre infini de valeurs positives et négatives, ce qui implique que la limite ne peut ni être positive ni être négative. Donc, on en déduit que nécessairement 
    \[\lim_{n \to \infty} \cos\left(n\right) = 0\]

    Par le même argument, on en déduit aussi que 
    \[\lim_{n \to \infty} \sin\left(n\right) = 0\]
    
    Cependant, cela veut donc dire que 
    \[\lim_{n \to \infty} \underbrace{\left(\sin^2\left(n\right) + \cos^2\left(n\right)\right)}_{= 1 \ \forall n \in \mathbb{N}} = 0\]
    
    Ce qui est une contradiction.
    \qed
    \end{subparag}
\end{parag}

\begin{parag}{Théorème (convergence des suites monotones)}
    Toute suite croissante qui est majorée converge vers le supremum de son ensemble. De la même manière, toute suite décroissante qui est minorée converge vers l'infimum de son ensemble.

    De plus, toute suite croissante qui n'est pas majorée tend vers $+\infty$ (diverge). De manière similaire, toute suite décroissante qui n'est pas minorée tend vers $-\infty$ (diverge).

    \begin{subparag}{Notation}
        $\left(a_n\right)\uparrow$ veut dire que $\left(a_n\right)$ est croissante.

        $\left(a_n\right)\downarrow$ veut dire que $\left(a_n\right)$ est décroissante.
    \end{subparag}

    \begin{subparag}{Preuve}
        Soit $\left(a_n\right)\uparrow$ une suite majorée. Alors il existe $\ell = \sup\left\{a_n, n \in \mathbb{N}\right\}$. En d'autres mots, $a_n \leq \ell$ pour tout $n$, et:
        \[\forall \epsilon > 0, \exists n_0 \in \mathbb{N} \text{ tel que } 0 \leq \ell - a_{n_0} \leq \epsilon\]

        Puisque $\left(a_n\right)$ est croissante, et puisque $\ell$ est un majorant, on sait que $\forall n \geq n_0$, alors $\ell \geq a_n \geq a_{n_0}$. Ainsi: 
        \[\forall n \geq n_0 \mathspace 0 \leq \ell - a_n \leq \ell - a_{n_0} \leq \epsilon \over{\implies}{\text{déf}} \lim_{n \to \infty} a_n = \ell \]

        Soit $\left(a_n\right)\uparrow$ une suite non majorée. Alors, $\forall A > 0$, $\exists n_0$ tel que $a_{n_0} \geq A$. De plus, on sait que $\left(a_n\right)$ est croissante. Donc, $a_n \geq a_{n_0} \geq A$ pour tout $n \geq n_0$. Par définition, on a donc que 
        \[\lim_{n \to \infty} a_n = \infty\]
        
        Le cas de $\left(a_n\right)\downarrow$ est similaire.

        \qed
    \end{subparag}
    
\end{parag}

\end{document}
