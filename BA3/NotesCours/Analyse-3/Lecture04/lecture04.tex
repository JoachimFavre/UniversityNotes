% !TeX program = lualatex
% Using VimTeX, you need to reload the plugin (\lx) after having saved the document in order to use LuaLaTeX (thanks to the line above)

\documentclass[a4paper]{article}

% Expanded on 2022-10-17 at 08:17:25.

\usepackage{../../style}

\title{Analyse}
\author{Joachim Favre}
\date{Lundi 17 octobre 2022}

\begin{document}
\maketitle

\lecture{4}{2022-10-17}{La gauche c'est positif}{
\begin{itemize}[left=0pt]
    \item Beaucoup d'exemples de champs qui dérivent et qui ne dérivent pas de potentiels.
    \item Définition de bord, d'adhérence et de domaine régulier.
    \item Explication et preuve du théorème de Green.
\end{itemize}

}

\parag{Intuition}{
    Si je fais un tour à vélo et que j'ai toujours le vent contraire, c'est que le vent ne dérivait pas d'un potentiel.

    En effet, mathématiquement, le total du vent sur notre tour fermé à vélo $\Gamma$ donne: 
    \[\int_{\Gamma} \text{vent}\left(\gamma\left(t\right)\right) \dotprod \gamma'\left(t\right) dt < 0 \mathspace \text{ puisque } \text{vent}\left(\gamma\left(t\right)\right) \dotprod \gamma'\left(t\right) dt < 0, \forall t\]
}

\parag{Exemple 1}{
    Soit le champ suivant: 
    \[F\left(x, y\right) = \left(-y, x\right) \in C^{\infty}\left(\mathbb{R}^2; \mathbb{R}^2\right)\]
    
    La première étape est de calculer le rotationnel: 
    \[\rot F = \frac{\partial F_2}{\partial x} - \frac{\partial F_1}{\partial y} = 1 - \left(-1\right) = 2 \neq 0\]
    
    Donc, $F$ ne dérive pas d'un potentiel.
}

\parag{Exemple 2}{
    Considérons le champ suivant: 
    \[F\left(x, y\right) = \left(\frac{-y}{x^2 + y^2}, \frac{x}{x^2 + y^2}\right) \in C^1\left(\Omega; \mathbb{R}^2\right), \mathspace \text{où } \Omega = \mathbb{R}^2 \setminus \left\{\left(0, 0\right)\right\}\]

    À nouveau, commençons par calculer le rotationnel: 
    \autoeq{\unexpanded{\rot F = \frac{\partial }{\partial x} \left(\frac{x}{x^2 + y^2}\right) - \frac{\partial }{\partial y} \frac{-y}{x^2 + y^2} = \frac{x^2 + y^2 - 2x^2 + x^2 + y^2 - 2y^2}{\left(x^2 + y^2\right)^2} = 0}}
    
    Si $\Omega$ était étoilé (ou simplement connexe), nous pourrions directement conclure que $F$ dérive d'un potentiel. Cependant, en l'occurrence, il a un trou au milieu, ce qui fait qu'il n'est pas simplement connexe.
    
    La deuxième étape est de garder espoir et essayer d'utiliser la méthode de la variation de la constante pour trouver $f$ telle que $F = \nabla f$. Mais, malheureusement, la troisième étape est de remarquer que ça ne marche pas et de perdre espoir. Ceci nous amène à l'étape suivante.

    La quatrième étape est de démontrer que notre champ ne dérive pas d'un potentiel en trouvant une courbe simple fermée $\Gamma$ telle que: 
    \[\int_{\Gamma} F \dotprod d \ell \neq 0\]
    
    Pour se faire, on peut simplement prendre le cercle unité: 
    \autoeq{\unexpanded{\int_{\Gamma} F \dotprod d \ell = \int_{0}^{2\pi} F\left(\cos\left(t\right), \sin\left(t\right)\right) \dotprod \left(-\sin\left(t\right), \cos\left(t\right)\right) dt = \int_{0}^{2\pi} \left(-\sin\left(t\right), \cos\left(t\right)\right) \dotprod \left(-\sin\left(t\right), \cos\left(t\right)\right)dt = \int_{0}^{2\pi} \left(\sin^2\left(t\right) + \cos^2\left(t\right)\right)dt = \int_{0}^{2\pi} dt = 2\pi}}
    ce qui n'est pas nul.
    
    Ainsi, nous avons bien montré que $F$ ne dérive pas d'un potentiel.
}

\parag{Exemple 3}{
    Considérons à nouveau le champ vectoriel suivant, mais sur un domaine différent:
    \[F\left(x, y\right) = \left(\frac{-y}{x^2 + y^2}, \frac{x}{x^2 + y^2}\right) \in C^1\left(\Omega; \mathbb{R}^2\right), \mathspace \text{où } \Omega = \left\{\left(x, y\right) | y > 0\right\}\]

    Nous n'avons pas changé notre champ, donc $\rot F = 0$. Cependant, maintenant, nous domaine est étoilé, ce qui nous permet de conclure par le théorème de la caractérisation par le rotationnel que $F$ dérive d'un potentiel.

    \subparag{Remarque}{
        Cet exemple nous montre que la géométrie du domaine est importante.
    }
}

\parag{Exemple 4}{
    Prenons le champ vectoriel suivant: 
    \[F\left(x, y\right) = \left(\frac{x}{x^2 + y^2}, \frac{y}{x^2 + y^2} + 1\right)\]
    avec $\Omega = \mathbb{R}^2 \setminus \left\{\left(0, 0\right)\right\}$.

    Nous pouvons trouver que $\rot F = 0$, mais que $\Omega$ n'est pas simplement connexe. 

    Passons maintenant à la deuxième étape:  Supposons que $F = \nabla f$. Alors, nous aurions: 
    \autoeq{\unexpanded{\frac{\partial f}{\partial x} = \frac{x}{x^2 + y^2} \implies f\left(x, y\right) = \int\frac{x}{x^2 + y^2}dx + c\left(y\right) = \frac{1}{2} \log\left|x^2 + y^2\right| + c\left(y\right) = \frac{1}{2} \log\left(x^2 + y^2\right) + c\left(y\right)}}

    Regardons maintenant la dérivée selon $y$: 
    \[\frac{\partial f}{\partial y} = \frac{y}{x^2 + y^2} + c'\left(y\right) \over{=}{déf $f$}  \frac{y}{x^2 + y^2} + 1 \implies c'\left(y\right) = 1 \implies c\left(y\right) = y + c_2\]
    
    Nous avons ainsi trouvé que $F$ dérive de: 
    \[f\left(x, y\right) = \frac{1}{2} \log\left(x^2 + y^2\right) + y + c_2\]
}

\parag{Exemple 5}{
    Considérons le champ vectoriel suivant: 
    \[F\left(x, y, z\right) = \left(\frac{x}{\left(x^2 + y^2\right)^2} + \frac{x}{\left(x^2 + z^2\right)^2}, \frac{y}{\left(x^2 + y^2\right)^2} + \frac{y}{\left(y^2 + z^2\right)^2}, \frac{z}{\left(x^2 + z^2\right)^2} + \frac{z}{\left(y^2 + z^2\right)^2}\right)\]
    appartenant à $C^1\left(\Omega; \mathbb{R}^3\right)$, avec:
    \[\Omega = \mathbb{R}^3 \setminus \left\{\left(x, y, z\right) | \left(x = 0 \text{ et } y = 0\right) \text{ ou } \left(x = 0 \text{ et } z = 0\right) \text{ ou } \left(y = 0 \text{ et } z = 0\right)\right\}\]

    En d'autres mots, $\Omega$ est $\mathbb{R}^3$ sans les axes.

    La première étape est de calculer le rotationnel, ce qui nous donne (nous pouvons accélérer ce calcul en se rendant compte de la symétrie entre chaque composante de $\rot F$): 
    \[\rot F = 0\]
    
    Cependant, notre domaine n'est pas simplement connexe, donc nous passons à l'étape 2. Si $F = \nabla f$, alors nous avons: 
    \[\frac{\partial f}{\partial x} = \frac{x}{\left(x^2 + y^2\right)^2} + \frac{x}{\left(x^2 + z^2\right)^2} \implies f\left(x, y, z\right) = -\frac{1}{2} \frac{1}{x^2 + y^2} - \frac{1}{2} \frac{1}{x^2 + z^2} + c\left(y, z\right)\]
    
    Considérons maintenant $y$: 
    \autoeq{\unexpanded{\frac{\partial f}{\partial y} = \frac{y}{\left(x^2 + y^2\right)^2} + \frac{\partial c}{\partial y} \left(y, z\right) \over{=}{déf $f$} \frac{y}{\left(x^2 + y^2\right)^2} + \frac{y}{\left(y^2 + z^2\right)^2} \implies \frac{\partial c\left(y, z\right)}{\partial y} = \frac{y}{\left(y^2 + z^2\right)^2} \implies c\left(y, z\right) = \int \frac{y}{\left(y^2 + z^2\right)^2} dy + c_2\left(z\right) \implies f\left(x, y, z\right) = -\frac{1}{2} \frac{1}{x^2 + y^2} - \frac{1}{2} \frac{1}{x^2 + z^2} - \frac{1}{2} \frac{1}{y^2 + z^2} + c_2\left(z\right)}}
    
    Utilisons finalement $z$: 
    \autoeq{\unexpanded{\frac{\partial f}{\partial z} = \frac{z}{\left(x^2 + z^2\right)^2} + \frac{z}{\left(y^2 + z^2\right)^2} + c_2'\left(z\right) \over{=}{déf $f$} \frac{z}{\left(x^2 + z^2\right)^2} + \frac{z}{\left(y^2+ z^2\right)^2} \implies c_2'\left(z\right) = 0 \implies c_2\left(z\right) = c_3}}
    
    Nous trouvons ainsi finalement que notre fonction $F$ dérive du potentiel suivant: 
    \[f\left(x, y, z\right) = -\frac{1}{2}\left(\frac{1}{x^2 + y^2} + \frac{1}{x^2 + z^2} + \frac{1}{y^2 + z^2}\right) + c_3\]
}

\subsection{Théorème de Green}
\parag{Définition: Bord}{
    Soit $\Omega \in \mathbb{R}^n$ un ouvert. Alors, le \important{bord} de $\Omega$, noté $\partial \Omega$, est défini par: 
    \[\partial \Omega = \left\{x \in \mathbb{R}^n | \forall r > 0, B_r\left(x\right) \cap \Omega \neq \o \text{ et } B_r\left(x\right) \cap \Omega^C \neq \o\right\}\]
    où $B_r\left(x\right)$ est la boule ouverte de rayon $r$ centrée en $x$.
    
    \subparag{Exemple}{
        Par exemple, sur l'image suivante, les points rouges ne font pas partie de $\partial \Omega$, mais le point vert oui:
        \svghere[0.5]{BordDefinition.svg}
    }
}

\parag{Définition: Adhérence}{
    Soit $\Omega \in \mathbb{R}^n$ un ouvert. L'\important{adhérence} de $\Omega$, noté $\bar{\Omega}$, est donné par: 
    \[\bar{\Omega} = \Omega \cup \partial \Omega\]
}

\parag{Exemple}{
    Considérons par exemple le domaine ouvert suivant, celui d'un disque de rayon 1: 
    \[\Omega = \left\{\left(x, y\right) \in \mathbb{R}^2 | x^2 + y^2 < 1\right\}\]
    
    Alors, nous avons: 
    \[\partial \Omega = \left\{\left(x, y\right) |x^2 + y^2 = 1\right\}, \mathspace \bar{\Omega} = \left\{\left(x, y\right) | x^2 + y^2 \leq 1\right\}\]
}

\parag{Définition: Domaine régulier}{
    $\Omega \in \mathbb{R}^2$ est un \important{domaine régulier} s'il existe des ouverts bornés $\Omega_0, \ldots, \Omega_k$ tels que:
    \begin{itemize}
        \item $\displaystyle\bar{\Omega_i} \subseteq \Omega_0$ pour tout $1 \leq i \leq k$.
        \item $\displaystyle\bar{\Omega_i} \cap \bar{\Omega_j} = \o$ pour $1 \leq i < j \leq k$.
        \item $\displaystyle\partial \Omega_i$ est une courbe régulière par morceau simple fermée pour tout $0 \leq i \leq k$.
        \item $\displaystyle \Omega = \Omega_0 \setminus \bigcup_{i=1}^k \Omega_i$
    \end{itemize}
    
    Le bord $\partial \Omega$ est orienté positivement si les paramétrisations choisies laissent le domaine $\Omega$ à gauche.

    \subparag{Intuition}{
        Notre domaine régulier $\Omega$ est défini par un ensemble $\Omega_0$ avec des ``trous'' $\Omega_1, \ldots, \Omega_k$.
        \svghere[0.5]{DomaineRegulierExemple.svg}

        Il est intéressant de remarquer que le bord ``des trous'' ne tournent pas dans le même sens que le bord de ``l'extérieur''.
    }

    \subparag{Remarque}{
        Donner une orientation au bord $\partial \Omega$ permet d'enlever l'ambigüité qu'on avait sur le signes des intégrales: une intégrale ne dépend plus de la paramétrisation maintenant.
    }
}

\parag{Théorème de Green}{
    Soit $\Omega \subseteq \mathbb{R}^2$ un domaine régulier avec le bord $\partial \Omega$ orienté positivement. Soit aussi $F \in C^1\left(\Omega; \mathbb{R}^2\right)$.

    Alors: 
    \[\iint_{\Omega} \rot F\left(x, y\right) dxdy = \int_{\partial\Omega} F \dotprod d \ell \]
    
    Nous relions donc une intégrale multiple (de dimension 2) à une intégrale le long d'une courbe. 

    \subparag{Remarque}{
        Si $F$ dérive d'un potentiel, alors le théorème de Green se lit: 
        \[0 = 0\]
    }

    \subparag{Preuve triangle isocèle rectangle}{
        Nous considérons le cas d'un isocèle rectangle, dont les sommets sont en $\left(0, 0\right)$, $\left(0, 1\right)$ et en $\left(1, 0\right)$.
        \svghere[0.6]{TriangleTheoremeGreen.svg}
        Nous avons donc: 
        \autoeq{\unexpanded{\Omega = \left\{\left(x, y\right) \in \mathbb{R}^2 | 0 < x < 1 \text{ et } 0 < y < 1 - x\right\} = \left\{\left(x, y\right)\in \mathbb{R}^2 | 0 < y < 1 \text{ et } 0 < x < 1 - y\right\}}}
        
        Ceci nous donne que: 
        \autoeq{\unexpanded{\iint_{\Omega} \rot F dxdy = \iint_{\Omega} \partial_x F_2 dxdy - \iint_{\Omega} \partial_y F_1 dxdy = \int_{0}^{1} \int_{0}^{1-y} \partial_x F_2 dxdy - \int_{0}^{1} \int_{0}^{1-x} \partial_y F_1 dydx = \int_{0}^{1} \underbrace{F_2\left(1 - y, y\right)dy}_{t = y} - \int_{0}^{1} \underbrace{F_2\left(0, y\right)dy}_{t = 1 - y} \fakeequal - \int_{0}^{1} \underbrace{F_1\left(x, 1- x\right)dx}_{t = 1 - x} + \int_{0}^{1} \underbrace{F_1\left(x, 0\right)dx}_{t = x}}}


        En faisant les changements de variables mentionnés sous chaque intégrale, nous obtenons que c'est égal à (en faisant attention aux bornes pour avoir les signes correctement):
        \autoeq[s]{\unexpanded{\int_{0}^{1} F_1\left(t, 0\right)dt + \int_{0}^{1} \left[-F_1\left(1-t, t\right) + F_2\left(1 - t, t\right)\right]dt \fakeequal + \int_{0}^{1} F_2\left(0, 1-t\right)\left(-1\right)dt}}

        Ce qui est exactement:
        \[\int_{\Gamma_1} F \dotprod d \ell + \int_{\Gamma_2} F \dotprod d \ell + \int_{\Gamma_3} F \dotprod d \ell\]
        comme attendu.
    }

    \subparag{Preuve triangle quelconque}{
        Le cas des triangles généraux sera fait dans la cinquième série d'exercices.
    }
    
    \subparag{Preuve ouvert quelconque}{
        On considère un $\Omega$ général. Dans l'idée, nous pouvons trianguler notre domaine, l'approximer avec des triangles de plus en plus fins.
        \svghere[0.5]{TheoremGreenTriangulisation.svg}

        On trouve ainsi, approximativement: 
        \autoeq{\unexpanded{\iint_{\Omega} \rot F \approx \sum_{\text{triangles}}^{} \iint_{\text{triangle}} \rot F = \sum_{\text{triangles}}^{} \int_{\partial \text{triangle}} F \dotprod d \ell = \int_{\text{''extérieur''}} F \dotprod d \ell \approx \int_{\partial \Omega} F \dotprod d \ell}}
        
        En effet, chaque ligne à ``l'intérieur'' de notre triangulation est toujours au bord d'exactement deux triangles, et une fois de manière positive et une fois de manière négative. Ceci nous montre que la somme des intégrales sur les bords nous donne bien l'intégrale des bords de l'extérieur.
    }
    
}

\parag{Exemple}{
    Soit le domaine suivant: 
    \[\Omega = B_1\left(0\right) = \left\{\left(x, y\right) \in \mathbb{R}^2 | x^2 + y^2 < 1\right\}\]
    
    Nous prenons aussi le champ suivant: 
    \[F\left(x, y\right) = \left(-y, x\right)\]
    
    Nous avions déjà calculé que $\rot F = 2$. Cela nous donne: 
    \[\iint_{\Omega} \rot F \left(x,y\right) dxdy = \iint_{\Omega} 2dxdy = 2\iint_{\Omega}dxdy = 2 \text{Aire}\left(\Omega\right) = 2\pi\]
    
    De plus, on peut poser la paramétrisation de $\partial \Omega$ suivante:
    \[\begin{split}
    \gamma: \left[0, 2\pi\right] &\longmapsto \partial \Omega \\
    t &\longmapsto \left(\cos\left(t\right), \sin\left(t\right)\right)
    \end{split}\]
    
    Nous avons bien que $\partial \Omega$ est orienté positivement, ainsi: 
    \[\int_{\partial \Omega} F \dotprod d \ell = \int_{0}^{2\pi} \left(-\sin\left(t\right), \cos\left(t\right)\right) \dotprod \left(-\sin\left(t\right), \cos\left(t\right)\right) = \int_{0}^{2\pi} 1dt = 2\pi\]
    
    Le théorème de Green est donc bien vérifié dans ce cas.
}


\end{document}
