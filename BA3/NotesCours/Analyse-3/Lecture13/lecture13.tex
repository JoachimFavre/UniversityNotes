% !TeX program = lualatex
% Using VimTeX, you need to reload the plugin (\lx) after having saved the document in order to use LuaLaTeX (thanks to the line above)

\documentclass[a4paper]{article}

% Expanded on 2022-12-19 at 08:17:31.

\usepackage{../../style}

\title{Analyse 3}
\author{Joachim Favre}
\date{Lundi 19 décembre 2022}

\begin{document}
\maketitle

\lecture{13}{2022-12-19}{Venez tous en Analyse 4}{
\begin{itemize}[left=0pt]
    \item Application des transformées de Fourier à l'équation de la chaleur.
    \item Résumé du cours.
\end{itemize}

}

\parag{Exemple 4: Équation de la chaleur}{
    Nous considérons une barre de métal infinie qu'on chauffe à certains endroits. La température dans la barre change forcément en fonction du temps, car elle se propage. Nous voulons savoir comment $u\left(x, t\right)$ évolue (où $u$ est la température, $x$ est la position et $t$ et le temps). Si les voisins d'un point sont plus froid, alors il va se refroidir; et inversement. Ceci est représenté mathématiquement par la dérivée seconde de la fonction. On trouve ainsi que:
    \[\frac{\partial u}{\partial t}\left(x, t\right) = \frac{\partial^2 u}{\partial x^2} \left(x, t\right) \]
    avec la condition initiale $u\left(x, 0\right) = f\left(x\right)$, qui est la température initiale de la barre.

    Notons $\hat{u}\left(\alpha, t\right)$, la transformée de Fourier spatiale (par rapport à $x$) de $u$, i.e: 
    \[\hat{u}\left(\alpha, t\right) = \frac{1}{\sqrt{2\pi}} \int_{-\infty}^{\infty} u\left(x, t\right) e^{-i\alpha x}dx\]
    
    Nous pouvons remarquer que: 
    \autoeq{\unexpanded{\mathcal{F}\left(\frac{\partial u}{\partial t}\right)\left(\alpha, t\right) = \frac{1}{\sqrt{2\pi}} \int_{-\infty}^{\infty} \frac{\partial u}{\partial t} \left(x, t\right) e^{-i \alpha x} dx = \frac{\partial }{\partial t} \left[\frac{1}{\sqrt{2\pi}} \int_{-\infty}^{\infty} u\left(x, t\right) e^{-i \alpha x}\right]dx = \frac{\partial}{\partial t} \hat{u}\left(\alpha, t\right)}}
    
    De plus, par les propriétés des transformées de Fourier: 
    \[\mathcal{F}\left(\frac{\partial^2 u}{\partial x^2} \right)\left(\alpha, t\right) = -\alpha ^2 \hat{u}\left(\alpha, t\right)\]
    
    Ainsi, en prenant une transformée de Fourier de chaque côté, notre équation devient: 
    \[\frac{\partial \hat{u}}{\partial t} \left(\alpha, t\right) = -\alpha ^2 \hat{u}\left(\alpha, t\right)\]
    
    Ceci est une EDL1 que nous pouvons résoudre facilement: 
    \[\hat{u}\left(\alpha, t\right) = c\left(\alpha\right) e^{-\alpha ^2}\]

    $c\left(\alpha\right)$ est une constante selon le temps, mais pourrait varier spatialement. Pour la trouver, nous pouvons utiliser la condition initiale $u\left(x, 0\right) = f\left(x\right)$: 
    \[c\left(\alpha\right) = \hat{u}\left(\alpha, 0\right) = \hat{f}\left(\alpha\right) \implies \hat{u}\left(\alpha, t\right) = \hat{f}\left(\alpha\right) e^{-\alpha ^2 t}\]
    
    Maintenant, nous pouvons appliquer la transformée de Fourier inverse: 
    \[u\left(x, t\right) = \mathcal{F}^{-1}\left(\hat{f}\left(\alpha\right) e^{-\alpha ^2 t}\right)\left(x\right) = \frac{1}{\sqrt{2\pi}} f * \mathcal{F}^{-1}\left(e^{-\alpha ^2 t}\right)\left(x\right)\]
    
    Nous devons trouver la transformée de Fourier inverse de $e^{-\alpha ^2 t}$. Elle ressemble à $e^{-\frac{\alpha ^2}{2}}$, donc essayons de le mettre en valeur: 
    \[e^{- \alpha ^2 t} = e^{- \frac{\left(\sqrt{2t} \alpha\right)^2}{2}} = \mathcal{F}\left(e^{-\frac{x^2}{2}}\right)\left(\sqrt{2t} \alpha\right) = \mathcal{F}\left(\frac{1}{\sqrt{2t}} e^{-\left(\frac{x}{\sqrt{2t}}\right)^2 \frac{1}{2}}\right)\left(\alpha\right) = \mathcal{F}\left(\frac{1}{\sqrt{2t}} e^{-\frac{-x^2}{4t}}\right)\left(\alpha\right)\]

    On trouve ainsi finalement que: 
    \autoeq{\unexpanded{u\left(x, t\right) = \frac{1}{\sqrt{2\pi}} \left(f\left(x\right) * \frac{1}{\sqrt{2t}} e^{- \frac{x^2}{4t}}\right)\left(x\right) = \frac{1}{2\sqrt{\pi t}} \int_{-\infty}^{\infty} f\left(x, y\right) e^{-\frac{y^2}{4t}}dy = \frac{1}{2\sqrt{\pi t}} \int_{-\infty}^{\infty} \exp\left(- \frac{\left(x- y\right)^2}{4t}\right) f\left(y\right)dy}}
}

\section{Résumé}
\subsection{Opérateurs différentiels}
\parag{Définition: Gradient}{
    Soit $f: \Omega \mapsto \mathbb{R}$ une fonction. Le \important{gradient} de $f$ est: 
    \[\nabla f\left(x\right) = \left(\frac{\partial f}{\partial x_1} \left(x\right), \ldots, \frac{\partial f}{\partial x_n} \left(x\right)\right) \in \mathbb{R}^n\]
}


\parag{Définition: Laplacien}{
    Soit $f: \Omega \mapsto \mathbb{R}$ une fonction. Le \important{Laplacien} de $f$ est: 
    \[\Delta f\left(x\right) = \frac{\partial^{2} f}{\partial x_1^{2}}\left(x\right) + \ldots + \frac{\partial^{2} f}{\partial x_n^{2}}\left(x\right) = \sum_{i=1}^{n} \frac{\partial^{2} f}{\partial x_i^{2}}\left(x\right) \in \mathbb{R}\]
}


\parag{Définition: Divergence}{
    Soient $\Omega \subset \mathbb{R}^n$ et $F = \left(F_1, \ldots, F_n\right): \Omega \mapsto \mathbb{R}^n$, alors la \important{divergence} de $F$ est définie par:
    \[\Div F\left(x\right) = \frac{\partial F_1}{\partial x_1} \left(x\right) + \ldots + \frac{\partial F_n}{\partial x_n} \left(x\right) \in \mathbb{R}\]
}


\parag{Définition: Rotationnel}{
    Soient $\Omega \in \mathbb{R}^n$ et $F = \left(F_1, \ldots, F_n\right) : \Omega \mapsto \mathbb{R}^n$.

    Si $n = 2$, alors le \important{rotationnel} de $F$ est donné par: 
    \[\rot F\left(x\right) = \frac{\partial F_2}{\partial x_1} \left(x\right) - \frac{\partial F_1}{\partial x_2} \left(x\right) \in \mathbb{R}\]
    
    Si $n = 3$, alors il est donné par: 
    \[\rot F\left(x\right) = \left(\frac{\partial }{\partial x_1} , \frac{\partial }{\partial x_2} , \frac{\partial }{\partial x_3} \right) \wedge \left(F_1, F_2, F_3\right)\]
}

\subsection{Intégrales curvilignes}
\parag{Définition: Courbe régulière}{
    Un sous-ensemble $\Gamma \subseteq \mathbb{R}^n$ est une \important{courbe régulière} s'il existe une fonction:
    \[\begin{split}
    \gamma: \left[a, b\right] &\longmapsto \mathbb{R}^n \\
    t &\longmapsto \gamma\left(t\right) = \left(\gamma_1\left(t\right), \ldots, \gamma_n\left(t\right)\right)
    \end{split}\]
    
    Telle que:
    \begin{enumerate}
        \item $\gamma\left(\left[a, b\right]\right) = \Gamma$
        \item $\gamma$ est continue sur $\left[a, b\right]$ et différentiable sur $\left]a, b\right[ $.
        \item $\left\|\gamma'\left(t\right)\right\| \neq 0$ pour tout $t \in \left[a, b\right]$.
    \end{enumerate}
    
    Un tel $\gamma$ s'appelle une paramétrisation de $\Gamma$.

    \subparag{Normale}{
        Soit $\Omega \in \mathbb{R}^2$ un domaine régulier de bord orienté positivement. La normale extérieure d'une courbe $\gamma : \left[a, b\right] \mapsto \partial \Omega$ est donnée par: 
        \[\nu_{\gamma\left(t\right)} = \frac{\left(\gamma_2'\left(t\right), -\gamma'_1\left(t\right)\right)}{\left\|\gamma\left(t\right)\right\|}, \mathspace \forall t \in \left[a, b\right]\]
    }
    
}

\parag{Définition: Types de courbes}{
    Une courbe \important{régulière par morceau} est des morceaux de courbes régulières qui sont collés ensembles. Cela revient à autoriser un nombre fini d'endroits où la dérivée s'annule.

    Une courbe \important{simple} est une courbe qui ne s'intersecte pas.

    Une courbe \important{fermée} est une courbe qui s'arrête là où elle commence.
}


\parag{Définition: Intégrale curviligne}{
    Soient $\Omega \subset \mathbb{R}^n$ un ensemble ouvert, $\Gamma \subset \Omega$ une courbe régulière, et $\gamma : \left[a, b\right] \mapsto \Gamma$ une paramétrisation de cette courbe.

    Pour $f \in C^0\left(\Omega\right)$, nous définissons l'intégrale curviligne de $f$ le long de $\Gamma$:
    \[\int_\Gamma f d \ell = \int_{a}^{b} f\left(\gamma\left(t\right)\right) \left\|\gamma'\left(t\right)\right\| dt\]
    
    Pour un champ vectoriel, $F \in C^0\left(\Omega, \mathbb{R}^n\right)$, nous définissons: 
    \[\int_\Gamma F \dotprod d \ell = \int_{a}^{b} F\left(\gamma\left(t\right)\right) \dotprod \gamma'\left(t\right) dt\]

    La deuxième intégrale dépend du sens de parcours.
}

\parag{Définition: Dériver d'un potentiel}{
    Soient $\Omega \subset \mathbb{R}^n$ un ensemble ouvert, et $F \in C^{0}\left(\Omega;\mathbb{R}^n\right)$.

    Si $F = \nabla f$ pour une $f \in C^1\left(\Omega\right)$, also le champ vectoriel $F$ \important{dérive du potentiel} $f$.
}

\parag{Proposition}{
    Si $F \in C^{1}\left(\Omega; \mathbb{R}^n\right)$ dérive du potentiel $f \in C^{1}\left(\Omega\right)$ et $\Gamma$ est une courbe de régulière de paramétrisation $\gamma : \left[a, b\right] \mapsto \Omega$, alors: 
    \[\int_{\Gamma} F \dotprod d \ell = f\left(\gamma\left(b\right)\right) - f\left(\gamma\left(a\right)\right)\]
}

\parag{Corollaire}{
    Soit $F \in C^{1}\left(\Omega; \mathbb{R}^n\right)$ une fonction dérivant d'un potentiel, et $\Gamma$ une courbe de régulière. Si $\Gamma$ est fermée, alors: 
    \[\int_{\Gamma} F \dotprod d \ell = 0\]
}


\parag{Proposition}{
    Soient $n \in \left\{2, 3\right\}$, $\Omega \subset \mathbb{R}^n$ un ensemble ouvert, et $F \in C^1\left(\Omega; \mathbb{R}^n\right)$.

    Si $F$ dérive d'un potentiel, alors $\rot F = 0$.
}

\parag{Théorème: Caractérisation du rotationnel}{
    Soient $n = 2, 3$, $\Omega \subset \mathbb{R}^n$ un ensemble ouvert \textit{simplement connexe}, et $F \in C^1\left(\Omega; \mathbb{R}^n\right)$.

    $F$ dérive d'un potentiel si et seulement si $\rot F = 0$.
}

\parag{Théorème: Caractérisation par les courbes}{
    Soient $\Omega \subset \mathbb{R}^n$ un ensemble ouvert et $F \in C^1\left(\Omega; \mathbb{R}^n\right)$.

    $F$ dérive d'un potentiel si et seulement si $\int_{\Gamma} F \dotprod d \ell = 0$ pour toute courbe régulière par morceaux simple \textit{fermée} $\Gamma$.
}


\parag{Théorème de Green}{
    Soit $\Omega \subseteq \mathbb{R}^2$ un domaine régulier avec le bord $\partial \Omega$ orienté positivement (c'est-à-dire qu'on garde le domaine à gauche quand on parcourt $\partial \Omega$). Soit aussi $F \in C^1\left(\Omega; \mathbb{R}^2\right)$.

    Alors: 
    \[\iint_{\Omega} \rot F\left(x, y\right) dxdy = \int_{\partial\Omega} F \dotprod d \ell \]
}


\parag{Théorème de la divergence dans $\mathbb{R}^2$}{
    Soit $\Omega \subset \mathbb{R}^2$ régulier, $F \in C^1\left(\Omega;\mathbb{R}^2\right)$ un champ vectoriel, et $\nu_P$ la normale extérieure à $\Omega$ en $P \in \partial \Omega$. Alors, nous avons: 
    \[\iint_{\Omega} \Div F dxdy = \int_{\partial \Omega} \left(F \dotprod \nu\right) d \ell \]

    Notez qu'il n'y a pas d'ambigüité sur le signe puisque $F \dotprod \nu$ est un scalaire, et le problème vient quand on intègre un champ vectoriel.
}

\subsection{Intégrales de surfaces}
\parag{Définition: Surface régulière}{
    $\Sigma \subset \mathbb{R}^3$ est une \important{surface régulière} si:
    \begin{enumerate}
        \item Il existe $\Omega \subset \mathbb{R}^2$, un ouvert borné, avec une frontière $\partial \Omega$ simple et régulière par morceaux (donc que $\Omega$ est un domaine ``sans trou'') et $\sigma: \bar{\Omega}\mapsto \mathbb{R}^3$ une fonction $C^1\left(\bar{\Omega};\mathbb{R}^3\right)$ injective sur $\Omega$ (mais pas nécessairement injective sur $\bar{\Omega}$) telle que: 
        \[\sigma\left(\bar{\Omega}\right) = \Sigma\]
        
        \item $\left\|\sigma_u \wedge \sigma_v\right\| \neq 0$ pour tout $\left(u, v\right) \in \Omega$ (où $\sigma_u = \frac{\partial \sigma}{\partial u} $ et $\sigma_v = \frac{\partial \sigma}{\partial v} $).
    \end{enumerate}

    \subparag{Normales}{
        Les deux vecteurs normaux unitaires au point $P = \sigma\left(u, v\right)$ sont donnés par: 
        \[\pm \frac{\sigma_u \wedge \sigma_v}{\left\|\sigma_u \wedge \sigma_v\right\|}\]
    }
}


\parag{Définition: Intégrale de surface}{
    Soit $\Omega \in \mathbb{R}^3$ un ouvert, contenant $\Sigma$ une surface régulière. Soient aussi $f \in C^0\left(\Omega\right)$ un champ scalaire et $F \in C^{0}\left(\Omega; \mathbb{R}^3\right)$ un champ vectoriel. Soient finalement $A \subset \mathbb{R}^2$ un ouvert et $\sigma: \bar{A} \mapsto \Sigma$ une paramétrisation régulière.

    Alors, nous définissons l'intégrale d'un champ scalaire par: 
    \[\iint_{\Sigma} f dS = \iint_{A} f\left(\sigma\left(u, v\right)\right) \left\|\sigma_{u} \wedge \sigma_v\right\| dudv\]
    
    Et, nous définissions l'intégrale d'un champ vectoriel par: 
    \[\iint_{\Sigma} F \dotprod dS = \iint_{A} F\left(\sigma\left(u, v\right)\right) \dotprod \left(\sigma_u \wedge \sigma_v\right) dudv\]

    Le signe de la deuxième intégrale dépend de l'orientation de notre surface.
}


\parag{Théorème de la divergence dans $\mathbb{R}^3$}{
    Soient $\Omega \subset \mathbb{R}^3$ régulier, $F \in C^1\left(\bar{\Omega}, \mathbb{R}^3\right)$ et $\nu$ un champ de normales extérieures à $\Omega$.

    Alors, nous avons: 
    \[\iiint_{\Omega} \Div\left(F\right) dxdydz = \iint_{\partial \Omega} \left(F \dotprod \nu\right)dS\]
}


\parag{Théorème de Stokes}{
    Soit $\Sigma \subset \mathbb{R}^3$ une surface régulière par morceaux orientable et $F \in C^1\left(\Omega; \mathbb{R}^3\right)$, où $\Omega \subset \mathbb{R}^3$ est un ouvert contenant $\Sigma \cup \partial \Sigma$.

    Alors: 
    \[\iint_{\Sigma} \left(\rot F\right) \dotprod dS = \int_{\partial \Sigma} F \dotprod d \ell \]
    où le signe est donné par les paramétrisations (on utilise la paramétrisation de l'intégrale de gauche pour trouver celle de l'intégrale de droite).
}

\subsection{Séries de Fourier}
\parag{Définition: Série de Fourier}{
    Soit $f: \mathbb{R} \mapsto \mathbb{R}$ une fonction $T$-périodique. Alors on a la série de Fourier:
    \[F f\left(x\right) = \sum_{n=-\infty}^{\infty} c_n e^{i \frac{2\pi}{T}nx} = \frac{a_0}{2} + \sum_{n=1}^{\infty} \left(a_n\cos\left(\frac{2\pi}{T}nx\right) + b_n \sin\left(\frac{2\pi}{T} nx\right)\right)\]
    
    Les coefficients des séries de Fourier sont donnés par:
    \[c_n = \frac{1}{T} \int_{0}^{T} f\left(x\right) e^{-i \frac{2\pi}{T} nx} dx\]
    \[a_n = \frac{2}{T} \int_{0}^{T} f\left(x\right) \cos\left(\frac{2\pi}{T} n x\right)dx\] 
    \[b_n = \frac{2}{T} \int_{0}^{T} f\left(x\right) \sin\left(\frac{2\pi}{T} nx\right) dx\]
    
    Nous avons les conversions suivantes: 
    \[\begin{systemofequations} a_n = c_n + c_{-n} \\ b_n = i\left(c_n - c_{-n}\right) \end{systemofequations} \iff \begin{systemofequations} c_0 = \frac{a_0}{2} \\ c_n = \frac{a_n - i b_n}{2}, & \text{si } n \geq 1 \\ c_{-n} = \frac{a_n + i b_n}{2}, & \text{si } n \geq 1 \end{systemofequations}\]
}

\parag{Théorème de Dirichlet}{
    Soit $f: \mathbb{R} \mapsto \mathbb{R}$ une fonction $T$-périodique et continue sur $\left]0, T\right[ $, alors: 
    \[F f\left(x\right) = f\left(x\right), \mathspace \forall x \in \left]0, T\right[ \]
}

\parag{Théorème: Identité de Parseval}{
    Soit $f: \mathbb{R} \mapsto \mathbb{R}$ une fonction $T$-périodique et $C_{morc}^{1}\left(\left[0, T\right]\right)$. Alors: 
    \[\frac{1}{T} \int_{0}^{T} f\left(x\right)^2 dx = \sum_{n=-\infty}^{\infty} \left|c_n\right|^2 = \frac{1}{2} \left[\frac{a_0^2}{2}+ \sum_{n=1}^{\infty} \left(a_n^2 + b_n^2\right)\right]\]
}


\parag{Définition: Série de Fourier en cosinus}{
    Soit $f \in C_{morc}^1\left(\left[0, L\right]\right)$.

    La série de Fourier en sinus de $f$ est obtenue en calculant la série de Fourier sur une extension impaire de $f$ sur $\left[-L, L\right]$, et similairement pour les séries de Fourier en cosinus avec une extension paire.
}

\parag{Proposition: Dérivée}{
    Soit $f: \mathbb{R} \mapsto \mathbb{R}$ une fonction continue et $T$-périodique telle que $f'$ existe (sauf éventuellement en un nombre fini de points) sur $\left[0, T\right]$ et telle que $f' \in C^1_{morc}\left(\left[0, T\right]\right)$.

    Alors:
    \[F\left(f'\right)\left(x\right) = \sum_{n=-\infty}^{\infty} \left(c_n e^{i \frac{2\pi}{T} nx}\right)'\]
}

\subsection{Transformées de Fourier}
\parag{Définition: Transformées de Fourier}{
    Soit $f: \mathbb{R} \mapsto \mathbb{R}$ une fonction telle que $\int_{-\infty}^{\infty} \left|f\right| < +\infty$. Alors, on définit la transformée de Fourier et son inverse: 
    \[\mathcal{F}\left(\alpha\right) = \frac{1}{\sqrt{2\pi}} \int_{-\infty}^{\infty} f\left(x\right) e^{-i \alpha x} dx\]
    \[\mathcal{F}^{-1}\left(\alpha\right) = \frac{1}{\sqrt{2\pi}} \int_{-\infty}^{\infty} f\left(x\right) e^{i \alpha x} dx\]
    
}

\parag{Propriétés}{
    Soit $f, g: \mathbb{R} \mapsto \mathbb{R}$ deux fonctions intégrables en valeur absolue sur $\left]-\infty, \infty\right[ $. Alors:
    \begin{enumerate}
        \item $\mathcal{F}\left(a f\left(x\right) + b g\left(x\right)\right)\left(\alpha\right) = a \mathcal{F}\left(f\right)\left(\alpha\right) + b \mathcal{F}\left(g\right)\left(\alpha\right)$
        \item $\mathcal{F}\left(f\left(x + b\right)\right)\left(\alpha\right) = e^{i b \alpha} \mathcal{F}\left(f\right)\left(\alpha\right)$
        \item $\mathcal{F}\left(f\left(cx\right)\right)\left(\alpha\right) = \frac{1}{\left|c\right|} \mathcal{F}\left(f\right)\left(\frac{\alpha}{c}\right)$
        \item $\mathcal{F}\left(f'\right)\left(\alpha\right) = i\alpha \mathcal{F}\left(f\right)\left(\alpha\right)$
    \end{enumerate}

    Dans l'autre sens (c'est-à-dire pour $\mathcal{F}^{-1}$, mais aussi plutôt pour la variable $\alpha$ plutôt que la variable $x$), $i$ devient simplement $-i$ (il faut juste faire attention de multiplier la constante au bon endroit).
}


\parag{Théorème: Identité de Parseval}{
    Soit $f: \mathbb{R} \mapsto \mathbb{C}$ une fonction intégrable en valeur absolue et au carré sur $\left]-\infty, \infty\right[ $. Alors: 
    \[\int_{-\infty}^{\infty} \left|f\left(x\right)\right|^2 dx = \int_{-\infty}^{\infty} \left|\hat{f}\left(\alpha\right)\right|^2 d \alpha\]
}

\parag{Définition: Produit de convolution}{
    Soient $f, g : \mathbb{R} \mapsto \mathbb{R}$ des fonctions intégrables en valeur absolue sur $\left]-\infty, \infty\right[ $. Alors, le \important{produit de convolution} de $f$ et $g$, noté $f * g$, est la fonction:
    \[\begin{split}
    f * g: \mathbb{R} &\longmapsto \mathbb{R} \\
    x &\longmapsto \int_{-\infty}^{\infty} f\left(x - t\right)g\left(t\right)dt
    \end{split}\]
}

\parag{Proposition: Convolutions et transformée}{
    Soient $f, g: \mathbb{R} \mapsto \mathbb{R}$ des fonctions intégrables en valeur absolue sur $\left]-\infty, \infty\right[ $. Alors:
    \[\mathcal{F}\left(f * g\right) = \sqrt{2\pi} \mathcal{F}\left(f\right) \mathcal{F}\left(g\right)\]
    \[\mathcal{F}\left(fg\right) = \frac{1}{\sqrt{2\pi}} \mathcal{F}\left(f\right) * \mathcal{F}\left(g\right)\]
}

\parag{Application}{
    Nous pouvons appliquer les séries et transformées de Fourier à la résolution d'équations différentielles ordinaires.
}

\end{document}
