% !TeX program = lualatex
% Using VimTeX, you need to reload the plugin (\lx) after having saved the document in order to use LuaLaTeX (thanks to the line above)

\documentclass[a4paper]{article}

% Expanded on 2022-11-28 at 08:15:55.

\usepackage{../../style}

\title{Analyse 3}
\author{Joachim Favre}
\date{Lundi 28 novembre 2022}

\begin{document}
\maketitle

\lecture{10}{2022-11-28}{L'identité de Provençal le Gaulois}{
\begin{itemize}[left=0pt]
    \item Il faut regarder Kaamelott, c'est une série très sympa.
    \item Explication et preuve de l'identité de Parseval.
    \item Définition des séries de Fourier en sinus et en cosinus.
    \item Explication de comment calculer les coefficients de Fourier d'une fonction à partir de ceux de sa dérivée, et de comment l'appliquer à la résolution d'équation différentielles ordinaires.
\end{itemize}

}

\begin{parag}{Théorème: Identité de Parseval}
    Soit $f: \mathbb{R} \mapsto \mathbb{R}$ une fonction $T$-périodique et $C_{morc}^{1}\left(\left[0, T\right]\right)$. Alors: 
    \[\frac{1}{T} \int_{0}^{T} f\left(x\right)^2 dx = \sum_{n=-\infty}^{\infty} \left|c_n\right|^2 = \frac{1}{2} \left[\frac{a_0^2}{2}+ \sum_{n=1}^{\infty} \left(a_n^2 + b_n^2\right)\right]\]

    \begin{subparag}{Preuve}
        Prouvons uniquement le cas où $T = 2\pi$, et $f$ est continue.

        On trouve ainsi: 
        \autoeq{\frac{1}{2\pi} \int_{0}^{2\pi} f\left(x\right)^2 dx = \frac{1}{2\pi} \int_{0}^{2\pi} \left(f\left(x\right) \sum_{n=-\infty}^{\infty} c_n e^{inx}\right)dx = \sum_{n=-\infty}^{\infty} c_n \underbrace{\left(\frac{1}{2\pi} \int_{0}^{2\pi} f\left(x\right)e^{inx}\right)}_{= \bar{c_n}} = \sum_{n=\infty}^{\infty} c_n \bar{c_n} = \sum_{n=-\infty}^{\infty} \left|c_n\right|^2}
        
        On peut maintenant convertir cette série avec les coefficients réels: 
        \autoeq{\sum_{n=-\infty}^{\infty} \left|c_n\right|^2 = \left|c_0\right|^2 + \sum_{n=1}^{\infty} \left(\left|c_n\right|^2 + \left|c_{-n}\right|^2\right) = \frac{a_0^2}{4} + \sum_{n=1}^{\infty} \left(\frac{a_n^2 + b_n^2}{4} + \frac{a_n^2 + b_n^2}{4}\right) = \frac{1}{2} \left(\frac{a_0^2}{2} + \sum_{n=1}^{\infty} \left(a_n^2 + b_n^2\right)\right)}
    \end{subparag}
\end{parag}

\begin{parag}{Application: Calcul de série}
    Nous avons déjà vu comment utiliser les séries de Fourier afin de calculer des séries dans la neuvième série d'exercices, mais voyons maintenant comment appliquer l'identité de Parseval pour ce même objectif.

    Nous voulons calculer la valeur de $\sum_{n=1}^{\infty} \frac{1}{n^2}$. On pose $f\left(x\right)$ comme la fonction $2\pi$ périodique telle que $f\left(x\right) = x$ sur $\left[-\pi, \pi\right[ $. Nous trouvons alors: 
    \[c_n = \frac{1}{2\pi} \int_{0}^{2\pi} f\left(x\right) e^{-inx} dx = \frac{1}{2\pi} \int_{-\pi}^{\pi} xe^{-inx}dx\]
    
    Si $n = 0$, alors nous intégrons une fonction impaire sur un intervalle symétrique, donc $c_0 = 0$. Si $n \neq 0$, alors on trouve: 
    \autoeq{c_n = \frac{1}{2\pi} \left[\frac{x e^{-inx}}{-in}\right]^\pi_{-\pi} - \frac{1}{2\pi} \int_{-\pi}^{\pi} \frac{e^{-inx}}{-in} dx = \frac{1}{2\pi} \left(\frac{\pi e^{-in\pi} + \pi e^{in\pi}}{-in}\right) + \frac{1}{2\pi i n } \underbrace{\int_{-\pi}^{\pi} e^{-inx} dx}_{= 0} = \frac{e^{in \pi}}{-in} = i \frac{\left(-1\right)^n}{n}}
    
    
    On trouve donc que $\left|c_n\right|^2 = \frac{1}{n^2}$, ce qui nous donne:
    \[\sum_{n=-\infty}^{\infty} \left|c_n\right|^2 = \underbrace{\left|c_0\right|^2}_{= 0} + 2\sum_{n=1}^{\infty} \frac{1}{n^2} = \sum_{n=1}^{\infty} \frac{2}{n^2}\]

    Et, par l'identité de Parseval: 
    \[\sum_{n=-\infty}^{\infty} \left|c_n\right|^2 = \frac{1}{2\pi} \int_{0}^{2\pi} f\left(x\right)^2 dx = \frac{1}{2\pi} \int_{-\pi}^{\pi} x^2 dx = \frac{1}{\pi} \int_{0}^{\pi} x^2 dx = \frac{1}{\pi} \left[\frac{x^3}{3}\right]^{\pi}_0 = \frac{1}{\pi} \cdot  \frac{\pi^3}{3} = \frac{\pi^2}{3}\]
    
    Ainsi, cela nous donne que: 
    \[\sum_{n=1}^{\infty} \frac{2}{n^2} = \frac{\pi^2}{3} \implies \zeta\left(2\right) = \sum_{n=1}^{\infty} \frac{1}{n^2} = \frac{\pi^2}{6}\]
\end{parag}

\begin{parag}{Remarque}
    Soit $f: \mathbb{R} \mapsto \mathbb{R}$ une fonction $T$-périodique et $C_{morc}^1\left(\left[0, T\right]\right)$. 

    Si $f$ est paire, alors $f\left(x\right)\sin\left(\frac{2\pi}{T} nx\right)$ est impaire et $T$-périodique. Ainsi, puisque les $b_n$ sont calculés en intégrant cette fonction entre $-\frac{T}{2}$ et $\frac{T}{2}$, alors on trouve que, pour tout $n$: 
    \[b_n = 0\]
    
    De manière similaire, si $f$ est impaire, alors $f\left(x\right) \cos\left(\frac{2\pi}{T} nx\right)$ est impaire et $T$-périodique. Ainsi, pour tout $n$: 
    \[a_n = 0\]

    Regarder la parité de notre fonction peut donc nous permettre de diviser par deux le nombre de calculs.
\end{parag}

\subsection{Séries de Fourier en sinus et en cosinus}
\begin{parag}{Définition: Série de Fourier en cosinus}
    Soit $f \in C_{morc}^1\left(\left[0, L\right]\right)$.

    La \important{série de Fourier en cosinus} de $f$ est: 
    \[F_c f\left(x\right) = \frac{\widetilde{a}_0}{2} + \sum_{n=1}^{\infty} \widetilde{a}_n \cos\left(\frac{\pi}{L} nx\right)\]
    
    Avec: 
    \[\widetilde{a}_n = \frac{2}{L} \int_{0}^{L} f\left(x\right) \cos\left(\frac{\pi}{L}nx\right)dx\]
\end{parag}

\begin{parag}{Définition: Série de Fourier en sinus}
    Soit $f \in C_{morc}^1\left(\left[0, L\right]\right)$.

    La \important{série de Fourier en sinus} de $f$ est: 
    \[F_s f\left(x\right) = \sum_{n=1}^{\infty} \widetilde{b}_n \sin\left(\frac{\pi}{L} nx\right)\]
    
    Avec: 
    \[\widetilde{b}_n = \frac{2}{L} \int_{0}^{L} f\left(x\right) \sin\left(\frac{\pi}{L}nx\right)dx\]
\end{parag}

\begin{parag}{Propriété}
    Soit $f \in C_{morc}^1\left(\left[0, L\right]\right)$. Si $x \in \left]0, L\right[ $, alors: 
    \[F_c f\left(x\right) = F_s f\left(x\right) = \lim_{h \to 0^+} \frac{f\left(x - h\right) + f\left(x + h\right)}{2}\]

    De plus, on a: 
    \[F_c f\left(0\right) = \lim_{x \to 0^+} f\left(x\right), \mathspace F_c f\left(L\right) = \lim_{x \to L^-} f\left(x\right)\]
    \[F_s f\left(0\right) = F_s f\left(L\right) = 0\]

    \begin{subparag}{Remarque}
        En particulier, si $f$ est continue, alors, pour $x \in \left]0, L\right[ $: 
        \[F_c f\left(x\right) = F_s f\left(x\right) = f\left(x\right)\]

    \end{subparag}

    \begin{subparag}{Observation}
        Là où les séries de Fourier ``classiques'' permettent de représenter n'importe quelle fonction périodique sur $\mathbb{R}$, les séries de Fourier en sinus et en cosinus permettent uniquement de représenter une fonction sur un intervalle fini.
    \end{subparag}
    

    \begin{subparag}{Preuve}
        Considérons les deux fonctions $f_+$ et $f_-$ définies par: 
        \begin{functionbypart}{f_{\pm}\left(x\right)} 
            f\left(x\right), \mathspace \text{si } x \in \left]0, L\right[  \\
            \pm f\left(-x\right), \mathspace \text{si } x \in \left]-L, 0\right[  \\
            0, \mathspace \text{si } x = 0
        \end{functionbypart}
        étendues par $2L$-périodicité.

        En d'autres mots, $f_+$ est une extension paire de $f$ et $f_-$ est une extension impaire de $f$, les deux telles que $f_{\pm} = f$ sur $\left]0, L\right] $. 

        Puisque $f_+$ est paire, ses coefficients de sinus sont 0. De plus, on trouve que:
        \[F f_+ = \frac{a_0}{2} + \sum_{n=1}^{\infty} a_n \cos\left(\frac{2\pi}{2L} nx\right)\]
        avec: 
        \autoeq{a_n = \frac{2}{2L} \int_{0}^{2L} f_+\left(x\right)\cos\left(\frac{\pi}{L}nx\right)dx = \frac{1}{L} \int_{-L}^{0} f_+\left(x\right)\cos\left(\frac{\pi}{L}nx\right)dx + \frac{1}{L} \int_{0}^{L} f_+\left(x\right)\cos\left(\frac{\pi}{L}nx\right)dx = \frac{2}{L}\int_{0}^{L} \underbrace{f_+\left(x\right)}_{= f\left(x\right)} \cos\left(\frac{\pi}{L} nx\right) dx = \frac{2}{L} \int_{0}^{L} f\left(x\right) \cos\left(\frac{\pi}{L} nx\right)dx = \widetilde{a}_n}
        
        Et on trouve ainsi que, pour $x \in \left]0, L\right[ $: 
        \[F_c f\left(x\right) = F f_{+}\left(x\right) = \lim_{h \to 0^+} \frac{f\left(x - h\right) + f\left(x + h\right)}{2}\]

        De plus: 
        \[F_C f\left(0\right) = \lim_{h \to 0^+} \frac{\overbrace{f_+\left(0 - h\right)}^{= f\left(h\right)} + f\left(0 + h\right)}{2} = \lim_{h \to 0^+} f\left(h\right)\]
        \[F_C f\left(L\right) = \lim_{h \to 0^+} \frac{\overbrace{f_+\left(L + h\right)}^{= f\left(L - h\right)} + f\left(L - h\right)}{2} = \lim_{h \to 0^+} f\left(L - h\right)\]
        comme attendu.

        La preuve pour $F_s f$ est similaire et utilise $F f_-$.

        \qed
    \end{subparag}
\end{parag}

\begin{parag}{Exemple}
    Prenons $f\left(x\right) = x$ pour $ \in \left[0, 1\right] $.

    Nous trouvons que, pour $n \neq 0$: 
    \[\widetilde{a}_n = 2 \int_{0}^{1} x\cos\left(\pi n x\right)dx = \frac{2}{\pi^2 n^2} \left(-1 + \left(-1\right)^n\right)\]
    
    Et ainsi:
    \begin{functionbypart}{\widetilde{a}_n}
    1, \mathspace n = 0 \\
    0, \mathspace n \geq 2 \text{ pair} \\
    \frac{-4}{\pi^2 n^2}, \mathspace n \text{ impair}
    \end{functionbypart}

    De manière similaire, on peut trouver que: 
    \[\widetilde{b}_n = \frac{2}{\pi n}\left(-1\right)^n\]
    
    Cela implique que, sur $x \in \left]0, 1\right[ $: 
    \[x = \frac{1}{2} + \sum_{k=0}^{\infty} \frac{-4}{\pi^2 \left(2k + 1\right)^2} \cos\left(\pi\left(2k + 1\right)x\right) = \sum_{n=1}^{\infty} \frac{2}{\pi n} \left(-1\right)^n \sin\left(\pi n x\right)\]
    où la première série est la série de Fourier en cosinus et la deuxième celle en sinus.
\end{parag}

\subsection[Application aux EDO]{Application aux équations différentielles ordinaires}
\begin{parag}{Remarque}
    La résolution d'équation différentielle (l'équation de la chaleur en particulier) était la motivation originale de Fourier.
\end{parag}

\begin{parag}{Proposition}
    Soit $f: \mathbb{R} \mapsto \mathbb{R}$ une fonction continue et $T$-périodique telle que $f'$ existe (sauf éventuellement en un nombre fini de points) sur $\left[0, T\right]$ et telle que $f' \in C^1_{morc}\left(\left[0, T\right]\right)$.

    Alors: 
    \[F f'\left(x\right) = \sum_{n=-\infty}^{\infty} c_n i \frac{2\pi}{T}n e^{i \frac{2\pi}{T}nx}\]
    
    \begin{subparag}{Remarque 1}
        Quand on dérive, on peut donc simplement dériver l'exponentielle à l'intérieur.
    \end{subparag}

    \begin{subparag}{Remarque 2}
        Il en suit que: 
        \[F f'\left(x\right) = \sum_{n=1}^{\infty} \left[\frac{2\pi}{T} n b_n \cos\left(\frac{2\pi}{T} nx\right) - \frac{2\pi}{T} n a_n \sin\left(\frac{2\pi}{T} nx\right)\right]\]

        Ainsi, si les coefficients de Fourier de $f$ sont $c_n$ (et $a_n$, $b_n$), alors: 
        \[c'_n \over{=}{déf} i \frac{2\pi}{T} n c_n, \mathspace a_n' = \frac{2\pi}{T} n b_n, \mathspace b_n' = -\frac{2\pi}{T} n a_n\]
        
        De plus, si $f: \mathbb{R} \mapsto \mathbb{R}$ est $T$-périodique, \textit{continue} et telle que $f' \in C^1_{morc}\left(\left[0, T\right]\right)$, alors on peut calculer les coefficients de $f$ à partir de ceux de $f'$.
    \end{subparag}
    
    \begin{subparag}{Idée de preuve}
        Dans l'idée, puisque $f\left(x\right) = F f\left(x\right)$, on trouve:
        \autoeq{f'\left(x\right) = \left(F f\left(x\right)\right)' = \left(\sum_{n=-\infty}^{\infty} c_n e^{i \frac{2\pi}{T} nx}\right)' \over{=}{\dagger} \sum_{n=-\infty}^{\infty} c_n \left(e^{i \frac{2\pi}{T} nx}\right)' = \sum_{n=-\infty}^{\infty} c_n i \frac{2\pi}{T} n e^{i \frac{2\pi}{T}nx}}
        
        L'égalité $\dagger$ est celle qu'il faudrait justifier formellement afin d'avoir une preuve correcte. Nous ne le ferons pas dans ce cours.
    \end{subparag}
\end{parag}

\begin{parag}{Exemple}
    Soit $f\left(x\right) = \left|x\right|$ sur $\left[-1, 1\right]$ étendue de telle manière à être $2$-périodique. Nous avons alors: 
    \begin{functionbypart}{f'\left(x\right)}
    1, \mathspace \text{si } 0 < x < 1 \\
    -1, \mathspace \text{si } -1 < x < 0 \\
    \end{functionbypart}

    Nous posons $f'\left(-1\right) = -1$ et $f'\left(0\right) = 0$ (ce choix est arbitraire et ne change pas notre résultat). Pour $f'$, nous avons déjà calculé que: 
    \begin{functionbypart}{c_n'}
    0, \mathspace \text{ si $n$ est pair} \\
    \frac{2}{i\pi n}, \mathspace \text{ si $n$ est impair}
    \end{functionbypart}
    
    Par la propriété du paragraphe précédent, on trouve: 
    \[c_n' = i \frac{2\pi}{T} n c_n = i \pi n c_n \iff c_n = \frac{1}{i \pi n} c'_n\]
    
    Ainsi, si $n\neq 0$, nous obtenons que:
    \begin{functionbypart}{c_n}
    0, \mathspace \text{ si $n$ est pair} \\
    \frac{-2}{\pi^2 n^2}, \mathspace \text{ si $n$ est impair}
    \end{functionbypart}

    Cependant, si $n = 0$, alors $c'_n = 0$ et donc on n'a pas d'information sur $c_0$. C'est normal, puisque c'était une constante qui a été enlevée par la dérivée. Nous sommes forcés de la calculer: 
    \[c_0 = \frac{1}{2}\int_{0}^{2} f\left(x\right)dx = \frac{1}{2}\]
    
    Et donc: 
    \[F f\left(x\right) = \frac{1}{2} + \sum_{n \text{ impair}}^{} \frac{-2}{\pi^2 n^2} e^{inx}\]
\end{parag}

\begin{parag}{Application à une EDO}
    Nous voulons trouver une fonction $f: \mathbb{R} \mapsto \mathbb{R}$ $2\pi$-périodique telle que: 
    \[f'\left(x\right) - f\left(x\right) = h\left(x\right)\]
    où $h: \mathbb{R} \mapsto \mathbb{R}$ est une fonction continue, $2\pi$-périodique et $C_{morc}^1\left(0, 2\pi\right)$.
    
    \begin{subparag}{Résolution}
        L'idée est de déterminer $f$ à partir de $F f$. Posons ainsi: 
        \[F f\left(x\right) = \sum_{n=-\infty}^{\infty} c_n e^{inx} \implies F f'\left(x\right) = \sum_{n=-\infty}^{\infty} c_n in e^{inx}\]
        
        Supposons aussi que: 
        \[F h\left(x\right) = \sum_{n=-\infty}^{\infty} \hat{c}_n e^{inx}\]
        
        Comme $f, f'$ et $h$ sont continues, on peut remplacer chaque fonction par sa série de Fourier: 
        \autoeq{f' - f  = h \iff F f' - F f = F h \iff \sum_{n = -\infty}^{\infty} c_n in e^{inx} - \sum_{n=-\infty}^{\infty} c_n e^{inx} = \sum_{n=-\infty}^{\infty} \hat{c}_n e^{inx} \iff \sum_{n=-\infty}^{\infty} c_n\left(in - 1\right)e^{inx} = \sum_{n=-\infty}^{\infty} \hat{c}_n e^{inx}}
        
        Puisque c'est vrai pour tout $x \in \left[0, 2\pi\right]$, et puisque les coefficients de Fourier sont uniques, cette équation est \textit{équivalente} à:  
        \[f' - f = h \iff c_n\left(in - 1\right) = \hat{c}_n \iff c_n = \frac{\hat{c}_n}{in - 1}, \mathspace \forall n\]

        Or, puisque $a_n = c_n + c_{-n}$: 
        \autoeq{a_n = \frac{\hat{c}_n}{in - 1} - \frac{\hat{c}_{-n}}{in + 1} = \frac{\hat{c}_n \left(in + 1\right) - \hat{c}_n\left(in - 1\right)}{-n^2 - 1} = -\frac{1}{n^2 + 1} \left(\hat{c}_n + \hat{c}_{-n}\right) - \frac{n}{n^2 + 1} i \left(\hat{c}_n - \hat{c}_{-n}\right) = -\frac{\hat{a}_n + n \hat{b}_n}{n^2 + 1}}
        où $\hat{a}_n$ et $\hat{b}_n$ sont les coefficients réels de $h\left(x\right)$.

        Nous pouvons alors trouver de manière similaire que: 
        \[b_n = i\left(c_n - c_{-n}\right) = \ldots = - \frac{\hat{b}_n - n \hat{a}_n}{n^2+ 1}\]
        
    \end{subparag}
    
    \begin{subparag}{Application}
        Prenons un cas concret pour $h$: 
        \[h\left(x\right) = 2 - \cos\left(x\right) + 2\sin\left(x\right) + \sin\left(2x\right) - 2 \cos\left(3x\right)\]
        
        $h$ est donné sous la forme de sa série de Fourier, donc on trouve: 
        \[\hat{a}_0 = 2\cdot 2 = 4, \mathspace \hat{a}_1 = -1, \mathspace \hat{a}_2 = 0, \mathspace\hat{a}_3 = -2, \mathspace \hat{a}_n = 0, \ \forall n \geq4\]
        \[\hat{b}_1 = 2, \mathspace\hat{b}_2 = 1, \mathspace \hat{b}_n = 0, \ \forall n \geq 3\]

        Et ainsi, par la formule trouvée précédemment, on obtient que: 
        \[a_0 = -4, \mathspace a_1 = -\frac{1}{2}, \mathspace a_2 = -\frac{2}{5}, \mathspace a_3 = \frac{1}{5}\] 
        \[b_1 = \frac{-\hat{b}_1 - \hat{a}_1}{2} = -\frac{3}{2}, \mathspace b_2 = -\frac{1}{5}, \mathspace b_3 = -\frac{3}{5}\]
        et avec $a_n = 0 = b_n$ pour tout $n \geq 4$.

        On trouve donc que: 
        \autoeq{f\left(x\right) = F f\left(x\right) = \frac{a_0}{2} + \sum_{n=1}^{\infty} \left[a_n \cos\left(nx\right) + b_n \sin\left(nx\right)\right] = -2 - \frac{1}{2}\cos\left(x\right) - \frac{2}{5} \cos\left(2x\right) + \frac{1}{5} \cos\left(3x\right) \fakeequal - \frac{3}{2}\sin\left(x\right) - \frac{1}{5} \sin\left(2x\right) - \frac{3}{5}\sin\left(3x\right)}
        
        Nous avons ainsi résolu une équation différentielle sans calculer une seule intégrale (ce qui rend sa résolution beaucoup moins marrante).
    \end{subparag}
\end{parag}



\end{document}
