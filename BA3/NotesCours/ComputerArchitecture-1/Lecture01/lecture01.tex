% !TeX program = lualatex
% Using VimTeX, you need to reload the plugin (\lx) after having saved the document in order to use LuaLaTeX (thanks to the line above)

\documentclass[a4paper]{article}

% Expanded on 2022-09-20 at 17:07:12.

\usepackage{../../style}

\title{Computer architecture}
\author{Joachim Favre}
\date{Mardi 20 septembre 2022}

\begin{document}
\maketitle

\lecture{1}{2022-09-20}{Datapath and control path}{
\begin{itemize}[left=0pt]
    \item Definition of datapath and control path.
    \item Explanation of the ALU and the register file.
    \item Explanation of the different component of the control path.
\end{itemize}

}

\section{Processor architecture}
\subsection{Introduction to processors}
\parag{Introduction}{
    In a program, we can see that there are things which affect the data (such as operations (+, \texttt{<<}, \ldots) and assignments (=)), and stuff that define the program flow (such as loops).

    Operations such as \texttt{temp = data \& mask} can be rewritten as a very simple structure such as \texttt{and \$r5, \$r1, \$r3} where the structure is:
    \begin{center}
        [operation name] [destination], [left operand], [right operand]
    \end{center}
    where the \texttt{\$rx} are registers in memories. Such instructions are harder to grasp for humans but easier for hardware. We can make simlar instructions for instantiation and branching, leading to an assembly language.
}

\parag{Datapath and control path}{
    The \important{datapath} is everything which concerns operations, assignments and more generally anything which treats variables. The \important{control path} is everything which works with the flow control and external devices.
}

\subsubsection{Datapath}
\parag{ALU}{
    The Arithmetic-Logic Unit (ALU) takes three inputs (two operands and a selection input to determine which operation to perform) and it produces one output. It has the following symbol:
    \imagehere[0.4]{ALU.png}
}

\parag{Register file}{
    The register file holds data, and can hand two registers at the same time (since we need two values for operations, and we have no disandvatage at giving two values at the same time). It can also write values. It has the following symbol:
    \imagehere[0.4]{RegisterFile.png}

    The input W specifies the value to write, AW the address where to write it and Wr whether the value in the W input should be written. AA and AB are the adresses of the two values we want to get, and A and B are the outputs.
}


\subsubsection{Control path}
\parag{Requirements}{
    The control path needs to ensure correct reading and sequencing of the program. It requires a Program Counter (PC) to know where it is in the program, which can automatically increase by one at each iteration. It also needs Instruction Memory, to know all the instructions, and to be able to give the one we are currently looking at. It finally needs Control Logic, which takes an instruction, deciphers it, and interacts with the datapath in order to produce the expected result. Note that the control logic needs a way to override the automatic incrementation of the value stored in the program counter, for branching; this can be done through a multiplexer.
    \imagehere[0.7]{ControlPath.png}

    If the program syntax is not very strict, the control logic will have a very complicated task.
}


\parag{Conclusion}{
    Putting everything together we get the following big structure:
    \imagehere[0.7]{GeneralProcessor.png}

    Note that we need to add data memory to increase the space. The register file is a fast memory which cannot store a lot of data.
}

\end{document}
