% !TeX program = lualatex

\documentclass[a4paper]{article}

% Expanded on 2022-04-05 at 15:18:05.

\usepackage{../../style}

\title{AICC-2}
\author{Joachim Favre}
\date{Mardi 05 avril 2022}

\begin{document}
\maketitle

\lecture{13}{2022-04-05}{Modulo world (less nice than Juice WRLD)}{
\begin{itemize}[left=0pt]
    \item Definition of congruence classes and $\mathbb{Z} / m\mathbb{Z}$.
    \item Definition of the operations of $\mathbb{Z} / m\mathbb{Z}$, and explanation of their properties.
    \item Definition of the multiplicative inverse, and proof that it is unique if it exists.
    \item Proof that, if there exists a $\left[b\right]_m$ such that $\left[a\right]_m x = \left[b\right]_m$ has a unique solution, then $\left[a\right]_m$ has an inverse, and thus this equation has a single solution for any $\left[b\right]_m$.
\end{itemize}
}

\subsection{$\mathbb{Z}$ modulo $m$}

\begin{parag}{Definition: Congruence class}
    Let $m \in\mathbb{Z}$, where $m > 1$. Also, let $a \in \mathbb{Z}$. We define the \important{congruence class} of $a$ mod $m$ to be: 
    \[\left[a\right]_m \over{=}{def} \left\{i \in\mathbb{Z} \text{ such that } \congruent{i}{a}{m}\right\}\]
    
    \begin{subparag}{Remark 1}
        Those are the set of integers which have the same colours, on the drawings we did earlier.
    \end{subparag}
    
    \begin{subparag}{Remark 2}
        If $m$ is implicit, then we write $a$ instead of $\left[a\right]_m$.
    \end{subparag}
    

    \begin{subparag}{Example}
        For example, $\left[18\right]_2 = \left[-2\right]_2 = \ldots$ is the set of even integers. 
    \end{subparag}
\end{parag}

\begin{parag}{Equivalences}
    The following propositions are equivalent:
    \begin{enumerate}
        \item $\left[a\right]_m = \left[b\right]_m$
        \item $\congruent{a}{b}{m}$
        \item $a \Mod m = b \Mod m$ 
        \item $\left(a - b\right) \Mod m = 0$
    \end{enumerate}
\end{parag}

\begin{parag}{Example}
    Since $-13 - 5 = -18$ is a multiple of $9$, we get: 
    \[\left[-13\right]_9 = \left[5\right]_9\]
    
    However, since $13 - 5 = 8$ is not a multiple of $9$, we get: 
    \[\left[13\right]_9 \neq \left[5\right]_9\]
\end{parag}

\begin{parag}{Definition: $\mathbb{Z}$ mod $m$}
    The set of all congruence classes modulo $m$, $\mathbb{Z}$ mod $m$, is denoted by $\mathbb{Z} / m\mathbb{Z}$ or $\mathbb{Z}_m$

    \begin{subparag}{Remark}
        Every element has a \important{unique representation} in reduced form: 
        \[\left[r\right]_m, \mathspace \text{where } 0 \leq r \leq m-1\]
    \end{subparag}
    
    
    \begin{subparag}{Example}
        For instance: 
        \[\mathbb{Z} / 3\mathbb{Z} = \left\{\left[0\right]_3, \left[1\right]_3, \left[2\right]_3\right\}\]
    \end{subparag}
\end{parag}

\begin{parag}{Operations in $\mathbb{Z} / m\mathbb{Z}$}
    We define the \important{sum} the following way:
    \[\left[a\right]_m + \left[b\right]_m = \left[a + b\right]_m\]
    
    Similarly, we can define the product as: 
    \[\left[a\right]_m \left[b\right]_m = \left[ab\right]_m\]
    
    Both definitions hold regardless of the choice of the representative in the set (if we are considering $\left[0\right]_2 + \left[1\right]_2$, it does not matter if we pick $\left[0 + 1\right]_2$ or $\left[24 + 45\right]_2$). 

    \begin{subparag}{Example}
        Let's draw operation tables for $\mathbb{Z} / 3 \mathbb{Z}$:
        \begin{center}
        \begin{tabular}{c|ccc}
            $+$ & 0 & 1 & 2 \\
            \hline
            0 & 0 & 1 & 2 \\
            1 & 1 & 2 & 0 \\
            2 & 2 & 0 & 1
        \end{tabular}
        \hspace{2em}
        \begin{tabular}{c|ccc}
            $\cdot$ & 0 & 1 & 2 \\
            \hline
            0 & 0 & 0 & 0 \\
            1 & 0 & 1 & 2 \\
            2 & 0 & 2 & 1
        \end{tabular}
        \end{center}
    \end{subparag}
\end{parag}

\begin{parag}{Proprieties of addition}
    \begin{itemize}[left=0pt]
        \item Associativity: $\displaystyle \left(\left[a\right]_m + \left[b\right]_m\right) + \left[c\right]_m = \left[a\right]_m + \left(\left[b\right]_m + \left[c\right]_m\right) = \left[a\right]_m + \left[b\right]_m + \left[c\right]_m$
            
        \item Commutativity: $\left[a\right]_m + \left[b\right]_m = \left[b\right]_m + \left[a\right]_m$
            
        \item Existence of additive identity: $\left[a\right]_m + \left[0\right]_m = \left[0\right]_m + \left[a\right]_m = \left[a\right]_m$.
            
        \item Existence of additive inverse: $\left[-a\right]_m + \left[a\right]_m = \left[0\right]_m$
    \end{itemize}
\end{parag}

\begin{parag}{Properties of multiplication}
    \begin{itemize}[left=0pt]
        \item Associativity: $\displaystyle \left(\left[a\right]_m \cdot \left[b\right]_m\right) \cdot \left[c\right]_m = \left[a\right]_m \cdot \left(\left[b\right]_m \cdot \left[c\right]_m\right) = \left[a\right]_m \cdot \left[b\right]_m \cdot \left[c\right]_m$
            
        \item Commutativity: $\left[a\right]_m \cdot \left[b\right]_m = \left[b\right]_m \cdot \left[a\right]_m$
            
        \item Existence of multiplicative identity: $\left[a\right]_m \cdot \left[1\right]_m = \left[1\right]_m \cdot \left[a\right]_m = \left[a\right]_m$.
    \end{itemize}

    However, we can realise that there is not always a multiplicative inverse.
\end{parag}

\begin{parag}{Distributivity}
    We also have the following property, that links both our operations: 
    \[\left[a\right]_m \left(\left[b\right]_m + \left[c\right]_m\right) = \left[a\right]_m \left[b\right]_m + \left[a\right]_m \left[c\right]_m\]
\end{parag}

\begin{parag}{Definition}
    Let $k \in \mathbb{N}$, where $k \geq 1$. We define the following notation: 
    \[k\left[a\right]_m = \underbrace{\left[a\right]_m + \ldots + \left[a\right]_m}_{\text{$k$ times}}\]
    
    \begin{subparag}{Remark}
        We can prove that: 
        \[k\left[a\right]_m = \left[k\right]_m \left[a\right]_m = \left[ka\right]_m\]
    \end{subparag}
\end{parag}

\begin{parag}{Definition: Multiplicative inverse}
    If it exists, the multiplicative inverse of $\left[a\right]_m$ is an element $\left[b_m\right]$ such that:
    \[\left[a\right]_m \left[b\right]_m = \left[b\right]_m \left[a\right]_m = \left[1\right]_m\]
\end{parag}

\begin{parag}{Theorem: Unicity of the multiplicative inverse}
    If it exists, the multiplicative inverse is unique. 

    \begin{subparag}{Remark}
        We can take advantage of this notation to write the inverse as $\left(\left[a\right]_m\right)^{-1}$.
    \end{subparag}

    \begin{subparag}{Proof}
        Since we are working in the modulo $m$ world, let us make it implicite. In other words, let us write $\left[x\right]_m = x$.

        Let's suppose for contradiction that there exists two numbers $b \neq c$ such that they are both inverse of $a$, i.e:
        \[ab = ac = 1\]
        
        Then: 
        \[ab = ac \iff bab = bac \iff 1\cdot b = 1\cdot c \iff b = c\]

        Which is our contradiction.

        \qed
    \end{subparag}
\end{parag}

\begin{parag}{Example}
    Let's take the following multiplication table, in $\mathbb{Z} / 4\mathbb{Z}$:
    \begin{center}
    \begin{tabular}{c|cccc}
        $\cdot$ & 0 & 1 & 2 & 3 \\
        \hline
        0 & 0 & 0 & 0 & 0 \\
        1 & 0 & 1 & 2 & 3 \\
        2 & 0 & 2 & 0 & 2 \\
        3 & 0 & 3 & 2 & 1
    \end{tabular}
    \end{center}

    We see that $\left[1\right]_4$ is the inverse of $\left[1\right]_4$, and $\left[3\right]_4$ is the inverse of $\left[3\right]_4$. However, neither $\left[0\right]_4$ nor $\left[2\right]_4$ have an inverse.
\end{parag}

\begin{parag}{Definition: Power}
    Let $k \in \mathbb{N}^*$. We define: 
    \[\left(\left[a\right]_m\right)^k = \underbrace{\left[a\right]_m \cdots \left[a\right]_m}_{\text{$k$ times}}\]
    
    When $k = 0$, we define: 
    \[\left(\left[a\right]_m\right)^{0} = \left[1\right]_m\]
    
    \begin{subparag}{Remark}
        Note that we should not try to work with negative powers since this is very dangerous: inverses do not always exist.
    \end{subparag}

    \begin{subparag}{Example}
        Let us compute the following: 
        \[\left(\left[3\right]_7\right)^{12} = \left(\left(\left[3\right]_7\right)^{2}\right)^{6} = \left(\left[3^2\right]_7\right)^6 = \left(\left[2\right]_7\right)^{6} = \left(\left[2^3\right]_7\right)^2 = \left(\left[1\right]_7\right)^2 = \left[1\right]_7\]
    \end{subparag}
\end{parag}

\begin{parag}{Proposition}
    Let $\left[a\right]_m$ be a congruence class which has an inverse. Then, there exist no number $k \in \mathbb{N}$ such that: 
    \[\left(\left[a\right]_m\right)^k = 0\]

    \begin{subparag}{Proof}
        Let's suppose for contradiction that there exists a $k$ such that $\left(\left[a\right]_m\right)^k = 0$.

        We know by hypothesis that $\left[a\right]_m$ has an inverse, thus let us multiply both sides of our equation by it: 
        \[0 = \left(\left[a\right]_m\right)^{-1} \cdot \left(\left[a\right]_m\right)^{k} = \left(\left[a\right]_m\right)^{-1} \cdot \left[a\right]_m \cdot \left(\left[a\right]_m\right)^{k-1} = \left(\left[a\right]_m\right)^{k-1}\]
        
        Continuing this $k-1$ times, we get: 
        \[0 = \left(\left[a\right]_m\right)^{0} = 1\]
        
        Which is our contradiction.

        \qed
    \end{subparag}
    
\end{parag}

\begin{parag}{Function terminology}
    Let a function $f: \mathcal{E} \mapsto \mathcal{F}$. We call $\mathcal{E}$ to be the \important{domain}, $\mathcal{F}$ to be the \important{codomain}, $f\left(\mathcal{E}\right)$ to be the image.

    We can draw the following diagrams to state if a function if is injective, surjective, or bijective:
    \imagehere[0.7]{FunctionInjectivitySurjectivityBijectivity.png}

    The pigeonhole principle \textit{(Principle des tiroirs et des chaussettes de Dirichlet} in French, if we have more pigeons than holes, then necessarily there is at least a hole containing at least two pigeons) tells us that:

    \begin{itemize}
        \item $f$ is injective implies that $\left|\mathcal{E}\right| \leq \left|\mathcal{F}\right|$.
        \item $f$ is surjective implies that $\left|\mathcal{E}\right| \geq \left|\mathcal{F}\right|$.
        \item $f$ is bijective implies that $\left|\mathcal{E}\right| = \left|\mathcal{F}\right|$.
    \end{itemize}
\end{parag}

\begin{parag}{Theorem}
    In $\mathbb{Z} / m\mathbb{Z}$, the following three propositions are equivalent:
    \begin{enumerate}
        \item $\left[a\right]_m$ has an inverse.
        \item For all $\left[b\right]_m$, $\left[a\right]_m x = \left[b\right]_m$ has a unique solution.
        \item There exists a $\left[b\right]_m$ such that $\left[a\right]_m x = \left[b\right]_m$ has a unique solution
    \end{enumerate}

    \begin{subparag}{Proof}
        The equivalence between (1) and (2), and the fact that (2) implies (3) are considered trivial and left as an exercise to the reader. We will only prove that (3) implies (2).

        Let us do our proof by the contrapositive. In other words, we suppose that there exists a $b$ such that $a x = b$ has either no solution or multiple solutions, and we want to prove that it implies that this is true for all $b$.

        By hypothesis, we know that there exists $\widetilde{b}$ such that $a x = \widetilde{b}$ has no or multiple solutions. Let us consider the case where this equation has no solution. Let us consider the mapping $x \mapsto a x$. Since we know that $x$ can take $m$ values (between $0$ and $m-1$) and $ax$ can take $m - 1$ values (between $0$ and $m-1$, except for $\widetilde{b}$ since its equation had no solution), we know that it implies by the pigeon hole principle that $\exists b^*$ such that $ax = b^*$ has multiple solutions.

        In any cases, we have found a $b^*$ such that $ax = b^*$ has multiple solutions. Let's call those solutions $x_1$ and $x_2$, and define $h = x_1 - x_2$. Then: 
        \[a h = a\left(x_1 - x_2\right) = b^* - b^* = 0\]

        This yields that, for no $b$, $ax = b$ has a unique solution: either it has 0 solution, either it has at least 2. Indeed, if $z$ is a solution (meaning $az = b^*$), then $z + h$ (which is different from $z$ since $x_1 \neq x_2 \iff h \neq 0$) is also a solution: 
        \[a\left(z + h\right) = az + ah = b^* + 0 = b^*\]
        
        This concludes our proof.

        \qed
    \end{subparag}
\end{parag}

\begin{parag}{Example 1}
    We want to find the solution to the following equation, if it exists: 
    \[\left[4\right]_9 x = \left[3\right]_9\]
    
    Since we work modulo $9$, we can try the $9$ guesses. Let us draw the following table:
    \begin{center}
    \begin{tabular}{c|ccccccccc}
        $x$ & 0 & 1 & 2 & 3 & 4 & 5 & 6 & 7 & 8 \\
        \hline
        $\left[4\right]_9 x$ & 0 & 4 & 8 & 3 & 7 & 2 & 6 & 1 & 5
    \end{tabular}
    \end{center}
    
    We now have two ways of solving our equation. The first one is to see in the table that $x = \left[3\right]_9$ implies $\left[4\right]_9 x = \left[3\right]_9$. The second one, is seeing that the inverse to $\left[4\right]_9$ is $\left[7\right]_3$; thus: 
    \[x = \left[7\right]_9 \left[3\right]_9 = \left[3\right]_9\]
\end{parag}

\begin{parag}{Example 2}
    Again, we want to find the solution to the following equation, if it exists: 
    \[\left[2\right]_7 x + \left[3\right]_7 = \left[1\right]_7 \]
    
    We can work a bit with it: 
    \[\left[2\right]_7 x + \left[3\right]_7 = \left[1\right]_7 \iff \left[2\right]_7 x + \left[3\right]_7 - \left[3\right]_7 = \left[1\right]_7 + \left[-3\right]_7 = \left[1\right]_7 + \left[4\right]_7 = \left[5\right]_7\]

    We need to find the multiplicative inverse of $2$ mod 7. By guessing, we see it is given by $\left[4\right]_7$. Thus: 
    \[x = \left[4\right]_7 \left[5\right]_7 = \left[20\right]_7 = \left[6\right]_7\]
\end{parag}

\begin{parag}{Example 3}
    Let us now find the solution to the following equation, if it exists: 
    \[\left[3\right]_9 x + \left[2\right]_9 = \left[5\right]_9 \iff \left[3\right]_9 x = \left[3\right]_9\]
    
    We see that $\left[3\right]_9$ does not have a multiplicative inverse mod $9$, so our equation either has no solution, or multiple solutions. Since $x = 1$ is a solution, we know there are multiple solutions.
\end{parag}


\end{document}
