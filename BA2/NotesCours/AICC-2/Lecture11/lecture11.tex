% !TeX program = lualatex

\documentclass[a4paper]{article}

% Expanded on 2022-03-29 at 15:31:42.

\usepackage{../../style}

\title{AICC-2}
\author{Joachim Favre}
\date{Mardi 29 mars 2022}

\begin{document}
\maketitle

\lecture{11}{2022-03-29}{Differences in Analysis, now divisions in AICC\ldots}{
\begin{itemize}[left=0pt]
    \item Explanation of Euclid's Division algorithm, and definition of the modulo operation.
    \item Definition of the congruence, and proof of it being an equivalence relation.
    \item Proof of the properties of modulos and congruences.
\end{itemize}

}

\subsection{Modulo arithmetic}
\begin{parag}{Introduction}
    We notice that in the set of integers, $\mathbb{Z}$, we can add, subtract and multiply, but we cannot divide (in general). We will dig deeper in this question.
\end{parag}

\begin{parag}{Euclid's Division Algorithm}
    For all integers $a$ and $m \neq 0$, there exist unique integers $q$ (the quotient) and $r$ (the remainder) such that: 
    \[a = mq + r\]
    where $0 \leq r < \left|m\right|$
    
    \begin{subparag}{Example}
        Let's take $m = 4$, then:
        \[7 = 4\cdot 1 + 3\] 
        \[-2 = 4\left(-1\right) + 2\]
    \end{subparag}
\end{parag}


\begin{parag}{Signs}
    Let's do some Euclidean divisions. Let's divide 8 by 3, then $-8$ by 3, then $8$ by $-3$, and finally $-8$ by $-3$:
    \[{\color{red}8} = {\color{red}3} \cdot 2 + 2, \mathspace {\color{red}-8} = {\color{red}3} \cdot \left(-3\right) + 1\]
    \[{\color{red}8} = {\color{red}\left(-3\right)} \cdot \left(-2\right) + 2, \mathspace {\color{red}-8} = {\color{red}\left(-3\right)} \cdot 3 + 1\]
    
    To do those divisions, we know the red numbers, and we look for our quotient and remainder, such that $0 \leq r < \left|m\right|$.

    \begin{subparag}{Remark}
        Note that, in many programming languages, the signs are not always implemented the same way. Thus, if $a$ and $m$ are positive, there is no problem, but if one of them is negative, then the result may be positive or negative depending on the language.

        Indeed, we could say: 
        \[8 = 3\cdot 3 - 1, \mathspace -8 = 3\left(-2\right) - 2\]
        \[8 = \left(-3\right)\left(-3\right) - 1, \mathspace -8 = \left(-3\right)\cdot 2 - 2\]

        This goes against our definition, since we want $0 \leq r < \left|m\right|$, but this could also be a definition that makes sense.
    \end{subparag}
\end{parag}

\begin{parag}{Mod operation}
    By $r = a \Mod m$, we denote the remainder when dividing $a$ by $m$. Thus: 
    \[a = mq + r \implies a \Mod m = r\]
\end{parag}

\begin{parag}{Example 1}
    If we split a pie with 8 slices for 3 people, then $8 \Mod 3 = 2$ slices will be left over.
\end{parag}

\begin{parag}{Example 2}
    If someone leaves home at 7 a.m. for a 40-hour trip, they arrive at $\left(7 + 40\right) \Mod 24 = 23$ the day after.
\end{parag}

\begin{parag}{Congruence}
    Two integers $a$ and $b$ are called \important{congruent} modulo $m$ if $m \divides a - b$, i.e. if $m$ divides $a - b$. This is denoted by: 
    \[\congruent{a}{b}{m}\]
    
    \begin{subparag}{Equivalent definitions}
        In fact, we can see that all the following statements are equivalent:
        \begin{itemize}
            \item $\congruent{a}{b}{m}$
            \item $\left(a - b\right) \Mod m = 0$
            \item $a \Mod m = b \Mod m$
            \item $r_a = r_b$
        \end{itemize}
    \end{subparag}

    \begin{subparag}{Example}
        Let us take $m = 4$. Then, we can draw the following number line, where numbers of the same colour belong to the same class:
        \svghere{CongruenceClassNumberLine.svg}
    \end{subparag}
\end{parag}

\begin{parag}{Example}
    If we know that $\congruent{x}{3}{5}$, then clearly $x$ is not a multiple of $5$. Indeed, $x \Mod 5 = 3 \neq 0$.

    However, clearly, if $\congruent{x}{25}{5}$, then $x$ is a multiple of 5 since $\congruent{25}{0}{5}$.
\end{parag}

\begin{parag}{Definition: Equivalence relation}
    On a set, a \important{equivalence relation} is a relation (denoted by $\sim$) which satisfies:
    \begin{enumerate}
        \item Reflexivity: $a \sim a$
        \item Symmetry: $a \sim b \implies b \sim a$
        \item Transitivity: $a \sim b \land b \sim c \implies a \sim c$
    \end{enumerate}
\end{parag}

\begin{parag}{Lemma: Equivalence relation}
    Congruence modulo $m$ is an equivalence relation on the set of integers.
\end{parag}

\begin{parag}{Theorem: Properties}
    If $\congruent{a}{a'}{m}$ and $\congruent{b}{b'}{m}$, then:
    \begin{enumerate}
        \item $\congruent{a + b}{a' + b'}{m}$
        \item $\congruent{ab}{a'b'}{m}$
        \item $\congruent{a^n}{\left(a'\right)^n}{m}$
    \end{enumerate}

    \vspace{1em}

    An equivalent stating of this theorem is: 
    \begin{enumerate}
        \item $\left(a + b\right) \Mod m = \left(a \Mod m + b \Mod m\right) \Mod m$
        \item $\left(ab\right) \Mod m = \left(\left(a \Mod m\right)\left(b \Mod m\right)\right) \Mod m$
        \item $a^n \Mod m = \left(\left(a \Mod m\right)^n\right) \Mod m$
    \end{enumerate}

    We have to be careful to see that the mod does not get distributed into the $n$ of the exponent.

    \begin{subparag}{Proof}
        First, looking at the number line, we realise that $a \Mod m = \left(a + m k\right) \Mod m$.

        Then, we can write $a = m q_a + r_a$ and $b = m q_b + r_b$, which gives us: 
        \begin{multiequality}
        \left(a + b\right) \Mod m =\ & \left(m\left(q_a + q_b\right) + r_a + r_b\right) \Mod m  \\
        =\ & \left(r_a + r_b\right) \Mod m  \\
        =\ & \left(a \Mod m + b \Mod m\right) \Mod m 
        \end{multiequality}
        
        Doing something similar with multiplications: 
        \begin{multiequality}
        \left(ab\right) \Mod m =\ & \left(\left(m q_a + r_a\right)\left(m q_b + r_b\right)\right) \Mod m  \\
        =\ & \left(m^2 q_a q_b + mq_a r_b + mq_b r_a + r_a r_b\right) \Mod m  \\
        =\ & \left(m\left(m q_a q_b + q_a r_b + q_b r_a\right) + r_a r_b\right) \Mod m  \\
        =\ & \left(r_a r_b\right)\Mod m  \\
        =\ & \left(\left(a \Mod m\right)\left(b \Mod m\right)\right) \Mod m
        \end{multiequality}

        Finally, for the powers, we can use what we have just found (since $n \in \mathbb{N}$):
        \begin{multiequality}
        a^{n} \Mod m =\ & a\cdot a \cdots a \Mod m  \\
        =\ & \left(a \Mod m\right)\left(a \Mod m\right)\cdots\left(a \Mod m\right) \Mod m  \\
        =\ & \left(a \Mod m\right)^n \Mod m 
        \end{multiequality}

        \qed
    \end{subparag}
\end{parag}

\begin{parag}{Example}
    For example: 
    \[\left(123\cdot 97\right)\Mod 2 = \left(\left(123 \Mod 2\right) \left(97 \Mod 2\right)\right) \Mod 2 = \left(1 \cdot 1\right)\Mod2 = 1\]
    
    Also: 
    \[\left(\left(1234 \cdot 333\right) + 41\left(76 + 5\right)\right)\Mod 2 = \left(0\cdot 1 + 1\left(0 + 1\right)\right) \Mod 2 = 1\]
    
    Let's say we now wonder if $2 + 2^{1000}$ is divisible by $3$: 
    \begin{multiequality}
    \left(2 + 2^{1000}\right) \Mod3 =\ & \left(2 \Mod 3 + \left(2 \Mod 3\right)^{1000}\right) \Mod 3  \\
    =\ & \left(2 + \left(-1 \Mod 3\right)^{1000}\right) \Mod 3  \\
    =\ & \left(2 + \left(-1\right)^{1000} \Mod 3\right) \Mod 3  \\
    =\ & \left(2 + 1\right) \Mod 3  \\
    =\ & 0 
    \end{multiequality}

    Thus, we can conclude that $2 + 2^{1000}$ is indeed divisible by 3.
    
    Using a similar trick, let's compute: 
    \[\left(9^{1000} + 9^{10^{6}}\right) \Mod5 = \left(\left(-1\right)^{1000} + \left(-1\right)^{10^{6}}\right) \Mod5 = \left(1 + 1 \right) \Mod 5 = 2\]
    
    Thus, $9^{1000} + 9^{10^6}$ is not divisible by 5.
\end{parag}

\begin{parag}{Even numbers}
    We know that: 
    \[1234 = 1\cdot 10^{3} + 2\cdot 10^{2} + 3\cdot 10^1 + 4\]
    
    Thus, taking a mod 2: 
    \begin{multiequality}
    1234 \Mod 2 =\ & \left(1 \cdot \left(10 \Mod 2\right)^3 + 2 \left(10 \Mod 2\right)^2 + 3 \cdot \left(10 \Mod 2\right) + 4\right) \Mod 2  \\
    =\ & \left(0 + 0 + 0 + 4\right) \Mod 2  \\
    =\ & 0 
    \end{multiequality}
    
    Thus, only the last number matters to say if a number is even in base 10.
\end{parag}

\begin{parag}{Division by 9}
   We know that $\congruent{10}{1}{9}$, thus: 
   \[\congruent{10^n}{1^n = 1}{9}\]
   
   This implies that:
   \begin{multiequality}
   1234 \Mod 9 =\ & \left(1 \cdot \left(10 \Mod 9\right)^3 + 2 \cdot \left(10 \Mod 9\right)^2 + 3 \cdot \left(10 \Mod 9\right)^1 + 4 \Mod9\right) \Mod 9  \\
   =\ & \left(1 + 2 + 3 + 4\right) \Mod 9  \\
   =\ & 10 \Mod 9  \\
   =\ & 1 
   \end{multiequality}
   
   In other words, to know if a number is divisible by 9, we can only add up the digits.
\end{parag}

\begin{parag}{Back to Diffie-Hellman}
    For Diffie-Hellman, we needed to compute quickly big exponents modulo a given number. Let's see an example of how to do that more quickly.

    To compute $28^{42} \Mod 67$, we can do it successively. Indeed: 
    \[28^{42} \Mod 67 = \left(\cdots \left(\left(\left(\left(28^2 \Mod 67\right) \cdot 28\right) \Mod 67\right) \cdot 28\right) \Mod 67 \cdots \right)\Mod 67\]
    
    This is $\Theta\left(n\right)$, but there is even an algorithm named fast-exponentiation which reaches $\Theta\left(\log\left(n\right)\right)$.
\end{parag}




\end{document}
