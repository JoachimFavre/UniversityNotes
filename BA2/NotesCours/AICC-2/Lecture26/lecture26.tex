\documentclass[a4paper]{article}

% Expanded on 2022-05-31 at 18:12:32.

\usepackage{../../style}

\title{AICC 2}
\author{Joachim Favre}
\date{Mardi 31 mai 2022}

\begin{document}
\maketitle

\lecture{26}{2022-05-31}{Reed-Solomon codes}{
\begin{itemize}[left=0pt]
    \item Definition of polynomials over a field.
    \item Explanation of Lagrange interpolation, to make a polynomial going through some points.
    \item Proof of the Fundamental Theorem of Algebra.
    \item Explanation of the construction of Reed-Solomon codes.
    \item Proof of properties of Reed-Solomon codes.
\end{itemize}

}

\subsection{Polynomials over $\mathbb{F}_q$}
\parag{Definition}{
    Let $\bvec{u} = \left(u_1, \ldots, u_m\right) \in \mathbb{F}_q^m$ be a vector. 

    We define the following polynomial: 
    \[p_{\bvec{u}}\left(x\right) = u_1 + u_2 x + \ldots + u_m x^{m-1}, \mathspace \forall x \in \mathbb{F}_q\]
}

    
\parag{Example}{
    Let's pick $m = 3$, $\mathbb{F}_7$ and $\bvec{u} = \left(1, 2, 3\right) \in \mathbb{F}_7^3$. Then: 
    \[p_{\bvec{u}}\left(x\right) = 1 + 2x + 3x^2\]

    We can then evaluate it for some values: 
    \[p_{\bvec{u}}\left(0\right) = 1\] 
    \[p_{\bvec{u}}\left(1\right) = 1 + 2 + 3 = 6\] 
    \[p_{\bvec{u}}\left(2\right) = 1 + 2\cdot 2 + 3\cdot 2^2 = 1 + 4 + 12 = 17 = 3\]
    

    A good way to think about this, is to see a polynomial as a table:
    \begin{center}
    \begin{tabular}{c|ccccccc}
        $x$ & 0 & 1 & 2 & 3 & 4 & 5 & 6 \\
        \hline
        $p_{\bvec{u}}\left(x\right)$ & 1 & 6 & 3 & 6 & 1 & 2 & 2
    \end{tabular}
    \end{center}
}

\parag{Definition: Degree of a polynomial}{
    The \important{degree of a polynomial} is the highest exponent $i$ for which $x^i$ has a non-zero coefficient.

    By convention, we define the degree of the polynomial $0$ to be $-\infty$: 
    \[\deg\left(0\right) = -\infty\]
    
    \subparag{Example}{
        For instance: 
        \[\deg\left(1\right) = 0\] 
        \[\deg\left(x + 2\right) = 1\] 
        \[\deg\left(x^5 + x^2 + 1\right) = 5\]
        \[\deg\left(5x^5 + x^2 + 2\right) = 5\]
    }
}

\parag{Polynomial interpolation}{
    Given $\left\{\left(a_1, y_1\right), \ldots, \left(a_k, y_k\right)\right\}$, where $\left(a_i, y_i\right) \in \mathbb{F}_q^2$, we want to find a polynomial such that it goes through all those points. We do not care what it does between two points. 

    In other words, we want to find a polynomial $p_{\bvec{u}}\left(x\right)$ of degree $\deg\left(p_{\bvec{u}}\left(x\right)\right) \leq k - 1$ such that: 
    \[p_{\bvec{u}}\left(a_i\right) = y_i, \mathspace i = 1, \ldots, k\]
    
    Finding such a polynomial is equivalent to finding its $\bvec{u}$. 

    \subparag{Example}{
        For instance, on $\mathbb{F} = \mathbb{R}$, we can illustrate this problem as:
        \imagehere[0.5]{PolynomialInterpolation.png}
    }
    

    \subparag{Lagrange Interpolation}{
        The solution to this problem is Lagrange interpolation: 
        \begin{enumerate}
            \item First, we make a polynomial $Q_i\left(x\right)$ with coefficients in $\mathbb{F}_5$ and degree less than or equal to $k-1$ such that: 
                \begin{functionbypart}{Q_i\left(x\right)}
                1, \mathspace \text{when } x = a_i \\
                0, \mathspace \text{when } x = a_j \text{ for } j \neq i
                \end{functionbypart}

                To make such a polynomial, we notice that the following almost has the right property, it is equal to 0 when $x = a_j$ for $j\neq i$, but it is not always equal to 1 when $x = a_i$: 
                \[\prod_{j \neq i}^{} \left(x - a_j\right)\]
                 
                So, to get 1 when $x = a_i$, we only need to normalise it: 
                \[Q_i\left(x\right) = \frac{\prod_{j \neq i}^{} \left(x - a_j\right)}{\prod_{j\neq i}^{} \left(a_i - a_j\right)}\]

                As required, this is of degree less than or equal to $k-1$.
                
            \item Second, we can pick: 
                \[p\left(x\right) = \sum_{i=1}^{k} y_i Q_i\left(x\right)\]
                
              This does indeed give us that:
              \[p\left(a_j\right) = \sum_{i=1}^{k} y_i Q_i\left(x\right) = 0 + \ldots + 0 + y_j \cdot 1 + 0 + \ldots + 0 = y_j\]
              
              Also, adding polynomials does not change their degrees, so we do get that $\deg\left(p\right) \leq k -1$.

        \end{enumerate}

        This technique always works and gives us a polynomial of degree at most $k - 1$. This is the best polynomial if we want to minimise the degree.
    }
}

\parag{Example}{
    Let $\mathbb{F} = \mathbb{F}_5$, $k = 3$, and the set of the following points: 
    \[\left\{\left(a_1, y_1\right), \left(a_2, y_2\right), \left(a_3, y_3\right)\right\} = \left\{\left(2, 3\right), \left(3, 4\right), \left(4, 0\right)\right\}\]
    
    Let's first find our polynomials $Q_i\left(x\right)$: 
    \[Q_1\left(x\right) = \frac{\left(x - a_2\right)\left(x - a_3\right)}{\left(a_1 - a_2\right)\left(a_1 - a_3\right)}\]
    
    This is indeed equal to 0 when $x = a_2$ or $x = a_3$, but equal to 1 when $x = a_1$. Also, we notice that we are never dividing by zero. This gives us: 
    \[Q_1\left(x\right) = \frac{\left(x - 3\right)\left(x-4\right)}{\left(2 - 3\right)\left(2-4\right)} = \frac{x^2 + 3x + 2}{2} = 3\left(x^2 + 3x + 2\right) = 3x^2 + 4x + 1\]

    Using a similar reasoning, we find: 
    \[Q_2\left(x\right) = \frac{\left(x - a_1\right)\left(x - a_3\right)}{\left(a_2 - a_1\right)\left(a_2 - a_3\right)} = 4x^2 + x + 2\] 
    \[Q_3\left(x\right) = \frac{\left(x - a_1\right)\left(x - a_2\right)}{\left(a_3 - a_1\right)\left(a_3 - a_2\right)} = 3x^2 + 3\]
    
    We can finally form our resulting polynomial: 
    \begin{multiequality}
        p\left(x\right) =\ & y_1 Q_1\left(x\right) + y_2 Q_2\left(x\right) + y_3 Q_3\left(x\right) \\
        =\ & 3\left(3x^2 + 4x + 1\right) + 4\left(4x^2 + x + 2\right) \\
        =\ & 25x^2 + 16x + 11 \\
        =\ & x + 1 \\
        =\ & p_{\left(1, 1, 0\right)}\left(x\right) 
    \end{multiequality}
    
    The fact that the coefficient of the $x^2$ cancels out is a coincidence. 
}

\parag{Definition: Roots of a polynomial}{
    The \important{roots} of a polynomial $p\left(x\right)$ over $\mathbb{F}_q$ are numbers $b \in \mathbb{F}_q$ such that: 
    \[p\left(b\right) = 0\]
    
    \subparag{Example 1}{
        Let us pick $\mathbb{F} = \mathbb{F}_3$, and the polynomial $p\left(x\right) = \left(x - 1\right)\left(x - 2\right)$. We can see that: 
        \[p\left(x\right) = x^2 - 3x + 2 = x^2 + 2\]
        
        We can list all possibilities: 
        \begin{center}
        \begin{tabular}{c|ccc}
            $x$ & 0 & 1 & 2 \\
            \hline
            $p\left(x\right)$ & 2 & 0 & 0
        \end{tabular}
        \end{center}
        
        So, we have found that the roots of $\left(x-1\right)\left(x-2\right)$ over $\mathbb{F}_3$ are $1$ and $2$.
    }

    \subparag{Example 2}{
        Let us pick $\mathbb{F} = \mathbb{F}_3$ and the polynomial $p\left(x\right) = x^2 + 1$. Again, let's draw a table:
        \begin{center}
        \begin{tabular}{c|ccc}
            $x$ & 0 & 1 & 2 \\
            \hline
            $p\left(x\right)$ & 1 & 2 & 2
        \end{tabular}
        \end{center}
        
        So, we can see that $x^2 + 1$ has no root over $\mathbb{F}_3$.
    }
    
    \subparag{Example 3}{
        Let us pick $\mathbb{F} = \mathbb{F}_5$ and the following polynomial: 
        \[p\left(x\right) = \left(x - 1\right)^2 = x^2  - 2x + 1\]
        
        As usual, we draw a table:
        \begin{center}
        \begin{tabular}{c|ccccc}
            $x$ & 0 & 1 & 2 & 3 & 4 \\
            \hline
            $p\left(x\right)$ & 1 & 0 & 1 & 4 & 4
        \end{tabular}
        \end{center}
        
        This polynomial thus has one root over $\mathbb{F}_5$, which is 1.
    }
}

\parag{Fundamental theorem of algebra}{
    Let $p\left(x\right)$ be a polynomial over $\mathbb{F}$ of degree at most $m-1$, meaning $\deg\left(p\left(x\right)\right) \leq m -1$.

    If this is not the all-zero polynomial ($\exists x \in \mathbb{F} : p\left(x\right) \neq 0$), then the number of its distinct roots is at most $m - 1$. 

    \subparag{Remark}{
        We know this theorem for $\mathbb{F} = \mathbb{R}$, but it is valid for any field. Indeed, for instance, $p\left(x\right) = ax^2 + bx + c$ can have 0, 1 or 2 roots over $\mathbb{R}$.
    }

    \subparag{Proof}{
        We are going to do this proof by the contrapositive, meaning that we want to show that if $p\left(x\right)$ has at least $m$ roots, then $p\left(x\right) = 0$ for all $x \in \mathbb{F}$.

        Let $b_1, \ldots, b_m$ be the first $m$ of those distinct roots. We consider the following mapping:
        \[\begin{split}
        \psi: \mathbb{F}^m &\longmapsto \mathbb{F}^m \\
        \bvec{u} &\longmapsto \left(p_{\bvec{u}}\left(b_1\right), \ldots, p_{\bvec{u}}\left(b_m\right)\right)
        \end{split}\]
        
        Clearly, the map $\psi$ is onto. In other words, for every $\left(y_1, \ldots, y_m\right)$, there exists a $\bvec{u}$ for which $\left(y_1, \ldots, y_m\right) = \left(p_{\bvec{u}}\left(b_1\right), \ldots, p_{\bvec{u}}\left(b_m\right)\right)$ by Lagrange. Also, since we are mapping $\mathbb{F}^m$ to $\mathbb{F}^m$, which have the same number of elements, $\psi$ being onto means that it is one-to-one. This implies that there exists a unique $\bvec{u} \in \mathbb{F}^m$, such that: 
        \[\left(p_{\bvec{u}}\left(b_1\right), \ldots, p_{\bvec{u}}\left(b_m\right)\right) = \left(0, \ldots, 0\right)\]
        
        However, since $\bvec{u} = \bvec{0}$ works, we know it is the only one. This implies that: 
        \[p_{\bvec{u}}\left(x\right) = 0, \mathspace \forall x \in\mathbb{F}\]
        as requested.

        \qed
    }
}

\subsection{Reed-Solomon codes}
\parag{Reed-Solomon codes}{
    The Reed-Solomon codes (RS codes) are constructed using the following algorithm:
    \begin{enumerate}
        \item We choose a finite field $\mathbb{F} = \mathbb{F}_q$ (where $q$ is a prime power).
        \item We choose $n$ and $k$ such that:  
        \[0 < k \leq n \leq q\]
        \item We choose $n$ \textit{distinct} elements: 
        \[a_1, \ldots, a_n \in \mathbb{F}\]
        \item The codeword associated to the information word $\bvec{u} \in \mathbb{F}^k$ is given by the following mapping: 
            \[\bvec{u} \mapsto \bvec{c} = \left(p_{\bvec{u}}\left(a_1\right), \ldots, p_{\bvec{u}}\left(a_n\right)\right) \in \mathbb{F}\]
    \end{enumerate}
    
    We notice that $p_{\bvec{u}}\left(x\right)$ is a polynomial of degree $\deg\left(p_{\bvec{u}}\left(x\right)\right) \leq k - 1$.

    \subparag{Remark}{
        As we will see, Reed-Solomon codes have very interesting properties. However, their weak part is that we want $n \leq \left|\mathbb{F}_q\right| = q$, so we need a big field if we want to work with a big $n$. With $q$ being a prime number, the computations are rather easy since we are working modulo $q$, but in computers we prefer to use $q = 2^k$, which make the computations even harder.

    }
    
}

\parag{Example}{
    Let us pick $\mathbb{F} = \mathbb{F}_3$, $n = 3, k = 2$, $a_1 = 2, a_2 = 1$ and $a_3 = 0$. We can draw the following table:
    \begin{center}
    \begin{tabular}{c|c|c}
        $\bvec{u}$ & $p_{\bvec{u}}\left(x\right)$ & $\bvec{c}$ \\
        \hline
        00 & $0$ & 000 \\
        01 & $x$ & 210 \\
        02 & $2x$ & 120 \\
        10 & $1$ & 111 \\
        11 & $1+x$ & 021 \\
        12 & $1+2x$ & 201 \\
        20 & $2$ & 222 \\
        21 & $2+x$ & 102 \\
        22 & $2+2x$ & 012
    \end{tabular}
    \end{center}
}


\parag{Properties}{
    \begin{enumerate}[left=0pt]
        \item RS codes are linear.
        \item RS codes have dimension $k$, meaning they have $q^k$ different codewords.
        \item RS codes are MDS, meaning $d_{min}\left(\mathcal{C}\right) = n - k + 1$.
    \end{enumerate}
    
    \subparag{Proof property 1}{
        Let $\bvec{u}, \bvec{v} \in \mathbb{F}^k$ and $\alpha \in \mathbb{F}$. Let us consider the codeword resulted by $\left(\alpha \bvec{u} + \bvec{v}\right)$:
        \begin{multiequality}
        p_{\alpha \bvec{u} + \bvec{v}}\left(x\right) =\ & \left(\alpha u_1 + v_1\right) + x\left(\alpha u_2 + v_2\right) + x^2 \left(\alpha u_3 + v_3\right) + \ldots \\
        =\ & \alpha\left(u_1 + u_2 x + u_3 x^2 + \ldots\right) + v_1 + v_2 x + v_3 x^2 + \ldots \\
        =\ & \alpha p_{\bvec{u}}\left(x\right) + p_{\bvec{v}}\left(x\right) 
        \end{multiequality}
        
        We see that: 
        \begin{multiequality}
            \bvec{c}_{\alpha \bvec{u} + \bvec{v}} =\ & \left(p_{\alpha \bvec{u} + \bvec{v}}\left(a_1\right), \ldots, p_{\alpha \bvec{u} + \bvec{v}}\left(a_n\right)\right) \\
        =\ & \left(\alpha p_{\bvec{u}}\left(a_1\right) + p_{\bvec{v}}\left(a_1\right), \ldots, \alpha p_{\bvec{u}}\left(a_n\right) + p_{\bvec{v}}\left(a_n\right)\right) \\
        =\ & \alpha\left(p_{\bvec{u}}\left(a_1\right), \ldots, p_{\bvec{u}}\left(a_n\right)\right) + \left(p_{\bvec{v}}\left(a_1\right), \ldots, p_{\bvec{v}}\left(a_n\right)\right) \\
        =\ & \alpha \bvec{c}_{\bvec{u}} + \bvec{c}_{\bvec{v}} 
        \end{multiequality}
        
        Thus, the sum of any two codewords is again a codeword, showing that the code is linear.

        \qed
    }
    
    \subparag{Proof property 2}{
        We want to show that each $\bvec{u}$ leads to a different codeword. This would indeed imply that our code has dimension $k$.

        Let's suppose for contradiction that $\bvec{u} \neq \bvec{v}$ lead to the same codeword, meaning that: 
        \begin{multiequation}
        & \left(p_{\bvec{u}}\left(a_1\right), \ldots, p_{\bvec{u}}\left(a_n\right)\right) = \left(p_{\bvec{v}}\left(a_1\right), \ldots, p_{\bvec{v}}\left(a_n\right)\right)  \\
        \iff & \left(p_{\bvec{u}}\left(a_1\right), \ldots, p_{\bvec{u}}\left(a_n\right)\right) - \left(p_{\bvec{v}}\left(a_1\right), \ldots, p_{\bvec{v}}\left(a_n\right)\right) = \bvec{0}  \\
        \iff & \left(p_{\bvec{u}}\left(a_1\right) - p_{\bvec{v}}\left(a_1\right), \ldots, p_{\bvec{u}}\left(a_n\right) - p_{\bvec{v}}\left(a_n\right)\right) = \bvec{0}
        \end{multiequation}
        
        However, we showed in the first proof that this is equivalent to: 
        \[\left(p_{\bvec{u} - \bvec{v}}\left(a_1\right), \ldots, p_{\bvec{u} - \bvec{v}}\left(a_n\right)\right) = \bvec{0}\]
        
        We know $p_{\bvec{u} - \bvec{v}}\left(x\right)$ has degree $\deg\left(p\left(x\right)\right) \leq k - 1$, and we want it to have $n$ zeroes. But, since $k \leq n$, this implies that this is the zero polynomial and thus that $\bvec{u} - \bvec{v} = \bvec{0}$, which is a contradiction to $\bvec{u} \neq \bvec{v}$.

        \qed
    }

    \subparag{Proof property 3}{
        We want to show that $d_{min}\left(\mathcal{C}\right) \geq n - k + 1$. By Singleton's bound it would indeed imply that $d_{min}\left(\mathcal{C}\right) = n - k + 1$, and thus that our code is MDS.

        Let's suppose for contradiction that $d_{min} \leq n - k$.

        For a linear code, we know that the minimum distance is the minimum weight of the code: 
        \[d_{min}\left(\mathcal{C}\right) = \min_{\bvec{c} \neq \bvec{0}} w\left(\bvec{c}\right)\]

        Let $\bvec{c} = \left(p_{\bvec{u}}\left(a_1\right), \ldots, p_{\bvec{u}}\left(a_n\right)\right) \neq \bvec{0}$ be one of the non-zero codeword that has the minimum weight. Using our assumption and what we just found, we have: 
        \[w\left(\bvec{c}\right) = d_{min} \leq n - k\]
        
        This means that $\bvec{c}$ must contain at least $k$ zeros, and thus $p_{\bvec{u}}\left(x\right)$ must have at least $k$ roots. However, by the fundamental theorem of algebra, this means that $\bvec{u} = \bvec{0}$ and thus that $\bvec{c} = \bvec{0}$, which is a contradiction.

        \qed
    }
}

\parag{Usefulness}{
    Reed-Solomon Codes are very interesting since they are MDS, linear, les us choose various parameters (we can pick an arbitrary $\mathbb{F}_q$ and $0 < k < n \leq q$), they have a nice construction and they have elegant and efficient decoding algorithms. 

    They have many practical applications, such as storage devices (for CDs, DVDs, QR Codes, \ldots), digital radio and television broadcast, high-speed modems (such as ADSL, xDSL, \ldots), deep space exploration modems (to communicate with Voyager 2, \ldots) and wireless mobile communication (including cellular phones and microwave links).
}


\end{document}
