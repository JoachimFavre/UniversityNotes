\documentclass[a4paper]{article}

% Expanded on 2022-02-23 at 12:51:27.

\usepackage{../../style}

\title{Analyse 2}
\author{Joachim Favre}
\date{Mercredi 23 février 2022}

\begin{document}
\maketitle

\lecture{2}{2022-02-23}{D'autres équations avec des différences, facile!}{
\begin{itemize}[left=0pt]
    \item Fin des EDVS, et définition de leur solution maximale.
    \item Définition des EDL1.
    \item Définition des équation homogène associée, démonstration du principe de superposition des solutions, explication de la méthode de la variation de la constante; puis utilisation de tous ces principes pour démontrer la méthode pour trouver la solution générale à une EDL1.
\end{itemize}
}

\parag{Résumé pour les EDVS}{
    Pour résoudre une EDVS, donc une équation sous la forme $f\left(y\right) y' = g\left(x\right)$ où $f: I \mapsto \mathbb{R}$ et $g: J \mapsto \mathbb{R}$ sont continues, on pose l'équation: 
    \[\int f\left(y\right) dy = \int g\left(x\right) dx\]
    
    Puisqu'on a une constante des deux côtés, il nous suffit de prendre une seule constante. 
}

\parag{Exemple}{
    Résolvons l'équation suivante: 
    \[\frac{y'\left(x\right)}{y^2\left(x\right)} = 1\]
    
    C'est une EDVS, car $f\left(y\right) = \frac{1}{y^2}$ est continue sur $\mathbb{R}^*_+$ ou sur $\mathbb{R}^*_-$, donc on peut considérer chaque intervalle séparément, et $g\left(x\right) = 1 : \mathbb{R} \mapsto \mathbb{R}$.

    Posons nos intégrales: 
    \[\int \frac{dy}{y^2} = \int dx \implies -\frac{1}{y} = x + C, \mathspace \forall C \in \mathbb{R}\]
    
    Ainsi, on peut résoudre notre équation pour obtenir: 
    \[y = -\frac{1}{x + C}, \mathspace \forall C \in \mathbb{R}\]

    C'est la solution générale sur $\left]-\infty, -C\right[ $ et $\left]-C, +\infty\right[ $ ($y$ n'est pas continue en $-C$, donc elle n'est clairement pas dérivable). Il ne faut pas oublier qu'une solution générale est une fonction et un intervalle.

    Supposons maintenant qu'on cherche une solution telle que $y\left(0\right) = b_0 \in \mathbb{R}^*$: 
    \[y\left(0\right) = -\frac{1}{C} = b_0 \implies C = -\frac{1}{b_0}\]
    
    Ainsi: 
    \[y\left(x\right) = - \frac{1}{x - \frac{1}{b_0}} = \frac{b_0}{1 - x b_0}\]
    
    Si $b_0 > 0 \implies \frac{1}{b_0} > 0$, notre solution particulière est: 
    \[y\left(x\right) = -\frac{1}{x - \frac{1}{b_0}} \mathspace \text{sur } \left]-\infty, \frac{1}{b_0}\right[ \ni 0 \]
    
    Si $b_0 < 0$, notre solution particulière est donnée par: 
    \[y\left(x\right) = - \frac{1}{x - \frac{1}{b_0}} \mathspace \text{sur } \left] \frac{1}{b_0}, +\infty\right[ \ni 0\]

    On choisit l'intervalle de manière à ce que $0$ soit dedans.
}

\parag{Définition: Solution maximale}{
    Une \important{solution maximale} d'une EDVS avec la condition initiale $y\left(x_0\right) = b_0$ où $x_0 \in J$ et $b_0 \in I$ est une fonction $y\left(x\right)$ de classe $C^1$ satisfaisant l'équation, la condition initiale, et qui est définie sur le plus grand intervalle possible. 

    Le théorème des EDVS nous dit que si $f\left(y\right) \neq 0$ sur $I$, alors il existe une unique solution maximale. Toute solution avec la même condition initiale est une restriction de la solution maximale.

    \subparag{Remarque}{
        Dans l'exemple ci-dessus, nous avons trouvé les solutions maximales pour les conditions initiales $x_0 = 0$ et $b_0 \in \mathbb{R}^*$, $b_0 > 0$ ou $b_0 < 0$.
    }
}

\subsection[EDL1]{Équations différentielles linéaires du premier ordre}
\parag{Définition: EDL1}{
    Soit $I \subset \mathbb{R}$ un intervalle ouvert. Nous appelons \important{équation différentielle linéaire du premier ordre} (EDL1) une équation de la forme suivante: 
    \[y'\left(x\right) + p\left(x\right) y\left(x\right) = f\left(x\right)\]
    où $p, f : I \mapsto \mathbb{R}$ sont continues.


    Une \important{solution} est une fonction $y : I \mapsto \mathbb{R}$ de classe $C^1$ satisfaisant l'équation.
}

\parag{Équation homogène associée}{
    Commençons par considérer l'équation suivante (qui est plus facile): 
    \[y'\left(x\right) + p\left(x\right) y\left(x\right) = 0\]
    
    Cette équation s'appelle \important{l'équation homogène associée} à l'EDL1 $y'\left(x\right) + p\left(x\right)y\left(x\right) = f\left(x\right)$.

    Nous avons deux cas. Soit $y\left(x\right) = 0\ \forall x \in \mathbb{R}$, soit $\frac{y'\left(x\right)}{y\left(x\right)} = -p\left(x\right)$ (qui est une EDVS).

    Continuous à travailler sur le deuxième cas: 
    \[\int \frac{dy}{y} = -\int p\left(x\right)dx \implies \log\left|y\right| = -P\left(x\right) + C_1, \mathspace C_1 \in \mathbb{R}\]
    où $P\left(x\right)$ est une primitive de $p\left(x\right)$.

    Ainsi: 
    \[\left|y\right| = e^{-P\left(x\right) + C_1} = \underbrace{e^{C_1}}_{C_2 > 0} e^{-P\left(x\right)} \implies y\left(x\right) = C_2 e^{-P\left(x\right)}, \mathspace C_2 \in \mathbb{R}^* \]
    
    Cependant, puisque $y\left(x\right) = 0$ est aussi une solution, on obtient que la solution générale de l'équation homogène associée sur $I \subset \mathbb{R}$ est: 
    \[y\left(x\right) = Ce^{-P\left(x\right)}, \mathspace \forall x \in I, \forall C \in \mathbb{R}\]
}

\parag{Principe de superposition de solutions}{
    Soit $I \subset \mathbb{R}$ un intervalle ouvert, et $p, f_1, f_2 : I \mapsto \mathbb{R}$ des fonction continues.

    Supposons que $v_1 : I \mapsto \mathbb{R}$ de classe $C^1$ est une solution de l'équation: 
    \[y' + p\left(x\right) y = f_1\left(x\right)\]
    
    Supposons aussi que $v_2 : I \mapsto \mathbb{R}$ de classe $C^1$ est une solution de l'équation: 
    \[y' + p\left(x\right) y = f_2\left(x\right)\]
    
    Alors, pour tout couple $C_1, C_2 \in\mathbb{R}$, la fonction $v\left(x\right) =  C_1 v_1\left(x\right) + C_2 v_2\left(x\right)$ est une solution de l'équation: 
    \[y' + p\left(x\right)y = C_1 f_1\left(x\right) + C_2 f_2\left(x\right)\]
    
    \subparag{Vérification}{
        Nous pouvons facilement vérifier notre équation: 
        \begin{multiequality}
            v'\left(x\right) + p\left(x\right) v\left(x\right) =\ & C_1 v_1'\left(x\right) + C_2 v_2'\left(x\right) + p\left(x\right)\left(C_1 v_1\left(x\right) + C_2 v_2\left(x\right)\right) \\
            =\ & C_1\left(v_1'\left(x\right) + p\left(x\right)v_1\left(x\right)\right) + C_2\left(v_2'\left(x\right) + p\left(x\right) v_2\left(x\right)\right) \\
            =\ &  C_1 f_1\left(x\right) + C_2 f_2\left(x\right)
        \end{multiequality}

        Ce qui termine notre démonstration.

        \qed
    }
}

\parag{Méthode de la variation de constante}{
    Nous cherchons une solution particulière de $y'\left(x\right) + p\left(x\right)y\left(x\right) = f\left(x\right)$ où $p, f : I\mapsto \mathbb{R}$ sont des fonctions continues, sous la forme suivante:
    \[v\left(x\right) = C\left(x\right)e^{-P\left(x\right)}\]
    où $P\left(x\right)$ est une primitive de $p\left(x\right)$ sur $I$ et $C\left(x\right)$ est une fonction inconnue de $x$.

    Nous appelons ceci un \textit{Ansatz}. On suppose que notre solution est d'une certaine forme et on espère que ça nous amène à une solution (en l'occurrence, on prend une solution similaire à celle qu'on avait trouvée pour les équations homogènes associées).

    Si $v\left(x\right)$ est une solution de l'équation, alors: 
    \[v'\left(x\right) + p\left(x\right) v\left(x\right) = f\left(x\right)\implies C'\left(x\right) e^{-P\left(x\right)} + C\left(x\right) e^{-P\left(x\right)}\left(-p\left(x\right)\right) + p\left(x\right)C\left(x\right)e^{-P\left(x\right)} = f\left(x\right)\]
    

    Ceci nous permet donc d'obtenir que: 
    \[C'\left(x\right)e^{-P\left(x\right)} = f\left(x\right) \implies C'\left(x\right) = f\left(x\right)e^{P\left(x\right)} \implies C\left(x\right) = \int f\left(x\right) e^{P\left(x\right)}dx\]
    
    Nous avons donc trouvé une solution particulière de l'équation, qui est: 
    \[v\left(x\right) = \left(\int f\left(x\right) e^{P\left(x\right)} dx\right) e^{-P\left(x\right)}\]
    où $P\left(x\right)$ est une primitive de $p\left(x\right)$ sur $I$.
}

\parag{Exemple}{
    Résolvons l'équation différentielle suivante: 
    \[y' + y = 5x + 1\]
    
    C'est une EDL1 avec $p\left(x\right) = 1$ et $f\left(x\right) = 5x + 1$, où $p, f : \mathbb{R} \mapsto \mathbb{R}$ sont continues. On trouve que $P\left(x\right) = x$ est une primitive (quelconque, donc sans constante) de $p\left(x\right)$.

    Nous savons que la solution générale de l'équation homogène associée $y' + y = 0$ est : 
    \[y\left(x\right) = Ce^{-P\left(x\right)} = Ce^{-x}, \mathspace \forall x \in \mathbb{R}, \forall C \in \mathbb{R}\]
    
    Pour trouver une solution particulière de l'équation $y' + y = 5x + 1$ on calcule: 
    \[C\left(x\right) = \int f\left(x\right) e^{P\left(x\right)}dx = \int \left(5x + 1\right) e^{x} dx = 5\int xe^{x} dx + \int e^x dx\]

    Nous pouvons calculer la première intégrale par partie:
    \[C\left(x\right) = 5xe^x - 5 \int e^x dx + \int e^x dx = 5xe^x - 4e^x\]

    On peut prendre une constante arbitraire, donc nous prenons $C = 0$. 

    Ainsi, on a trouvé une solution particulière de $y' + y = 5x + 1$: 
    \[v\left(x\right) = \left(5x e^x - 4e^x\right)e^{-x} = 5x - 4\]
    
    Nous pouvons vérifier que c'est bien une solution: 
    \[v'\left(x\right) + v\left(x\right) = 5 + 5x - 4 = 5x + 1\]
    comme attendu.
}

\parag{Proposition pour les EDL1}{
    Soient $p, f : I \mapsto \mathbb{R}$ des fonctions continues. Supposons que $v_0 : I \mapsto \mathbb{R}$ est une solution particulière de l'équation suivante: 
    \[y'\left(x\right) + p\left(x\right) y\left(x\right) = f\left(x\right)\]

    Alors, la solution générale de cette équation est: 
    \[v\left(x\right) = v_0\left(x\right) + Ce^{-P\left(x\right)}, \mathspace \forall C \in \mathbb{R}\]
    où $P\left(x\right)$ est une primitive de $p\left(x\right)$ sur $I$.
    
    \subparag{Preuve}{
        Nous allons montrer que toute solution de cette équation est de la forme $v_0\left(x\right) + Ce^{-P\left(x\right)}$.

        Soit $v_1\left(x\right)$ une solution de $y'\left(x\right) + p\left(x\right)y\left(x\right) = f\left(x\right)$. On a aussi que $v_0\left(x\right)$ est une solution de la même équation.

        Alors, d'après le principe de superposition de solutions, la fonction $v_1\left(x\right) - v_0\left(x\right)$ est une solution de l'équation:
        \[y'\left(x\right) + p\left(x\right)y\left(x\right) = f\left(x\right) - f\left(x\right) = 0\]

        Ainsi, $v_1\left(x\right) - v_0\left(x\right)$ est une solution de l'équation homogène: 
        \[y'\left(x\right) + p\left(x\right) y\left(x\right) = 0\]

        Cependant, c'est une EDVS, donc nous savons que la solution générale de cette équation homogène est: 
        \[v\left(x\right) = Ce^{-P\left(x\right)}, \mathspace C \in \mathbb{R} \text{ arbitraire}\]
        où $P\left(x\right)$ est une primitive de $p\left(x\right)$ sur $I$.

        On en déduit qu'il existe une valeur de $C \in \mathbb{R}$ telle que $v_1\left(x\right) - v_0\left(x\right) = Ce^{-P\left(x\right)}$. Ainsi, on obtient que la solution $v_1\left(x\right)$ est de la forme: 
        \[v_1\left(x\right) = v_0\left(x\right) + Ce^{-P\left(x\right)}\]
        
        Puisque $v_1\left(x\right)$ était une solution arbitraire, nous obtenons que l'ensemble de toutes les solutions de l'équation $y'\left(x\right) + p\left(x\right)y\left(x\right) = f\left(x\right)$ est: 
        \[v\left(x\right) = v_0\left(x\right) + Ce^{-P\left(x\right)}, \mathspace C \in \mathbb{R}, x \in I\]
        
        Donc, par définition, $v\left(x\right)$ est la solution générale.
        
        \qed
    }
}


\end{document}
