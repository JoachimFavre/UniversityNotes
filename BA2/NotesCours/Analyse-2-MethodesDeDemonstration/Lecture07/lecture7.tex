\documentclass[a4paper]{article}

% Expanded on 2022-03-14 at 11:56:33.

\usepackage{../../style}

\title{Méthodes de démonstration}
\author{Joachim Favre}
\date{Lundi 14 mars 2022}

\begin{document}
\maketitle

\lecture{7}{2022-03-14}{Les tiroirs et les chaussettes}{
\begin{itemize}[left=0pt]
    \item Explication du principe des tiroirs et des chaussettes de Dirichlet.
\end{itemize}

}

\parag{Méthode 6: Principe des tiroirs et des chaussettes de Dirichlet}{
    Si $\left(n+1\right)$ objets sont placé dans $n$ tiroirs, alors au moins un tiroir contient 2 objets ou plus.

    Plus généralement, si $n$ objets sont placés dans $k$ tiroirs, alors au moins un tiroir contient $\left\lceil \frac{n}{k} \right\rceil $ objets, ou plus.

    \subparag{Définition: Fonction plafond}{
        La fonction plafond est simplement un arrondi vers le haut, mais nous pouvons la définir formellement comme: 
        \[\left\lceil \frac{n}{k} \right\rceil \over{=}{déf} \min\left\{m \in\mathbb{N} \telque m \geq \frac{n}{k}\right\}\]

    }
    
}

\parag{Exemple 1}{
    Soient $a_1, a_2, a_3, a_4$ des nombres entiers. Alors, il existe 2 d'entre eux tels que leur différence est divisible par 3.

    \subparag{Preuve}{
        Soient $r_1, r_2, r_3, r_4$ les restes de division de $a_1, a_2, a_3, a_4$ par 3. Nous savons que $r_1, r_2, r_3, r_4 \in \left\{0, 1, 2\right\}$. Alors, nous savons qu'il existe $i, j$ avec $i \neq j$ pour lequel $r_i = r_j$, par le principe des tiroirs et des chaussettes de Dirichlet. Or, puisque $a_i$ et $a_j$ ont le même reste, nous savons que $a_i - a_j$ est divisible par 3. En effet, $a_i = 3m + r$ et $a_j = 3n + r$ où $m, n \in \mathbb{Z}$ et $r \in \left\{0, 1, 2\right\}$, donc:
        \[a_i - a_j = 3m + r - \left(3n + r\right) = 3\left(m - n\right)\]

        \qed
    }
}

\parag{Exemple 2}{
    Dans un groupe de $n$ personnes, où $n \geq 2$, il existe au moins 2 personnes avec le même nombre de connaissances dans le groupe.

    \subparag{Preuve}{
        Utilisions la disjonction de cas.

        \begin{enumerate}[left=0pt]
            \item Supposons qu'il y a quelqu'un qui ne connait personne dans le groupe.

                Alors, clairement, il n'existe personne qui connait tout le monde. On obtient donc que le nombre $K$ de connaissances pour chacun est entre 0 et $\left(n-2\right)$, ce qui représente $\left(n-1\right)$ possibilités. Par le principe des tiroirs de Dirichlet, puisqu'il y a $n$ personnes et $\left(n-1\right)$ possibilités, il existe au moins 2 personnes avec le même nombre de connaissances.
            \item Supposons maintenant que tout le monde connait au moins quelqu'un dans le groupe.

                Puisque personne ne connait personne, nous savons que $1 \leq K \leq n-1$. Ainsi, nous avons à nouveau $\left(n-1\right)$ possibilités pour $n$ personnes, et donc, par le principe des tiroirs, il existe au moins 2 personnes avec le même nombre de connaissances.
        \end{enumerate}
        
        \qed
    }
}


\end{document}
