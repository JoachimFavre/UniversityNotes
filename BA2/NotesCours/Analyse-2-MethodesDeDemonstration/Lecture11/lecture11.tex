% !TeX program = lualatex

\documentclass[a4paper]{article}

% Expanded on 2022-03-28 at 10:22:21.

\usepackage{../../style}

\title{Méthodes de démonstration}
\author{Joachim Favre}
\date{Lundi 28 mars 2022}

\begin{document}
\maketitle

\lecture{11}{2022-03-28}{Récurrence}{
\begin{itemize}[left=0pt]
    \item Explication de la méthode de démonstration dite par récurrence.
    \item Explication de la méthode de démonstration dite par récurrence généralisée.
\end{itemize}

}

\begin{parag}{Méthode 7: Récurrence}
    Soit $P\left(n\right)$ une proposition qui dépend de $n \in \mathbb{N}$, où $n \geq n_0$. Supposons que:
    \begin{enumerate}
        \item $P\left(n_0\right)$ est vraie.
        \item $P\left(n\right)$ implique que $P\left(n+1\right)$ pour tous $n \geq n_0$ naturels.
    \end{enumerate}
    
    Alors $P\left(n\right)$ est vraie pour tous $n \geq n_0$.

    \begin{subparag}{Remarque}
        Cette méthode de démonstration découle directement de l'axiome suivant (le principe fondamental de la récurrence):
    
        Soit $S \subset \mathbb{N}$. Si ce sous-ensemble est tel que $0 \in \mathbb{N}$ et pour tout $n \in S$ on a $\left(n + 1\right) \in S$, alors, $S = \mathbb{N}$.
    \end{subparag}
\end{parag}

\begin{parag}{Exemple}
    L'inégalité suivante tient pour tout $n \geq 1$ naturels: 
    \[1 + \frac{1}{2^2} + \ldots + \frac{1}{n^2} \leq 2 - \frac{1}{n}\]
    
    \begin{subparag}{Preuve}
        Puisque nous voulons montrer ceci pour tout $n$, alors c'est intéressant de le faire par récurrence.

        À l'examen, il est important de dire quelle est la proposition que nous essayons de démontrer, ainsi, nous voulons montrer $P\left(n\right)$, où:
        \[P\left(n\right): 1 + \frac{1}{2^2} + \ldots + \frac{1}{n^2} \leq 2 - \frac{1}{n}\]
        
        \begin{enumerate}[left=0pt]
            \item Commençons par l'initialisation. $P\left(1\right)$ est bien vraie car:
            \[1 \leq 2 - \frac{1}{1} = 1\]
        \item Faisons maintenant l'hérédité. Nous supposons que $P\left(n\right)$ est vraie (hypothèse de récurrence, HR), et nous voulons en déduire $P\left(n+1\right)$. Commençons par le côté gauche de $P\left(n+1\right)$: 
            \begin{multiequality}
                1 + \frac{1}{2^2} + \ldots + \frac{1}{n^2} + \frac{1}{\left(n + 1\right)^2} \over{\leq}{HR}\ & 2 - \frac{1}{n} + \frac{1}{\left(n+1\right)^2} \\
                =\ & 2 - \frac{n^2 + n + 1}{n\left(n + 1\right)^2}  \\
                =\ & 2 - \frac{n\left(n+1\right)}{n\left(n+1\right)^2} - \underbrace{\frac{1}{n\left(n+1\right)^2}}_{> 0}  \\
                \leq\ & 2 - \frac{1}{n + 1} 
            \end{multiequality}
            
            Nous avons donc aussi montré $P\left(n+1\right)$.
        \item Pour finir, il ne faut pas oublier la conclusion, qui est aussi importante. Puisque $P\left(1\right)$ est vraie et $P\left(n\right) \implies P\left(n+1\right)$, nous avons montré que $P\left(n\right)$ est vraie pour tout $n \geq 1$, où $n \in \mathbb{N}$.

            \qed
        \end{enumerate}
    \end{subparag}
\end{parag}

\begin{parag}{Récurrence généralisée}
    Soit $P\left(n\right)$ une proposition qui dépend de $n \in \mathbb{N}$, où $n \geq n_0$. Supposons qu'il existe un $k \in \mathbb{N}$ tel que:
    \begin{enumerate}
        \item $P\left(n_0\right), \ldots, P\left(n_0 + k\right)$ sont vraies.
        \item $\left\{P\left(n\right), \ldots, P\left(n + k\right)\right\}$ impliquent $P\left(n + k + 1\right)$ pour tout $n \geq n_0$.
    \end{enumerate}
    
    Alors, $P\left(n\right)$ est vraie pour tout $n \geq n_0$, où $n \in \mathbb{N}$.
\end{parag}

\begin{parag}{Définition: Suite de Fibonacci}
    La suite de Fibonacci $\left(f_n\right)$ est définie telle que: 
    \[f_1 = 1, \mathspace f_2 = 1, \mathspace f_{n+2  = f_n + f_{n+1}}\]
    
    Les premiers termes de cette suite sont: 
    \[1, \mathspace 1, \mathspace 2, \mathspace 3, \mathspace 5, \mathspace 8, \mathspace 13, \mathspace 21, \mathspace \ldots\]
\end{parag}

\begin{parag}{Exemple}
    Pour tout $n \geq 3$, nous avons: 
    \[3f_n = f_{n+2} + f_{n-2}\]
    
    \begin{subparag}{Preuve}
        Notre proposition $P\left(n\right)$ est définie comme: 
        \[P\left(n\right) : 3f_n = f_{n+2} + f_{n-2}, \mathspace \forall n \geq 3\]
        
        \begin{enumerate}[left=0pt]
            \item Faisons la base. $P\left(3\right)$ fonctionne car: 
            \[3\cdot 2 = 5 + 1\]

            De plus, $P\left(4\right)$ tient aussi: 
            \[3\cdot 3 = 8+1\]
            
        \item Passons maintenant à l'hérédité. Puisque nous utilisons le principe de la récurrence généralisée, nous supposons que $P\left(n\right)$ et $P\left(n + 1\right)$ sont vraies (HR), pour un $n \geq 3$. Nous voulons démontrer que $P\left(n+2\right)$ est vraie: 
            \begin{multiequality}
            3f_{n+2} \over{=}{déf}\ & 3f_n + 3f_{n+1} \\
            \over{=}{HR}\ & \left(f_{n + 2} + f_{n-2}\right) + \left(f_{n + 3} + f_{n-1}\right) \\
            =\ & \underbrace{f_{n+2} + f_{n+3}}_{= f_{n+4}} + \underbrace{f_{n-2} + f_{n-1}}_{= f_n} \\
            =\ & f_{n+4} + f_n 
            \end{multiequality}
         
            Ce qui montre bien que $P\left(n+2\right)$ est vraie.
        \item Pour conclure, nous avons montrer que $P\left(3\right)$ et $P\left(4\right)$ sont vraies, puis que $P\left(n\right) \land P\left(n + 1\right) \implies P\left(n+2\right)$. Ainsi, par récurrence généralisée, nous avons montré que $P\left(n\right)$ est vraie pour tout $n \geq 3$, où $n \in \mathbb{N}$.
        \end{enumerate}

        \qed
    \end{subparag}
\end{parag}



\end{document}
