\documentclass[a4paper]{article}

% Expanded on 2022-04-25 at 13:17:27.

\usepackage{../../style}

\title{DSD}
\author{Joachim Favre}
\date{Lundi 25 avril 2022}

\begin{document}
\maketitle

\lecture{15}{2022-04-25}{Becoming explicit}{
\begin{itemize}[left=0pt]
    \item Explanation of how simulation of HDL works.
    \item Explanation of explicit processes.
\end{itemize}

}

\subsection{Explicit processes}
\parag{Introduction}{
    For now, we have only seen implicit processes. To understand explicit processes, we first need to understand how HDL is simulated.
}

\parag{Simulation}{
    We need to simulate a parallel system on a sequential machine, a computer. Thus, this is not a trivial task.

    We realise that we only have a change in our gates when there is an event. This means that we can use an \textit{event-driven} approach.

    First, we need to define the concept of sensitivity. We see that a process is sensitive to some things. For instance, the output of an AND gate is sensitive to the input of the gate.

    When simulating, we can propagate events to processes that are sensitive to it, until all signals become stable. Note that there is no notion of time, since we are considering the zero-delay model. When we have an event, a sequential $\delta$-iterator is initialised for all signals in the process to the left of the \texttt{<=} operator (signals that may change). Also, these signals get now two values, namely the value seen before the even and the resulting $\delta$-value. We take as many $\delta$-iteration steps as we need to have the circuit reach stability (or stops if there are too many iterations (100k for instance) to block oscillations). Finally, the computer copies back the values of the $\delta$-variables back to signals, and the simulation continues until the next event.

    \subparag{Example}{
        Let's consider the following circuit:
        \imagehere[0.5]{DeltaSimulationCircuitExample.png}

        Also, let's consider the following timing diagram:
        \imagehere[0.5]{DeltaSimulationTimingDiagramExample.png}

        So, at the event, the simulation fills in the following table (where ``Tr'' stands for ``Triggered by an event'', ``Fi'' stand for ``Fire an event'' and ``St'' stands for ``Stable''): 
        \imagehere{DeltaSimulationTableExample.png}
    }

    Note that we can simulate this by adding the gate-delay to the fire action. However, they are not synthesizable since we cannot enforce a delay on a gate, they will be here but we cannot manipulate them.
}

\begin{filecontents*}[overwrite]{processSyntax.code}
<process_identifier>: PROCESS(<sensitivity_list>) IS
BEGIN 
    <process_body>
END PROCESS <process_identifier>;
\end{filecontents*}

\parag{Explicit processes}{
    Simulating implicit processes can be very computationally intensive (or at least was at the time VHDL was invented). This is why explicit processes where invented.

    To make one, we write inside of the description section of an architecture:
    \importcode{processSyntax.code}{vhdl}
    
    The sensitivity list must contain all the signals that will trigger the process. Thus, for instance, we can compare an implicit process, and its explicit version with an incomplete sensitivity list:
    \imagehere{ExpliciteProcessSensitivityList.png}

    Note that, however, the sensitivity list will be ignored when synthesizing the circuit. Thus, both will give the same result on a Gecko board.
}

\begin{filecontents*}[overwrite]{sequentialExplicitProcess.code}
example: PROCESS (A, B) IS
BEGIN
    Z <= NOT A;
    Z <= '0';
    Z <= NOT A AND B;
    Z <= A NOR B;
END PROCESS example;
\end{filecontents*}

\parag{Behaviour}{
    To understand the behaviour of an explicit process, we have to remember the $\delta$-iterator. Explicit processes are executed sequentially, and this is really important.

    For instance, let's look at the following code:
    \importcode{sequentialExplicitProcess.code}{vhdl}

    This is not a short-circuit, this process describes a nor gate since we are working sequentially. Note that the three first lines in the body are completely useless. Also, all values on the right hand side of the \texttt{<=} operator are always the values at $\delta$-iteration 0 (hence the values seen at the event).
}




\end{document}
