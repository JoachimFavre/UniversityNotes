\documentclass[a4paper]{article}

% Expanded on 2022-04-11 at 19:00:31.

\usepackage{../../style}

\title{DSD}
\author{Joachim Favre}
\date{Mercredi 13 avril 2022}

\begin{document}
\maketitle

\lecture{14}{2022-04-13}{Architectures}{
\begin{itemize}[left=0pt]
    \item Explanation of how to write the architecture of a circuit.
    \item Explanation of how to make signals and constants.
\end{itemize}

}

\parag{Architecture}{
    Each entity needs to have at least one architecture (functional description), but it can have multiple. Indeed, we can implement the same functionality with different circuits.
}

\begin{filecontents*}[overwrite]{architectureVDHL.code}
ARCHITECTURE <id_name> OF <entity_name> IS
    <declaration_section>
BEGIN
    <description_section>
END <id_name>;
\end{filecontents*}


\parag{Structure}{
    The architecture body looks like:
    \importcode{architectureVDHL.code}{VHDL}

    Each architecture has a unique name, and holds a reference to its corresponding entity. Also, they have a declaration section (which holds signals (wires), constants, components, and other stuff that are synthesisable) and a description section (implicite processes, explicite processes, or component instantiations). Every parts execute in parallel, not line by line.

    Most VHDL code written and used nowadays by designers is written following the RTL (Register Transfer Level) principle. This means that we separate sequential logic functions (flip-flops, for instance) from the combinatorial logic functions (the circuit logic of FSM, for example). This allows us to be more modular and to verify more easily that we did not make any mistake.
}


\parag{Example 1}{
    If we want to make a triple or gate in VHDL, we can write \texttt{Y <= A OR B OR C}. Note that the input of a gate is on the left hand side of the \texttt{<=} operator, and the inputs of the gates are on the right hand side.

    Note that VHDL knows the gates \texttt{NOR, AND, NAND, OR, NOR, XOR, XNOR}.
}

\begin{filecontents*}[overwrite]{fullExampleVHDL.code}
LIBRARY ieee;
USE ieee.std_logic_1164.all;

ENTITY exercise IS
PORT (
    A, B, C : IN std_logic_vector(1 DOWNTO 0);
    X : OUT std_logic_vector(1 DOWNTO 0)
);
END exercise;

ARCHITECTURE example OF exercise IS
BEGIN
    X <= (A AND NOT B) NOR C;
END example;
\end{filecontents*}


\parag{Example 2}{
    Let's consider the following code:
    \importcode{fullExampleVHDL.code}{VHDL}

    This realises a circuit mapping the LSB of \texttt{X} to \texttt{(A(0) AND NOT B(0)) NOR C(0)}, and the MSB of \texttt{X} to \texttt{(A(1) AND NOT B(1)) NOR C(1)}.
}

\begin{filecontents*}[overwrite]{architectureMoreComplex.code}
ARCHITECTURE gate OF full_adder IS
    SIGNAL s_Xor, s_And1, s_And2 : std_logic;
BEGIN
    Count <= s_And1 OR s_And2;
    s_And1 <= A AND B;
    Sum <= s_Xor XOR Cin;
    s_And2 <= s_Xor AND Cin;
    s_Xor <= A XOR B;
END gate1;
\end{filecontents*}

\begin{filecontents*}[overwrite]{architectureMoreComplexMoreConcise.code}
ARCHITECTURE gate OF full_adder IS
BEGIN
    Count <= (A AND B) OR (Cin AND (A XOR B));
    Sum <= (A XOR B) XOR Cin;
END gate1;
\end{filecontents*}

\parag{More complex example}{
    Let's write code for the following circuit:
    \imagehere[0.7]{VHDLArchitectureExampleCircuit.png}

    We need in between signals to be able to describe it. The syntax is:

    \begin{center}
        \texttt{SIGNAL <signal\_name>, <signal\_name>, ... : <signal\_type>;}
    \end{center}

    Each signal has a unique name, and it is a good practice to do camelCase, prefixed with \texttt{s\_}. Each signal has a type, and signals of the same type can be concatenated by separating their name by a comma. Note that signals do not have a pre-defined flow (we must thus not define if they are IN or OUT, since they do not flow).

    Putting everything together, we get:
    \importcode{architectureMoreComplex.code}{VHDL}

    Note that the order of the lines between the \texttt{BEGIN} and the \texttt{END} does not matter. In fact, we could also suppress wires:
    \importcode{architectureMoreComplexMoreConcise.code}{VHDL}
}

\begin{filecontents*}[overwrite]{SRLatchVHDL.code}
ARCHITECTURE srLatchArchitecture of srLatchEntity IS
SIGNAL s_Lamp, s_Y : std_logic;
BEGIN
    s_lamp <= (NOT S) NAND s_Y;
    s_Y <= (NOT R) NAND s_Lamp;
    Y <= s_Y;
    Lamp <= s_Lamp;
END srLatchArchitecture;
\end{filecontents*}

\parag{SR Latch}{
    Let us write the code for a SR-latch:
    \imagehere[0.5]{SRLatchCircuit.png}

    We need to introduce signals. Indeed, outputs are write-only and inputs are read-only (meaning that we cannot write on an output).
    \importcode{SRLatchVHDL.code}{VHDL}
}

\begin{filecontents*}[overwrite]{faultyCode.code}
ARHITECTURE common of doesItWork IS
    SIGNAL s_MySignal : std_logic;
BEGIN
    s_MySignal <= NOT (A(0)) AND A(1);
    Q <= s_MySignal;
    s_MySignal <= A(0) AND NOT A(1);
\end{filecontents*}

\parag{Short circuits}{
    Let's consider the following code:
    \importcode{faultyCode.code}{VHDL}

    We can access the value $i-1$\Th bit of \texttt{A} by doing \texttt{A(i)}. We have two times \texttt{s\_MySignal} on the left hand side of an assignment, so this yields the following circuit:
    \imagehere[0.7]{VHDLShortCircuitExample.png}

    This is a short circuit, and in real life it explodes \textit{(Professor screams ``boom'' and wakes up the whole class)}.
}

\parag{Constant}{
    To define a constant, we can use the following syntax:
    \begin{center}
        \texttt{CONSTANT <constant\_name> : <constant\_type> := <value>;}
    \end{center}

    It is a good practice to prefix constants prefixed with \texttt{c\_}. If we are defining a constant of type \texttt{std\_logic}, we have to give the values in between single quote: \texttt{'1'}. Similarly, if we are defining a constant of type \texttt{std\_logic\_vector}, we have to give the value in between double quotes: \texttt{"001"}. We can also represent hexadecimal using a capital \texttt{X}: \texttt{X"FA00"}. However, note that every character represents four bits, thus we cannot use \texttt{s\_example <= X"18"} if \texttt{s\_example} is 5 bits (we would be giving it 8 bits when it is expecting 5). However, we can use the glue operator, taking \texttt{s\_example <= "1" \& X"8"}. Also, we can use the macro \texttt{OTHERS}: \texttt{s\_example <= (23 => '1', OTHERS => '0')} puts every bit to 0, except for the \nth{24}. Finally, we can modify part of the bits by doing \texttt{s\_example(23 DOWNTO 16) <= X''80''}.

}



\end{document}
