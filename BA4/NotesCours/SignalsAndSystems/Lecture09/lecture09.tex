% !TeX program = lualatex
% Using VimTeX, you need to reload the plugin (\lx) after having saved the document in order to use LuaLaTeX (thanks to the line above)

\documentclass[a4paper]{article}

% Expanded on 2023-05-19 at 22:32:28.

\usepackage{../../style}

\title{Signals and systems}
\author{Joachim Favre}
\date{Vendredi 19 mai 2023}

\begin{document}
\maketitle

\lecture{9}{2023-05-19}{A new composition type}{
\begin{itemize}[left=0pt]
    \item Application of Laplace transform for solving differential equations.
    \item Definition of feedback composition of systems.
    \item Definition of three types of block diagrams: direct form, cascade form and parallel form.
\end{itemize}

}

\begin{parag}{Property: Stability}
    Let $\mathcal{H}$ be a LTI system with transfer function $H\left(s\right)$.

    $\mathcal{H}$ is stable if and only if the ROC of $H\left(s\right)$ includes the imaginary axis.

    \begin{subparag}{Proof}
        This comes directly from our earlier observations.
    \end{subparag}
\end{parag}

\begin{parag}{Property: Causality}
    Let $\mathcal{H}$ be a LTI system with transfer function $H\left(s\right)$.

    $\mathcal{H}$ is causal implies that the ROC of $H\left(s\right)$ is right-sided.
    
    \begin{subparag}{Proof}
        If $\mathcal{H}$ is causal, then its impulse response $h\left(t\right)$ is right-sided and thus $H\left(s\right)$ is also right-sided.

        \qed
    \end{subparag}
\end{parag}

\begin{parag}{Property: Rational function causality}
    Let $\mathcal{H}$ be a LTI system with a \textit{rational} transfer function $H\left(s\right) = \frac{P\left(s\right)}{Q\left(s\right)}$.

    $\mathcal{H}$ is causal \textit{if and only if} the ROC of $H\left(s\right)$ is right-sided.

    \begin{subparag}{Justification}
        It is possible to prove this theorem by doing a partial fraction decomposition of $H\left(s\right)$, and arguing that the inverse Laplace transform of the result, $h\left(t\right)$, must be right-sided. This indeed implies that $\mathcal{H}$ is causal.
    \end{subparag}
\end{parag}

\begin{parag}{Example}
    Let us consider the following impulse response: 
    \[h\left(t\right) = e^{-a\left(t - t_0\right)} u\left(t - t_0\right)\]
    
    We can see in a table that: 
    \[h\left(t\right) \laplacepair H\left(s\right) = \frac{e^{-s t_0}}{s + a}, \mathspace \cre\left(s\right) > -a\]
    
    We can see that the ROC of this transfer function is always right-sided. However, if $t_0 < 0$, the system is not causal. This shows that the previous theorem does not hold if $H\left(s\right)$ is not a rational function. In this case, we only have an implication, not an equivalence.
\end{parag}

\begin{parag}{Differential equations}
    We know the following property of Laplace transforms: 
    \[\frac{d}{dt} x\left(t\right) \laplacepair sX\left(s\right)\]
    
    We can  use this property to solve differential equations representing both stable and non-stable LTI systems (which is better than Fourier transforms).
\end{parag}

\begin{parag}{Example 1}
    Let's suppose that a LTI system satisfies the following equation: 
    \[y\left(t\right) + 3 \frac{d}{dt}y\left(t\right) + 2\frac{d^2}{dt^2}y\left(t\right) = x\left(t\right) + \frac{1}{2} \frac{d}{dt} x\left(t\right)\]
    
    We apply a Laplace transform on both sides, giving: 
    \[Y\left(s\right) + 3sY\left(s\right) + 2s^2 Y\left(s\right) = X\left(s\right) + \frac{1}{2} s X\left(s\right)\]
    
    We can arrange everything as: 
    \[H\left(s\right) = \frac{Y\left(s\right)}{X\left(s\right)} = \frac{1 + \frac{1}{2}s}{1 + 3s + 2s^2} = \frac{\frac{3}{2}}{2s + 1} - \frac{\frac{1}{2}}{s + 1}\]
    
    We need to know its ROC to get a unique solution for this equation; all three ROCs are coherent. The first is anti-causal, the second is neither causal nor anti-causal and the third is stable and causal system. 
    \imagehere[0.5]{LaplaceDifferentialEquationExample1ROCPlot.png}
\end{parag}

\begin{parag}{Example 2}
    Let's suppose that we know a causal LTI system satisfies the following equation: 
    \[\frac{d^2}{dt^2}y\left(t\right) - y\left(t\right) = \frac{d^2}{dt^2} x\left(t\right)\]
    
    Applying a Laplace transform on both sides, we get: 
    \[s^2 Y\left(s\right)-  Y\left(s\right) = s^2 X\left(s\right) \implies H\left(s\right) = \frac{Y\left(s\right)}{X\left(s\right)} = \frac{s^2}{s^2 - 1} = \frac{s}{2}\left(\frac{1}{s+1} + \frac{1}{s-1}\right)\]
    
    We see that we have two poles: $s = -1$ and $s = 1$. Since we know the LTI is causal, this implies that the ROC is right-sided and thus that: 
    \[\cre\left(s\right) > 1\]
\end{parag}

\begin{parag}{System composition}
    Since we have the same properties as Fourier transforms, everything works the same for Laplace transforms when doing system composition.

    A parallel composition yields a transfer function of:
    \[H\left(s\right) = H_1\left(s\right)H_2\left(s\right)\]
    
    A series composition yields a transfer function of: 
    \[H\left(s\right) = H_1\left(s\right) + H_2\left(s\right)\]
\end{parag}

\begin{parag}{Feedback composition}
    Let us consider the following new type of composition, named \important{feedback composition}:
    \imagehere[0.5]{FeedbackComposition.png}

    Note that the result of $\mathcal{H}_2$ is subtracted from $x\left(t\right)$ (there is a minus sign).

    \begin{subparag}{Transfer function}
        We see that we have the following system of equations: 
        \[\begin{systemofequations} Y\left(s\right) = H_1\left(s\right) W\left(s\right) \\ W\left(s\right) = X\left(s\right) - H_2\left(s\right)Y\left(s\right) \end{systemofequations}\]
        
        This implies that: 
        \autoeq{Y\left(s\right) = H_1\left(s\right) X\left(s\right) - H_1\left(s\right) H_2\left(s\right) Y\left(s\right) \implies H\left(s\right) = \frac{Y\left(s\right)}{X\left(s\right)} = \frac{H_1\left(s\right)}{1 + H_1\left(s\right)H_2\left(s\right)}}
    \end{subparag}
\end{parag}

\begin{parag}{Example 1: Feedback control system}
    Let us consider the following type of feedback composition systems:
    \imagehere[0.9]{FeedbackControlSystem.png}

    A system of this form is called a \important{feedback control system}. Its transfer function is: 
    \[H_{tot}\left(s\right) = \frac{H\left(s\right) G\left(s\right)}{1 + H\left(s\right)G\left(s\right)B\left(s\right)}\]

    This type of systems is used a lot in practice, though we often need to consider many other subtleties.

    \begin{subparag}{Example}
        An example of such a system is heating in a house. In that case, $x\left(t\right)$ is the temperature we aim, and $y\left(t\right)$ is the actual temperature in the house. The controller determines how hard the heater $\mathcal{H}$ should behave. The sensor verifies everything goes as planned by measuring the output, and using this to compute the error $e\left(t\right)$.
    \end{subparag}
\end{parag}

\begin{parag}{Example 2}
    Let us consider a causal LTI system which satisfies the equation: 
    \[\frac{dy\left(t\right)}{dt} + 3y\left(t\right) = x\left(t\right)\]
    
    This implies that: 
    \[H\left(s\right) = \frac{1}{s+3}, \mathspace \cre\left(s\right) > -3\]
    
    However, we can write this using a feedback composition of systems: 
    \[H\left(s\right) = \frac{\frac{1}{s}}{1 + \frac{3}{s}} = \frac{H_1\left(s\right)}{1 + H_2\left(s\right)H_1\left(s\right)}\]
    
    \imagehere[0.5]{BlockDiagramExample.png}

    However, we know that: 
    \[\int_{-\infty}^{t} x\left(\tau\right)d\tau \laplacepair \frac{1}{s}X\left(s\right)\]
    
    Thus, $\mathcal{H}_1$ is an integrator and $\mathcal{H}_2$ is a scaling. Writing differential equations using such diagrams is really interesting for their understanding, let us thus dig deeper in this.
\end{parag}

\begin{parag}{Block diagrams}
    We will develop three types of \important{block diagrams} hereinafter. They work for any causal LTI system with a rational transfer function.
\end{parag}

\begin{parag}{Direct form representation}
    Let us now consider a causal LTI system which satisfies the following differential equation: 
    \[\frac{d^2 y\left(t\right)}{dt^2} + 3\frac{dy\left(t\right)}{dt} + 2y\left(t\right) = x\left(t\right)\]
    
    Let us define the following intermediate signals:
    \[e\left(t\right) = \frac{d^2 y\left(t\right)}{dt^2}, \mathspace f\left(t\right) = \frac{dy\left(t\right)}{dt}\]
    
    This implies that:
    \[f\left(t\right) = \int_{-\infty}^{t} e\left(\tau\right)d\tau, \mathspace y\left(t\right) = \int_{-\infty}^{t} f\left(\tau\right)d\tau\]
    \[e\left(t\right) = x\left(t\right) - 3 f\left(t\right) -2 y\left(t\right)\]

    We can thus write this in the following form, named \important{direct form representation}:
    \imagehere[0.8]{DirectFormRepresentationExample.png}

    This is named direct form representation since we can very easily convert to and from the differential equation, even without having to compute the transfer function.
\end{parag}

\begin{parag}{Cascade form representation}
    Let us consider again a causal LTI system which satisfies the same differential equation: 
    \[\frac{d^2 y\left(t\right)}{dt^2} + 3\frac{dy\left(t\right)}{dt} + 2y\left(t\right) = x\left(t\right)\]

    We can find that: 
    \[H\left(s\right) = \left(\frac{1}{s+1}\right)\left(\frac{1}{s + 2}\right), \mathspace \cre\left(s\right) > -1\]

    However, using the same trick that we used in the example that made us dig into block diagrams, we notice that we can write both $\frac{1}{s+1}$ and $\frac{1}{s+2}$ as feedback composition of systems. We then only need to put them in series to get a multiplication:
    \imagehere[0.8]{CascadeFormExample.png}

    This is named the \important{cascade form}.
\end{parag}

\begin{parag}{Parallel form representation}
    Let us consider again a causal LTI system which satisfies the same differential equation: 
    \[\frac{d^2 y\left(t\right)}{dt^2} + 3\frac{dy\left(t\right)}{dt} + 2y\left(t\right) = x\left(t\right)\]

    Using partial fraction expansion, we can find that: 
    \[H\left(s\right) = \frac{1}{s+1} - \frac{1}{s+2}, \mathspace \cre\left(s\right) > -1\]

    We again use the same trick to write both fraction as a feedback composition of systems. Then, to add them, we can just put everything in parallel:
    \imagehere[0.6]{ParallelFormExample.png}

    This is named the \important{parallel form}.
\end{parag}

\end{document}
