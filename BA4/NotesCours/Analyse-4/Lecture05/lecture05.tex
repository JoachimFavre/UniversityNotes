% !TeX program = lualatex
% Using VimTeX, you need to reload the plugin (\lx) after having saved the document in order to use LuaLaTeX (thanks to the line above)

\documentclass[a4paper]{article}

% Expanded on 2023-03-23 at 15:15:44.

\usepackage{../../style}

\title{Analyse 4}
\author{Joachim Favre}
\date{Jeudi 23 mars 2023}

\begin{document}
\maketitle

\lecture{5}{2023-03-23}{''Ils ont pas bien nettoyé le tableau, il reste des traces de sang''}{
\begin{itemize}[left=0pt]
    \item Explication de la méthode de calcul du rayon de convergence d'une série de Taylor et de Laurent. 
    \item Définition de la série de Laurent, sa partie régulière et singulière, le résidu, le rayon de convergence, les points réguliers, les pôles d'ordre $m$ et les SEI.
    \item Exemple de calcul de séries de Laurent triviales.
\end{itemize}

}

\parag{Définition: Boule ouverte}{
    Soit $z_0 \in \mathbb{C}$ et $\epsilon >0$. La \important{boule ouverte} centrée en $z_0$ et de rayon $R$ est donnée par: 
    \[B_{R}\left(z_0\right) = \left\{z \in \mathbb{C} \suchthat \left|z - z_0\right| < R\right\}\]
}

\parag{Théorème}{
    Soit $\Omega \subset\mathbb{C}$ ouvert, $f: \Omega \mapsto \mathbb{C}$ holomorphe, $z_0 \in \Omega$ et $\epsilon > 0$ tel que:
    \[B_{\epsilon}\left(z_0\right) = \left\{z \in \mathbb{C} \suchthat \left|z - z_0\right| < \epsilon\right\} \subset \Omega\]

    Alors, $f$ est analytique dans cette boule $B_{\epsilon}\left(z_0\right)$, i.e: 
    \[f\left(z\right) = \sum_{n=0}^{\infty} c_n \left(z - z_0\right)^n, \mathspace \forall z \in B_{\epsilon}\left(z_0\right)\]
    
    Avec les coefficients donnés par: 
    \[c_n = \frac{f^{\left(n\right)} \left(z_0\right)}{n!}\]

    \subparag{Intuition}{
        Les fonctions holomorphes sont donc automatiquement analytique. Cela montre encore qu'elles sont beaucoup plus puissantes que les fonctions réelles (de classe $C^{\infty}\left(\mathbb{R}\right)$ par exemple).
    }

    \subparag{Réciproque}{
        La réciproque est vraie.

        En d'autres mots, si nous avons une fonction $f$ définie par 
        \[f\left(z\right) = \sum_{n=0}^{\infty} c_n \left(z - z_0\right)^n,\]
        et qu'elle est bien définie dans une boule $B_{\epsilon}\left(z_0\right)$ (elle converge partout), alors $f$ est holomorphe dans cette boule $B_{\epsilon}\left(z_0\right)$.

        De plus: 
        \[f'\left(z\right) = \sum_{n=1}^{\infty} n c_n \left(z- z_0\right)^{n-1}\]
    }
}

\parag{Remarque}{
    Par la FIC, nous pouvons exprimer $c_n$ à l'aide d'une intégrale: 
    \[c_n = \frac{f^{\left(n\right)}\left(z_0\right)}{n!} = \frac{1}{2\pi i} \int_{\Gamma} \frac{f\left(z\right)}{\left(z - z_0\right)^{n+1}}dz, \mathspace n \geq 0\]
    
    Cette formule n'est pas pratique pour calculer les coefficients, mais elle sera utile par la suite. En effet, nous voyons que, si nous pouvions prendre $n = -1$, alors $c_{-1}$ nous donnerait directement l'intégrale de notre fonction. Donc, si nous sommes en mesure de donner un sens à ce $c_{-1}$, nous serions en mesure de calculer n'importe quelle intégrale.
}

\parag{Définition: Rayon de convergence}{
    Le \important{rayon de convergence} est le plus grand nombre $R > 0$ tel que $f$ reste holomorphe dans $B_R\left(z_0\right)$.
}

\parag{Exemple 1}{
    Nous voulons trouver le rayon de convergence en $z_0 = 2 + i$ de la fonction: 
    \[f\left(z\right) = \frac{e^z}{\left(z^2 + 1\right)\left(z+1\right)}, \mathspace z \in \mathbb{C} \setminus \left\{\pm i, -1\right\}\]
    
    Nous pouvons faire le schéma suivant:
    \svghere[0.4]{ExempleRayonDeConvergence.svg}

    Nous voyons visuellement que $R = 2$.

    Cependant, si par exemple nous avions pris $z_0 = 0$, alors le rayon de convergence aurait été $R = 1$.
}

\parag{Exemple 2}{
    Considérons la fonction suivante: 
    \[f\left(z\right) = \log\left(z\right), \mathspace z \in \mathbb{C} \setminus \left\{x + yi \suchthat x \leq 0, y = 0\right\}\]
    
    Alors, en posant $z_0 = -2 + 2i$, nous trouvons un rayon de convergence de $R = 2$.

    \svghere[0.4]{ExempleRayonConvergenceLog.svg}
}

\parag{Corollaire: Règle de l'Hospital}{
    Soit $\Omega \subset \mathbb{C}$ un ouvert, $z_0 \in \Omega$, et $f, g : \Omega \mapsto\mathbb{C}$ des fonctions holomorphes telles que: 
    \[f\left(z_0\right) = g\left(z_0\right) = 0, \mathspace g'\left(z_0\right) \neq 0\]
    
    Alors, nous avons: 
    \[\lim_{z \to z_0} \frac{f\left(z\right)}{g\left(z\right)} = \frac{f'\left(z_0\right)}{g'\left(z_0\right)}\]

    \subparag{Preuve}{
        Ce corollaire découle directement du théorème précédent (d'où le fait que ce soit un corollaire), et il sera démontré en exercices.
    }
    
    \subparag{Remarque}{
        Ce corollaire est facilement généralisable: si $f\left(z_0\right) = f'\left(z_0\right) = \ldots = f^{\left(n\right)}\left(z_0\right) = 0$, $g\left(z_0\right) = g'\left(z_0\right) = \ldots = g^{\left(n\right)}\left(z_0\right) = 0$ et $g^{\left(n+1\right)}\left(z_0\right) \neq 0$, alors:
    \[\lim_{z \to z_0} \frac{f\left(z\right)}{g\left(z\right)} = \frac{f^{\left(n+1\right)}\left(z_0\right)}{g^{\left(n+1\right)}\left(z_0\right)}\]
    }
}

\parag{Observation}{
    Il serait sympa de pouvoir faire des séries de puissances (à la manière des séries de Taylor) autour de n'importe quel point. Nous pouvons le faire déjà partout où $f$ est holomorphe, mais nous voudrions généraliser cela à des points $z_0$ sur lequel $f$ n'est pas holomorphe ou même pas définie.

    Cette idée n'est pas complètement absurde. Considérons la fonction suivante, qui est holomorphe dans $\mathbb{C}$: 
    \[f\left(z\right) = e^{z}\]
    
    Nous savons qu'elle a une série de Taylor: 
    \[f\left(z\right) = 1 + z + \frac{z^2}{2} + \frac{z^3}{3!} + \ldots, \mathspace \forall z \in \mathbb{C}\]
    
    Considérons maintenant $\frac{e^z}{z}$, en divisant par $z$ des deux côtés: 
    \[\frac{f\left(z\right)}{z} = \frac{1}{z} + 1 + \frac{z}{2} + \frac{z^2}{3!} + \ldots, \mathspace \forall z \neq 0\]
    
    Considérons maintenant $\frac{e^z}{z^2}$: 
    \[\frac{e^z}{z^2} = \frac{1}{z^2} + \frac{1}{z} + \frac{1}{2} + \frac{z}{3!} + \frac{z^2}{4!} + \ldots, \mathspace \forall z \neq 0\]
    
    Nous obtenons des sommes qui ressemblent aux séries de Taylor, mais avec des termes non-polynomiaux. Cependant, nous avons tout de même trouvé que $\frac{e^z}{z}$ et $\frac{e^z}{z^2}$ admettent un développement en série autour de $z_0 = 0$ et avec un rayon de convergence infini.

    Nous allons donc développer de la théorie autour de développements de cette formes, appelés les séries de Laurent.
}

\parag{Définition: Série de Laurent}{
    Soit $\Omega \subset \mathbb{C}$ ouvert, $z_0 \in \Omega$, et $f: \Omega \setminus \left\{z_0\right\} \mapsto \mathbb{C}$. 

    La \important{série de Laurent} de $f$ en $z_0$, si elle existe, est: 
    \[L_{z_0} f\left(z\right) = \sum_{n=-\infty}^{\infty} c_n\left(z - z_0\right)^n = \ldots + c_{-1}\left(z-z_0\right)^{-1} + c_0 + c_1 \left(z-z_0\right)^1 + \ldots\]

    \subparag{Remarque}{
        Il existe une formule générale pour les coefficients $c_n$, mais elle n'est pas du tout pratique. Nous allons donc plutôt passer par les séries de Taylor pour les trouver.
    }
    
}

\parag{Définition: Partie régulière et singulière}{
    Soit $\Omega \subset \mathbb{C}$ ouvert, $z_0 \in \Omega$, et $f: \Omega \setminus \left\{z_0\right\} \mapsto \mathbb{C}$ dont la série de Laurent existe. 
    
    La \important{partie régulière} de la série de Laurent est: 
    \[\sum_{n=0}^{\infty} c_n\left(z - z_0\right)^n\]
    
    L'autre partie, la \important{partie singulière} de la série de Laurent, est: 
    \[\sum_{n=-\infty}^{-1} c_n\left(z-z_0\right)^n\]
}


\parag{Définition: Résidu}{
    Soit $\Omega \subset \mathbb{C}$ ouvert, $z_0 \in \Omega$, et $f: \Omega \setminus \left\{z_0\right\} \mapsto \mathbb{C}$ dont la série de Laurent existe. 

    Le \important{résidu} de $f$ en $z_0$ est le coefficient $c_{-1}$ de la série de Laurent de $f$ en $z_0$. Nous notons: 
    \[c_{-1} = \Res_{z_0}\left(f\right)\]

    \subparag{Remarque}{
        Nous avons réussi à faire du sens d'un $c_{-1}$, comme voulu. Nous y reviendrons.
    }
}

\parag{Théorème de Laurent}{
    Soit $\Omega \subset \mathbb{C}$ ouvert, $z_0 \in \Omega$, et $f: \Omega \setminus \left\{z_0\right\} \mapsto \mathbb{C}$ holomorphe.

    Alors, $f$ admet un développement en série de Laurent. En d'autres mots, $\exists \epsilon >0$ tel que, $\forall z \in B_{\epsilon}\left(z_0\right) \setminus \left\{z_0\right\}$:
    \[f\left(z\right) = \sum_{n=-\infty}^{\infty} c_n \left(z-z_0\right)^n  = L_{z_0} f\left(z\right)\]
}


\parag{Définition: Rayon de convergence}{
    Soit $\Omega \subset \mathbb{C}$ ouvert, $z_0 \in \Omega$, et $f: \Omega \setminus \left\{z_0\right\} \mapsto \mathbb{C}$. 

    Le \important{rayon de convergence}, $R > 0$, est le plus grand $\epsilon > 0$ tel que: 
    \[f\left(z\right) = L_{z_0} f\left(z\right), \mathspace \forall z \in B_R\left(z_0\right) \setminus \left\{z_0\right\}\]

    \subparag{Remarque}{
        Le calcul d'un tel rayon de convergence se fait exactement de la même manière que pour les séries de Taylor, sauf que nous ignorons la potentielle singularité en $z_0$.
    }
    
}

\parag{Définition: Point régulier}{
    Si la partie singulière de $L_{z_0}f$ est nulle, alors $z_0$ est appelé un \important{point régulier}. Dans ce cas, notre série de Laurent est une série de Taylor.
}

\parag{Définition: Pôle}{
    Si la partie singulière de $L_{z_0} f$ commence à la puissance $-m$ pour un certain $m \in \mathbb{N}^*$, $z_0$ est appelé un \important{pôle} d'ordre $m$.

    Dans ce cas, la partie singulière a un nombre fini de termes: 
    \[L_{z_0} f\left(z\right) = \sum_{n=-m}^{\infty} c_n \left(z-z_0\right)^n, \mathspace c_{-m} \neq 0\]
}

\parag{Définition: Singularité essentielle isolée}{
    Si la partie singulière de $L_{z_0}f$ admet un nombre infini de termes $c_{-m} \neq 0$, alors $z_0$ est appelé une \important{singularité essentielle isolée} (SEI).
}

\parag{Exemple 1}{
    Nous voulons trouver la série de Laurent et son rayon de convergence, la nature de $z_0$ et le résidu en $z_0 = 0$ pour la fonction: 
    \[f\left(z\right) = e^z\]
    
    Nous connaissons la série de Taylor de l'exponentielle: 
    \[f\left(z\right) = 1 + z + \ldots = \sum_{n=0}^{\infty} \frac{z^n}{n!}, \mathspace \forall z \in \mathbb{C}\]
    
    Nous trouvons donc que $L_{0}f\left(z\right) = T_{0} f\left(z\right)$. Puisque le rayon de convergence de la série de Taylor est $R = \infty$, c'est aussi le cas pour la série de Laurent. De plus, puisqu'il n'y a pas de partie singulière, $z_0$ est un point régulier. Finalement, le résidu est simplement: 
    \[\Res_0\left(f\right) = c_{-1} = 0\]
}

\parag{Exemple 2}{
    Nous voulons trouver la série de Laurent et son rayon de convergence, la nature de $z_0$ et le résidu en $z_0 = 0$ pour la fonction: 
    \[f\left(z\right) = \frac{1}{z}\]
    
    On remarque trivialement que notre fonction est directement sous la forme de la série de Laurent: 
    \[c_{-1} = 1, \mathspace c_n = 0, \forall\ n \neq 1\]
    
    Clairement, le rayon de convergence est $R = \infty$: pour tout $z \neq 0$, nous avons bien $\frac{1}{z} = \frac{1}{z}$. De plus, par définition, $z_0$ est un pôle d'ordre 1. Finalement, le résidu est donné par $\Res_0\left(f\right) = c_{-1} = 1$.
}


\end{document}
