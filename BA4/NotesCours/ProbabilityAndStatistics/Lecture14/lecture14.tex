% !TeX program = lualatex
% Using VimTeX, you need to reload the plugin (\lx) after having saved the document in order to use LuaLaTeX (thanks to the line above)

\documentclass[a4paper]{article}

% Expanded on 2023-04-06 at 13:18:30.

\usepackage{../../style}

\title{Probability and statistics}
\author{Joachim Favre}
\date{Jeudi 06 avril 2023}

\begin{document}
\maketitle

\lecture{14}{2023-04-06}{Linearity of expectation}{
\begin{itemize}[left=0pt]
    \item Proof that expectation is linear for any random variables.
    \item Proof that variance is linear for any independent random variables.
\end{itemize}

}

\begin{parag}{Theorem}
    Let $X, Y$ be random variables. If the required expectations exist, then: 
    \[\exval\left[g\left(X, Y\right)\right] = \exval_X\left[\exval\left[g\left(X, Y\right) | X = x\right]\right]\] 
    where $\exval_X$ represent expectation according to the distribution of $X$. 

    This means that we can freeze a random variable ($X$ here), and then average this result over this variable.

    \begin{subparag}{Proof}
        For the discrete case, we have that: 
        \autoeq{\exval_X\left[\exval\left[g\left(X, Y\right) | X = x\right]\right] = \sum_{x}^{} \exval\left[g\left(X, Y\right) | X = x\right]f_X\left(x\right) = \sum_{x}^{} \sum_{y}^{} g\left(x, y\right) f_{Y|X}\left(y|x\right) f_X\left(x\right) = \sum_{x}^{} \sum_{y}^{} g\left(x, y\right) f_{X, Y}\left(x, y\right) = \exval\left[g\left(X, Y\right)\right]}
    \end{subparag}
\end{parag}

\begin{parag}{Properties}
    Let $X$ and $Y$ be random variables. The following equality \textit{always} (even if they are dependent) holds: 
    \[\exval\left(X + Y\right) = \exval\left(X\right) + \exval\left(Y\right)\]
    \[\Var\left(X + Y\right) = \Var\left(X\right) + \Var\left(Y\right) + 2\Cov\left(X, Y\right)\]
    
    If moreover $X$ and $Y$ are independent: 
    \[\Var\left(X + Y\right) = \Var\left(X\right) + \Var\left(Y\right)\]
    
    \begin{subparag}{Proof 1}
        We want to show the first equality.

        By definition of $\exval\left[g\left(x, y\right)\right]$, we need $f_{X, Y}\left(x, y\right)$ to compute $\exval\left(X + Y\right)$. Thus, let us pick $Z = g\left(X, Y\right) = X + Y$. By definition: 
        \autoeq{\exval\left[X + Y\right] = \exval\left(Z\right) = \sum_{z}^{} z\prob\left(g\left(X, Y\right) = z\right) = \sum_{z}^{} z \prob\left(X + Y = z\right) = \sum_{z}^{} z \sum_{x}^{} \prob\left(X + Y = z | X = x\right)f_X\left(x\right) = \sum_{x}^{} \left[\sum_{z}^{} z \prob\left(X + Y = z | X = x\right)\right] f_X\left(x\right)}
        
        We consider the inner sum, by using the change of variable $u = z - x$: 
        \autoeq{\sum_{z}^{} z \prob\left(X + Y = z | X = x\right) = \sum_{z}^{} z \prob\left(Y = z - x | X = x\right) = \sum_{u}^{} \left(u + x\right) f_{Y|X}\left(u|x\right) = \sum_{u}^{} u f_{Y|X} \left(u|x\right) + x \sum_{u}^{} f_{Y|X}\left(u|x\right) = \exval\left(Y|X=x\right) + x}
        since $u$ is just a dummy variable.

        Coming back to our sum, we find 
        \autoeq{\exval\left[X + Y\right] = \sum_{x}^{} \left[\sum_{z}^{} z \prob\left(X + Y = z | X = x\right)\right] f_X\left(x\right) = \sum_{x}^{} \left[\exval\left(Y|X=x\right) + x\right] f_X\left(x\right) = \exval\left(Y\right) + \sum_{x}^{} x f_X\left(x\right) = \exval\left(Y\right) + \exval\left(X\right)}
    \end{subparag}

    \begin{subparag}{Proof 2}
        We want to show the first equality in a different way.

        We know that:
        \[\exval\left(g\left(X, Y\right)\right) = \sum_{x, y}^{} g\left(x, y\right)f_{X, Y}\left(x, y\right)\]
        
        Thus: 
        \autoeq{\exval\left(X + Y\right) = \sum_{x,y}^{} \left(x+y\right)f_{X, Y}\left(x, y\right) = \sum_{x}^{} x \sum_{y}^{} f_{X, Y}\left(x, y\right) + \sum_{y}^{} y \sum_{x}^{} f_{X, Y}\left(x, y\right) = \sum_{x}^{} x f_X\left(x\right) + \sum_{y}^{} y f_Y\left(y\right) = \exval\left(X\right) + \exval\left(Y\right)}
        
        This proof is simpler.

        \qed
    \end{subparag}
\end{parag}


\end{document}
